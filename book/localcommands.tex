%add all your local new commands to this file

\newcommand{\smiley}{:\)}

% ===========

% to get the U+Hexdecimal abbreviations looking good
\newcommand{\uni}[1]
{\@{\small \fontspec[Letters=Uppercase]{LinLibertineO}U+#1}\@}

\newcommand{\unif}[1]
{\@{\footnotesize \fontspec[Letters=Uppercase]{LinLibertineO}U+#1}\@} % small for footnote

% ===========

% use charisSIL in suitable size to fit other text
\newcommand{\charis}[1]
{{\small \fontspec{CharisSIL}#1}}

% use a Tamil font
\newcommand{\tamil}[1]
{{\fontspec{Noto Sans Tamil}#1}}

% use a Tamil font
\newcommand{\sinhala}[1]
{{\Large \fontspec{Sinhala MN}#1}}


% ===========

% mark lonely diacritics with a dotted circle
\newcommand{\dia}[1]
{{\fontspec{CharisSIL}{\large ◌}\symbol{"#1}}} % circle before diacritics

\newcommand{\diareverse}[1]
{{\fontspec{CharisSIL}\symbol{"#1}{\large ◌}}} % circle after diacritic

\newcommand{\diaf}[1]
{{\footnotesize \fontspec{CharisSIL}{\small ◌}\symbol{"#1}}} % small for footnote

% ===========

% scale texttt fontsize, for regular text and for footnotes
\renewcommand{\texttt}[1]
{{\small\ttfamily #1}}

\newcommand{\textttf}[1]
{{\scriptsize\ttfamily #1}} % small for footnote

% ===========

% manually specifying space before and after knitr chunks
\renewenvironment{knitrout}
  {\vspace{-0.5em} } % what happens before the code chunk
  {\vspace{-0.5em} } % what happens after the code chunk

