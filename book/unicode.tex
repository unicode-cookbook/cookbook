\documentclass[output=book,nonflat,modfonts,
colorlinks, citecolor=brown,
% draft,draftmode   
		]{langsci/langscibook}
\usepackage[]{graphicx}\usepackage[]{color}
%% maxwidth is the original width if it is less than linewidth
%% otherwise use linewidth (to make sure the graphics do not exceed the margin)
\makeatletter
\def\maxwidth{ %
  \ifdim\Gin@nat@width>\linewidth
    \linewidth
  \else
    \Gin@nat@width
  \fi
}
\makeatother

\definecolor{fgcolor}{rgb}{0.345, 0.345, 0.345}
\newcommand{\hlnum}[1]{\textcolor[rgb]{0.686,0.059,0.569}{#1}}%
\newcommand{\hlstr}[1]{\textcolor[rgb]{0.192,0.494,0.8}{#1}}%
\newcommand{\hlcom}[1]{\textcolor[rgb]{0.678,0.584,0.686}{\textit{#1}}}%
\newcommand{\hlopt}[1]{\textcolor[rgb]{0,0,0}{#1}}%
\newcommand{\hlstd}[1]{\textcolor[rgb]{0.345,0.345,0.345}{#1}}%
\newcommand{\hlkwa}[1]{\textcolor[rgb]{0.161,0.373,0.58}{\textbf{#1}}}%
\newcommand{\hlkwb}[1]{\textcolor[rgb]{0.69,0.353,0.396}{#1}}%
\newcommand{\hlkwc}[1]{\textcolor[rgb]{0.333,0.667,0.333}{#1}}%
\newcommand{\hlkwd}[1]{\textcolor[rgb]{0.737,0.353,0.396}{\textbf{#1}}}%
\let\hlipl\hlkwb

\usepackage{framed}
\makeatletter
\newenvironment{kframe}{%
 \def\at@end@of@kframe{}%
 \ifinner\ifhmode%
  \def\at@end@of@kframe{\end{minipage}}%
  \begin{minipage}{\columnwidth}%
 \fi\fi%
 \def\FrameCommand##1{\hskip\@totalleftmargin \hskip-\fboxsep
 \colorbox{shadecolor}{##1}\hskip-\fboxsep
     % There is no \\@totalrightmargin, so:
     \hskip-\linewidth \hskip-\@totalleftmargin \hskip\columnwidth}%
 \MakeFramed {\advance\hsize-\width
   \@totalleftmargin\z@ \linewidth\hsize
   \@setminipage}}%
 {\par\unskip\endMakeFramed%
 \at@end@of@kframe}
\makeatother

\definecolor{shadecolor}{rgb}{.97, .97, .97}
\definecolor{messagecolor}{rgb}{0, 0, 0}
\definecolor{warningcolor}{rgb}{1, 0, 1}
\definecolor{errorcolor}{rgb}{1, 0, 0}
\newenvironment{knitrout}{}{} % an empty environment to be redefined in TeX

\usepackage{alltt}

%%%%%%%%%%%%%%%%%%%%%%%%%%%%%%%%%%%%%%%%%%%%%%%%%%%%
%%%                                              %%%
%%%          additional packages                 %%%
%%%                                              %%%
%%%%%%%%%%%%%%%%%%%%%%%%%%%%%%%%%%%%%%%%%%%%%%%%%%%%

% put all additional commands you need in the 
% following files. If you do not know what this might 
% mean, you can safely ignore this section

\usepackage{verbatim}

%%%%%%%%%%%%%%%%%%%%%%%%%%%%%%%%%%%%%%%%%%%%%%%%%%%%
%%%                                              %%%
%%%                 Metadata                     %%%
%%%          fill in as appropriate              %%%
%%%                                              %%%
%%%%%%%%%%%%%%%%%%%%%%%%%%%%%%%%%%%%%%%%%%%%%%%%%%%%

\title{The Unicode Cookbook \linebreak for Linguists}
\subtitle{Managing writing systems using orthography profiles}
\BackTitle{Change backtitle in  localmetadata.tex}
\BackBody{Change backbody in  localmetadata.tex}
% \dedication{Change dedication in localmetadata.tex}
\typesetter{Change typesetter in localmetadata.tex}
\proofreader{Sandra Auderset, Aleksandrs Berdičevskis, Rosey Billington, Simon Cozens, Aniefon Daniel, Bev Erasmus, Amir Ghorbanpour, David Lukeš, Hugh Paterson III, Stephen Pepper, Katja Politt, Felix Rau, James Tauber, Luigi Talamo, Jeroen van de Weijer, Viola Wiegand, Alena Witzlack-Makarevich, and Esther Yap}
\author{Steven Moran \and Michael Cysouw}
\renewcommand{\lsISBNdigital}{978-3-96110-090-3 }                     
\renewcommand{\lsISBNhardcover}{978-3-96110-091-0}                     
\renewcommand{\lsSeries}{tmnlp}  
\renewcommand{\lsSeriesNumber}{10}  
\renewcommand{\lsID}{176}  

% add all extra packages you need to load to this file  

\usepackage{amsmath} 
\usepackage{unicode-math}
\setmathfont{Asana-Math.otf} % this looks much better for formulas
\usepackage[libertine]{newtxmath}

\usepackage{enumitem} % some additional possibilities for enumerations
\setitemize{noitemsep}
\setenumerate{noitemsep}

\usepackage{tabularx}
\usepackage{booktabs} % nice lines in tables
\usepackage{xtab} % xtabular for better multipage tables
\xentrystretch{-0.16} % squeeze more lines on a page
\usepackage{array} % for better handling of columns
\newcolumntype{L}[1]{>{\footnotesize\raggedright\let\newline\\\arraybackslash\hspace{0pt}}p{#1}}


\widowpenalty=10000
\clubpenalty=10000

% I have tried to use monospaced numbers in the TOC, but this does not work...
%\settocstylefeature{\fontspec[Numbers=Monospaced]{LinLibertineO}}
%\settocfeature{\fontspec[Numbers=Monospaced]{LinLibertineO}}

% other ideas that also don't give the desired results
%\usepackage{tocloft}
%\renewcommand{\cftXpagefont}{\fontspec[Numbers=Monospaced]{LinLibertineO}}


%%%%%%%%%%%%%%%%%%%%%%%%%%%%%%%%%%%%%%%%%%%%%%%%%%%%
%%%                                              %%%
%%%           Examples                           %%%
%%%                                              %%%
%%%%%%%%%%%%%%%%%%%%%%%%%%%%%%%%%%%%%%%%%%%%%%%%%%%% 

\usepackage{lsp-gb4e} 

%% to add additional information to the right of examples, uncomment the following line
% \usepackage{jambox}
%% if you want the source line of examples to be in italics, uncomment the following line
% \renewcommand{\exfont}{\itshape}

\usepackage{listings}

\lstset{ %
  backgroundcolor=\color{white},   % choose the background color; you must add \usepackage{color} or \usepackage{xcolor}
  basicstyle=\footnotesize\ttfamily,        % the size of the fonts that are used for the code 
  keywordstyle=\color{blue!60!black},       % keyword style
  language=XML,                 % the language of the code 
  stringstyle=\color{green!60!black},     % string literal style 
  morekeywords={token,xlink:href, Action, Value, Cursor,LogEvent}
} 
 
%% hyphenation points for line breaks
%% Normally, automatic hyphenation in LaTeX is very good
%% If a word is mis-hyphenated, add it to this file
%%
%% add information to TeX file before \begin{document} with:
%% %% hyphenation points for line breaks
%% Normally, automatic hyphenation in LaTeX is very good
%% If a word is mis-hyphenated, add it to this file
%%
%% add information to TeX file before \begin{document} with:
%% %% hyphenation points for line breaks
%% Normally, automatic hyphenation in LaTeX is very good
%% If a word is mis-hyphenated, add it to this file
%%
%% add information to TeX file before \begin{document} with:
%% \include{localhyphenation}
\hyphenation{
affri-ca-te
affri-ca-tes
com-ple-ments
}
\hyphenation{
affri-ca-te
affri-ca-tes
com-ple-ments
}
\hyphenation{
affri-ca-te
affri-ca-tes
com-ple-ments
}
%add all your local new commands to this file

\newcommand{\smiley}{:\)}

% to get the U+Hexdecimal abbreviations looking good
\newcommand{\uni}[1]
{\@{\small \fontspec[Letters=Uppercase]{LinLibertineO}U+#1}\@}

\newcommand{\unif}[1]
{\@{\footnotesize \fontspec[Letters=Uppercase]{LinLibertineO}U+#1}\@}


\bibliography{localbibliography}

\lstset{
  literate={ĉ}{{\^c}}1
           {á}{{\'a}}1
           {ã̌}{{\'a}}1
}

%%%%%%%%%%%%%%%%%%%%%%%%%%%%%%%%%%%%%%%%%%%%%%%%%%%%
%%%                                              %%%
%%%             Frontmatter                      %%%
%%%                                              %%%
%%%%%%%%%%%%%%%%%%%%%%%%%%%%%%%%%%%%%%%%%%%%%%%%%%%%
\IfFileExists{upquote.sty}{\usepackage{upquote}}{}
\begin{document}
\maketitle
\frontmatter
\addchap{Preface}
\begin{refsection}

%content goes here

\printbibliography[heading=subbibliography]
\end{refsection}


\addchap{Acknowledgments}
\begin{refsection}

%content goes here

\printbibliography[heading=subbibliography]
\end{refsection}


% \addchap{List of abbreviations}
\begin{refsection}

%content goes here

\printbibliography[heading=subbibliography]
\end{refsection}


\tableofcontents
\mainmatter%

%%%%%%%%%%%%%%%%%%%%%%%%%%%%%%%%%%%%%%%%%%%%%%%%%%%%
%%%                                              %%%
%%%             knitr settings                   %%%
%%%                                              %%%
%%%%%%%%%%%%%%%%%%%%%%%%%%%%%%%%%%%%%%%%%%%%%%%%%%%%

% Here are settings for knitr, which will be removed in the .tex file
% spacing of code-chunks is set with the "knitrout" environment in localcommands.tex



%%%%%%%%%%%%%%%%%%%%%%%%%%%%%%%%%%%%%%%%%%%%%%%%%%%%
%%%                                              %%%
%%%             Chapters                         %%%
%%%                                              %%%
%%%%%%%%%%%%%%%%%%%%%%%%%%%%%%%%%%%%%%%%%%%%%%%%%%%%
% 
% \chapter{Writing Systems} \label{writing_systems}

% \section{Introduction} % \label{introduction}

Writing systems arise and develop in a complex mixture of cultural,
technological and practical pressures. They tend to be highly conservative, in
that people who have learned to read and write in a specific way (however
impractical or tedious) are mostly unwilling to change their habits, e.g.~they
tend to resist spelling reforms. In all literate societies there exists a strong
socio-political mainstream that tries to force unification of writing (for
example by strongly enforcing ``right'' from ``wrong'' writing in schools).
However, there is also a large community of users who take as many liberties in
their writing as they can get away with.

For example, the writing of tone diacritics in Yoruba is often proclaimed to be
the right way to write, although many users of Yoruba writing seem to be
perfectly fine with leaving them out. As pointed out by the proponents of the
official rules, there are some homographs when leaving out the tone diacritics
\citet[44]{Olumuyiw2013}. However, writing systems (and the languages they
represent) are normally full of homophones, which is normally not a problem at
all for speakers of the language. More importantly, writing is not just a purely
functional tool, but just as importantly it is a mechanism to signal social
affiliation. By showing that you \textit{know the rules} of expressing yourself
in writing, others will more easily accept you as a worthy participant in their
group. And that just as well holds for obeying to the official rules when
writing a job application, as for obeying to the informal rules when writing an
SMS to classmates in school. The case of Yoruba writing is an exemplary case, as
even after more than a century of efforts to standardize the writing systems,
there is still a wide range of variation in daily use \citet{Olumuyiw2013}.

The sometimes cumbersome and sometimes illogical structure, and the enormous
variability of existing writing systems is a fact of life scholars have to
accept and should try to adapt to as good as possible. Our plea here is a
proposal for a formalization to do exactly that.

When considering the worldwide linguistic diversity, including all
lesser-studied and endangered languages, there exist numerous different
orthographies using symbols from the same scripts. For example, there are
hundreds of orthographies using Latin-based alphabetic scripts. All of these
orthographies use the same symbols, but these symbols differ in meaning and
usage throughout the various orthographies. To be able to computationally use
and compare different orthographies, we need a way to specify all orthographic
idiosyncrasies in a computer-readable format (a process called
\textsc{tailoring} in Unicode parlance). We call such specifications
\textsc{orthography profiles}. Ideally, these specifications have to be
integrated into so-called Unicode locale descriptions, though we will argue that
in practice this is often not the most useful solution for the kind of problems
arising in the daily practice of linguistics. Consequently, a central goal of
this paper is to flesh out the linguistic challenges for locale descriptions,
and work out suggestions to improve their structure for usage in a linguistic
context. Conversely, we also aim to improve linguists' understanding and
appreciation for the accomplishments of the Unicode Consortium in the
development of the Unicode Standard.

The necessity to computationally use and compare different orthographies most
forcefully arises in the context of language comparison. Concretely, in our
current research our goal is to develop quantitative methods for language
comparison and historical analysis in order to investigate worldwide linguistic
variation and to model the historical and areal processes that underlie
linguistic diversity,
cf.~\citet{Steiner_etal2011,List2012,List2012a,ListMoran2013,MoranProkic2013}.
In this work, it is crucial to be able to flexibly process across numerous
resources with different orthographies. In many cases even different resources
on the \textit{same} language use different orthographic conventions. Another
orthographic challenge that we encounter regularly in our linguistic practice is
electronic resources on a particular language that claim to follow a specific
orthographic convention (often a resource-specific convention), but on closer
inspection such resources are almost always not consistently encoded. Thus a
second goal of our orthography profiles is to allow for an easy specification of
orthographic conventions, and use such profiles to check consistency and to
report errors to be corrected.

A central step in our proposed solution to this problem is the tailored grapheme
separation of strings of symbols, a process we call \textsc{grapheme
tokenization}. Basically, given some strings of symbols (e.g.~morphemes, words,
sentences) in a specific source, our first processing step is to specify how
these strings have to be separated into graphemes, considering the specific
orthographic conventions used in a particular source document. Our experience is
that such a graphemic tokenization can be performed without extensive in-depth
knowledge about the phonetic and phonological details of the language in
question. For example, the specification that $<$ou$>$ is a grapheme of English
is a much easier task than to specify what exactly the phonetic values of this
grapheme are in any specific occurrence in English words. Grapheme separation is
a task that can be performed relatively reliably and with limited availability
of time and resources (compare, for example, the task of creating a complete
phonetic or phonological normalization).

Although grapheme tokenization is only one part of the solution, it is an
important and highly fruitful processing step. Given a grapheme tokenization,
various subsequent tasks become easier, like (a) temporarily reducing the
orthography in a processing pipeline, e.g.~only distinguishing high versus low
vowels; (b) normalizing orthographies across sources (often including temporary
reduction of oppositions), e.g.~specifying an (approximate) mapping to the
International Phonetic Alphabet; (c) using co-occurrence statistics across
different languages (or different sources in the same language) to estimate the
probability of grapheme matches, e.g.~with the goal to find regular sound
changes between related languages or transliterations between different sources
in the same language.

Before we deal with these proposals, in the first part of this paper (Sections
\ref{encoding} through \ref{ipa-meets-unicode}) we give an extended introduction
to the notion of encoding (Section \ref{encoding}) and writing systems, both
from a linguistic perspective and from the perspective of the Unicode Consortium
(Section \ref{terminology}). We consider the Unicode Standard to be a
breakthrough (and ongoing) development that fundamentally changed the way we
look at writing systems, and we aim to provide here a slightly more in-depth
survey of the many techniques that are available in the standard. A good
appreciation for the solutions that the Unicode Standard also allows for a
thorough understanding of possible pitfalls that one might encounter when using
it (Section \ref{unicode-pitfalls}). As an example of the current
state-of-the-art, we discuss the rather problematic marriage of the
International Phonetic Alphabet (IPA) with the Unicode Standard (Section
\ref{ipa-meets-unicode}).

The second part of the paper (Sections \ref{orthography-profiles} and
\ref{use-cases}) describes our proposals for how to deal with the Unicode
Standard in the daily practice of (comparative) linguists. First, we discuss the
challenges of characterizing a writing system. To solve these problems, we
propose the notions of orthography profiles, closely related to Unicode locale
descriptions (Section \ref{orthography-profiles}). Finally, we discuss practical
issues with actual examples (Section \ref{use-cases}). We provide reference
implementation of our proposals in R and in Python, available as open-source
libraries.

The following conventions are followed in this paper. All phonemic and phonetic
representations are given in the International Phonetic Alphabet (IPA), unless
noted otherwise \citep{IPA2005}. Standard conventions are used for
distinguishing between graphemic < >, phonemic / / and phonetic [ ]
representations. For character descriptions, we follow the notational
conventions of the Unicode Standard \citep{Unicode2014}. Character names are
represented in small capital letters (e.g.~\textsc{latin small letter schwa})
and code points are expressed as U\emph{+n} where \emph{n} is a four to six
digit hexadecimal number, e.g.~U+0256, which can be rendered as the glyph <ə>.

% % ==========================
\chapter{The Unicode approach}
\label{the-unicode-approach}
% ==========================

\section{Background}

The conceptualization and terminology of writing systems was rejuvenated by
the development of the Unicode Standard, with major input from Mark Davis,
co-founder and long-term president of the Unicode Consortium. For many years,
``exotic'' writing systems and phonetic transcription systems on personal
computers were constrained by the American Standard Code for Information
Interchange (ASCII) character encoding scheme, based on the Latin script, which
only allowed for a strongly limited number of different symbols to be encoded.
This implied that users could either use and adopt the (extended) Latin alphabet
or they could assign new symbols to the small number of code points in the ASCII
encoding scheme to be rendered by a specifically designed font
\citep{BirdSimons2003}. In this situation, it was necessary to specify the font
together with each document to ensure the rightful display of its content. To
alleviate this problem of assigning different symbols to the same code points,
in the late 80s and early 90s the Unicode Consortium set itself the ambitious
goal of developing a single universal character encoding to provide a unique
number, a code point, for every character in the world's writing systems.
Nowadays, the Unicode Standard is the default encoding of the technologies that
support the World Wide Web and for all modern operating systems, software and
programming languages.

\section{The Unicode Standard}

The Unicode Standard represents a massive step forward because it aims to
eradicate the distinction between universal (ASCII) versus language-particular
(font) by adding as much language-specific information as possible into the
universal standard. However, there are still language/resource-specific
specifications necessary for the proper usage of Unicode, as will be discussed
below. Within the Unicode structure many of these specifications can be captured
by so-called \textsc{Unicode Locales}, so we are moving to a new distinction
of universal (Unicode Standard) versus language-particular (Unicode Locale).
The major gain is much larger compatibility on the universal level (because
Unicode standardizes a much greater portion of writing system diversity), and
much better possibilities for automated processing on the language-particular
level (because Unicode Locales are machine-readable specifications).

Each version of the Unicode Standard (\citealp{Unicode2018}, as of writing at
version 11.0.0) consists of a set of specifications and guidelines that include (i) a
core specification, (ii) code charts, (iii) standard annexes and (iv) a
character database.\footnote{All documents of the Unicode Standard are available
at: \url{http://www.unicode.org/versions/latest/}. For a quick survey of the use
of terminology inside the Unicode Standard, their glossary is particularly
useful, available at: \url{http://www.unicode.org/glossary/}. For a general
introduction to the principles of Unicode, Chapter 2 of the core specification,
called \textsc{general structure}, is particularly insightful. Unlike 
many other documents in the Unicode Standard, this general introduction is
relatively easy to read and illustrated with many interesting examples from
various orthographic traditions from all over the world.} The \textsc{core
specification} is a book aimed at human readers that describes the formal
standard for encoding multilingual text. The \textsc{code charts} provide a
human-readable online reference to the character contents of the Unicode
Standard in the form of PDF files. The \textsc{Unicode Standard Annexes (UAX)}
are a set of technical standards that describe the implementation of the Unicode
Standard for software development, web standards, and programming languages.
The \textsc{Unicode Character Database (UCD)} is a set of computer-readable text
files that describe the character properties, including a set of rich character
and writing system semantics, for each character in the Unicode Standard. In
this section, we introduce the basic Unicode concepts, but we will leave out
many details. Please consult the above-mentioned full documentation for a more
detailed discussion. Further note that the Unicode Standard is exactly that,
namely a standard. It normatively describes notions and rules to be followed. In
the actual practice of applying this standard in a computational setting, a
specific implementation is necessary. The most widely used implementation of the
Unicode Standard is the \textsc{International Components for Unicode (ICU)},
which offers C/C++ and Java libraries implementing the Unicode
Standard.\footnote{More information about the ICU is available here:
\url{http://icu-project.org}.}

\section{Character encoding system}
\label{character-encoding-system}

The Unicode Standard is a \textsc{character encoding system} whose
goal is to support the interchange and processing of written characters and
text in a computational setting.\footnote{An insightful reviewer notes that the term \textsc{encoding} is used for both sequences of code points and text encoded as bit patterns. Hence a Unicode-aware programmer might prefer to say that UTF-8, UTF-16, etc., are Unicode encoding systems. The issue is that the Unicode Standard introduces a layer of indirection between characters and bit patterns, i.e.\ the code point, which can be encoded differently by different encoding systems.} Underlyingly, the character encoding is
represented by a range of numerical values called a \textsc{code space}, which
is used to encode a set of characters. A \textsc{code point} is a unique
non-negative integer within a code space (i.e.~within a certain numerical
range). In the Unicode Standard character encoding system, an \textsc{abstract
character}, for example the \textsc{latin small letter p}, is mapped to a
particular code point, in this case the decimal value 112, normally represented in
hexadecimal, which then looks in Unicode parlance as
\uni{0070}.
%\footnote{Hexadecimal (base-16) 0070 is equivalent to decimal
%(base-10) 112, which can be calculated by considering that $(0\cdot16^3) +
%(0\cdot16^2) + (7\cdot16^1) + (0\cdot16^0) = 7\cdot16 = 112$. Underlyingly,
%computers may treat this code point as an 8-bit binary (base-2) sequence (11100000), as
%can be seen by calculating that $(1\cdot2^7) + (1\cdot2^6) + (1\cdot2^5) +
%(0\cdot2^4) + (0\cdot2^3) + (0\cdot2^2) + (0\cdot2^1) + (0\cdot2^0) = 64 + 32 +
%16 = 112$.} 
That encoded abstract character is rendered on a computer screen (or
printed page) as a \textsc{glyph}, e.g.\ <p>, depending on the \textsc{font} and
the context in which that character appears.

In Unicode Standard terminology, an (abstract) \textsc{character} is the basic
encoding unit. The term \textsc{character} can be quite confusing due to its
alternative definitions across different scientific disciplines and because in
general the word \textsc{character} means many different things to different
people. It is therefore often preferable to refer to Unicode characters simply
as \textsc{code points}, because there is a one-to-one mapping between Unicode
characters and their numeric representation. In the Unicode approach, a
character refers to the abstract meaning and/or general shape, rather than a
specific shape, though in code tables some form of visual representation is
essential for the reader's understanding. Unicode defines characters as
abstractions of orthographic symbols, and it does not define visualizations for
these characters (although it does present examples). In contrast, a
\textsc{glyph} is a concrete graphical representation of a character as it
appears when rendered (or rasterized) and displayed on an electronic device or
on printed paper. For example, <g {\large \textit{g}} \textbf{g}
{\fontspec{ArialMT} {\small g} \textit{g} \textbf{g}}> are different glyphs of the
same character, i.e.~they may be rendered differently depending on the
typography being used, but they all share the same code point. From the
perspective of Unicode they are \textit{the same thing}. In this approach, a
\textsc{font} is then simply a collection of glyphs connected to code points.
Allography is not specified in Unicode (barring a few exceptional cases, due
to legacy encoding issues), but can be specified in a font as a
\textsc{contextual variant} (aka presentation form).

Each code point in the Unicode Standard is associated with a set of
\textsc{character properties} as defined by the Unicode character property
model.\footnote{The character property model is described in
\url{http://www.unicode.org/reports/tr23/}, but the actual properties are
described in \url{http://www.unicode.org/reports/tr44/}. A simplified overview
of the properties is available at: 
\url{http://userguide.icu-project.org/strings/properties}. The actual code
tables listing all properties for all Unicode code points are available at: 
\url{http://www.unicode.org/Public/UCD/latest/ucd/}.} Basically, those
properties are just a long list of values for each character. For example, code
point \uni{0047} has the following properties (among many others): 
\begin{itemize}
	\item Name: LATIN CAPITAL LETTER G 
	\item Alphabetic: YES 
	\item Uppercase: YES 
	\item Script: LATIN 
	\item Extender: NO 
	\item Simple\_Lowercase\_Mapping: 0067 
\end{itemize}

These properties contain the basic information of the Unicode Standard and they
are necessary to define the correct behavior and conformance required for
interoperability in and across different software implementations (as defined in
the Unicode Standard Annexes). The character properties assigned to each code
point are based on each character's behavior in real-world writing
traditions. For example, the corresponding lowercase character to \uni{0047} is
\uni{0067}.\footnote{Note that the relation between uppercase and lowercase is in
many situations much more complex than this, and Unicode has further
specifications for those cases.} Another use of properties is to define the
script of a character.\footnote{The Glossary of Unicode Terms defines the term \textsc{script} as
a ``collection of letters and other written signs used to represent textual
information in one or more writing systems. For example, Russian is written with
a subset of the Cyrillic script; Ukrainian is written with a different subset.
The Japanese writing system uses several scripts.''} In practice, script is
simply defined for each character as the explicit \textsc{script} property in
the Unicode Character Database.

One frequently referenced property is the \textsc{block} property, which is
often used in software applications to impose some structure on the large number
of Unicode characters. Each character in Unicode belongs to a specific block.
These blocks are basically an organizational structure to alleviate the
administrative burden of keeping Unicode up-to-date. Blocks consist of
characters that in some way belong together, so that characters are easier to
find. Some blocks are connected with a specific script, like the Hebrew block or
the Gujarati block. However, blocks are predefined ranges of code points, and
often there will come a point after which the range is completely filled. Any
extra characters will have to be assigned somewhere else. There is, for example,
a block \textsc{Arabic}, which contains most Arabic symbols. However, there is
also a block \textsc{Arabic Supplement}, \textsc{Arabic Presentation Forms-A}
and \textsc{Arabic Presentation Forms-B}. The situation with Latin symbols is
even more extreme. In general, the names of blocks should not be taken as a
definitional statement. For example, many IPA symbols are not located in the
aptly-named block \textsc{IPA extensions}, but in other blocks
(see Section~\ref{pitfall-no-complete-ipa-block}).

\section{Grapheme clusters}

There are many cases in which a sequence of characters (i.e.~a sequence of more
than one code point) represents what a user perceives as an individual unit in a
particular orthographic writing system. For this reason the Unicode Standard
differentiates between \textsc{abstract character} and \textsc{user-perceived
character}. Sequences of multiple code points that correspond to a single
user-perceived characters are called \textsc{grapheme clusters} in Unicode parlance.
Grapheme clusters come in two flavors: (default) grapheme clusters and tailored
grapheme clusters.

The (default) \textsc{grapheme clusters} are locale-independent graphemes,
i.e.~they always apply when a particular combination of characters occurs
independent of the writing system in which they are used. These character
combinations are defined in the Unicode Standard as functioning as one
\textsc{text element}.\footnote{The Glossary of Unicode Terms defines \textsc{text element} as:
``A minimum unit of text in relation to a particular text process, in the
context of a given writing system. In general, the mapping between text elements
and code points is many-to-many.''} The simplest example of a grapheme cluster
is a base character followed by a letter modifier character. For example, the
sequence <n> + <\dia{0303}> (i.e.~\textsc{latin small letter n} at \uni{006E}, followed
by \textsc{combining tilde} at \uni{0303}) combines visually into <ñ>, a
user-perceived character in writing systems like that of Spanish. In effect, what the
user perceives as a single character actually involves a multi-code-point
sequence. Note that this specific sequence can also be represented with a single
so-called \textsc{precomposed code point}, the \textsc{latin small letter n with
tilde} at \uni{00F1}, but this is not the case for all multi-code-point
character sequences. A solution to the problem of multiple encodings for the
same text element was developed early on in the Unicode Standard. It is called 
\textsc{canonical equivalence}, e.g.~for <ñ>, the sequence \uni{006E} \uni{0303} 
should in all situations be treated identically to the precomposed \uni{00F1}. By doing so, 
Unicode can also support special or precomposed characters in legacy character sets. 
To determine canonical equivalence, the Unicode Standard offers different kinds of normalization to either 
decompose precomposed characters (called \textsc{NFD} for \textsc{normalization form
canonical decomposition}) or to combine sequences of code points into precomposed 
characters (called \textsc{NFC} for \textsc{normalization form canonical composition}).\footnote{See the 
Unicode Standard Annex \#15, Unicode Normalization Forms (\url{http://unicode.org/reports/tr15/}), 
which provides a detailed description of normalization algorithms and illustrated examples.} 
In current software development practice, NFC seems to be preferred in most situations
and is widely proposed as the preferred canonical form. We discuss Unicode normalization 
in detail in Section \ref{pitfall-canonical-equivalence}.

More difficult for text processing, because less standardized, is what the
Unicode Standard terms \textsc{tailored grapheme clusters}.\footnote{\url{http://unicode.org/reports/tr29/}} 
Tailored grapheme clusters are locale-dependent graphemes, i.e.~such combination of characters do
not function as text elements in all situations. Examples include the sequence
<c>~+~<h> for the Slovak digraph <ch> and the sequence <ky> in the Sisaala
practical orthography, which is pronounced as IPA /tʃ/ \citep{Moran2006}. These grapheme
clusters are \textsc{tailored} in the sense that they must be specified on a
language-by-language or writing-system-by-writing-system basis. They are also 
grapheme clusters in these orthographies for processes such as collation (i.e.\ sorting).\footnote{\url{https://www.unicode.org/glossary/\#collation}}

The Unicode Standard provides technical specifications for creating locale specific data
in so-called \textsc{Unicode Locales}, i.e.~specifications 
that define a set of language-specific elements (e.g.~tailored grapheme
clusters, collation order, capitalization equivalence), as well as other special
information, like how to format numbers, dates, or currencies. Locale
descriptions are saved in the \textsc{Common Locale Data Repository
(CLDR)},\footnote{More information about the CLDR can be found here:
\url{http://cldr.unicode.org/}.} a repository of
language-specific definitions of writing system properties, each of which
describes specific usages of characters. Each locale can be encoded in a
document using the \textsc{Locale Data Markup Language (LDML)}. LDML is an XML
format and vocabulary for the exchange of structured locale data. Unicode Locale
Descriptions allow users to define language- or even resource-specific writing
systems or orthographies.\footnote{The Glossary of Unicode Terms defines \textsc{writing
system} only very loosely, as it is not a central concept in the Unicode
Standard. A writing system is, ``A set of rules for using one or more scripts to
write a particular language. Examples include the American English writing
system, the British English writing system, the French writing system, and the
Japanese writing system.''} However, Unicode Locales have various drawbacks 
for the daily practice of scientific linguistic research in a multilingual setting.

% \chapter{Unicode pitfalls}
\label{unicode-pitfalls}

% ==========================
\section{Wrong it ain't}
\label{wrong-it-is-not}
% ==========================

In this chapter we describe some of the most common pitfalls that we have
encountered when using the Unicode Standard in our own work, or in discussion
with other linguists. This section is not meant as a criticism of the decisions
made by the Unicode Consortium; rather we aim to highlight where the
technical aspects of the Unicode Standard diverge from many users'
intuitions. What have sometimes been referred to as problems or inconsistencies
in the Unicode Standard are mostly due to legacy compatibility issues, which can
lead to unexpected behavior by linguists using the standard. However, there are
also some cases in which the Unicode Standard has made decisions that
theoretically could have been made differently, but for some reason or another
(mostly very good reasons) were accepted as they are now. We call such behavior that
executes without error but does something different than the user
expected -- often unknowingly -- a \textsc{pitfall}.

In this context, it is important to realize that the Unicode Standard was not
developed to solve linguistic problems per se, but to offer a consistent
computational environment for written language. In those cases in which the
Unicode Standard behaves differently than expected, we think it is important not
to dismiss Unicode as wrong or deficient, because our
experience is that in almost all cases the behavior of the Unicode Standard has
been particularly well thought through. The Unicode Consortium has a 
wide-ranging view of matters and often examines important practical use cases
that are not normally considered from a linguistic point of view. Our general
guideline for dealing with the Unicode Standard is to accept it as it is, and
not to tilt at windmills. Alternatively, of course, it is possible to actively
engage in the development of the standard itself, an effort that is highly
appreciated by the Unicode Consortium.

% ==========================
\section{Pitfall: Characters are not glyphs}
\label{pitfall-characters-are-not-glyphs}
% ==========================

A central principle of Unicode is the distinction between character and glyph. A
character is the abstract notion of a symbol in a writing system, while a glyph
is the visual representation of such a symbol. In practice, there is a complex
interaction between characters and glyphs. A single Unicode character may of
course be rendered as a single glyph. However, a character may also be a piece
of a glyph, or vice-versa. Actually, all possible relations between glyphs and
characters are attested.

First, a single character may have different contextually determined glyphs. For
example, characters in writing systems like Hebrew and Arabic have different
glyphs depending on where they appear in a word. Some letters in Hebrew change
their form at the end of the word, and in Arabic, primary letters have four
contextually-sensitive variants (isolated, word initial, medial and final).

Second, a single character may be rendered as a sequence of multiple glyphs. For
example, in Tamil one Unicode character may result in a combination of a
consonant and vowel, which are rendered as two adjacent glyphs by fonts that
support Tamil, e.g.\ \textsc{tamil letter au} at \uni{0B94} represents a single 
character <\tamil{ஔ}>, composed of two glyphs <\tamil{ஓ}> and <\tamil{ன}>. Perhaps confusingly, 
in the Unicode Standard there are also two individual characters, 
i.e.\ \textsc{tamil letter oo} at \uni{0B93} and 
\uni{0BA9} \textsc{tamil letter nnna}, each of which is a glyph. Another example is 
Sinhala \textsc{sinhala vowel sign kombu deka} at \uni{0DDB} <\sinhala{ෛ}>, which is 
visually two glyphs, each represented by \textsc{sinhala vowel sign kombuva} 
at \uni{0DD9} <\sinhala{ ෙ}>.

Third, a single glyph may be a combination of multiple characters. For example, 
the ligature <fi>, a single glyph, is the result of two
characters, <f> and <i>, that have undergone glyph substitution by font
rendering (see also Section~\ref{pitfall-faulty-rendering}). Like
contextually-determined glyphs, ligatures are (intended) artifacts of text
processing instructions. Finally, a single glyph may be a part of a
character, as exemplified by diacritics like the diaeresis <\dia{0308}> in <ë>.

Further, the rendering of a glyph is dependent on the font being used. For
example, the Unicode character \textsc{latin small letter g} appears as <g> and
<{\fontspec{Courier}g}> in the Linux Libertine and Courier fonts, respectively,
because their typefaces are designed differently. Furthermore, the font face may
change the visual appearance of a character, for example Times New Roman
two-story <{\fontspec{Times New Roman}a}> changes to a single-story glyph in italics
<\emph{\fontspec{Times New Roman}a}>. This becomes a real problem for some
phonetic typesetting (see Section~\ref{pitfall-ipa-homoglyphs}).

In sum, character-to-glyph mappings are complex technical issues that the
Unicode Consortium has had to address in the development of the Unicode
Standard. However, they can can be utterly confusing for the lay user because visual
rendering does not (necessarily) indicate logical encoding.

% ==========================
\section{Pitfall: Characters are not graphemes}
\label{pitfall-characters-are-not-graphemes}
% ==========================

The Unicode Standard encodes characters. This becomes most clear with the notion of grapheme.
From a linguistic point of view, graphemes are the basic building blocks of a
writing system (see Section~\ref{linguistic-terminology}). It is extremely
common for writing systems to use 
combinations of multiple symbols (or letters) as a single grapheme, such as <sch>, <th> or <ei>.
There is no way to encode such complex graphemes using the Unicode Standard.

The Unicode Standard deals with complex graphemes only inasmuch as they consist of
base characters with diacritics (see
Section~\ref{pitfall-different-notions-of-diacritics} for a discussion of the
notion of diacritic). The Unicode Standard calls such combinations \textit{grapheme
clusters}. Complex graphemes consisting of multiple base characters,
like <sch>, are called \textit{tailored grapheme clusters} (see
Chapter~\ref{the-unicode-approach}).

There are special Unicode characters that 
glue together characters into larger tailored grapheme clusters,
specifically the \textsc{zero width joiner} at \uni{200D} and the
\textsc{combining grapheme joiner} at \uni{034F}. However, these characters are
confusingly named (cf.~Section~\ref{pitfall-names}). Both code points actually do
not join characters, but explicitly separate them. The zero-width joiner (ZWJ)
can be used to solve special problems related to ordering (called \textit{collation}
in Unicode parlance). The combining grapheme joiner (CGJ) can be used to
separate characters that are not supposed to form ligatures. 

To solve the issue of tailored grapheme clusters, Unicode offers some assistance
in the form of Unicode Locales.\footnote{\url{http://cldr.unicode.org/locale_faq-html}} 
However, in the practice of
linguistic research, this is not a real solution. To address this issue, we propose to
use orthography profiles (see Chapter~\ref{orthography-profiles}). Basically,
both orthography profiles and Unicode Locales offer a way to specify
tailored grapheme clusters. For example, for English one could specify that <sh>
is such a cluster. Consequently, this sequence of characters is then always
interpreted as a complex grapheme. For cases in which this is not the right
decision, like in the English word \textit{mishap}, the \textsc{zero width
joiner} at \uni{200D} has to be entered between <s> and <h>.

% ==========================
\section{Pitfall: Missing glyphs}
\label{pitfall-missing-glyphs}
% ==========================

The Unicode Standard is often praised (and deservedly so) for solving many of
the perennial problems with the interchange and display of the world's writing
systems. Nevertheless, a common complaint from users is that some symbols do not display 
correctly, i.e.\ \textit{not at all} or from a \textit{fall back font}, e.g.\ a 
rectangle <▯>, question mark <?>, or the Unicode replacement character <�>. 
The user's computer does not have the fonts 
installed that map the desired glyphs to Unicode characters. Therefore the glyphs cannot be displayed.
This is not the Unicode Standard's fault because it is a character 
encoding system and not a font. Computer-internally everything works as expected; 
any handling of Unicode code points works independently of how they 
are displayed on the screen. So although users might see
alien faces on display, they should not fret because everything is still 
technically in order below the surface.

There are two obstacles regarding missing glyphs. One is practical: 
designing glyphs includes many different considerations and 
it is a time-consuming process, especially when done well. 
Traditional expectations of what specific characters should look like need
to be taken into account when designing glyphs. Those expectations are often not
well documented, and it is mostly up to the knowledge and expertise of the font
designer to try and conform to them. Furthermore, the number of characters 
supported by Unicode is vast. Therefore, most designers 
produce fonts that only include glyphs for certain parts of the Unicode
Standard. 

The second obstacle is technical: the maximum number of glyphs that can be 
defined by the TrueType font standard and the OpenType specification 
(ISO/IEC 14496-22:2015) is 65,535. The current version of the Unicode Standard 
contains 137,374 characters. Thus, no single font can provide individual 
glyphs for all Unicode characters.

A simple solution to missing glyphs is to install additional fonts
providing additional glyphs. For broad coverage, there is the Noto font family, a project developed by Google, 
which covers over 100 scripts and nearly 64,000 characters.\footnote{\url{https://www.google.com/get/noto/}} 
The Unicode Consortium also provides, but does not endorse, an extensive list of fonts and font libraries online.\footnote{\url{http://unicode.org/resources/fonts.html}}

For the more exotic characters there is often not much choice. We have had success using 
Michael Everson's \textsc{Everson Mono} font, which has 9,756 different glyphs (not including 
Chinese)\footnote{\url{http://www.evertype.com/emono/}} and with the somewhat older \textsc{Titus Cyberbit Basic} font 
by Jost Gippert and Carl-Martin Bunz. It includes 10,044 different glyphs (not including 
Chinese).\footnote{\url{http://titus.fkidg1.uni-frankfurt.de/unicode/tituut.asp}} 

We also suggest installing at least one \textsc{fall-back
font}, which provides glyphs that show the user some information about
the underlying encoded character. Apple computers have such a font (which is
invisible to the user), which is designed by Michael Everson and made available
for other systems through the Unicode Consortium.\footnote{
\url{http://www.unicode.org/policies/lastresortfont\_eula.html}} Further, the
\textsc{GNU Unifont} is a clever way to produce bitmaps approximating the
intended glyph of each available character.\footnote{\url{http://unifoundry.com/unifont.html}} Finally,
SIL International provides a \textsc{SIL Unicode BMP Fallback
Font}. This font does not show a real 
glyph, but instead shows the hexadecimal code inside a box
for each character, so a user can at least see the Unicode code point of the
character intended for display.\footnote{\url{http://scripts.sil.org/UnicodeBMPFallbackFont}}

% ==========================
\section{Pitfall: Faulty rendering}
\label{pitfall-faulty-rendering}
% ==========================

A similar complaint to missing glyphs, discussed previously, is that while 
a glyph might be displayed, it does not look right. There are two
reasons for unexpected visual display, namely automatic font substitution and
faulty rendering. Like missing glyphs, any such problems are independent from
the Unicode Standard. The Unicode Standard only includes very general
information about characters and leaves the specific visual display for others to
decide on. Any faulty display is thus not to be blamed on the Unicode
Consortium, but on a complex interplay of different mechanisms happening in a
computer to turn Unicode code points into visual symbols. We will only sketch a
few aspects of this complex interplay here.

Most modern software applications (like Microsoft Word) offer some approach to
\textsc{automatic font substitution}. This means that when a text is written in
a specific font (e.g.~Times New Roman) and an inserted Unicode character does not
have a glyph within this font, then the software application will automatically
search for another font to display the glyph. The result will be that this
specific glyph will look slightly different from the others. This mechanism
works differently depending on the software application; only limited
user influence is usually expected and little feedback is given. This may be rather
frustrating to font-aware users.

% \footnote{For example, Apple Pages does not give any feedback that a font is being replaced, and the user does not seem to have any influence on the choice of replacement (except by manually marking all occurrences). In contrast, Microsoft Word does indicate the font replacement by showing the name in the font menu of the font replacement. However, Word simply changes the font completely, so any text written after the replacement is written in a different font as before. Both behaviors leave much to be desired.}

Another problem with visual display is related to so-called \textsc{font
rendering}. Font rendering refers to the process of the actual positioning of
Unicode characters on a page of written text. This positioning is actually a
highly complex challenge and many things can go wrong in the process. Well-known
rendering difficulties, like proportional glyph size or ligatures, are reasonably
well understood by developers. Nevertheless, the positioning of multiple diacritics relative to
a base character is still a widespread problem. Especially problematic is when 
more than one diacritic is supposed to be placed above (or
below) another. Even within the Latin script vertical placement 
often leads to unexpected effects in many modern software applications. 
The rendering problems arising in Arabic and in many scripts of Southeast
Asia (like Devanagari or Burmese) are even more complex. 

To understand why these problems arise it is important to realize that there are
basically three different approaches to font rendering. The most widespread is
Adobe's and Microsoft's \textsc{OpenType} system. This approach makes it
relatively easy for font developers, but the font itself does not include all
details about the precise placement of individual characters. For those details,
additional script descriptions are necessary. All such systems can lead to
unexpected behavior.\footnote{For more details about OpenType, see
\url{http://www.adobe.com/products/type/opentype.html} and
\url{http://www.microsoft.com/typography/otspec/}. Additional systems for
complex text layout are, among others, Microsoft's DirectWrite
(\url{https://msdn.microsoft.com/library/dd368038.aspx}) and the open-source
project HarfBuzz (\url{http://www.freedesktop.org/wiki/Software/HarfBuzz/}).}
Alternative systems are \textsc{Apple Advanced Typography} (AAT) and the
open-source \textsc{Graphite} system produced and maintained by the Non-Roman Script Initiative of SIL International 
(SIL).\footnote{More information about AAT can be found at:
\url{https://developer.apple.com/fonts/}. \newline Graphite is described
in detail at:
\url{http://scripts.sil.org/default}.}
In these systems, a larger burden is placed on the description inside
the font.

There is no complete solution to the problems arising from faulty font rendering.
Switching to another software application that offers better handling is the
only real alternative, but this is normally not an option for daily work. 
Font rendering is developing quickly in the software industry, so we can expect 
the situation to only get better. % In the meantime one 
% can try to correct faulty layout by tweaking baseline and/or kerning (when such 
% option are available).

% ==========================
\section{Pitfall: Blocks}
\label{pitfall-blocks}
% ==========================

The Unicode code space is subdivided into blocks of contiguous code points. For
example, the block called \textsc{Cyrillic} runs from \uni{0400} till
\uni{04FF}. These blocks arose as an attempt at ordering the enormous number of
characters in Unicode, but the idea of blocks very quickly ran into problems.
First, the size of a block is fixed, so when a block is full, a new block will
have to be instantiated somewhere else in the code space. For example, this
led to the blocks \textsc{Cyrillic Supplement}, \textsc{Cyrillic Extended-A}
(both of which are already full) and \textsc{Cyrillic Extended-B}. Second,
when a specific character already exists, it is not duplicated in another
block, although the name of the block might indicate that a specific symbol
should be available there. In general, names of blocks are just an approximate
indication of the kind of characters that will be in the block.

The problem with blocks arises because finding the right character among the
thousands of Unicode characters is not easy. Many software applications present
blocks as a primary search mechanism, because the block names suggest where to
look for a particular character. However, when a user searches for an IPA
character in the block \textsc{IPA Extensions}, then many IPA characters will not
be found there. For example, the velar nasal <ŋ> is not part of the block
\textsc{IPA Extensions} because it was already included as \textsc{latin small letter
eng} at \uni{014B} in the block \textsc{Latin Extensions-A}.

In general, finding a specific character in the Unicode Standard is often non-trivial. 
The names of the blocks can help, but they are not (and were never supposed
to be) a foolproof structure. It is neither the goal nor the aim of the Unicode
Consortium to provide a user interface to the Unicode Standard. If one often
encounters the problem of needing to find a suitable character, there are
various other useful services for end-users available.\footnote{The Unicode
website offers a basic interface to the code charts at:
\url{http://www.unicode.org/charts/index.html}. As a more flexible interface, we
particularly like PopChar from Ergonis Software, available for both Mac and
Windows. There are also various free websites that offer search interfaces
to the Unicode code tables, like \url{http://unicode-search.net} or
\url{http://unicode-search.net}. Another useful approach for searching for characters
using shape matching \citep{Belongie2002} is: \url{http://shapecatcher.com}.}

% ==========================
\section{Pitfall: Names}
\label{pitfall-names}
% ==========================

The names of characters in the Unicode Standard are sometimes misnomers and
should not be misinterpreted as definitions. For example, the \textsc{combining
grapheme joiner} at \uni{034F} does not join characters into larger graphemes
(see Section~\ref{pitfall-characters-are-not-graphemes}) and the \textsc{latin
letter retroflex click} \uni{01C3} is actually not the IPA symbol for a
retroflex click, but for an alveolar click (see
Section~\ref{pitfall-ipa-homoglyphs}). In a sense, these names can be seen as
errors. However, it is probably better to realize that such names are just
convenience labels that are not going to be changed. Just like the block names
(Section~\ref{pitfall-blocks}), the character names are often helpful, but they
are not supposed to be definitions.

The actual intended meaning of a Unicode code point is a combination of the
name, the block and the character properties (see
Chapter~\ref{the-unicode-approach}). Further details about the underlying intentions 
with which a character should be used
are only accessible by perusing the actual decisions of the Unicode Consortium.
All proposals, discussions and decisions of the Unicode Consortium are publicly
available. Unfortunately there is not (yet) any way to easily find everything
that is ever proposed, discussed and decided in relation to a specific
code point of interest, so many of the details are often somewhat
hidden.\footnote{All proposals and other documents that are the basis of Unicode
decisions are available at: \url{http://www.unicode.org/L2/all-docs.html}. The
actual decisions that make up the Unicode Standard are documented in the minutes
of the Unicode Technical Committee, available at: 
\url{http://www.unicode.org/consortium/utc-minutes.html}.}

% ==========================
\section{Pitfall: Homoglyphs}
\label{pitfall-homoglyphs}
% ==========================

Homoglyphs are visually indistinguishable glyphs (or highly similar glyphs) that
have different code points in the Unicode Standard and thus different character
semantics. As a principle, the Unicode Standard does not specify how a character
appears visually on the page or the screen. So in most cases, a different
appearance is caused by the specific design of a font, or by user-settings like
size or boldface. Taking an example already discussed in
Section~\ref{character-encoding-system}, the following symbols <g {\large \textit{g}}
\textbf{g} {\fontspec{ArialMT} {\small g} \textit{g} \textbf{g}}> are different
glyphs of the same character, i.e.~they may be rendered differently depending on
the typography being used, but they all share the same code point (viz.
\textsc{latin small letter g} at \uni{0067}). In contrast, the symbols
<{\fontspec{EversonMono}AАΑᎪᗅᴀꓮ𐊠𝖠𝙰}> are all different code points,
although they look highly similar -- in some cases even sharing exactly the same
glyph in some fonts. All these different A-like characters include the following
code points in the Unicode Standard:

\begin{itemize}
	\item[] <{\fontspec{EversonMono}A}> \textsc{latin capital letter a}, at \uni{0041} 
	\item[] <{\fontspec{EversonMono}А}> \textsc{cyrillic capital letter a}, at \uni{0410} 
	\item[] <{\fontspec{EversonMono}Α}> \textsc{greek capital letter alpha}, at \uni{0391} 
	\item[] <{\fontspec{EversonMono}Ꭺ}> \textsc{cherokee letter go}, at \uni{13AA} 
	\item[] <{\fontspec{EversonMono}ᗅ}> \textsc{canadian syllabics carrier gho}, at \uni{15C5} 
	\item[] <{\fontspec{EversonMono}ᴀ}> \textsc{latin small letter capital a}, at \uni{1D00} 
	\item[] <{\fontspec{EversonMono}ꓮ}> \textsc{lisu letter a}, at \uni{A4EE} 
%	\item[] <{\fontspec{EversonMono}A}> \textsc{fullwidth latin capital letter a}, at \uni{FF21} 
	\item[] <{\fontspec{EversonMono}𐊠}> \textsc{carian letter a}, at \uni{102A0} 
%	\item[] <{\fontspec{EversonMono}̀}> \textsc{old italic letter a}, at \uni{10300} 
	\item[] <{\fontspec{EversonMono}𝖠}> \textsc{mathematical sans-serif capital a}, \uni{1D5A0} 
	\item[] <{\fontspec{EversonMono}𝙰}> \textsc{mathematical monospace capital a}, at \uni{1D670} 
\end{itemize}

The existence of such homoglyphs is partly due to legacy compatibility, but for
the most part these characters are simply different characters that happen to
look similar.\footnote{A particularly nice interface to look for homoglyphs is
\url{http://shapecatcher.com}, based on the principle of recognizing shapes
\citep{Belongie2002}.} Yet, they are suppose to behave differently from the
perspective of a font designer. For example, when designing a Cyrillic font, the
<A> will have different aesthetics and different traditional expectations
compared to a Latin <A>. Thus, the Unicode Standard has character properties 
associated with each code point which define certain expectations, e.g.\ characters 
belong to different blocks, they have different lower case variants (see 
Section \ref{character-encoding-system}).

Homoglyphs are a widespread problem for consistent encoding. Although for
most users it looks like the words <voces> and <νοсеѕ> are nearly identical, in 
fact they do not share any code points.\footnote{The first words
consists completely of Latin characters: \unif{0076}, \unif{006F},
\unif{0063}, \unif{0065} and \unif{0073}. The second is a mix of Cyrillic
and Greek characters: \unif{03BD}, \unif{03BF}, \unif{0041}, \unif{0435}
and \unif{0455}.} For computers these two words are completely different
entities. Sometimes when users with Cyrillic or Greek keyboards have to type
some Latin-based orthography, they mix similar looking Cyrillic or Greek
characters into their text, because those characters are so much easier to type.
Similarly, when users want to enter an unusual symbol, they normally search by
visual impression in their favorite software application, and just pick
something that looks reasonably alike to what they expect the glyph to look
like.

It is very easy to make errors during text entry and add characters that are 
not supposed to be included. Our proposals for orthography profiles (see
Chapter~\ref{orthography-profiles}) are a method for checking the consistency of 
any text. In situations in which interoperability is important, we consider it 
crucial to add such checks in any workflow.

% ==========================
\section{Pitfall: Canonical equivalence}
\label{pitfall-canonical-equivalence}
% ==========================

For some characters, there is more than one possible encoding in the Unicode
Standard. This means that for the computer
there exists multiple different entities, which for the user, may be visually the same. This
leads to, for example, problems with search. The computer searches for specific 
code points and by design does not return all visually similar characters.
As a solution, the Unicode Standard includes a notion of \textsc{canonical
equivalence}. Different encodings are explicitly declared as equivalent in the
Unicode Standard code tables. Further, to harmonize all encodings in a specific
piece of text, the Unicode Standard proposes a mechanism of
\textsc{normalization}. The process of normalization and the 
Unicode Normalization Forms are described 
in detail in the Unicode Standard Annex \#15 online.\footnote{\url{http://unicode.org/reports/tr15/}} 
Here we provide a brief summary of that material as it pertains to canonical equivalence.

Consider for example the characters and following Unicode code points:
\begin{enumerate}
	\def\labelenumi{\arabic{enumi}.} 
	\item <Å> \textsc{latin capital letter a with ring above} \uni{00C5} 
	\item <Å> \textsc{angstrom sign} \uni{212B}
	\item <Å> \textsc{latin capital letter a} \uni{0041}
	+ \textsc{combining ring above} \uni{030A}
\end{enumerate}

\noindent The character, represented here by glyph <Å>, is encoded in the Unicode Standard
in the first two examples by a single-character sequence; each is assigned a
different code point. In the third example, the glyph is encoded in a
multiple-character sequence that is composed of two character code points. All
three sequences are \textsc{}, i.e.~they are strings that
represent the same abstract character and because they are not distinguishable
by the user, the Unicode Standard requires them to be treated the same in
regards to their behavior and appearance. Nevertheless, they are encoded
differently. For example, if one were to search an electronic text (with
software that does not apply Unicode Standard normalization) for
\textsc{angstrom sign} (\uni{212B}), then the instances of \textsc{latin 
capital letter a with ring above} (\uni{00C5}) would not be found.

In other words, there are equivalent sequences of Unicode characters that should
be normalized, i.e.~transformed into a unique Unicode-sanctioned representation
of a character sequence called a \textsc{normalization form}. Unicode provides a
Unicode Normalization Algorithm, which puts combining marks
into a specific logical order and it defines decomposition and composition
transformation rules to convert each string into one of four normalization
forms. We will discuss here the two most relevant normalization forms: NFC and
NFD.

The first of the three characters above is considered the \textsc{Normalization
Form C (NFC)}, where \textsc{C} stands for composition. When the process of NFC
normalization is applied to the characters in 2 and 3, both 
are normalized into the \textsc{pre-composed} character sequence in 1. Thus all
three canonical character sequences are standardized into one composition form
in NFC. The other frequently encountered Unicode normalization form is the
\textsc{Normalization Form D (NFD)}, where \textsc{D} stands for decomposition.
When NFD is applied to the three examples above, all three, including
importantly the single-character sequences in 1 and 2, are normalized into the
\textsc{decomposed} multiple-sequence of characters in 3. Again, all three are
then logically equivalent and therefore comparable and syntactically
interoperable.

As illustrated, some characters in the Unicode Standard have alternative
representations (in fact, many do), but the Unicode Normalization Algorithm can
be used to transform certain sequences of characters into canonical
forms to test for equivalency. To determine equivalence, each
character in the Unicode Standard is associated with a combining class, which is
formally defined as a character property called \textsc{canonical combining
class} which is specified in the Unicode Character Database. The combining class
assigned to each code point is a numeric value between 0 and 254 and is used by
the Unicode Canonical Ordering Algorithm to determine which sequences of
characters are . Normalization forms, as very briefly
described above, can be used to ensure character equivalence by ordering
character sequences so that they can be faithfully compared.

It is very important to note that any software application that is Unicode
Standard compliant is free to change the character stream from one
representation to another. This means that a software application may compose,
decompose or reorder characters as its developers desire; as long as the
resultant strings are  to the original. This might lead to
unexpected behavior for users. Various players, like the Unicode Consortium, the
W{\large 3}C, or the TEI recommend NFC in most user-directed situations, and
some software applications that we tested indeed seem to automatically convert
strings into NFC.\footnote{See the summary of various recommendation here:
\url{http://www.win.tue.nl/~aeb/linux/uc/nfc_vs_nfd.html}.} This means in
practice that if a user, for example, enters <a> and <\dia{0300}>, i.e.~\textsc{latin
small letter a} at \uni{0061} and \textsc{combining grave accent} at \uni{0300},
this might be automatically converted into <à>, i.e.~\textsc{latin small letter
a with grave} at \uni{00E0}.\footnote{The behavior of software applications can
be quite erratic in this respect. For example, Apple's TextEdit does not do any
conversion on text entry. However, when you copy and paste some text inside the
same document in rich text mode (i.e.~RTF-format), it will be transformed into
NFC on paste. Saving a document does not do any conversion to the glyphs on
screen, but it will save the characters in NFC.}


% ==========================
\section{Pitfall: Absence of canonical equivalence}
\label{pitfall-absence-of-equivalence}
% ==========================

Although in most cases canonical equivalence will take care of alternative
encodings of the same character, there are some cases in which the Unicode
Standard decided against equivalence. This leads to identical characters that
are not equivalent, like <ø> \textsc{latin small letter o with stroke} at
\uni{00F8} and <o̷> a combination of \textsc{latin small letter o} at \uni{006F}
with \textsc{combining short solidus overlay} at \uni{0037}.
The general rule followed is that extensions of Latin characters that are
connected to the base character are not separated as combining diacritics. For
example, characters like <ŋ ɲ ɳ> or <ɖ ɗ> are obviously derived from <n> and <d>
respectively, but they are treated like new separate characters in the Unicode
Standard. Likewise, characters like <ø> and <ƈ> are not separated into a base 
character <o> and <c> with an attached combining diacritic.

Interestingly, and somewhat illogically, there are three elements which are
directly attached to their base characters, but which are still treated as
separable in the Unicode Standard. Such characters are decomposed (in NFD
normalization) into a base character with a combining diacritic. However, it is
these cases that should be considered the exceptions to the rule. These three 
elements are the following:

\begin{itemize}

  \item <\dia{0327}>: the \textsc{combining cedilla} at \uni{0327} \newline 
        This diacritic is
        for example attested in the precomposed character <ç> \textsc{latin
        small letter c with cedilla} at \uni{00E7}. This <ç> will thus be
        decomposed in NFC normalization.
  \item <\dia{0328}>: the \textsc{combining ogonek} at \uni{0328} \newline 
        This diacritic is
        for example attested in precomposed <ą> \textsc{latin small letter a
        with ogonek} at \uni{0105}. This <ą> will thus be decomposed in NFC
        normalization.
  \item <\dia{031B}>: the \textsc{combining horn} at \uni{031B} \newline 
        This diacritic is for
        example attested in precomposed <ơ> \textsc{latin small letter o with
        horn} at \uni{01A1}. This <ơ> will thus be decomposed in NFC
        normalization. 

\end{itemize}

There are further combinations that deserve special care because it is actually
possible to produce identical characters in different ways without them being
. In these situations, the general rule holds, namely that
characters with attached extras are not decomposed. However, in the following
cases the extras actually exist as combining diacritics, so there is also 
the possibility to construct a character by using a base character with those 
combining diacritics.

\begin{itemize}
  
  \item First, there are the combining characters designated as \textit{combining
        overlay} in the Unicode Standard, like <\dia{0334}>
        \textsc{combining tilde overlay} at \uni{0334} or <\dia{0335}>
        \textsc{combining short stroke overlay} at \uni{0335}. There are many
        characters that look like they are precomposed with such an overlay,
        for example <\charis{ɫ~ᵬ~ᵭ~ᵱ}> or <\charis{ƚ~ɨ~ɉ~ɍ}>, or also the
        example of <ø> given at the start of this section. However, they are 
        not decomposed in NFD normalization.
  \item Second, the same situation also occurs with combining characters
        designated as \textit{combining hook}, like 
        <{\fontspec{CharisSIL}{\large ◌}}\symbol{"0321}> \textsc{combining
        palatalized hook below} at \uni{0321}. This element seems to occur in
        precomposed characters like <\charis{ᶀ~ᶁ~ᶂ~ᶄ}>. However, they are 
        not decomposed in NFD normalization.
        
\end{itemize}

To harmonize the encoding in these cases it is not sufficient to use Unicode 
normalization. Additional checks are necessary, for example by using orthography 
profiles (see Chapter~\ref{orthography-profiles}).


% ==========================
\section{Pitfall: Encodings}
\label{encodings}
% ==========================

% Section "Pitfall: File formats" may profit most from a more transparent terminology, distinguishing levels of encoding, rather than talking about how texts "appear inside some kind of computer file".

Before we discuss the pitfall of different file formats in Section \ref{pitfall-file-formats}, it is pertinent to point out that the common usage of the term \textsc{encoding} unfortunately does not distinguish between \textit{encoded} sequences of code points and text \textit{encoded} as bit patterns. Recall, a code point is simply a numerical representation of some defined entity; in other words, a code point is a character encoding-specific unique identifier or ID. In the Unicode Standard encoding, code points are numbers that serve as unique identifiers, each of which is associated with a set of character properties defined by the Unicode Consortium in the Unicode Character Database.\footnote{\url{https://www.unicode.org/ucd/}} The number of each code point can be \textit{encoded} in various formats, including as a decimal integer (e.g.\ 112), as an 8-bit binary sequence (01110000) or hexadecimal (0070). This example Unicode code point, \uni{0070}, represents \textsc{latin small letter p} and its associated Unicode properties, such as it belongs to the category Letter, Lowercase [Ll], in the Basic Latin block, and that its title case and upper case is associated with code point \uni{0050}.\footnote{See also Chapter \ref{the-unicode-approach}.}

% 0070;LATIN SMALL LETTER P;Ll;0;L;;;;;N;;;0050;;0050

The other meaning of encoding has to do with the fact that computers represent data and instructions in patterns of bits. A bit pattern is a combination of binary digits arranged in a sequence. And how these sequences are carved up into bit patterns is determined by how they are \textit{encoded}. Thus the term \textsc{encoding} is used for both sequences of code points and text encoded as bit patterns. Hence a Unicode-aware programmer might prefer to say that UTF-8, UTF-16, etc., are Unicode encoding systems because they determine how sequences of bit patterns are determined, which are then mapped to characters.\footnote{UTF stands for Unicode Transformation Format. It a method for translating numbers into binary data and vice versa. There are several different UTF encoding formats, e.g.\ UTF-8 is a variable-length encoding that uses 8-bit code units, is compatible with ASCII, and is common on the web. UTF-16 is also variable-length, uses 16-bit code units, and is used system-internally by Windows and Java. See further discussion under \textit{Code units} in Section \ref{pitfall-file-formats}. For more in-depth discussion, refer to the Unicode Frequently Asked Questions and additional sources therein: \url{http://unicode.org/faq/utf_bom.html}.} The terminological issue here is that the Unicode Standard introduces a layer of indirection between characters and bit patterns, i.e.\ the code point, which can be encoded differently by different encoding systems.

% Characters encoded, but not seen.
Note also that all computer character encodings include so-called \textsc{control characters}, which are non-printable sometimes action-inducing characters, such as the null character, bell code, backspace, escape, delete, and line feed. Control characters can interact with encoding schemes. For example, some programming languages make use of the null character to mark the end of a string. Line breaks are part of the text, and as such as covered by the Unicode Standard. But they can be problematic because line breaks differ from operating system to operating system in how they are encoded. These variants are discussed in Section \ref{pitfall-file-formats}.

% This section should also mention that line breaks are actually part of the text, and as such also covered by the Unicode standard. Again regarding line breaks, the recommendation "that everybody use this encoding [only LF] whenever possible" (page 31) seems less well motivated (and specific) than most other recommendations in the book. Some common formats, e.g.\ CSV (as specified by  RFC 4180 [4]), specify "CRLF" as line break. Thus, the perceived "strong tendency" towards "LF" may be just that.


% ==========================
\section{Pitfall: File formats}
\label{pitfall-file-formats}
% ==========================

Unicode is a character encoding standard, but characters of course 
appear inside some kind of computer file. The most basic Unicode-based file
format is pure line-based text, i.e.~strings of Unicode-encoded characters
separated by line breaks (note that these line breaks are what for most people
intuitively corresponds to paragraph breaks). Unfortunately, even within this
apparently basic setting there exists a multitude of variants. In general these
different possibilities are well-understood in the software industry, and
nowadays they normally do not lead to any problems for the end user. However,
there are some situations in which a user is suddenly confronted with cryptic
questions in the user interface involving abbreviations like LF, CR, BE, LE or
BOM.\@ Most prominently this occurs with exporting or importing data in several
software applications from Microsoft. Basically, there are two different issues
involved. First, the encoding of line breaks and, second, the encoding of the
Unicode characters into code units and the related issue of endianness.

\subsubsection*{Line breaks}

The issue with \textsc{line breaks} originated with the instructions necessary
to direct a printing head of a physical printer to a new line. This involves two
movements, known as \textsc{carriage return} (CR, returning the printing head to
the start of the line on the page) and \textsc{line feed} (LF, moving the
printing head to the next line on the page). Physically, these are two different
events, but conceptually together they form one action. In the history of
computing, various encodings of line breaks have been used (e.g.~CR+LF, LF+CR,
only LF, or only CR). Currently, all Unix and Unix-derived systems use only LF
as code for a line break, while software from Microsoft still uses a combination
of CR+LF.\@ Today, most software applications recognize both options, and
are able to deal with either encoding of line breaks (until rather recently this
was not the case, and using the wrong line breaks would lead to unexpected
errors). Our impression is that there is a strong tendency in software
development to standardize on the simpler ``only LF'' encoding for line
breaks, and we suggest that everybody should use this encoding whenever possible.

\subsubsection*{Code units}

The issue with \textsc{code units} stems from the question how to separate a
stream of binary ones and zeros, i.e.~bits, into chunks representing Unicode
characters. A code unit is the sequence of bits used to encode a single
character in an encoding. The Unicode Standard offers three different approaches, 
called UTF-32, UTF-16 and UTF-8, that are intended for different use cases.\footnote{The
letters UTF stand for \textsc{Unicode Transformation Format}, but the notion of
``transformation'' is a legacy notion that does not have meaning anymore.
Nevertheless, the designation UTF (in capitals) has become an official
standard designation, but should probably best be read as simply ``Unicode
Format''.} The details of this issue are extensively explained in section 2.5 of
the Unicode Core Specification \citep{Unicode2018}. 

Basically, \textsc{UTF-32} encodes each character in 32 bits (32 \textit{bi}nary
uni\textit{ts}, i.e.~32 zeros or ones) and is the most disk-space-consuming
variant of the three. However, it is the most efficient encoding
processing-wise, because the computer simply has to separate each character
after 32 bits. 

In contrast, \textsc{UTF-16} uses only 16 bits per character, which is
sufficient for the large majority of Unicode characters, but not for all of
them. A special system of \textsc{surrogates} is defined within the Unicode
Standard to deal with these additional characters. The effect is a more
disk-space efficient encoding (approximately half the size), while adding a
limited computational overhead to manage the surrogates. 

Finally, \textsc{UTF-8} is a more complex system that dynamically encodes each
character with the minimally necessary number of bits, choosing either 8, 16 or
32 bits depending on the character. This represents again a strong reduction in
space (particularly due to the high frequency of data using erstwhile ASCII
characters, which need only 8 bits) at the expense of even more computation
necessary to process such strings. However, because of the ever growing
computational power of modern machines, the processing overhead is in most
practical situations a non-issue, while saving on space is still useful,
particularly for sending texts over the Internet. As a result, UTF-8 has become
the dominant encoding on the World Wide Web. We suggest that everybody uses
UTF-8 as their default encoding.

A related problem is a general issue about how to store information in computer
memory, which is known as \textsc{endianness}. The details of this issue go
beyond the scope of this book. It suffices to realize that there is a difference
between \textsc{big-endian} (BE) storage and \textsc{little-endian} (LE)
storage. The Unicode Standard offers a possibility to explicitly indicate what
kind of storage is used by starting a file with a so-called \textsc{byte order
mark} (BOM). However, the Unicode Standard does not require the use of BOM,
preferring other non-Unicode methods to signal to computers which kind of
endianness is used. This issue only arises with UTF-32 and UTF-16 encodings.
When using the preferred UTF-8, using a BOM is theoretically possible, but
strongly dispreferred according to the Unicode Standard. We suggest that
everyone tries to prevent the inclusion of BOM in their data.

% ==========================
%\section{Pitfall: Software}
%\label{software}
% ==========================

% issues we've encountered in our computing environments
%  save as UTF-8
%  conversion between software programs (or when is something really UTF-8?)
%  NFC / NFD in copy & paste(!)


% ==========================
\section{Pitfall: Incomplete implementations}
\label{incomplete-implementations}
% ==========================
Another pitfall that we encounter when using the Unicode Standard is its incomplete implementation in different standards and programming languages, e.g.\ SQL, XML, XLST, Python. For example, although the Unicode Standard mandates that the comparison of Unicode text be done using normalized text, this is not the case with the equality operator ``=='' in Python. Furthermore, it is not always transparent what the operating system or specific software applications do when text is being copied and pasted. For example, copy and pasting the character sequence \uni{0061} \textsc{latin small letter a} <a> and \uni{0301} \textsc{combining acute accent} <\dia{0301}>, visually <á>, into the text editor TextWrangler leaves the sequence decomposed as two characters. But when pasting the decomposed sequence into RStudio, and other software programs, the sequence becomes precomposed as \uni{00E1} \textsc{latin small letter a with acute}, i.e.\ <á>.

% One Unicode pitfall which may be worth adding is the problem with incomplete implementations of the Unicode Standard. This is a problem most big-enough standards suffer from, e.g.\ SQL, XML, XSLT. So although the Unicode Standard mandates that comparison of Unicode text should always be done using normalized text, this is not how the equality operator "==" is implemented for Unicode text in the Python programming language.


% ==========================
\section{Recommendations}
\label{recommendations}
% ==========================

Summarizing the pitfalls discussed in this chapter, we propose the following 
recommendations:

\begin{itemize}
   \item To prevent strange boxes instead of nice glyphs, always install a few
         fonts with a large glyph collection and at least one fall-back font
         (see Section~\ref{pitfall-missing-glyphs}).
   \item Unexpected visual impressions of symbols do not necessarily mean that
         the actual encoding is wrong. It is mostly a problem of faulty
         rendering (see Section~\ref{pitfall-faulty-rendering}).
   \item Do not trust the names of code points as a definition of the character
         (see Section~\ref{pitfall-names}). Also do not trust Unicode blocks as
         a strategy to find specific characters (see
         Section~\ref{pitfall-blocks}).
   \item To ensure consistent encoding of texts, apply Unicode normalization
         (NFC or NFD, see Section~\ref{pitfall-canonical-equivalence}).
   \item To prevent remaining inconsistencies after normalization, for example 
         stemming from homoglyphs (see Section~\ref{pitfall-homoglyphs}) 
         or from missing canonical equivalence in the Unicode Standard
         (see Section~\ref{pitfall-absence-of-equivalence}), 
         use orthography profiles (see Chapter~\ref{orthography-profiles}).
   \item To deal with tailored grapheme clusters
         (Section~\ref{pitfall-characters-are-not-graphemes}), use Unicode Locale 
         Descriptions, or orthography profiles 
         (see Chapter~\ref{orthography-profiles}).
   \item As a preferred file format, use Unicode Format UTF-8 in 
         Normalization Form Composition (NFC) with LF line endings, 
         but without byte order mark (BOM), whenever possible (see 
         Section~\ref{pitfall-file-formats}). This last nicely cryptic 
         recommendation has T-shirt potential:
  
\end{itemize}

\begin{center}
  I prefer it
  
  \textbf{UTF-8 NFC LF no BOM}
\end{center}



% % ==========================
\chapter{The International Phonetic Alphabet}
\label{the-international-phonetic-alphabet}
% ==========================

In this chapter we present a brief history of the IPA 
(Section~\ref{IPAhistory}), which dates 
back to the late 19th century, not long after the creation of the first 
typewriter with a QWERTY 
keyboard. An understanding of the IPA and its premises and principles 
(Section~\ref{IPApremises-principles}) leads to a better 
appreciation of the challenges that the International Phonetic Association 
faced when digitally encoding the IPA's set of symbols and 
diacritics (Section~\ref{EncodingIPA}). Occurring a little over a hundred years after 
the inception of the IPA, its encoding was a major challenge 
(Section~\ref{need-for-multilingual-environment}); many 
linguists have encountered the pitfalls when the two are used together 
(Chapter~\ref{ipa-meets-unicode}).

% ==========================
\section{Brief history}
\label{IPAhistory}
% ==========================

Established in 1886, the \textsc{international phonetic association} (henceforth
\textit{Association}) has long maintained a standard alphabet, the
\textsc{international phonetic alphabet} or IPA, which is a
standard in linguistics to transcribe sounds of spoken languages. It was
first published in 1888 as an international system of phonetic transcription for
oral languages and for pedagogical purposes. It contained phonetic values for
English, French and German. Diacritics for length and nasalization were already
present in this first version, and the same symbols are still used today. 
%\footnote{Also referred to as API, for \textit{Association Phonétique Internationale}.} 

Originally, the IPA was a list of symbols with pronunciation examples
using words in different languages. In 1900 the symbols were first organized into
a chart and were given phonetic feature labels, e.g.\ for manner of
articulation among others \textit{plosives}, \textit{nasales}, \textit{fricatives}, for place of
articulation among others \textit{bronchiales}, \textit{laryngales}, \textit{labiales} and for vowels
e.g.\ \textit{fermées}, \textit{mi-fermées}, \textit{mi-ouvertes}, \textit{ouvertes}. Throughout the last
century, the structure of the chart has changed with increases in phonetic
knowledge. Thus, similar to notational systems in other scientific disciplines,
the IPA reflects facts and theories of phonetic knowledge that have developed
over time. It is natural then that the IPA is modified occasionally to
accommodate scientific innovations and discoveries. In fact, updates are part of the
Association's mandate. These changes are captured in revisions to the IPA chart.\footnote{For a detailed history, 
we refer the reader to:
\url{https://en.wikipedia.org/wiki/History\_of\_the\_International\_Phonetic_Alphabet}.}

Over the years there have been several revisions, but mostly minor ones. Articulation 
labels -- what are often called \textit{features} even though the IPA
deliberately avoids this term -- have changed, e.g.\ terms like \textit{lips}, \textit{throat}
or \textit{rolled} are no longer used. Phonetic symbol values have changed, e.g.\
voiceless is no longer marked by <h>. Symbols have been dropped, e.g.\ the
caret diacritic denoting `long and narrow' is no longer used. And many symbols
have been added to reflect contrastive sounds found in the world's very diverse
phonological systems. The use of the IPA is guided by principles outlined in 
the \textit{Handbook of the International Phonetic Association} \citep{IPA1999}, 
henceforth simply called \textit{Handbook}. 

Today, the IPA is designed to meet practical linguistic needs and is used to
transcribe the phonetic or phonological structure of languages. It is also used
increasingly as a foreign language learning tool, as a standard pronunciation
guide and as a tool for creating practical orthographies of previously unwritten
languages. The IPA suits many linguists' needs because:

\begin{itemize}

	\item it is intended to be a set of symbols for representing all possible
       sounds in the world's (spoken) languages;
	\item its chart has a linguistic basis (and specifically a phonological bias)
       rather than just being a general phonetic notation scheme;
	\item its symbols can be used to represent distinctive feature
       combinations;\footnote{Although the chart uses traditional manner and
       place of articulation labels, the symbols can be used as a representation
       of any defined bundle of features, binary or otherwise, to define
       phonetic dimensions.}
	\item its chart provides a summary of linguists' agreed-upon phonetic 
	knowledge.

\end{itemize}

Several styles of transcription with the IPA are possible, as illustrated in the
\textit{Handbook}, and they are all valid.\footnote{For an illustration of
the differences, see the 29 languages and their transcriptions in the
\textit{Illustrations of the IPA} \citep[41--154]{IPA1999}.} Therefore, there are 
different but equivalent transcriptions, or as noted by \citet[64]{Ladefoged1990a}, 
``perhaps now that the Association has been explicit in its eclectic approach, outsiders to the
Association will no longer speak of \textit{the} IPA transcription of a given
phenomenon, as if there were only one approved style.'' Clearly not all
phoneticians agree, nor are they likely to ever completely agree, on all aspects of the
IPA or on transcription approaches and practices in general. As noted above, 
there have been several revisions in the IPA's long history, but the current version (2005) is
strikingly similar to the 1926 version, which shows the viability of the IPA as a
common standard for linguistic transcription.

% ==========================

\section{Premises and principles}
\label{IPApremises-principles}
% ==========================
\subsection*{Premises}
\label{IPApremises}

Any IPA transcription is based on two premises: (i) that it is possible to
describe the acoustic speech signal (sound waves) in terms of sequentially
ordered discrete segments, and, (ii) that each segment can be characterized by
an articulatory target.

Once spoken language data are segmented, the IPA provides symbols to
unambiguously represent phonetic details. However, since phonetic detail could
potentially include anything, e.g.\ something like ``deep voice'', the IPA
restricts phonetic detail to linguistically relevant aspects of speech.
Phonological considerations thus inextricably play a roll in transcription. In
other words, phonetic observations beyond quantitative acoustic analysis are
always made in terms of some phonological framework.

Today, the IPA chart reflects a linguistic theory grounded in principles of
phonological contrast and in knowledge about the attested linguistic variation.
This fact is stated explicitly in several places, including in the
\textit{Report on the 1989 Kiel convention} published in the \textit{Journal of
the International Phonetic Association} \citep[67--68]{International1989report}:

\begin{quote}
The IPA is intended to be a set of symbols for representing all the possible 
sounds of the world's languages. The representation of these sounds uses a set 
of phonetic categories which describe how each sound is made. These categories 
define a number of natural classes of sounds that operate in phonological rules 
and historical sound changes. The symbols of the IPA are shorthand ways of 
indicating certain intersections of these categories.
\end{quote}

\noindent and in the \textit{Handbook} \citep[18]{IPA1999}: 

\begin{quote}
% The general value of the symbols in the chart is listed below. In each case 
[...] a symbol can be regarded as a shorthand equivalent to a phonetic
description, and a way of representing the contrasting sounds that occur in a
language. Thus [m] is equivalent to ``voiced bilabial nasal'', and is also a way
of representing one of the contrasting nasal sounds that occur in English and
other languages. [...] When a symbol is said to be suitable for the
representation of sounds in two languages, it does not necessarily mean that the
sounds in the two languages are identical.
\end{quote}

\noindent From its earliest days the Association aimed to provide ``a separate sign for
each distinctive sound; that is, for each sound which, being used instead of
another, in the same language, can change the meaning of a word''
\citep[27]{IPA1999}. Distinctive sounds became later known as \textsc{phonemes}
and the IPA has developed historically into a notational device with a strictly
segmental phonemic view. A phoneme is an abstract theoretical notion derived
from an acoustic signal as produced by speakers in the real world. Therefore the
IPA contains a number of theoretical assumptions about speech and how to
transcribe speech in written form. 

% TODO: cite Moran 2018

% ==========================
\subsection*{Principles}
\label{IPAprinciples}
% ==========================

Essentially, transcription has two parts: a text containing symbols and a set 
of conventions for interpreting those symbols (and their combinations). 
The symbols of the IPA distinguish between letter-like symbols and
diacritics (symbol modifiers). The use of the letter-like symbols to represent 
a language's sounds is guided by the principle of contrast; where two words 
are distinguishable by phonemic contrast, those contrasts should be transcribed 
with different letter symbols (and not just diacritics). Allophonic distinction 
falls under the rubric of diacritically-distinguished symbols, e.g.\ the 
difference in English between an aspirated /p/ in [pʰæt] and 
an unreleased /p/ in [stop̚]. 

\begin{itemize}

	\item Different letter-like symbols should be used whenever
          a language employs two sounds contrastively.
	\item When two sounds in a language are not known to be contrastive, the same
          symbol should be used to represent these sounds. Diacritics may
          be used to distinguish different articulations when necessary.
\end{itemize}          
          
\noindent Yet, in some situations diacritics are used to mark phonemic
contrasts. The \textit{Handbook} recommends to limit the use of phonemic
diacritics to the following situations: 

\begin{itemize}

 	\item denoting length, stress and pitch;
	\item obviating the design of a (large) number of new symbols when a 
		  single diacritic suffices (e.g.\ nasalized vowels, aspirated stops). 
               
\end{itemize}	

The interpretation of the IPA symbols in specific usage is not trivial. Although
the articulatory properties of the IPA symbols themselves are rather succinctly
summarized by the normative description in the \textit{Handbook}, it is common
in practical applications that the transcribed symbols do not precisely
represent the phonetic value of the sounds that they represent. So an IPA symbol
<t> in one transcription is not always the same as an IPA <t> in another
transcription (or even within a single transcription). The interpretation of any
particular <t> is mostly a language-specific convention (or even
resource-specific and possibly even context-specific), a fact which --
unfortunately -- is in most cases not made explicit by users of the IPA.

There are different reasons for this difficulty to interpret IPA symbols, all
officially sanctioned by the IPA. An important principle of the IPA is that
different representations resulting from different underlying analyses are
allowed. Because the IPA does not provide phonological analyses for specific
languages, the IPA does not define a single correct transcription system.
Rather, the IPA aims to provide a resource that allows users to express any
analysis so that it is widely understood. Basically, the IPA allows for both a 
\textit{narrow} phonetic transcription and a \textit{broad} phonological transcription. 
A narrow phonetic transcription may freely use all symbols in the IPA 
chart with direct reference to their phonetic value, i.e.\ the transcriber can 
indicate with the symbols <ŋ͡m> that the phonetic value of the attested sound 
is a simultaneous labial and velar closure which is voiced and contains nasal 
airflow, independently of the phonemic status of this sound in the language in 
question. 

% examples from Wells \url{https://www.phon.ucl.ac.uk/home/wells/transcription-ELL.pdf}

In contrast, the basic goal of a broad phonemic transcription is to distinguish all
words in a language with a minimal number of transcription symbols
\citep[19]{Abercrombie1964}. A phonemic transcription includes the conventions
of a particular language's phonological rules. These rules determine the
realization of that language's phonemes. For example, in the transcription of
German, Dutch, English and French a symbol <t> might be used for the voiceless
plosive in the alveolar and dental areas. This symbol is sufficient for a succinct
transcription of these languages because there is no further phonemic
subdivisions in this domain in either of these languages. However, the
language-specific realization of this consonant is closer to [t̪ʰ], [t], [tʰ]
and [t̪], respectively. Similarly, the five vowels of Greek can be represented
phonemically in IPA as /ieaou/, though phonetically they are closer to [iεaɔu].
The Japanese five-vowel system can also be transcribed in IPA as
/ieaou/, while the phonetic targets in this case are closer to [ieaoɯ].

Note also that there can be different systems of phonemic transcription for the
same variety of a language, so two different resources on the ``same'' language
might use different symbols that represent the same sound. The differences may
result from the fact that more than one phonetic symbol may be appropriate for a
phoneme, or the differences may be due to different phonemic analyses, e.g.\
Standard German's vowel system is arguably contrastive in length, tenseness or
vowel quality. Finally, even within a single phonemic transcription a specific
symbol can have different realizations because of allophonic variation which 
might not be explicitly transcribed.

In sum, there are three different reasons why phonemically-based IPA 
transcription is difficult to interpret:

\begin{itemize}
  
   \item A symbol represents the phonemic status, and not necessarily the
         precise phonetic realization. So, different transcriptions might use 
         the same symbol for different underlying sounds.
   \item Any symbol that is used for a specific phoneme is not necessarily
         unique, so different transcriptions might use different symbols for the
         same underlying sound.
   \item Allophonic variation can be disregarded in phonemic transcription, so
         the same symbol within a single transcription can represent different
         sounds.
  
\end{itemize}

Ideally, all such implicit conventions of a phonemic transcription would be
explicitly codified. This could very well be performed by using an orthography
profile (see Chapter~\ref{orthography-profiles}), linking the selected phonemic 
transcription symbols to narrow phonetic transcriptions, possibly also including 
specifications of contextual interpretation.

% ==========================

\section{IPA encodings}
\label{EncodingIPA}
% ==========================

In 1989, an IPA revision convention was held in Kiel, Germany. As in previous meetings, 
there were changes made to the repertoire of phonetic symbols in the IPA chart, which 
reflected what had been discovered, described and cataloged by linguists about the 
phonological systems in the world's languages in the interim. Personal computers 
were also becoming more commonplace, and linguists were using them to create databases. 
A cogent example is the UCLA Phonological 
Segment Inventory Database (UPSID; \citealt{Maddieson1984}), which was expanded 
\citep{MaddiesonPrecoda1990} and then encoded and distributed in a computer program 
\citep{MaddiesonPrecoda1992}.\footnote{It could be installed via 
floppy disk on an IBM PC, or compatible, running 
DOS with 1MB free disk space and 360K available RAM.} The programmers used 
only ASCII characters to maximize compatibility (e.g.\ <kpW> for $[$kpʷ$]$), but 
were faced with unavoidable arbitrary mappings between ASCII letters and 
punctuation, and the more than 900 segment types documented 
in their sample of world's languages' phonological systems. The developers 
devised a system of base characters with secondary diacritic marks 
(e.g.\ in the previous example <kp>, the base character, is modified with <W>). 
This encoding approach is 
also used in SAMPA and X-SAMPA (Section~\ref{sampa-xsampa}) and in the 
ASJP.\footnote{See the ASJP use case in the online supplementary 
materials to this book: \url{https://github.com/unicode-cookbook/recipes}.} 
But before UPSID, SAMPA and ASJP, IPA was encoded with numbers.
 
% \footnote{The marking of tone was extended (from characteristically Africanist practice) to include a second system for marking linguistic tones (the `Chao tone letters'), a much used convention based on musical staff to describe pitch in by Yuen Ren Chao (1892--1982).} 

\subsection*{IPA Numbers}
Prior to the Kiel Convention for the modern revision of the IPA in 1989,
\citet{Wells1987} collected and published practical approaches to coded
representations of the IPA, which dealt mainly with the assignment of characters
on the keyboard to IPA symbols. The process of assigning standardized computer
codes to phonetic symbols was given to the \textit{workgroup on computer
coding} (henceforth \textit{working group}) at the Kiel Convention. This working
group was given the following tasks
\citep{Esling1990,EslingGaylord1993}: 

\begin{itemize}
	\item determining how to represent the IPA numerically
	\item developing a set of numbers to refer to the IPA symbols unambiguously
	\item providing each symbol a unique name (intended to provide a mnemonic description of that character's shape)
\end{itemize}

\noindent The identification of IPA symbols with unique identifiers was 
a first step in formalizing the IPA computationally because it would give 
each symbol an unambiguous numerical identifier called an \textsc{ipa number}. 
The numbering system was to be comprehensive enough to support future revisions 
of the IPA, including symbol specifications and diacritic placement. The 
application of diacritics was also to be made explicit. 

Although the Association had never officially approved a set of names 
for the IPA symbols, each IPA symbol received a unique \textsc{ipa name}. 
Many symbols already had an informal name (or two) used by linguists, but 
consensus on symbol names was growing due to the recent publication of the 
\textit{Phonetic Symbol Guide} \citep{PullumLadusaw1986}. Thus most of the 
IPA symbol names were taken from that source \citep[31]{IPA1999}.

The working group decided insightfully that the computing-coding convention 
for the IPA should be independent of computer environments or formats, 
i.e.\ the \textsc{ipa number} was not meant to be encoded at the bit pattern level.
The working group report's declaration includes the following explanatory 
remarks \citep[82]{International1989report}:

\begin{quote}
The recommendation of a 7-bit ASCII or 8-bit extended-ASCII coding system 
would be short-sighted in view of development towards 16-bit and 32-bit 
processors. In fact, any specific recommendations would tie the Association 
to a stage of technological development which is bound to be outdated long 
before the next revision of the handbook.
\end{quote}

\noindent The coding convention was not meant to address the engineering 
aspects of the actual encoding in computers (cf.\ \cite{Anderson1984}). However, 
it was meant to serve as a basis for a interchange standard for 
creating mapping tables from various computer encodings, fonts, phonetic-character-set 
software, etc., to common IPA numbers, and therefore symbols.\footnote{Remember, at 
this time in the late 1980s there was no stable multilingual computing environment. 
But some solution was needed because scholars were increasingly using personal 
computers for their research and many were quickly adopting electronic mail or 
discussion boards like Usenet as a medium for international exchanges. 
Most of these systems ran on 8-bit hardware systems using a 7-bit ASCII character encoding.}

Furthermore, the assignment of computer codes to IPA symbols was meant to
represent an unbiased formulation. The Association here played the role of an
international advisory body and it stated that it should not recommend a
particular existing system of encoding. In fact, during this time there were a
number of coding systems used (see Section~\ref{encoding}), but none of them had
a dominant international position. The differences between systems were also
either too great or too subtle to warrant an attempt at combining them
\citep{International1989report}.

The working group assigned each IPA symbol to a unique three-digit IPA number. 
Encoded in this number scheme implicitly is information
about the status of each symbol (see below). The IPA numbers were listed with the
IPA symbols and they were also illustrated in IPA chart form (see
\cite[84]{EslingGaylord1993} or \cite[App. 2]{IPA1999}). The numbers were
assigned in linear order (e.g.\ [p] 101, [b] 102, [t] 103...) following the IPA
revision of 1989 and its update in 1996. Although the numbering scheme still 
exists, in practice it is superseded by the Unicode codification of symbols.

The working group made the decision that no IPA symbol, past or present, 
could be ignored. The comprehensive inclusion of all IPA symbols was to 
anticipate the possibility that some symbols might be added, withdrawn, 
or reintroduced into current or future usage. For example, in the 1989 
revision voiceless implosives <~ƥ,~ƭ,~ƈ,~ƙ,~ʠ~> were added; in the 1993 
revision they were removed. Ligatures like <~ʧ,~ʤ~> are included as formerly 
recognized IPA symbols; they are assigned to the 200 series of IPA numbers 
as members of the group of symbols formerly recognized by the IPA. To ensure 
backwards compatibility, legacy IPA symbols would retain an IPA number and 
an IPA name for reference purposes. As we discuss below, this decision is 
later reflected in the Unicode Standard as many legacy IPA symbols still reside in 
the \textsc{IPA Extensions} block.

The IPA number is expressed as a three-digit number. The first digit 
indicates the symbol's category \citep{Esling1990,EslingGaylord1993}:

\begin{itemize}
	\item 100s for accepted IPA consonants
	\item 200s for former IPA consonants and non-IPA symbols
	\item 300s for vowels
	\item 400s for segmental diacritics
	\item 500s for suprasegmental symbols
	\item 600s-800s for future specifications
	\item 900s for escape sequences
\end{itemize}

After a symbol is categorized, it is assigned a number sequentially, e.g.\ [i]
301, [e] 302, [ɛ] 303. The system allows for the addition of new symbols within
the various series by appending them, e.g.\ [\charis{ⱱ}] 184. Former non-IPA
symbols (or often-used but non-official IPA symbols) for consonants, vowels and
diacritics are numbered from 299 backwards. For example, the voiceless and
voiced postalveolar affricates and fricatives <~č,~ǰ,~š,~ž~> are assigned the
IPA numbers 299, 298, 297 and 296, respectively, because they are not sanctioned
IPA symbols.

The assignment of the IPA numbers to IPA symbols provided the basis for uniquely
identifying the set of past and present IPA symbols as a type of computational
representational standard of the IPA. Within each revision of the IPA, the
coding defines a closed and clearly defined set of characters. The benefits of
this standardization are clear in at least two ways: it is used in translation
tables that reference ASCII representations of the IPA, and this early
computational representation of the IPA became the basis for X-SAMPA and for the
inclusion of the IPA into the Unicode Standard version 1.0.

% ==========================
\subsection*{SAMPA and X-SAMPA}
\label{sampa-xsampa}
% ==========================

True to the working group's aim, the IPA numbers provided a mechanism for 
an interchange standard for creating mapping tables to various 
computer encodings. For example, the IPA coding system was used as a mapping 
system in the creation of SAMPA \citep{Wells_etal1992}, an ASCII representation 
of the IPA symbols. 

For a long time, linguists, like all other computer users, were
limited to ASCII-encoded 7-bit characters, which only includes Latin characters,
numbers and some punctuation and symbols. Restricted to these standard character
sets that lacked IPA support or other language-specific graphemes that they
needed, linguists devised their own solutions.\footnote{Early work addressing
the need for a universal computing environment for writing systems and their
computational complexity is discussed in \citet{Simons1989}. A more recent survey of
practical recommendations for language resources, including notes on encoding,
can be found in \citet{BirdSimons2003}.} For example, some chose to represent
unavailable graphemes with substitutes, e.g.~the combination of <ng> to
represent <ŋ>. Tech-savvy linguists redefined selected characters from a
character encoding by mapping custom-made fonts to specific code points.\footnote{For 
example, SIL's popular font SIL IPA 1990.} However,
one linguist's electronic text would not render properly on another linguist's
computer without access to the same font. Furthermore, if two character encodings
defined two character sets differently, then data could not be reliably and
correctly displayed. This is a commonly encountered example of the non-interoperability of
data and data formats.

One solution was the ASCII-ification of the IPA, which simply involved 
defining keyboard-able sequences consisting of ASCII combinations as IPA symbols. 
For example, \citet{Wells1987} provides an in-depth description of IPA
codings from country-to-country. Later ASCII-IPAs include Kirshenbaum (created
in 1992 in a Usenet group and named after its lead developer who was at
Hewlett-Packard Laboratories) and Worldbet (published by
\citet{Hieronymus1993}, who was at AT\&T Laboratories). 
The most successful effort was SAMPA (Speech Assessment
Methods Phonetic Alphabet), which was created between 1988--1991 in Europe to 
represent IPA symbols with ASCII
character sequences \citep{Wells1987,Wells_etal1992}, using e.g.\ <p\textbackslash> 
for [ɸ]. SAMPA was developed by a group of speech scientists from nine countries 
in Europe and it constituted the ASCII-IPA symbols needed for phonemic transcription 
of the principal European languages \citep{Wells1995}. It is still widely 
used in language technology.

Two problems with SAMPA are that (i) it is only a partial encoding of the IPA
and (ii) it encodes different languages in separate data tables, instead of
using a universal alphabet, like IPA.\@ SAMPA tables were developed as part of a
European Commission-funded project to address technical problems like electronic
mail exchange (what is now simply called email). SAMPA is essentially a hack to
work around displaying IPA characters, but it provided speech technology and
other fields a basis that has been widely adopted and often still used in code.
So, SAMPA is a collection of tables to be compared, instead of a large universal
table representing all languages. 

An extended version of SAMPA, called X-SAMPA, set out to include every symbol,
including all diacritics, in the IPA chart \citep{Wells1995}. X-SAMPA is
considered more universally applicable because it consists of one table that
encodes all characters in IPA. In line with the principles of the IPA, SAMPA and
X-SAMPA include a repertoire of symbols. These symbols are intended to represent
phonemes rather than all allophonic distinctions. Additionally, both
ASCII-ifications of IPA -- SAMPA and X-SAMPA -- are
(reportedly) uniquely parsable \citep{Wells1995}. However, like the IPA, X-SAMPA
has different notations for encoding the same phonetic phenomena (cf.\ Section~
\ref{pitfall-multiple-options-ipa}).

SAMPA and X-SAMPA have been widely used for speech technology and as an encoding
system in computational linguistics. In fact, they are still used in popular
software packages that require ASCII input, e.g.~RuG/L04 and SplitsTree4.\footnote{See
\url{http://www.let.rug.nl/kleiweg/L04/} and \url{http://www.splitstree.org/},
respectively.}

% ==========================

\section{The need for a single multilingual environment}
\label{need-for-multilingual-environment}
% ==========================

In hindsight it is easy to lose sight of how impactful 30 years of technological
development have been on linguistics, from theory development using quantitative
means to pure data collection and dissemination. But at the end of the 1970s,
virtually no ordinary working linguist was using a personal computer
\citep{Simons1996}. Personal computer usage, however, dramatically increased
throughout the 1980s. By 1990, dozens of character sets were in common use. They
varied in their architecture and in their character repertoires, which made
things a mess. 

During the 1980s, it became increasingly clear that an adequate solution 
to the problem of multilingual computing environments was needed. Linguists 
were on the forefront of addressing this issue because they faced these 
challenges head-on by wishing to publish and communicate electronic text 
with phonetic symbols which were not included in basic ASCII. One 
only needs to look at facsimiles of older electronic documents to see exotic 
symbols written in by hand after the preparation of typed version.

%There were two major players in the universal character set race:
%Unicode and the International Organization for Standardization (ISO).

%Long familiar were linguists already with the distinction between function 
%and form. Even in the context of the computer implementation of writing systems, 
%the necessity to distinguish form and function had been made \citep{Becker1984}. 
%The computer industry, on the other hand, did not consider, ignored, or simply 
%did not encode this principle when creating operating systems like MS-DOS, which 
%were limited to 256 code points (due to computer hardware architecture) and 
%encoded with one-to-one mappings from character codes to graphemes.

% This standard became the basis for a proposal to include the IPA in the first version of the Unicode Standard. Decisions by the Computer Coding working group and work they continued after the 1989 Kiel Convention were adopted by the International Phonetic Association. These decisions are directly reflected in the Unicode Standard's encoding of IPA, seeing as it was the Association who submitted the script proposal to the Unicode Consortium.

A major benefit of the standardization of the IPA in a computational
representation by the Kiel working group is that it provided the basis for a
formal proposal to be submitted to various international standards
organizations, several of which were trying to tackle (and in a sense win) the
multilingual computing environment problem (cf.\ Section~\ref{encoding}).
Basically, everyone -- from corporations to governments to language scientists
-- wanted a single unified multilingual character encoding set for all the
world's writing systems, even if they did not understand or appreciate the
challenges involved in creating and adopting a solution.

Industry was starting to tackle the issues involved in developing a single
multilingual computing environment on a variety of fronts, including the then
new technology of bitmap fonts and the creation of Font Manager and Script
Manager by Apple \citep{Apple1985,Apple1986,Apple1988}. As noted above, around
this time linguists were developing work-arounds such as SAMPA, so that they
could communicate IPA transcription and use ASCII-based software. Some linguists
formalized the issues of multilingual text processing from a computational
perspective \citep{Anderson1984,Becker1984,Simons1989}. The study of writing
systems was also being invigorated \citep[11--15]{Sampson1985} by the
computational challenges in making computers work in a multilingual environment.
The engineering problems and solutions had been spelled out years before, e.g.\
a two-byte encoding for multilingual text \citep{Anderson1984}. Although
languages vary to an astounding extent (cf.\ \cite{EvansLevinson2009}), writing
systems are quite similar formally and the issue of formal representation of the
world's orthographic systems had already been addressed \citep{Simons1989}. 

%A major obstacle in creating a single encoding multilingual environment from 
%the perspective of writing systems involves the distinction between function 
%and form \citep{Becker1984} This distinction is so central to basic linguistic 
%theory and that trained linguists and semiologists take it as second nature. 
%A central challenge in developing a universal character set was to combine a 
%technological solution with a formalization of writing systems proper.\footnote{Of 
%course there were additional practical issues to overcome, e.g.\ funding, creating 
%the formal and technological proposal, deciding which characters and writing systems 
%to include initially, while setting precedence of how to add new ones in the future.}


% "The set of IPA symbols and their numbers were used to draw up an entity set within SGML by the Text Encoding Initiative (TEl). The name of each entity is formed by 'IPA' preceding the number, e.g.\ IPA304 is the rEIentity name of lower-case A. These symbols can be processed as IPA symbols and represented on paper and screen with the appropriate local font by modifying the :entity replacement text. The advantage of the SGML entity set is that it is independent or the character set being used."
% "A TEl writing system declaration (wsd) has been drawn up for the IPA symbols."
% A TEl writing system declaration (wsd) has been drawn up for the IPA symbols. This document gives information about the symbol and its IPA function, as well as its encoding in the accompanying SGML document and in UnicodelUCS and in AFII. The writing system declaration can be read as a text d9cument or processed by machines in an SMGL process.

After the Kiel Convention in 1989, the working group assisted
the International Phonetic Association in representing the IPA to the
\textsc{international organization for standardization} (ISO) and to the \textsc{text
encoding initiative} (TEI) \citep{EslingGaylord1993}. The working group's
formalization of the IPA, i.e.\ a full listing of agreed-upon computer codings
for phonetic symbols, was used in developing writing system descriptions, which
were at the time being solicited for scripts to be included in the new
multilingual international character encoding standards. The working group for
ISO/IEC 10646 and Unicode were two such initiatives.

In the historical context of the IPA being considered for inclusion in 
ISO/IEC 10646, it is important to realize that there were a variety of 
sources (i.e.\ not just from the Association) which submitted character 
proposals for phonetic alphabets. These proposals, including the one from the 
Association via the Kiel working group, were considered as a whole by 
the ISO working groups that were responsible for incorporating a phonetic 
script into the universal character set (UCS). The ISO working groups that 
were responsible for assigning a phonetic character set then made their 
own submissions as part of a review process by ISO for approval based on 
both informatics and phonetic criteria \citep[86]{EslingGaylord1993}. 

Character set ISO/IEC 10646 was approved by ISO, including the phonetic
characters submitted to them in May 1993. The set of IPA characters were
assigned UCS codes in 16-bit representation (in hexadecimal) and were published
as Tables 2 and 3 in \cite{EslingGaylord1993}, which include a graphical
representation of the IPA symbol, its IPA name, phonetic description, IPA
number, UCS code and AFII code.\footnote{The Association for Font
Information Interchange (AFII) was an international database of glyphs created
to promote the standardization of font data required to produce ISO/IEC 10646.} When the
character sets of ISO/IEC 10646 and the Unicode Standard later converged (see
Chapter~\ref{the-unicode-approach}), the IPA proposal was
included in the Unicode Standard Version 1.0 -- largely as we know it
today.

With subsequent revisions to the IPA, one might have expected that the Unicode
Consortium would update the Unicode Standard in a way that is in line with the
development of linguistic insights. However, updates that go against the principle
of maintaining backwards compatibility lose out, i.e.\ it is more important to
deal with the pitfalls created along the way than it is to change the standard.
Therefore, many of the pitfalls we encounter today when using Unicode IPA are
historic relics that we have to come to grips with.

% https://en.wikipedia.org/wiki/Uralic_Phonetic_Alphabet
% http://www.unicode.org/conference/bulldog.html
% http://www-01.sil.org/computing/computing_environment.html

It was a long journey, but the goal of achieving a single multilingual computing
environment has largely been accomplished. As such, we should not dismiss the IPA numbers 
or pre-Unicode encoding attempts, such as SAMPA/X-SAMPA, as misguided. The parallels 
between the IPA numbers and Unicode Code points, for example, are striking because both the IPA and 
the Unicode Consortium came up with the solution of an additional layer of indirection (an abstraction layer) 
between symbols/characters and encoding on the bit pattern level. SAMPA/X-SAMPA is also still useful 
as an input method for IPA in ASCII and required by some software.

Current users of the Unicode Standard must cope
with the pitfalls that were dug along the way, as will be discussed in the next
chapter. As the Association foresightfully remarked about Unicode:

\begin{center} 

\textit{``When this character set is in wide use, \\
it will be the normal way to encode IPA symbols.''}

\ \\

\citep[164]{IPA1999}.

\end{center}


% ==========================

\begin{comment}
\section{Unicode and ISO 10646}
% ==========================

In the late 1980s, a universal character set was being developed by what 
is now referred to as the Unicode Consortium (officially incorporated in January 1991).
This consortium 
consisted largely, although not entirely, of major US corporations, with 
the aim of overcoming the inoperability of different coded character sets 
and their costly hinderance for developing multilingual software development 
and for internationalization efforts. Commercial importance of course drove 
the early inclusion of Latin, non-Latin, and some exotic scripts; see the 
table of commercial importance as measured by GDP of countries using certain 
writing systems \citep[2]{Becker1988}.

The original Unicode manifesto is \cite{Becker1988}.\footnote{http://www.unicode.org/history/unicode88.pdf} 
Its aim was for a reliable international multilingual text encoding standard 
that would encompass all scripts of the world, or in the author's own words, 
``a new, world-wide ASCII''. An in-depth history of Unicode, highlighting 
interesting facts like its first text prototypes at Apple and its incorporation 
into TrueType, is retold online.\footnote{\url{http://www.unicode.org/history/earlyyears.html}}

Unicode 88 provided the basic principles for the Unicode Standard's design -- 
pushing for 16-bit representations of characters with a clear distinction 
between characters and glyphs. Some of the contents of this status proposal 
of 1988 were reworked for inclusion in the early Unicode Standard pre-publication 
drafts and by August 1990, the proposal was in a (very) rough draft format. Its 
editors and the Unicode Working Group (the predecessor to the Unicode Technical 
Committee) worked together to lay out the proposed standard's structure and 
content. At this time, the proposal contained no code charts nor block descriptions. 

% http://www.unicode.org/history/earlyyears.html
% During this period of time, in addition to his co-authoring of Apple KanjiTalk, Davis was involved in further discussions of a universal character set which were being prompted by the development of the Apple File Exchange.

The other major player in developing a universal character set was the ISO 
working group from the International Standards Organization (ISO), based 
in Europe, which was responsible for ISO/IEC 10646. This character set 
standard was composed in 1989 and a draft was published in 
1990.\footnote{\url{http://www.iso.org/iso/catalogue_detail.htm?csnumber=56921}} 
The \textit{Universal Multiple-Octet Coded Character Set} or simply UCS was the first 
officially standardized character encoding with the aim of including all 
characters from all writing 
systems.\footnote{\url{http://www.nada.kth.se/i18n/ucs/unicode-iso10646-oview.html}}

ISO/IEC 10646 is partly based on ISO/IEC 8859, a series of ASCII-based 
standard character encodings published in 1987 that use a single bit 8-byte 
character set. Each part of the standard, e.g.\ 8859-1, 8859-5, 8859-6, 
encodes characters to support different languages' writing systems, e.g.\ 
Latin-1 Western European, Latin/Cyrillic, Latin/Arabic, respectively. Being 
a joint effort by the International Organization for Standardization (ISO) 
and the International Electrotechnical Commission (IEC), the aim of the 
standard is reliable information exchange. So, again, issues of phonetic 
symbol encoding, typography, etc., were ignored -- or perhaps more properly 
put, not commercially driven at this early stage.

Intended for the major Western European languages, ISO/IEC 8859 was an 
extension of the ASCII character encoding standard, which included the 
English alphabet, numerals and computer control characters (e.g.\ beep, 
space, carriage return). By extending ASCII's 7-bit system to 8-bit, the 
character repertoire of each of ISO/IEC 8859 character set was doubled 
from 128 to 256 characters. Each character set defined a mapping between 
digital bit patterns and characters, which are visually rendered on screen 
as graphic symbols. ASCII was shared between ISO/IEC 8859 character sets, 
but the characters in the extra bit patterns were different. Thus an aim 
of the ISO working group responsible for ISO/IEC 10646 was to bring all 
characters in all writings systems into a single unified encoding.

In 1991, the Unicode Consortium and the ISO Working Group for ISO/IEC 10646 
decided to create a single universal standard for encoding multilingual 
text.\footnote{http://unicode.org/book/appC.pdf} 
The two character sets converged, resulting in mutually acceptable changes 
to both, and each group keeps versions of their respective character codes 
and encoding forms synchronized.\footnote{http://www.unicode.org/versions/} 
Although each standard has its own form of reference and the terminology in 
each may differ slightly, the practical difference is that the Unicode Standard 
is a formal implementation of ISO/IEC 10646 and imposes additional constraints 
on its implementation. The Unicode Standard includes character data, algorithms 
and specifications, outside the scope of ISO/IEC 10646, which ensure, when 
properly implemented in software applications and platforms, that characters 
are treated uniformly. 

The incorporation of the Unicode Standard into the international encoding 
standard ISO 10646 was approved by ISO as an International Standard in June 
1992.\footnote{http://www.unicode.org/versions/Unicode1.0.0/Notice.pdf} 
The joint Unicode and ISO/IEC 10646 standard has become \textit{the} universal 
character set and it is a single multilingual environment for the majority 
of the world's written languages. Its formal implementation has also been 
vital to the rise of a multi-lingual Internet.

\end{comment}


% \chapter{IPA meets Unicode}
\label{ipa-meets-unicode}

\section{Introducing the International Phonetic Alphabet (IPA)}
\label{introducing-the-international-phonetic-alphabet-ipa}

The International Phonetic Alphabet (IPA) is a common standard in linguistics to transcribe sounds of spoken language into some Latin-based characters (International Phonetic Association 1999). Although IPA is reasonably easily adhered to with pen and paper, it is not trivial to encode IPA characters electronically. Early work addressing the need for a universal computing environment for writing systems and their computational complexity is discussed in Simons 1989. For a long time, linguists (like all other computer users) were limited to ASCII-encoded 7-bit characters, which only includes Latin characters, numbers and some punctuation and symbols. Restricted to these standard character sets that lacked IPA support or other language-specific graphemes that they needed, linguists devised their own solutions (cf.~Bird and Simons 2003). For example, some chose to represent unavailable graphemes with substitutes, e.g.~the combination of to represent . Tech-savvy linguists redefined selected characters from a character encoding by mapping custom made fonts to those code points. However, one linguist's electronic text would not render properly on another linguist's computer without access to the same font. Further, if two character encodings defined two character sets differently, then data could not be reliably and correctly displayed. This is a common example of the non-interoperability of data and data formats.

To alleviate this problem, during the late 1980s, SAMPA (Speech Assessment Methods Phonetic Alphabet) was created to represent IPA symbols with 7-bit printable ASCII character sequences, e.g. for [ɸ]. Two problems with SAMPA are that (i) it is only a partial encoding of the IPA and (ii) it encodes different languages in separate data tables, instead of a universal alphabet, like IPA. SAMPA tables are derived from phonemes appearing in several European languages that were developed as part of a European Commission-funded project to address technical problems like electronic mail exchange (what is now simply called email). SAMPA is essentially a hack to work around displaying IPA characters, but it provided speech technology and other fields a basis that has been widely adopted and used in code. So, SAMPA was a collection of tables to be compared, instead of a large universal table representing all languages. An extended version of SAMPA, called X-SAMPA, set out to include every symbol in the IPA chart including all diacritics (Wells nd.). X-SAMPA was considered more universally applicable because it consisted of one table that encoded the set of characters that represented phones/segments in IPA across languages. SAMPA and X-SAMPA have been widely used for speech technology and as an encoding system in computational linguistics. Eventually, ASCII-encoding of the IPA became depreciated through the advent of the Unicode Standard. Note however that many popular software packages used for linguistic analyses still require ASCII input, e.g.~RuG/L04\footnote{http://www.let.rug.nl/kleiweg/L04/} and SplitsTree4.\footnote{http://www.splitstree.org/}

Still, there are a few pitfalls to be aware of when using the Unicode Standard to encode IPA. As we have said before, from a linguistic perspective it might look like the Unicode Consortium is making incomprehensible decisions, but it is important to realize that the consortium has tried and is continuing to try to be as consistent as possible across a wide range of use cases, and it does place linguistic traditions above other orthographic possibilities. In general, we strongly suggest linguists not to complain about any decisions in the Unicode Standard, but to try and understand the rationale of the Unicode Consortium (which in our experience is almost always well-conceived) and devise ways to work with any unexpected behaviour. Many of the current problems derive from the fact that the IPA is clearly historically based on the Latin script, but different enough from most other Latin-based writing systems to warrant special attention. This ambivalent status of the IPA glyphs (partly Latin, partly special) is unfortunately also attested in the treatment of IPA in the Unicode Standard. In retrospect, it might have been better to consider the IPA (and other transcription systems) to be a special or new kind of script within the Unicode Standard, and treat the obvious similarity to Latin glyphs as a historical relic. All IPA glyphs would then have their own code points, instead of the current situation in which some IPA glyphs have special code points, while others are treated as being identical to the `regular' Latin characters. Yet, the current situation, however unfortunate, is unlikely to change, so as linguists we will have to learn to deal with the specific pitfalls of IPA within the Unicode Standard. In this section, we will describe these pitfalls in some detail.

\section{Pitfall: There is no single complete Unicode code block for IPA}
\label{pitfall-there-is-no-single-complete-unicode-code-block-for-ipa}

The ambivalent nature of IPA glyphs arises because, on the one hand, the IPA uses Latin-based glyphs like , or . From this perspective, the IPA seems to be just another orthographic tradition using Latin characters, all of which do not get a special treatment within the Unicode Standard (just like e.g.~the French, German, or Danish orthographic traditions do not have a special status). On the other hand, the IPA uses many special symbols (like turned , mirrored and/or extended Latin glyphs ) not found in any other Latin-based writing system. For this reason, and already in the first version of the Unicode Standard (Version 1.0 from 1991), a special block with code points, called ``IPA Extensions'' was included. As noted in Section 4, Pitfall 1, the Unicode Standard code space is subdivided into character blocks, which generally encode characters from a single script. However, as is illustrated by the IPA, characters that form a single writing system may be dispersed across several different character blocks. With its diverse collection of symbols from various scripts and diacritics, the IPA is spread across over 13 blocks in the Unicode Standard:\footnote{This number of blocks depends on whether only IPA-sanctioned symbols are counted or if the phonetic symbols commonly found in the literature are also included, see Moran 2012, Appendix C.}
\begin{itemize}
	\item Basic Latin (30 characters), e.g. <a, b, c, d, e> 
	\item Latin-1 Supplement (4 characters): <æ, ç, ð, ø\textgreater{} 
	\item Latin Extended-A (3 characters): <ħ, ŋ, œ> 
	\item Latin Extended-B (5 characters): <ǀ, ǁ, ǂ, ǃ, ȵ> 
	\item IPA Extensions (70 characters), e.g.: <ɐ, ɑ, ɔ> 
	\item Spacing Modifier Letters (20 characters), e.g.: <ʰ ʷ ˥> 
	\item Combining Diacritical Marks (33 characters), e.g.: <̝ ̥ ̪> 
	\item Greek and Coptic (3 characters): <β, θ, χ> 
	\item Phonetic Extensions (2 characters): <ᴅ, ᴴ> 
	\item Phonetic Extensions Supplement (3 characters): <ᶑ , ᶾ, ᶣ> 
	\item Superscripts and Subscripts (1 character): <ⁿ> 
	\item Arrows (2 characters): \textless{}↑, ↓\textgreater{} 
	\item Latin Extended-C (1 character): <ⱱ> 
\end{itemize}

\section{Pitfall: There are many IPA homoglyphs in Unicode}
\label{pitfall-there-are-many-ipa-homoglyphs-in-unicode}

Another problem is the large number of homoglyphs, i.e.~different characters that have highly similar glyphs (or even completely identical, depending on the font rendering). For example, a speaker of Russian should ideally not use the CYRILLIC SMALL LETTER A at code point U+0430 for IPA transcriptions, but instead the LATIN SMALL LETTER A at code point U+0061, although visually they are mostly indistinguishable, and the Cyrillic character is more easily typed on a Cyrillic keyboard. Another example we commonly encounter is the use of LATIN SMALL LETTER G at U+0067, instead of the sanctioned Unicode Standard IPA character for the voiced velar stop LATIN SMALL LETTER SCRIPT G at U+0261. One begins to question whether this issue is at all apparent to the working linguist, or if they simply use the U+0067 because it is easily keyboarded and thus saves time, whereas the latter must be cumbersomely inserted as a special symbol in most software.\footnote{This issue was recently addressed by the International Phonetic Association, which has taken the stance that both the keyboard LATIN SMALL LETTER G and the LATIN SMALL LETTER SCRIPT G are valid input characters for the voiced velar plosive. Unfortunately, this decision further introduces ambiguity for linguists trying to adhere to a strict Unicode Standard IPA encoding.}

Furthermore, on the one hand even linguists are unlikely to distinguish between the LATIN SMALL LETTER SCHWA at code point U+0259 and LATIN SMALL LETTER TURNED E at U+01DD. On the other hand, non-linguists are unlikely to distinguish any semantic difference between an open back unrounded vowel , the LATIN SMALL LETTER ALPHA at U+0251, and the open front unrounded vowel , LATIN SMALL LETTER A at U+0061. But even among linguists this distinction leads to problems. For example, as pointed out by Mielke (2009), there is a problem stemming from the fact that about 75\% of languages are reported to have a five-vowel system (Maddieson 1984). Historically, linguistic descriptions tend not to include precise audio recording and measurements of formants, so this may lead one to ask if the many characters that are used in phonological description reflects a ``transcriptional bias''. The common use of in transcriptions could be in part due to the ease of typing the letter on an English keyboard (or for older descriptions, the typewriter). We found it to be exceedingly rare that a linguist uses for a low back unrounded vowel.\footnote{One example is Vidal 2001a:75, in which the author states: ``The definition of Pilagá /a/ as {[}+back{]} results from its behavior in certain phonological contexts. For instance, uvular and pharyngeal consonants only occur around /a/ and /o/. Hence, the characterization of /a/ and /o/ as a natural class of (i.e., {[}+back{]} vowels), as opposed to /i/ and /e/.''} They simply use $<$a$>$ as long as there is no opposition to $<$ɑ$>$.\footnote{See Thomason's Language Log post, ``Why I don't love the International Phonetic Alphabet'', at: http://itre.cis.upenn.edu/\textasciitilde{}myl/languagelog/archives/005287.html.} Making things even more problematic, there is an old typographic tradition that the double-story uses a single-story in italics. This leads to the unfortunate effect that in most well-designed fonts the italics of and use the same glyph. If this distinction has to be kept upright in italics, the only solution we can currently offer is to use `slanted' glyphs (i.e.~artificially italicised glyphs) instead of real italics (i.e.~special italics glyphs designed by a typographer).\footnote{For example, the widely used IPA font Doulos SIL (http://scripts.sil.org/cms/scripts/page.php?item\_id=DoulosSIL) does not have real italics. This leads some word-processing software, like Microsoft Word, to produce slanted glyphs instead. That particular combination of font and software application will thus lead to the desired effect. However, note that when the text is transferred to another font (i.e.~one that includes real italics) and/or to another software application (like Apple Pages, which does not perform slanting), then this visual appearance will be lost. In this case we are thus still in the pre-Unicode situation in which the choice of font and rendering software actually matters. The ideal solution from a linguistic point of view would be the introduction of a new IPA code point for a different kind of which explicitly specifies that it should still be rendered as a double-story character when italicized. After informal discussion with various Unicode players, our impression is that this highly restricted problem is not sufficiently urgent to introduce even more -like characters in Unicode (which already lead to much confusion, see Section 4, Pitfall 4). This is a clear situation in which the Unicode Consortium is not just thinking about linguists, but has a more wide-ranging practical view to consider.}

Some other homoglyphs related to encoding IPA in the Unicode Standard are:
\begin{itemize}
	\item The uses of the apostrophe has led to long discussions on the Unicode Standard email list. An English keyboard inputs \textless{}`\textgreater{} APOSTROPHE at U+0027, although the `preferred' Unicode apostrophe is the \textless{}'\textgreater{} RIGHT SINGLE QUOTATION MARK at U+2019. Yet the glottal stop/glottalization/ejective marker is another completely different character, the MODIFIER LETTER APOSTROPHE at U+02BC, but unfortunately looks mostly highly similar to U+2019. 
	\item There is ambiguous encoding of IPA segments within the Unicode Standard. An example is the U+02C1 MODIFIER LETTER REVERSED GLOTTAL STOP vs the U+02E4 MODIFIER LETTER SMALL REVERSED GLOTTAL STOP . Both are denoted in the Unicode Standard as the `pharyngealized diacritic' and both appear in various resources representing phonetic data online. This is thus an example for which the Unicode Standard does not solve the linguistic standardization problem. 
	\item There is at least one case in which the character name assigned by the Unicode Consortium does not match the IPA's description: in the Unicode Standard at U+01C3 is labeled LATIN LETTER RETROFLEX CLICK, but in IPA is an alveolar or postalveolar click (not retroflex). This naming is probably best seen as a simple error in the Unicode Standard. Note that most linguists simply seem to use or <ou>) while ligatures are single characters (e.g. <ʧ> LATIN SMALL LETTER TESH DIGRAPH at U+02A7). Ligatures arose in the context of printing easier-to-read texts, and are included in the Unicode Standard for reasons of legacy encoding. However, their usage is discouraged by the Unicode core specification. Specifically related to IPA, various phonetic combinations of characters (typically affricates) are available as single code-points in the Unicode Standard, but are designated as ``ligatures'' or ``digraphs'' (confusingly both names appear interchangeably). Such glyphs might be used by software to produce a pleasing display, but they should not be hard-coded into the text itself. In the context of IPA, characters like the following ligatures should thus \emph{not} be used. Instead a combination of two characters is preferred: 
	\item <ʣ> LATIN SMALL LETTER DZ DIGRAPH at U+02A3 (use <d> + <z> instead) 
	\item <ʧ> LATIN SMALL LETTER TESH DIGRAPH at U+02A7 (use <t> + <ʃ> instead) 
	\item <ʩ> LATIN SMALL LETTER FENG DIGRAPH at U+02A9 (use <f> + <ŋ> instead) 
\end{itemize}

However, there are a few Unicode characters that are historically ligatures, but which are today considered as simple characters in the Unicode Standard and thus should be used when writing IPA, namely:
\begin{itemize}
	\item <ɮ> LATIN SMALL LETTER LEZH at U+026E 
	\item <œ> LATIN SMALL LIGATURE OE at U+0153 
	\item <ɶ> LATIN LETTER SMALL CAPITAL OE at U+0276 
	\item <æ> LATIN SMALL LETTER AE at U+00E6 
\end{itemize}

\section{Pitfall: The notion of diacritic differs between IPA and Unicode}
\label{pitfall-the-ipa-notion-of-diacritics-is-not-the-same-as-the-unicode-standards-notion-of-diacritics}

Another pitfall is diacritics. The problem is that the meaning of the term `diacritics' as used by the IPA is not the same as it used in the Unicode Standard. Specifically, diacritics in the IPA-sense are either so-called \textsc{Combining Diacritical Marks} or \textsc{Spacing Modifier Letters} in the Unicode Standard. Crucially, Combining Diacritical Marks are by definition combined with the character before them (to form so-called default grapheme clusters, see Section 3). In contrast, Spacing Modifier Letters are by definition \emph{not} combined into grapheme clusters with the preceding character, but simply treated as separate letters. In the context of the IPA, the following IPA-diacritics are actually Spacing Modifier Letters in the Unicode Standard:
\begin{itemize}
	\item Length marks, namely <ː> MODIFIER LETTER TRIANGULAR COLON at U+02D0 and <ˑ> MODIFIER LETTER HALF TRIANGULAR COLON at U+02D1. 
	\item Tone letters, like <˥> MODIFIER LETTER EXTRA-HIGH TONE BAR at U+02E5, and others like this. 
	\item Superscript letters, like <ʰ> MODIFIER LETTER SMALL H at U+02B0 or <ˤ> MODIFIER LETTER SMALL REVERSED GLOTTAL STOP at U+02E4, and others like this. 
	\item <˞> MODIFIER LETTER RHOTIC HOOK at U+02DE. 
\end{itemize}

Although linguists might expect these characters to belong together with the character in front of them, at least for <ʰ> MODIFIER LETTER SMALL H at U+02B0 the Unicode Consortium's decision to treat it as a separate character is also linguistically correct, because according to the IPA it can be used both for aspiration (more precisely postaspiration following the base character) and preaspiration (preceding the base character). Note that there is a mechanism in Unicode to force separate characters to be combined (namely by using the ZERO WIDTH JOINER at U+200D), but this seems to be a rather impractical, and probably not enforceable solution to us.

\section{Pitfall: There is no unique diacritic ordering in IPA, nor in Unicode}
\label{pitfall-neither-the-ipa-nor-the-unicode-standard-enforce-a-unique-diacritic-ordering}

Also related to diacritics is the question of ordering. To our knowledge, the International Phonetic Association does not specify a specific ordering for diacritics that combine with phonetic base symbols; this exercise is left to the reasoning of the transcriber. However, such marks have to be explicitly ordered if sequences of them are to be interoperable and compatible. An example is a labialized aspirated alveolar plosive: <tʷʰ>. There is nothing holding linguists back from using <tʰʷ> instead (with exactly the same intended meaning). However, from a technical standpoint, these two sequences are different, e.g.~if both sequences are used in a document, searching for <tʷʰ> will not find any instances of <tʰʷ>, and vice versa. Likewise, a creaky voiced syllabic dental nasal can be encoded in various orders, e.g. <n̪̰̩>, <n̩̰̪> or <n̩̪̰>.

In accordance with the absence of any specification of ordering in the IPA, the Unicode Standard likewise does not propose any standard orderings. Both leave it to the user to be consistent. However, there is one aspect of ordering for which the Unicode Standard does present a canonical solution, which is uncontroversial from a linguistic perspective. Diacritics in the Unicode Standard (i.e.~Combining Diacritical Marks, see above) are classified in Canonical Combining Classes. In practice, the diacritics are distinguished by their position relative to the base character.\footnote{See http://unicode.org/reports/tr44/\#Canonical\_Combining\_Class\_Values for a detailed description.} When applying a Unicode normalization (NFC or NFD, see previous section), the diacritics in different positions are put in a specified order. This process therefore harmonizes the difference between different encodings, for example, of a midtone creaky voice vowel <ḛ̄>. This grapheme cluster can be encoded either as <e> + <̄> + <̰> or as <e> + <̰> + <̄> . To prevent this twofold encoding, the Unicode Standard specifies the second ordering as canonical (diacritics below before diacritics above).

When encoding a string according to the Unicode Standard, it is possible to do this either using the NFC (``composition'') or NFD (``decomposition'') normalization. Decomposition implies that precomposed characters (like <á> LATIN SMALL LETTER A WITH ACUTE at U+00E1) will be split into its parts. This might sound preferable for a linguistic analysis, as the different diacritics are separated from the base characters. However, note that most attached elements like strokes (e.g.~in the <ɨ>), retroflex hooks (e.g.~in <ʐ>) or rhotic hooks (e.g.~in <ɝ>) will not be decomposed, but strangely enough a cedilla (like in <ç>) will be decomposed. In general, Unicode decomposition does not behave like a feature decomposition as expected from a linguistic perspective. It is thus important to consider Unicode decomposition only as a technical procedure, and not assume that it is linguistically sensible.

Facing the problem of specifying a consistent ordering of diacritics while developing a large database of phonological inventories from the world's languages, Moran (2012: 540) defines a set of diacritic ordering conventions.\footnote{The most recent version of these conventions is online: http://phoible.github.io/conventions/} The conventions are influenced by the linguistic literature, though some ad-hoc decisions had to be taken. The goal was to explicitly define all character sequences so that the vast variety of phonemes found in descriptions of the world's language were normalized into consistent character sequences, e.g.~if one language description uses and another , when both are intended to be phonetically equivalent, then a decision to normalize to one form was taken. For example, when a character sequence contains more than one character in Spacing Modifier Letters, the order that is proposed is the following (where <\textbar{}> indicates ``or'' and <→> indicates ``precedes''):
\begin{itemize}
	\itemsep1pt\parskip0pt\parsep0pt 
	\item \textsc{Spacing Modifier Letters ordering:} ( unreleased <̚> \textbar{} lateral release <ˡ> \textbar{} nasal release <ⁿ>) → ( palatalized <ʲ>) → ( labialized <ʷ>) → ( velarized <ˠ>) → ( pharyngealized <ˤ>) → ( aspirated <ʰ> \textbar{} ejective <ʼ>) → ( long <ː> \textbar{} half-long <ˑ>) 
\end{itemize}

If a character sequence contains more than one diacritic below the base character, then the place feature is applied first (dental, laminal, apical, fronted, backed, lowered, raised), then the laryngeal setting (voiced, voiceless, creaky voice, breathy voice) and finally the syllabic or non-syllabic marker (for vowels, ATR gets put on between the place and laryngeal setting). So:
\begin{itemize}
	\itemsep1pt\parskip0pt\parsep0pt 
	\item \textsc{Combining Diacritical Marks (below) ordering:} ( dental <t̪> \textbar{} laminal <t̻> \textbar{} apical <t̺>) → ( fronted <u̟> \textbar{} backed <e̠>) →( lowered <e̞> \textbar{} raised <e̝>) → ( ATR <e̘ e̙>) → ( voiced <s̬> \textbar{} voiceless <n̥> \textbar{} creaky voice <b̰> \textbar{} breathy voice <b̤>) → ( syllabic <n̩> \textbar{} non-syllabic <e̯>) 
\end{itemize}

Character sequences with diacritics above the base character were not problematic in Moran 2012 because they include only the centralized, mid-centralized and nasalized combining characters. Moran (2012) marks tones as singletons with Space Modifier Letters, e.g. \textless{}˦\textgreater{} for a phonemic high tone, instead of accent diacritics, alleviating potential conflicts. Building on the work of Moran (2012), if a character sequence contains more than one diacritic above the base character, we propose:
\begin{itemize}
	\itemsep1pt\parskip0pt\parsep0pt 
	\item \textsc{Combining Diacritical Marks (above) ordering:} (centralized <ë> \textbar{} mid-centralized <e̽>) → (extra short <ĕ>) →( tone accents, e.g. <è> ) → ( Spacing Modifier Letters ) → ( tone letters, e.g. <e˦>) 
\end{itemize}
% \chapter{Practical recommendations}
\label{practical-recommendations}

% A section which is missing in something called a "Cookbook" would be
% $ practical recommendations on how to input Unicode characters. There are
% various character selection tools, shortcuts on the keyboard, the
% shapecatcher website references at several places, or the Wikipedia
% lists of glyphs and fileformat.info. Having all this in one section
% would be handy for the user. It is of course unrelated to the
% orthography profiles, but I imagine that many people will use this book
% as a primer on IPA+Unicode and actually disregard the last two chapters.
% For this group, such a summary would be useful.

In this chapter, we provide some practical recommendations for quickly finding information about the Unicode Standard, the International Phonetic Alphabet (IPA), and tools and resources for working with each. This chapter is meant to be a short guide for novice users who are not interested in the programmatic aspects presented in the next two chapters. Instead, we aim here at ordinary working linguists who want to read about the basics and who want to easily insert special characters into their digital documents and applications.

\section{Unicode}
We discussed the Unicode Consortium's approach to computationally encoding writing systems in detail in Section \ref{the-unicode-approach}. The common pitfalls that we have encountered when using the Unicode Standard are discussed in detail in Chapter \ref{unicode-pitfalls}. Together these chapters provide users with an in-depth background about the hurdles they may encounter when using the Unicode Standard for encoding linguistic data or for developing multilingual applications. The provided inforation is by no means exhaustive and it is aimed at both linguists and programmers. For general background information about Unicode and character encodings, try these resources:

\begin{itemize}
	\item \url{http://www.unicode.org/standard/WhatIsUnicode.html}
	\item \url{https://en.wikipedia.org/wiki/Unicode}
	\item \url{https://www.w3.org/International/articles/definitions-characters/}
\end{itemize}

% \section{Unicode character pickers}
For practical purposes, users need a way to insert special characters (i.e.\ characters that are not easily entered via their keyboards) into documents and software applications. There are a few basic approaches for inserting special characters. One way is to use software-specific functionality, when it is available. For example, Microsoft Word has an insert-special-symbol-or-character function that allows users to scroll through a table of special characters across different scripts. Special characters can be then inserted into the document by clicking on them. Another way is to install a system-wide application for special character insertion. We have long been fans of the PopChar application from Ergonis Software, which is a small program that can insert Unicode characters.\footnote{\url{http://www.ergonis.com/products/popcharx/}}

There are also web-based so-called Unicode character pickers available through the browser that allow for the creation and insert of special characters, which can then be copied \& pasted into documents or software applications. For example:

\begin{itemize}
	\item \url{https://unicode-table.com/en/}
	\item \url{https://r12a.github.io/pickers/}
\end{itemize}

Yet another option for special character insertion includes operating system-specific shortcuts. For example on the Mac, holding down a key on the keyboard for a second, say <u>, triggers a pop up with the options <û, ü, ù, ú, ū> which can then be chosen by number (1--5). This method is convenient for occasionally inserting special characters, but it is burdensome for inserting many characters. 

Therefore, there are many applications for installing language-specific or script-specific keyboards. These programs typically override keys or keystrokes on the user's keyboard and therefore allow them to quickly keyboard the insertion of specal characters (once the layout of the new keyboard is mastered). Many operating systems provide such keyboards as part of the base operating system, e.g.\ Microsoft Windows, Linux, and OS X, have special character Unicode eyboards pre-installed. Seach for how to access these pre-installed options on your operating system.

Lastly, there are third party applications that provide virtual keyboards. They can be language-specific or devoted specifically to IPA. Two popular programs are:

\begin{itemize}
	\item \url{https://keyman.com/}
	\item \url{http://scripts.sil.org/ipa-sil_keyboard}
\end{itemize}



\section{IPA}
In Chapter \ref{the-international-phonetic-alphabet} we described in detail the history of the International Phonetic Alphabet and how it became encoded in the Unicode Standard. In Chapter \ref{ipa-meets-unicode} we describe the resulting pitfalls from their marriage. These two chapters provide a very detailed overview of the challenges that users face when working with the two standards.

For general information about the International Phonetic Alphabet, the standard text is the \textit{Handbook of the International Phonetic Association: A Guide to the Use of the International Phonetic Alphabet} \cite{IPA2007}. It described in detail the principles and premises of the IPA that we summarize in Section \ref{IPApremises-principles}. The Association makes available information about itself  online\footnote{\url{https://www.internationalphoneticassociation.org/}} and it also makes available the International Phonetic Alphabet in chart form.\footnote{\url{https://www.internationalphoneticassociation.org/content/ipa-chart}} Wikipedia also has a comprehensive article about the IPA.\footnote{\url{https://en.wikipedia.org/wiki/International_Phonetic_Alphabet}}

There are several good Unicode IPA character pickers available on the Web, including:

\begin{itemize}
	\item \url{https://r12a.github.io/pickers/ipa/}
	\item \url{https://westonruter.github.io/ipa-chart/keyboard/}
	\item \url{http://ipa.typeit.org/}
\end{itemize}

\noindent Various linguistics departments also provide information about IPA fonts, software, and inserting Unicode IPA characters. Some good ones are:

\begin{itemize}
	\item \url{http://www.phon.ucl.ac.uk/resource/phonetics/}
	\item \url{https://www.york.ac.uk/language/current/resources/freeware/ipa-fonts-and-software/}
\end{itemize}	

Regarding fonts that display Unicode IPA correctly, many linguists turn to the IPA Unicode fonts developed by SIL Internnational. The complete SIL font list is available online.\footnote{\url{http://scripts.sil.org/SILFontList}} There is also a page that describes IPA transcription using the SIL fonts and provides an informative discussion on deciding which font to use.\footnote{\url{http://scripts.sil.org/ipahome}} Traditionally, IPA fonts popular with linguists were created and maintained by SIL International, so it is often the case in our experience that we encounter linguistics data in legacy IPA fonts, i.e.\ pre-Unicode fonts such as SIL IPA93.\footnote{\url{http://scripts.sil.org/FontFAQ_IPA93}} SIL International does a good job of describing how to convert from legacy IPA fonts to Unicode IPA. By far the most popular fonts by SIL International are Doulos SIL and Charis SIL:

\begin{itemize}
	\item \url{https://software.sil.org/doulos/}}
	\item \url{https://software.sil.org/charis/}
\end{itemize}	


\section{Advanced topics}
If you have made it this far, and you are eager to know more about the technological aspects of the Unicode Standard and how they relate to software programming, we recommend two light-hearted blog posts on the topic. The classic blog post about what programmers should know about the Unicode Standard is Joel Spolsky's ``The Absolute Minimum Every Software Developer Absolutely, Positively Must Know About Unicode and Character Sets (No Excuses!)''.\footnote{\url{https://www.joelonsoftware.com/2003/10/08/the-absolute-minimum-every-software-developer-absolutely-positively-must-know-about-unicode-and-character-sets-no-excuses/}} A more recent blogpost, with a bit more of the technical details, is by David C. Zentgraf and is titled, ``What Every Programmer Absolutely, Positively Needs To Know About Encodings And Character Sets To Work With Text''.\footnote{\url{http://kunststube.net/encoding/}} This post is aimed at software developers and uses the PHP language for examples.

Here are some online resources that we find particularly useful for finding more information about individual Unicode characters and also for converting between encodings:

\begin{itemize}
	\item \url{http://www.fileformat.info/}
	\item \url{https://unicodelookup.com/}
	\item \url{https://r12a.github.io/scripts/featurelist/}
	\item \url{https://r12a.github.io/app-conversion/}
\end{itemize}

For users of Python, see the standard documentation on how to use Unicode in your programming applications.\footnote{\url{https://docs.python.org/3/howto/unicode.html}} For \latex users, the TIPA package is useful for inserting IPA characters into your typeset documents. See these resources:

\begin{itemize}
	\item \url{http://www.tug.org/tugboat/tb17-2/tb51rei.pdf}
	\item \url{https://ctan.org/pkg/tipa}
	\item \url{http://ptmartins.info/tex/tipacheatsheet.pdf}
\end{itemize}	

Lastly, we leave you with some Unicode humor for making it this far:

\begin{itemize}
	\item \url{https://xkcd.com/380/}
	\item \url{https://xkcd.com/1137/}
	\item \url{http://www.commitstrip.com/en/2014/06/17/unicode-7-et-ses-nouveaux-emoji/}
	\item \url{http://www.i18nguy.com/humor/unicode-haiku.html}
\end{itemize}




% \chapter{Orthography profiles}
\label{orthography-profiles}

\section{Characterizing writing systems}
\label{characterizing-writing-systems}

At this point in the course of rapid ongoing developments, we are left with a
situation in which the Unicode Standard offers a highly detailed and flexible
approach to deal computationally with writing systems, but it has unfortunately
not influenced the linguistic practice very much. In many practical situations,
the Unicode Standard is far too complex for the day-to-day practice in
linguistics because it does not offer practical solutions for the down-to-earth
problems of many linguists. In this section, we propose some simple practical
guidelines and methods to improve on this situation.

Our central aims for linguistics, to be approached with a Unicode-based
solution, are: (i) to improve the consistency of the encoding of sources, (ii)
to transparently document knowledge about the writing system (including
transliteration), and (iii) to do all of that in a way that is easy and quick to
manage for many different sources with many different writing systems. The
central concept in our proposal is the \textsc{orthography profile}, a simple
tab-separated CSV text file, that characterizes and documents a writing system.
We also offer basic implementations in Python and R to assist with the
production of such files, and to apply orthography profiles for consistency
testing, grapheme tokenization and transliteration. Not only can orthography
profiles be helpful in the daily practice of linguistics, they also succinctly
document the orthographic details of a specific source, and, as such, might
fruitfully be published alongside sources (e.g.~in digital archives). Also, in
high-level linguistic analyzes in which the graphemic detail is of central
importance (e.g.~phonotactic or comparative-historical studies), orthography
profiles can transparently document the decisions that have been taken in the
interpretation of the orthography in the sources used.

Given these goals, Unicode locale descriptions (see Section~\ref{terminology})
might seem like the ideal orthography profiles. However, there are various
practical obstacles preventing the use of such locale descriptions in the daily
linguistic practice, namely: (i) the XML-structure is too verbose to easily and
quickly produce or correct manually, (ii) locale descriptions are designed for a
wide scope on information (like date formats or names of weekdays) most of which
is not applicable for documenting writing systems, and (iii) most crucially,
even if someone made the effort to produce a technically correct locale
description for a specific source at hand, then it is nigh impossible to deploy
the description. This is because a locale description has to be submitted to and
accepted by the Unicode Common Locale Data Repository. The repository is
(rightly so) not interested in descriptions that only apply to a limited set of
sources (e.g.~only a single specific dictionary).

The major challenge then is developing an infrastructure to identify the
elements that are individual graphemes in a source, specifically for the
enormous variety of sources using some kind of alphabetic writing system.
Authors of source documents (e.g.~dictionaries, wordlists, corpora) use a
variety of writing systems that range from their own idiosyncratic
transcriptions to already well-established practical or longstanding
orthographies. Although the IPA is one practical choice as a sound-based
normalization for writing systems (which can act as an interlingual pivot to
attain interoperability across writing systems), graphemes in each writing
system must also be identified and standardized if interoperability across
different sources is to be achieved. In most cases, this amounts to more than
simply mapping a grapheme to an IPA segment because graphemes must first be
identified in context (e.g.~is the sequence one sound or two sounds or both?)
and strings must be tokenized, which may include taking orthographic rules into
account (e.g.~between vowels is /n/ and after a vowel but before a consonant is
a nasalized vowel /ṽ/). In our experience, data from each source must be
individually tokenized into graphemes so that its orthographic structure is
identified and its contents can be extracted. To extract data for analysis, a
source-by-source approach is required before an orthography profile can be
created. For example, almost each available lexicon on the world's languages is
idiosyncratic in its orthography and thus requires lexicon-specific approaches
to identify graphemes in the writing system and to map graphemes to phonemes, if
desired.

Thus, our key proposal for the characterization of a writing system is to use a
grapheme tokenization as an inter-orthographic pivot. Basically, any source
document is tokenized by graphemes, and only then a mapping to IPA (or any other
orthographic conversion) is performed. An orthography profile then is a
description of the units and rules that are needed to adequately model a
graphemic tokenization for a language variety as described in a particular
source document. An orthography profile summarizes the Unicode (tailored)
graphemes and orthographic rules used to write a language (the details of the
structure and assumptions of such a profile will be presented in the next
section).

% TODO: add tsoshi figure here?

As an example of graphemic tokenization, note the three different levels of
technological and linguistic elements that interact in the hypothetical lexical
form <tsʰṍ̰shi>:

\begin{enumerate}
	\def\labelenumi{\arabic{enumi}.} 
	\item code points (10 text elements): t s ʰ o ̃ ̰ ´ s h i 
	\item grapheme clusters (7 text elements): t s ʰ ṍ̰ s h i 
	\item tailored grapheme clusters (4 text elements): tsʰ ṍ̰ sh i 
\end{enumerate}

In (1), the string <tsʰṍ̰shi> has been tokenized into ten Unicode code points
(using NFD normalization), delimited here by space. Unicode tokenization is
required because sequences of code points can differ in their visual and logical
orders. For example, <õ̰> is ambiguous to whether it is the sequence of + <̰> +
<̃> or + <̃> + <̰>. Although these two variants are visually homoglyphs,
computationally they are different. Unicode normalization should be applied to
this string to reorder the code points into a canonical order, allowing the data
to be treated canonically equivalently for search and comparison. In (2), the
Unicode code points have been logically normalized and visually organized into
grapheme clusters, as specified by the Unicode Standard. The combining character
sequence <õ̰> is normalized and visually grouped together. Note that, the
MODIFIER LETTER SMALL H at \uni{02B0}, is not grouped with. This is because it
belongs to Spacing Modifier Letters category in the Unicode Standard. These
characters are underspecified for the direction in which they modify a host
character. For example, can indicate either pre- or post-aspiration (whereas the
nasalization or creaky diacritic is defined in the Unicode Standard to apply to
a specified base character). Finally, to arrive at the graphemic tokenization in
(3), tailored grapheme clusters are needed (as for example specified in an
orthography profile). For example, this orthography profile would specify that
the sequence of characters, and form a single grapheme, and that and form a
grapheme. The orthography profile could also specify orthographic rules,
e.g.~when tokenization graphemes, in say English, the in the forms and should be
treated as distinct sequences depending on their contexts.

\section{Informal description}
\label{informal-description-of-orthography-profiles}

An orthography profile describes the Unicode code points, characters, graphemes
and orthographic rules in a writing system. An orthography profile is a
language-specific (and often even resource-specific) description of the units
and rules that are needed to adequately model a writing system. An important
assumption of our work is that we assume a resource is encoded in Unicode (or
has been converted to Unicode). Any data source that the Unicode Standard is
unable to capture, will also not be captured by an orthography profile.

Informally, an orthography profile specifies the graphemes (or, in Unicode
parlance, \textsc{tailored grapheme clusters}) that are expected to occur in any
data to be analyzed or checked for consistency. These graphemes are first
identified throughout the whole data (a step which we call
\textsc{tokenization}), and possibly simply returned as such, possibly including
error messages about any parts of the data that are not specified by the
orthography profile. Once the graphemes are identified, they might also be
changed into other graphemes (a step which we call \textsc{transliteration}).
When a grapheme has different possible transliterations, then these differences
should be separated by contextual specification, possibly down to listing
individual exceptional cases.

In practice, we foresee a workflow in which orthography profiles are iteratively
refined, while at the same time inconsistencies and errors in the data to be
tokenized are corrected. In some more complex use-cases there might even be a
need for multiple different orthography profiles to be applied in sequence (see
Section~\ref{use-cases} on various exemplary use-cases). The result of any such
workflow will normally be a cleaned dataset and an explicit description of the
orthographic structure in the form of an orthography profile. Subsequently, the
orthography profiles can be easily distributed in scholarly channels alongside
the cleaned data, for example in supplementary material added to journal papers
or in electronic archives.

\section{Formal specification}
\label{formal-specification-of-orthography-profiles}

The formal specifications of an orthography profile (or simply \textsc{profile}
for short) are the following:

\begin{enumerate}
	\def\labelenumi{\arabic{enumi}.} 
	\item \textsc{A profile is a} \textsc{Unicode UTF-8 encoded text file} (ideally using NFC, no-BOM, and LF; see Section~\ref{pitfall-file-formats}, Pitfall: File Formats) that includes the information pertinent to the orthography. 
	\item \textsc{A profile is a} \textsc{tab-separated CSV file with an obligatory header line}. A minimal profile can have just a single column, in which case there will of course no tabs, but the first line will still be the header. For all columns we assume the name in the header of the CSV file to be crucial. The actual ordering of the columns is unimportant. 
	\item \textsc{Lines starting with a hash \textless{}\#\textgreater{} are ignored.} Comments and metadata can be included inside the file, but only as complete lines in the profile, to be marked by lines starting with hash \textsc{\#} (\textsc{number sign} at \uni{0023}). Hashes somewhere else in the file are to be treated literally, i.e.~hashes are only to be ignored when they occur at the start of a line.\footnote{Comments that belong to specific lines will have to be put in a separate column of the CSV file, e.g.~add a column called \textsc{comments}. Further, if the content of a profile contains a hash at the start of a line, either reorder the columns so the hash does not occur at the start of the line, or add a dummy column in front of the data to not have the data start with a hash.} 
	\item \textsc{Metadata are given in commented lines at the beginning of the text file in a basic \textsc{tag: value} format. }Metadata about the orthographic description given in the orthography profile includes, minimally, (i) author, (ii) date, (iii) title, (iv) a stable language identifier encoded in BCP 47/ISO 639-3, and (v) bibliographic data for resource(s) that illustrate the orthography described in the profile. 
\end{enumerate}

The content of a profile consists of lines, each describing a grapheme of the
orthography, using the following columns:

\begin{enumerate}
	\def\labelenumi{\arabic{enumi}.} 
	\item \textsc{A minimal profile consists of a single column with header \textsc{graphemes}}, listing each of the different graphemes in a separate line. 
	\item \textsc{Optional columns called \textsc{left} and \textsc{right} can be used to specify the left and right context of the grapheme, respectively.} The same grapheme can occur multiple times with different contextual specifications, for example to distinguish different pronunciations depending on the context. 
	\item \textsc{The columns \textsc{grapheme}, \textsc{left} and \textsc{right} can use regular expression metacharacters.} If regular expressions are used, then all literal usage of the special symbols, like full stops <.> or dollar signs <\$> (so-called \textsc{metacharacters}) have to be explicitly escaped by adding a backslash before them (i.e.~use <.> or <\$>). Note that any specification of context automatically expects regular expressions, so it is probably better to always escape all regular expression metacharacters when used literally in the orthography, i.e.~the following symbols will need to be preceded by a backslash: {[} {]} ( ) \{ \} ~+ * . - ! ? \^{} \$ . 
	\item \textsc{An optional column called \textsc{class} can be used to specify classes of graphemes}, for example to define a class of vowels. Users can simply add ad-hoc identifiers in this column to indicate a group of graphemes, which can then be used in the description of the graphemes or the context. The identifiers should of course be chosen such that they do not conflate with any symbols used in the orthography themselves. Note that such classes only refer to the graphemes, not to the context. 
	\item \textsc{Columns describing transliterations for each graphemes can be added and named at will}. Often more than a single possible transliteration will be of interest. Any software application using these profiles should use the names of these columns to select a specific transliteration column. 
	\item \textsc{Any other columns can be added freely, but will mostly be ignored by any software application using the profiles}. As orthography profiles are also intended to be read and interpreted by humans, it is often highly useful to add extra information on the graphemes in further columns, like for example Unicode codepoints, Unicode names, frequency of occurrence, examples of occurrence, explanation of the contextual restrictions, or comments. 
\end{enumerate}

For the automatic processing of the profiles, the following technical standards
will be expected:

\begin{enumerate}
	\def\labelenumi{\arabic{enumi}.} 
	\item \textsc{Each line of a profile will be interpreted as a regular expression. }Software applications using profiles can also offer to interpret a profile in the literal sense to avoid the necessity for the user to escape regular expressions metacharacters in the profile. However, this only is possible when no contexts or classes are described, so this seems only useful in the most basic orthographies. 
	\item \textsc{The \textsc{class} column will be used to produce explicit \textsc{or} chains of regular expressions}, which will then be inserted in the \textsc{graphemes}, \textsc{left} and \textsc{right} columns at the position indicated by the class-identifiers. For example, a class \textsc{V} as a context specification might be replaced by a regular expression like: (a\textbar{}e\textbar{}i\textbar{}o\textbar{}u\textbar{}ei\textbar{}au). Only the graphemes themselves are included here, not any contexts specified for the elements of the class. 
	\item \textsc{The \textsc{left} and \textsc{right} contexts will be included into the regular expressions by using lookbehind and lookahead}. Basically, the actual regular expression syntax of lookbehind and lookahead is simply hidden to the users by allowing them to only specify the contexts themselves. Internally, the contexts in the columns \textsc{left} and \textsc{right} are combined with the column \textsc{graphemes} to form a complex regular expression like: (?\textless{}=left)graphemes(?=right). 
	\item \textsc{The regular expressions will be applied in the order as specified in the profile, from top to bottom.} A software implementation can offer help in figuring out the optimal ordering of the regular expressions, but should then explicitly report on the order used. 
\end{enumerate}

The actual implementation of the profile on some text-string will function as
follows:

\begin{enumerate}
	\def\labelenumi{\arabic{enumi}.} 
	\item \textsc{All graphemes are matched in the text before they are tokenized or transliterated}. In this way, there is no necessity for the user to consider `feeding' and `bleeding' situations, in which the application of a rule either changes the text so another rule suddenly applies (feeding) or prevents another rule to apply (`bleeding'). 
	\item \textsc{The matching of the graphemes can occur either globally or linearly. }From a computer science perspective, the most natural way to match graphemes from a profile in some text is by walking linearly through the text-string from left to right, and at each position go through all graphemes in the profile to see which one matches, then go to the position at the end of the matched grapheme and start over. This is basically how a finite state transducer works, which is a well-established technique in computer science. However, from a linguistic point of view, our experience is that most linguists find it more natural to think from a global perspective. In this approach, the first grapheme in the profile is matched everywhere in the text-string first, before moving to the next grapheme in the profile. Theoretically, these approaches will lead to different results, though in practice of actual natural language orthographies they almost always lead to the same result. Still, we suggest that any software application using orthography profiles should offer both approaches (i.e. \textsc{global} or \textsc{linear}) to the user. The approach used should be documented in the metadata as \textsc{tokenization method}. 
	\item \textsc{The matching of the graphemes can occur either in NFC or NFD. }By default, both the profile and the text-string to be tokenized should be treated as NFC (see section \ref{pitfall-canonical-equivalence}, Pitfall: Canonical equivalence, above). However, in some use-cases it turns out to be practical to treat both text and profile as NFD. This typically happens when very many different combinations of diacritics occur in the data. An NFD-profile can then be used to first check which individual diacritics are used, before turning to the more cumbersome inspection of all combinations. We suggest that any software application using orthography profiles should offer both approaches (i.e. \textsc{NFC} or \textsc{NFD}) to the user. The approach used should be documented in the metadata as \textsc{unicode normalization}. 
	\item \textsc{The text-string is always returned in tokenized form} by separating the matched graphemes by a user-specified symbols-string. Any transliteration will be returned on top of the tokenization. 
	\item \textsc{Leftover characters (i.e.~characters that are not matched by the profile) should be reported to the user as errors.} Typically, the unmatched character are replaced in the tokenization by a user-specified symbol-string. 
\end{enumerate}

Any software application offering to use orthography profile:

\begin{enumerate}
	\def\labelenumi{\arabic{enumi}.} 
	\item \textsc{should offer user-options} to specify:
	\begin{enumerate}
		\def\labelenumii{\arabic{enumii}.} 
		\item \textsc{the name of the column to be used for transliteration} (if any). 
		\item \textsc{the symbol-string to be inserted between graphemes.} Optionally, a warning might be given if the chosen string includes characters from the orthography itself. 
		\item \textsc{the symbol-string to be inserted for unmatched strings} in the tokenized and transliterated output. 
		\item \textsc{the tokenization method}, i.e.~whether the tokenization should proceed \textsc{global} or \textsc{linear}. 
		\item \textsc{unicode normalization}, i.e.~whether the text-string and profile should use \textsc{NFC} or \textsc{NFD}. 
	\end{enumerate}
	\item \textsc{might offer user-options }to:
	\begin{enumerate}
		\def\labelenumii{\arabic{enumii}.} \setcounter{enumii}{5} 
		\item \textsc{assist in the ordering of the graphemes.} In our experience, it makes sense to apply larger graphemes before shorter graphemes, and to apply graphemes with context before graphemes without context. Further, frequently relevant rules might be applied after rarely relevant rules (though frequency is difficult to establish in practice, as it depends on the available data). Also, if this all fails to give any decisive ordering between rules, it seems useful to offer linguists the option to reverse the ordering from any manual specified ordering, because linguists tend to write the more general rule first, before turning to exceptions or special cases. 
		\item \textsc{assist in dealing with upper and lower case characters.} It seems practical to offer some basic case matching, so characters like <a> and <A> are treated equally. This will be useful in many concrete cases, although the user should be warned that case matching does not function universally in the same way across orthographies. Ideally, users should prepare orthography profiles with all lowercase and uppercase variants explicitly mentioned, so by default no case matching should be performed. 
		\item \textsc{treat the profile literal}, i.e.~to not interpret regular expression metacharacters. Matching graphemes literally often leads to strong speed increase, and would allow users to not needing to worry about escaping metacharacters. However, in our experience all actually interesting use-cases of orthography profiles include some contexts, which automatically prevents any literal interpretation, so by default the matching should not be literal. 
	\end{enumerate}
	\item \textsc{should return the following information} to the user:
	\begin{enumerate}
		\def\labelenumii{\arabic{enumii}.} \setcounter{enumii}{8} 
		\item \textsc{the original text-strings to be processed in the used Unicode normalization}, i.e.~in either NFC or NFD as specified by the user. 
		\item \textsc{the tokenized strings}, with additionally any transliterated
        strings, if transliteration is requested. 
		\item \textsc{a survey of all errors encountered}, ideally both in which
        text-strings any errors occurred and which characters in the
        text-strings lead to errors. 
		\item \textsc{a reordered profile}, when any automatic reordering is offered 
	\end{enumerate}
\end{enumerate}

\section{Examples}

[Here should a few abstract short simple examples be added]

\ 

Note that to deal with ambiguous parsing cases, we can use the Unicode approach
using the zero width joiner. This is actually a non-joiner (the name is
confusing): the idea is to add this character into the text to identify cases in
which a sequence of characters is not supposed to be a complex grapheme (even
though the sequence is in the orthography profile)


% \include{chapters/implementation2}

%%%%%%%%%%%%%%%%%%%%%%%%%%%%%%%%%%%%%%%%%%%%%%%%%%%%
%%%                                              %%%
%%%             Backmatter                       %%%
%%%                                              %%%
%%%%%%%%%%%%%%%%%%%%%%%%%%%%%%%%%%%%%%%%%%%%%%%%%%%%

% There is normally no need to change the backmatter section
\backmatter
\phantomsection%this allows hyperlink in ToC to work
{\sloppy\printbibliography[heading=\lsReferencesTitle]}
\cleardoublepage

\phantomsection 
\addcontentsline{toc}{chapter}{\lsIndexTitle} 
\addcontentsline{toc}{section}{\lsNameIndexTitle}
\ohead{\lsNameIndexTitle} 
\printindex 
\cleardoublepage
  
%\phantomsection 
%\addcontentsline{toc}{section}{\lsLanguageIndexTitle}
%\ohead{\lsLanguageIndexTitle} 
%\printindex[lan] 
%\cleardoublepage
  
%\phantomsection 
%\addcontentsline{toc}{section}{\lsSubjectIndexTitle}
%\ohead{\lsSubjectIndexTitle} 
%\printindex[sbj]
%\ohead{} 

\end{document}

% you can create your book by running
% xelatex lsp-skeleton.tex
%
% you can also try a simple 
% make
% on the commandline
