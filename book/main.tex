% !BIB TS-program = biber
% !BIB program = biber
\documentclass[output=book,nonflat,modfonts,
% colorlinks, citecolor=brown,
citecolor=brown,
% draft,draftmode   
		]{langsci/langscibook}\usepackage[]{graphicx}\usepackage[]{color}
%% maxwidth is the original width if it is less than linewidth
%% otherwise use linewidth (to make sure the graphics do not exceed the margin)
\makeatletter
\def\maxwidth{ %
  \ifdim\Gin@nat@width>\linewidth
    \linewidth
  \else
    \Gin@nat@width
  \fi
}
\makeatother

\definecolor{fgcolor}{rgb}{0.345, 0.345, 0.345}
\newcommand{\hlnum}[1]{\textcolor[rgb]{0.686,0.059,0.569}{#1}}%
\newcommand{\hlstr}[1]{\textcolor[rgb]{0.192,0.494,0.8}{#1}}%
\newcommand{\hlcom}[1]{\textcolor[rgb]{0.678,0.584,0.686}{\textit{#1}}}%
\newcommand{\hlopt}[1]{\textcolor[rgb]{0,0,0}{#1}}%
\newcommand{\hlstd}[1]{\textcolor[rgb]{0.345,0.345,0.345}{#1}}%
\newcommand{\hlkwa}[1]{\textcolor[rgb]{0.161,0.373,0.58}{\textbf{#1}}}%
\newcommand{\hlkwb}[1]{\textcolor[rgb]{0.69,0.353,0.396}{#1}}%
\newcommand{\hlkwc}[1]{\textcolor[rgb]{0.333,0.667,0.333}{#1}}%
\newcommand{\hlkwd}[1]{\textcolor[rgb]{0.737,0.353,0.396}{\textbf{#1}}}%
\let\hlipl\hlkwb

\usepackage{framed}
\makeatletter
\newenvironment{kframe}{%
 \def\at@end@of@kframe{}%
 \ifinner\ifhmode%
  \def\at@end@of@kframe{\end{minipage}}%
  \begin{minipage}{\columnwidth}%
 \fi\fi%
 \def\FrameCommand##1{\hskip\@totalleftmargin \hskip-\fboxsep
 \colorbox{shadecolor}{##1}\hskip-\fboxsep
     % There is no \\@totalrightmargin, so:
     \hskip-\linewidth \hskip-\@totalleftmargin \hskip\columnwidth}%
 \MakeFramed {\advance\hsize-\width
   \@totalleftmargin\z@ \linewidth\hsize
   \@setminipage}}%
 {\par\unskip\endMakeFramed%
 \at@end@of@kframe}
\makeatother

\definecolor{shadecolor}{rgb}{.97, .97, .97}
\definecolor{messagecolor}{rgb}{0, 0, 0}
\definecolor{warningcolor}{rgb}{1, 0, 1}
\definecolor{errorcolor}{rgb}{1, 0, 0}
\newenvironment{knitrout}{}{} % an empty environment to be redefined in TeX

\usepackage{alltt}
\usepackage[]{graphicx}\usepackage[]{color}

%%%%%%%%%%%%%%%%%%%%%%%%%%%%%%%%%%%%%%%%%%%%%%%%%%%%
%%%                                              %%%
%%%          additional packages                 %%%
%%%                                              %%%
%%%%%%%%%%%%%%%%%%%%%%%%%%%%%%%%%%%%%%%%%%%%%%%%%%%%

% put all additional commands you need in the 
% following files. If you do not know what this might 
% mean, you can safely ignore this section

\usepackage{verbatim}
\usepackage{hologo}
\input{localmetadata.tex}
% add all extra packages you need to load to this file  
\usepackage{url}

% Math
\usepackage{amsmath} 
\usepackage{unicode-math}
\setmathfont{Asana-Math.otf} % this looks much better for formulas
\usepackage[libertine]{newtxmath}
% \usepackage[no-math]{fontspec}

% % override defaults from langsci.cls
% % to get straight quotes in code-snippets
% \setmonofont[
% 	Ligatures={Common},Scale=MatchLowercase,Path=\fontpath,
% 	BoldFont = FreeMonoBold_B.otf ,
% 	SlantedFont = FreeMonoOblique_B.otf ,		
% 	BoldSlantedFont = FreeMonoBoldOblique_B.otf 	
% 	]{FreeMono_B.otf}

% override defaults from langsci.cls
% to get straight quotes in code-snippets
\newfontfamily\myfont[
	Ligatures={Common},Scale=0.8,Path=\fontpath,
	BoldFont = FreeMonoBold_B.otf ,
	SlantedFont = FreeMonoOblique_B.otf ,		
	BoldSlantedFont = FreeMonoBoldOblique_B.otf 	
	]{FreeMono_B.otf}

% some additional possibilities for enumerations
\usepackage{enumitem}
\setitemize{noitemsep}
\setenumerate{noitemsep}

% sizes
\usepackage{moresize}

% large tables
\usepackage{tabularx}
\usepackage{booktabs} % nice lines in tables
\usepackage{xtab} % xtabular for better multipage tables
\xentrystretch{-0.16} % squeeze more lines on a page
\usepackage{array} % for better handling of columns
\usepackage{needspace} % for positioning tables on the page
\newcolumntype{L}[1]{>{\footnotesize\raggedright\let\newline\\\arraybackslash\hspace{0pt}}p{#1}}

\widowpenalty=10000
\clubpenalty=10000

% I have tried to use monospaced numbers in the TOC, but this does not work...
%\settocstylefeature{\fontspec[Numbers=Monospaced]{LinLibertineO}}
%\settocfeature{\fontspec[Numbers=Monospaced]{LinLibertineO}}

% other ideas that also don't give the desired results
%\usepackage{tocloft}
%\renewcommand{\cftXpagefont}{\fontspec[Numbers=Monospaced]{LinLibertineO}}


%%%%%%%%%%%%%%%%%%%%%%%%%%%%%%%%%%%%%%%%%%%%%%%%%%%%
%%%                                              %%%
%%%           Examples                           %%%
%%%                                              %%%
%%%%%%%%%%%%%%%%%%%%%%%%%%%%%%%%%%%%%%%%%%%%%%%%%%%% 

\usepackage{lsp-gb4e} 

%% to add additional information to the right of examples, uncomment the following line
% \usepackage{jambox}
%% if you want the source line of examples to be in italics, uncomment the following line
% \renewcommand{\exfont}{\itshape}

\usepackage{listings}

\lstset{ %
  backgroundcolor=\color{white},   % choose the background color; you must add \usepackage{color} or \usepackage{xcolor}
  basicstyle=\footnotesize\ttfamily,        % the size of the fonts that are used for the code 
  % keywordstyle=\color{blue!60!black},       % keyword style
  keywordstyle=\myfont,       % keyword style
  language=Python,                 % the language of the code 
  % stringstyle=\color{green!60!black},     % string literal style 
  stringstyle=\myfont,     % string literal style 
  morekeywords={token,xlink:href, Action, Value, Cursor,LogEvent}
} 

\input{localhyphenation.tex}
\input{localcommands.tex}
\bibliography{localbibliography}

\lstset{
  literate={ĉ}{{\^c}}1
           {á}{{\'a}}1
           {ã̌}{{\'a}}1
}

%%%%%%%%%%%%%%%%%%%%%%%%%%%%%%%%%%%%%%%%%%%%%%%%%%%%
%%%                                              %%%
%%%             Frontmatter                      %%%
%%%                                              %%%
%%%%%%%%%%%%%%%%%%%%%%%%%%%%%%%%%%%%%%%%%%%%%%%%%%%%
\IfFileExists{upquote.sty}{\usepackage{upquote}}{}
\begin{document}
\maketitle
\frontmatter
\include{chapters/preface}
\include{chapters/acknowledgments}
% \include{chapters/abbreviations}
\tableofcontents
\mainmatter%

%%%%%%%%%%%%%%%%%%%%%%%%%%%%%%%%%%%%%%%%%%%%%%%%%%%%
%%%                                              %%%
%%%             knitr settings                   %%%
%%%                                              %%%
%%%%%%%%%%%%%%%%%%%%%%%%%%%%%%%%%%%%%%%%%%%%%%%%%%%%

% Here are settings for knitr, which will be removed in the .tex file
% spacing of code-chunks is set with the "knitrout" environment in localcommands.tex



%%%%%%%%%%%%%%%%%%%%%%%%%%%%%%%%%%%%%%%%%%%%%%%%%%%%
%%%                                              %%%
%%%             Chapters                         %%%
%%%                                              %%%
%%%%%%%%%%%%%%%%%%%%%%%%%%%%%%%%%%%%%%%%%%%%%%%%%%%%

\chapter{Writing systems}
\label{writing_systems}

% ==========================
\section{Introduction}
\label{introduction}
% ==========================

Writing systems arise and develop in a complex mixture of cultural,
technological and practical pressures. They tend to be highly conservative, in
that people who have learned to read and write in a specific way -- however
impractical or tedious -- are mostly unwilling to change their habits. Writers
tend to resist spelling reforms. In all literate societies there exists a strong
socio-political mainstream that tries to force unification of writing (for
example by strongly enforcing ``right'' from ``wrong'' spelling in schools).
However, there are also communities of users who take as many liberties in
their writing as they can get away with.

For example, the writing of tone diacritics in Yoruba is often proclaimed to be
the right way to write, although many users of Yoruba orthography seem to be
perfectly fine with leaving them out. As pointed out by the proponents of the
official rules, there are some homographs when leaving out the tone diacritics
\citep[44]{Olumuyiw2013}. However, writing systems (and the languages they
represent) are normally full of homographs (and homophones), which is 
not a problem at all for speakers of the language. More importantly, writing is
not just a purely functional tool, but just as importantly it is a mechanism to
signal social affiliation. By showing that you \textit{know the rules} of
expressing yourself in writing, others will more easily accept you as a worthy
participant in their group -- whether it means following the official rules 
when writing a job application or conforming to the informal rules when writing text 
messages. The case of Yoruba writing is
an exemplary case, as even after more than a century of efforts to standardize
the writing systems, there is still a wide range of variation of writing in
daily use \citep{Olumuyiw2013}.

\subsubsection*{Formalizing orthographic structure}

The resulting cumbersome and often illogical structure of writing systems, and
the enormous variability of existing writing systems for the world's languages, 
is a fact of life that scholars have to accept and they should try to adapt to as well as
they can. Our goal in this book is a proposal for how to do exactly that: formalize 
knowledge about individual writing systems in a form that is easy to use 
for linguists in daily practice, and at the same time computer-readable for 
automated processing.

When considering worldwide linguistic diversity, including the many
lesser-studied and endangered languages, there exist numerous different
orthographies using symbols from the same scripts. For example, there are
hundreds of orthographies using Latin-based alphabetic scripts. All of these
orthographies use the same symbols, but these symbols differ in meaning and
usage throughout the various orthographies. To be able to computationally use
and compare different orthographies, we need a way to specify all orthographic
idiosyncrasies in a computer-readable format. We call such specifications
\textsc{orthography profiles}. Ideally, these specifications have to be
integrated into so-called Unicode Locales,\footnote{\url{http://cldr.unicode.org/locale_faq-html}} 
though we will argue that in practice this is often not the most useful solution for the kind of problems
arising in the daily practice of many linguists. Consequently, a central goal of this book is to flesh out the linguistic-specific challenges regarding Unicode Locales and to work out suggestions to simplify their structure for usage in a linguistic context. Conversely, we also aim to improve linguists' understanding and appreciation for the accomplishments of the Unicode Consortium in the development of the Unicode Standard.

The need to use computational methods to compare different orthographies arises most
forcefully in the context of language comparison. Concretely, the proper
processing of orthographies and transcription systems becomes critical for the
development of quantitative methods for language comparison and historical
reconstruction. In order to investigate worldwide linguistic variation and to
model the historical and areal processes that underlie linguistic
diversity, it is crucial that we are able to flexibly process numerous
resources in different orthographies. In many cases even different resources
on the same language use different orthographic conventions. Another
orthographic challenge that we encounter regularly in our linguistic practice is
electronic resources on a particular language that claim to follow a specific
orthographic convention (often a resource-specific convention), but on closer
inspection such resources are almost always not consistently encoded. Thus, a
second goal of our orthography profiles is to allow for an easy specification of
orthographic conventions, and use such profiles to check consistency and to
report errors to be corrected.

% \footnote{cf.~\citet{Steiner_etal2011,List2012,List2012a,ListMoran2013,MoranProkic2013}}

A central step in our proposed solution to this problem is the tailored grapheme
separation of strings of symbols, a process we call \textsc{grapheme
tokenization}. Basically, given some strings of symbols (e.g.~morphemes, words,
sentences) in a specific source, our first processing step is to specify how
these strings should be separated into graphemes, considering the specific
orthographic conventions used in a particular source document. Our experience is
that such a graphemic tokenization can often be performed reasonably accurately
without extensive in-depth knowledge about the phonetic and phonological details
of the language in question. For example, the specification that <ou> is a
grapheme of English is a much easier task than to specify what exactly the
phonetic values of this grapheme are in any specific occurrence in English
words. Grapheme separation is a task that can be performed relatively reliably
and with limited time and resources (compare, for example, the
daunting task of creating a complete phonetic or phonological normalization).

Although grapheme tokenization is only one part of the solution, it is an
important and highly fruitful processing step. Given a grapheme tokenization,
various subsequent tasks become easier, for instance (a) temporarily reducing the
orthography in a processing pipeline, e.g.~only distinguishing high versus low
vowels; (b) normalizing orthographies across sources (often including temporary
reduction of oppositions), e.g.~specifying an (approximate) mapping to the
International Phonetic Alphabet; (c) using co-occurrence statistics across
different languages (or different sources in the same language) to estimate the
probability of grapheme matches, e.g.~with the goal to find regular sound
changes between related languages or transliterations between different sources
in the same language.

\subsubsection*{Structure of this book}

Before we deal with these proposals we will first discuss the theoretical
background on text encoding, on the Unicode Standard, and on the International
Phonetic Alphabet. In the remainder of this chapter, we give an extended
introduction to the notion of encoding (Section~\ref{encoding}) and the
principles of writing systems from a linguistic perspective
(Section~\ref{linguistic-terminology}). In Chapter~\ref{the-unicode-approach}, we 
discuss the notions of encoding and writing systems from the perspective of the Unicode
Consortium. We consider the Unicode
Standard to be a breakthrough (and ongoing) development that fundamentally
changed the way we look at writing systems, and we aim to provide here a
slightly more in-depth survey of the many techniques that are available in the
standard. A good appreciation for the solutions that the Unicode Consortium has created 
allows for a thorough understanding of the possible pitfalls that one might
encounter when using the Unicode Standard in general (Chapter~\ref{unicode-pitfalls}). Linguists are more often interested in using the Unicode Standard with the International Phonetic Alphabet (IPA). We first provide a history of the development of the IPA and early attempts to encode it electronically (Chapter~\ref{the-international-phonetic-alphabet}) before we discuss the rather problematic marriage of the IPA with the Unicode Standard (Chapter~\ref{ipa-meets-unicode}).

In the second part of the book (Chapters~\ref{practical-recommendations}, \ref{orthography-profiles} \&
\ref{implementation}) we describe our proposals of how to deal with the Unicode
Standard in the daily practice of (comparative) linguistics. First, we provide some 
practical recommendations for using the Unicode Standard and IPA for ordinary 
working linguists and for computer programmers (Chapter~\ref{practical-recommendations}). 
Second, we discuss the
challenges of characterizing a writing system; to solve these problems, we
propose the notions of orthography profiles, closely related to Unicode locale
descriptions (Chapter~\ref{orthography-profiles}). Lastly, we provide an introduction to two open source libraries that we have developed, in Python and R, for working with linguistic data and orthography profiles (Chapter~\ref{implementation}).


\subsubsection*{Conventions}

The following conventions are adhered to in this book. All phonemic and phonetic
representations are given in the International Phonetic Alphabet (IPA), unless
noted otherwise \citep{IPA2015}. Standard conventions are used for
distinguishing between graphemic < >, phonemic / / and phonetic [ ]
representations. For character descriptions, we follow the notational
conventions of the Unicode Standard \citep{Unicode2018}. Character names are
represented in small capital letters (e.g.~\textsc{latin small letter schwa})
and code points are expressed as U\emph{+n}, where \emph{n} is a four to six
digit hexadecimal number, e.g.~\uni{0256}, which can be rendered as the glyph <ə>.

% ==========================
\section{Encoding}
\label{encoding}
% ==========================

There are many in-depth histories of the origin and development of writing
systems \citep[e.g.~][]{Robinson1995,Powell2012}, a story that we therefore will
not repeat here. However, the history of turning writing into machine-readable
code is not so often told, so we decided to offer a short survey of the major
developments of such encoding here.\footnote{Because of the recent
history as summarized in this section, we have used mostly rather ephemeral
internet sources. When not referenced by traditional literature in the
bibliography, we have used \url{http://www.unicode.org/history/} and various
Wikipedia pages for the information presented here. A useful survey of the
historical development of the physical hardware of telegraphy and
telecommunication is \citet{Huurdeman2003}. Most books that discuss the
development of encoding of telegraphic communication focus of cryptography,
e.g.~\citet{Singh1999}, and forego the rather interesting story of open,
i.e.~non-cryptographic, encoding that is related here.} This history turns out
to be intimately related to the history of telegraphic communication.

\subsubsection*{Telegraphy}

Writing systems have existed for roughly 6000 years, allowing people to exchange
messages through time and space. Additionally, to quickly bridge large geographic
distances, telegraphic systems of communication (from Greek \emph{τῆλε
γράφειν} `distant writing') have a long and widespread history since ancient
times. The most common telegraphic systems worldwide are so-called whistled
languages \citep{Meyer2015}, but also drumming languages \citep{Meyer2012} and
signaling by smoke, fire, flags, or even changes in water levels through
hydraulic pressure have been used as forms of telegraphy. 

Telegraphy was reinvigorated at the end of the eighteenth century through the
introduction of so-called semaphoric systems by Claude Chapelle to convey
messages over large distances. Originally, various specially designed
contraptions were used to send messages. Today, descendants of these systems are
still in limited use, for example utilizing flags or flashing lights. The
innovation of those semaphoric systems was that all characters of the
written language were replaced one-to-one by visual signals. Since then, all
telegraphic systems have adopted this principle,\footnote{Sound- and
video-based telecommunication take a different approach by ignoring
the written version of language and they directly encode sound waves or light
patterns.} namely that any language to be
transmitted first has to be turned into some orthographic system, which
subsequently is encoded for transmission by the sender, and then turned back
into orthographic representation at the receiver side. This of course implies 
that the usefulness of any such telegraphic
encoding completely depends on the sometimes rather haphazard structure of
orthographic systems.

In the nineteenth century, electric telegraphy led to a further innovation in
which written language characters were encoded by signals sent through a copper
wire. Originally, \textsc{bisignal codes} were used, consisting of two different
signals. For example, Carl Friedrich Gauss in 1833 used positive and negative
current \citep[282]{Mania2008}. More famous and influential, Samuel Morse in
1836 used long and short pulses. In those bisignal codes each character from the
written language was encoded with a different number of signals (between one and
five), so two different separators are needed: one between signals and one
between characters. For example, in Morse-code there is a short pause between
signals and a long pause between characters.\footnote{Actually, Morse-code also
includes an extra long pause between words. Interestingly, it took a long time
to consider the written word boundary -- using white-space -- as a bona-fide
character that should simply be encoded with its own code point. This happened
only with the revision of the Baudot-code (see below) by Donald Murray in 1901,
in which he introduced a specific white-space code. This principle has been
followed ever since.}

\subsubsection*{Binary encoding}

From those bisignal encodings, true \textsc{binary codes} developed with a fixed
length of signals per character. In such systems only a single separator between
signals is needed, because the separation between characters can be established
by counting until a fixed number of signals has passed.\footnote{Of course, no
explicit separator is needed at all when the timing of the signals is known, which is
the principle used in all modern telecommunication systems. An important modern
consideration is also how to know where to start counting when you did not catch
the start of a message, something that is known in Unicode as \textsc{self
synchronization}.} In the context of electric telegraphy, such a binary code
system was first established by Émile Baudot in 1870, using a fixed combination
of five signals for each written character.\footnote{True binary codes have a
longer history, going back at least to the Baconian cipher devised by Francis
Bacon in 1605. However, the proposal by Baudot was the quintessential proposal
leading to all modern systems.} There are $2^5 = 32$ possible combinations when
using five binary signals; an encoding today designated as 5-bit. These
codes are sufficient for all Latin letters, but of course they do not suffice
for all written symbols, including punctuation and digits. As a solution, the
Baudot code uses a so-called shift character, which signifies that from
that point onwards -- until shifted back -- a different encoding is used, allowing
for yet another set of 32 codes. In effect, this means that the Baudot code, and
the \textsc{International Telegraph Alphabet} (ITA) derived from it, had an
extra bit of information, so the encoding is actually 6-bit (with $2^6
= 64$ different possible characters). For decades, this encoding was the
standard for all telegraphy and it is still in limited use today.

To also allow for different uppercase and lowercase letters and for a large
variety of control characters to be used in the newly developing technology of
computers, the American Standards Association decided to propose a new 7-bit
encoding in 1963 (with 2\textsuperscript{7} = 128 different possible characters), 
known as the
\textsc{American Standard Code for Information Interchange} (ASCII), geared
towards the encoding of English orthography. With the ascent of other
orthographies in computer usage, the wish to encode further variations of Latin
letters (including German <ß> and various letters with diacritics, e.g.\ <è>) led the
Digital Equipment Corporation to introduce an 8-bit \textsc{Multinational
Character Set} (MCS, with 2\textsuperscript{8} = 256 different possible characters), first used with the introduction of the VT{\large 220} Terminal in 1983. 

Because 256 characters were clearly not enough for the unique representation of 
many different characters
needed in the world's writing systems, the ISO/IEC~8859 standard in 1987
extended the MCS to include 16 different 8-bit code pages. For example, part 5
was used for Cyrillic characters, part 6 for Arabic, and part 7 for
Greek. % \footnote{In effect, because $16 = 2^4$, this means that ISO/IEC~8859 was
%actually an $8+4=12$-bit encoding, though with very many duplicates by design,
%namely all ASCII codes were repeated in each 8-bit code page. To be precise,
%ISO/IEC~8859 used the 7-bit ASCII as the basis for each code page, and defined
%16 different 7-bit extensions, leading to $(1+16) \cdot {2^7} = 2,176$ possible
%characters. However, because of overlap and not-assigned codes points the actual
%number of symbols was much smaller.} 
This system was almost immediately 
understood to be insufficient and impractical, so various initiatives to extend
and reorganize the encoding started in the 1980s. This led, for example, to
various proprietary encodings from Microsoft (e.g.~Windows Latin 1) and Apple
(e.g.~Mac OS Roman), which one still sometimes encounters today. 

In the 1980s various people started to develop true
international code sets. In the United States, a group of computer scientists
formed the \textsc{unicode consortium}, proposing a 16-bit encoding in 1991
(with 2\textsuperscript{16} = 65,536 different possible characters). At the same time in
Europe, the \textsc{international organization for standardization} (ISO) was
working on ISO~10646 to replace the ISO/IEC~8859 standard. Their first draft of
the \textsc{universal character set} (UCS) in 1990 was 31-bit (with
theoretically 2\textsuperscript{31} = 2,147,483,648 possible characters, but because of some
technical restrictions only 679,477,248 were allowed). Since 1991, the Unicode
Consortium and the ISO jointly develop the \textsc{unicode standard}, or
ISO/IEC~10646, leading to the current system including the original 16-bit
Unicode proposal as the \textsc{basic multilingual plane}, and 16 additional
planes of 16-bit for further extensions (with in total (1+16)\times 2\textsuperscript{16} =
1,114,112 possible characters). The most recent version of the Unicode Standard
(currently at version number 11.0.0) was published in June 2018 and it defines
137,374 different characters \citep{Unicode2018}.

\ 

\noindent In the next section we provide a very brief overview of the linguistic
terminology concerning writing systems before turning to the slightly different
computational terminology in the subsequent chapter on the Unicode Standard. 

% ==========================
\section{Linguistic terminology}
\label{linguistic-terminology}
% ==========================

Linguistically speaking, a \textsc{writing system} is a symbolic system that
uses visible or tactile signs to represent language in a systematic way. The
term writing system has two mutually exclusive meanings. First, it may
refer to the way a particular language is written. In this sense the term refers
to the writing system of a particular language, as, for example, in \emph{the
Serbian writing system uses two scripts: Latin and Cyrillic}. Second, the term
writing system may also refer to a type of symbolic system as, for example, in
\emph{alphabetic writing system}. In this latter sense the term refers to how
scripts have been classified according to the way that they encode language, as
in, for example, \emph{the Latin and Cyrillic scripts are both alphabetic
writing systems}. To avoid confusion, this second notion of writing system
would more aptly have been called \textsc{script system}. 

\subsubsection*{Writing systems}

Focusing on the first sense of \textsc{writing system} described above, we distinguish 
between two different kinds of writing systems used for a particular language, namely transcriptions and
orthographies. First, \textsc{transcription} is a scientific procedure (and also
the result of that procedure) for graphically representing the sounds of human
speech at the phonetic level. It incorporates a set of unambiguous symbols to
represent speech sounds, including conventions that specify how these symbols
should be combined. A transcription system is a specific system of symbols and
rules used for transcription of the sounds of a spoken language variety. In
principle, a transcription system should be language-independent, in that it
should be applicable to all spoken human languages. The \textsc{International
Phonetic Alphabet} (IPA) is a commonly used transcription system that provides a
medium for transcribing languages at the phonetic level. However, there is a
long history of alternative kinds of transcription systems
\citep[see][]{Kemp2006} and today various alternatives are in widespread use
(e.g.~X-SAMPA and Cyrillic-based phonetic transcription systems). Many
users of IPA do not follow the standard to the letter, and many dialects based
on the IPA have emerged, e.g.~the Africanist and Americanist transcription
systems. Note that IPA symbols are also often used to represent language on a
phonemic level. It is important to realize that in this usage the IPA
symbols are not a transcription system, but rather an orthography (though with
strong links to the pronunciation). Further, a transcription system does not
need to be as highly detailed as the IPA.\ It can also be a system of broad sound
classes. Although such an approximative transcription is not normally used in
linguistics, it is widespread in technological approaches
\citetext{\citealp[Soundex and variants, e.g.~][391--392]{Knuth1973};
\citealp{postel1969,Beider2008}}, and it is sometimes fruitfully used in
automatic approaches to historical linguistics
\citep{Dolgopolsky1986,List2012esslli,Brown2013}.

Second, an \textsc{orthography} specifies the symbols, punctuations, and the
rules in which a specific language is written in a standardized way.
Orthographies are often based on a phonemic analysis, but they almost always
include idiosyncrasies because of historical developments (like sound changes or
loans) and because of the widely-followed principle of lexical integrity
(i.e.~the attempt to write the same lexical root in a consistent way, also when
synchronic phonemic rules change the pronunciation, as for example with final
devoicing in many Germanic languages). Orthographies are language-specific
(and often even resource-specific), although individual symbols or rules might
be shared between languages. A \textsc{practical orthography} is a strongly
phoneme-based writing system designed for practical use by speakers. The mapping
relation between phonemes and graphemes in practical orthographies is purposely
shallow, i.e.~there is mostly a systematic and faithful mapping from a phoneme
to a grapheme. Practical orthographies are intended to jumpstart written
materials development by correlating a writing system with the sound units of a
language \citep{MeinhofJones1928}. Symbols from the IPA are often used
by linguists in the development of such practical orthographies for languages
without writing systems, though this usage of IPA symbols should not be confused
with transcription (as defined above). 

Further, a \textsc{transliteration} is a mapping between two different
orthographies. It is the process of ``recording the graphic symbols of one
writing system in terms of the corresponding graphic symbols of a second writing
system'' \citep[396]{Kemp2006}. In straightforward cases, such a transliteration
is simply a matter of replacing one symbol with another. However, there are
widespread complications, like one-to-many or many-to-many mappings, which are
not always easy, or even possible, to solve without listing all cases
individually \citep[cf.~][Ch.~2]{Moran2012}.

\subsubsection*{Script systems}

Different kinds of writing systems are classified into script
systems. A \textsc{script} is a collection of distinct symbols as
employed by one or more orthographies. For example, both Serbian and Russian are
written with subsets of the Cyrillic script. A single language, like Serbian or
Japanese, can also be written using orthographies based on different scripts.
Over the years linguists have classified script systems in a variety of ways,
with the tripartite classification of logographic, syllabic, and alphabetic
remaining the most popular, even though there are at least half a dozen
different types of script systems that can be distinguished
\citep{Daniels1990,Daniels1996}.

Breaking it down further, a script consists of \textsc{graphemes}, which are writing 
system-specific minimally distinctive symbols (see below). Graphemes may consist of one or more 
\textsc{characters}. The term `character' is overladen. In the linguistic terminology of writing
systems, a character is a general term for any self-contained element
in a writing system. A second interpretation is used as a conventional term for a unit in the Chinese writing
system \citep{Daniels1996}. In technical terminology, a character}
refers to the electronic encoding of a component in a writing system that has semantic 
value (see Section \ref{character-encoding-system}). Thus in this work we must navigate 
between the general linguistic and technical terms for \textsc{character} 
and \textsc{grapheme} because of how these notions are defined and how they relate at the intersection 
between the International Phonetic Alphabet and the Unicode Standard (Chapter \ref{ipa-meets-unicode}).

Although in literate societies most people have a strong intuition
about what the characters are in their particular orthography or orthographies,
it turns out that the separation of an orthography into separate characters is
far from trivial. The widespread intuitive notion of a character is strongly
biased towards educational traditions, like the alphabet taught at schools, and
technological possibilities, like the available type pieces in a printer's job
case, the keys on a typewriter, or the symbols displayed in Microsoft Word's
symbol browser. In practice, characters often consist of multiple building
blocks, each of which could be considered a character in its own right. For
example, although a Chinese character may be considered to be a single basic
unanalyzable unit, at a more fine-grained level of analysis the internal
structure of Chinese characters is often comprised of smaller semantic and
phonetic units that should be considered characters \citep{Sproat2000}. In
alphabetic scripts, this problem is most forcefully exemplified by diacritics. 

A \textsc{diacritic} is a mark, or series of marks, that may be above, below, 
before, after, through, around, or between other characters \citep{Gaultney2002}. Diacritics are sometimes used to
distinguish homophonous words, but they are more often used to indicate a
modified pronunciation \citep[xli]{DanielsBright1996}. The central question is
whether, for example, <e>, <è>, <a> and <à> should be considered four
characters, or different combinations of three characters, i.e.\ <a>, <e>, and <\dia{0300}>. 
In general, multiple characters together can form another character, and it is not always possible to
decide on principled grounds what should be the basic building blocks of an
orthography.

For that reason, it is better to analyze an orthography as a collection of
graphemes. A \textsc{grapheme} is the basic, minimally distinctive symbol of a
particular writing system. It was modeled
after the term \textsc{phoneme} (an abstract representation of
a distinct sound in a specific language) and as such it represents a contrastive graphical unit in a
writing system (see \citealp{Kohrt1986} for a historical overview of the term
\textsc{grapheme}). Most importantly, a single grapheme regularly consists of multiple
characters, like <th>, <ou> and <gh> in English (note that each character in
these graphemes is also a separate grapheme in English). Such complex graphemes
are often used to represent single phonemes. So, a combination of characters is
used to represent a single phoneme. Note that the opposite is also found in
writing systems, in cases in which a single character represents a combination
of two or more phonemes. For example, <x> in English orthography sometimes represents a
combination of the phonemes /k/ and /s/, as in the word `index' [ˈɪnˌdɛks]; other times 
it is pronounced as /z/, such as in the words `Xerox' [ˈzɪrˌɑks]; and in `example' [ɪɡˈzæmpəl] 
it is a combination of /g/ and /s/. As one can see, there can be non-trivial mappings between graphemes and phonemes in 
orthographies like English, e.g.\ `give', `gin', `jingle', where the graphemes 
<g> and <j> and the phonemes /g/ and /dʒ/ have a complex mapping.

Further, conditioned or free variants of a grapheme are called
\textsc{allographs}. For example, the distinctive forms of Greek sigma are
conditioned, with <σ> being used word-internally and <ς> being used at the end
of a word. In sum, there are many-to-many relationships between phonemes and
graphemes as they are expressed in the myriad of language- and resource-specific
orthographies.

\subsubsection*{Summary}

\noindent This exposition of the linguistic terminology involved in describing writing
systems has been purposely brief. We have highlighted some of the linguistic
notions that are pertinent to, yet sometimes confused with, the technical
definitions developed for the computational processing of the world's writing
systems, which we describe in the next Chapter.



% ==========================
\chapter{The Unicode approach}
\label{the-unicode-approach}
% ==========================

\section{Background}

The conceptualization and terminology of writing systems was rejuvenated by
the development of the Unicode Standard, with major input from Mark Davis,
co-founder and long-term president of the Unicode Consortium. For many years,
``exotic'' writing systems and phonetic transcription systems on personal
computers were constrained by the American Standard Code for Information
Interchange (ASCII) character encoding scheme, based on the Latin script, which
only allowed for a strongly limited number of different symbols to be encoded.
This implied that users could either use and adopt the (extended) Latin alphabet
or they could assign new symbols to the small number of code points in the ASCII
encoding scheme to be rendered by a specifically designed font
\citep{BirdSimons2003}. In this situation, it was necessary to specify the font
together with each document to ensure the rightful display of its content. To
alleviate this problem of assigning different symbols to the same code points,
in the late 80s and early 90s the Unicode Consortium set itself the ambitious
goal of developing a single universal character encoding to provide a unique
number, a code point, for every character in the world's writing systems.
Nowadays, the Unicode Standard is the default encoding of the technologies that
support the World Wide Web and for all modern operating systems, software and
programming languages.

\section{The Unicode Standard}

The Unicode Standard represents a massive step forward because it aims to
eradicate the distinction between universal (ASCII) versus language-particular
(font) by adding as much language-specific information as possible into the
universal standard. However, there are still language/resource-specific
specifications necessary for the proper usage of Unicode, as will be discussed
below. Within the Unicode structure many of these specifications can be captured
by so-called \textsc{Unicode Locales}, so we are moving to a new distinction
of universal (Unicode Standard) versus language-particular (Unicode Locale).
The major gain is much larger compatibility on the universal level (because
Unicode standardizes a much greater portion of writing system diversity), and
much better possibilities for automated processing on the language-particular
level (because Unicode Locales are machine-readable specifications).

Each version of the Unicode Standard (\citealp{Unicode2018}, as of writing at
version 11.0.0) consists of a set of specifications and guidelines that include (i) a
core specification, (ii) code charts, (iii) standard annexes and (iv) a
character database.\footnote{All documents of the Unicode Standard are available
at: \url{http://www.unicode.org/versions/latest/}. For a quick survey of the use
of terminology inside the Unicode Standard, their glossary is particularly
useful, available at: \url{http://www.unicode.org/glossary/}. For a general
introduction to the principles of Unicode, Chapter 2 of the core specification,
called \textsc{general structure}, is particularly insightful. Unlike 
many other documents in the Unicode Standard, this general introduction is
relatively easy to read and illustrated with many interesting examples from
various orthographic traditions from all over the world.} The \textsc{core
specification} is a book aimed at human readers that describes the formal
standard for encoding multilingual text. The \textsc{code charts} provide a
human-readable online reference to the character contents of the Unicode
Standard in the form of PDF files. The \textsc{Unicode Standard Annexes (UAX)}
are a set of technical standards that describe the implementation of the Unicode
Standard for software development, web standards, and programming languages.
The \textsc{Unicode Character Database (UCD)} is a set of computer-readable text
files that describe the character properties, including a set of rich character
and writing system semantics, for each character in the Unicode Standard. In
this section, we introduce the basic Unicode concepts, but we will leave out
many details. Please consult the above-mentioned full documentation for a more
detailed discussion. Further note that the Unicode Standard is exactly that,
namely a standard. It normatively describes notions and rules to be followed. In
the actual practice of applying this standard in a computational setting, a
specific implementation is necessary. The most widely used implementation of the
Unicode Standard is the \textsc{International Components for Unicode (ICU)},
which offers C/C++ and Java libraries implementing the Unicode
Standard.\footnote{More information about the ICU is available here:
\url{http://icu-project.org}.}

\section{Character encoding system}
\label{character-encoding-system}

The Unicode Standard is a \textsc{character encoding system} whose
goal is to support the interchange and processing of written characters and
text in a computational setting.\footnote{An insightful reviewer notes that the term \textsc{encoding} is used for both sequences of code points and text encoded as bit patterns. Hence a Unicode-aware programmer might prefer to say that UTF-8, UTF-16, etc., are Unicode encoding systems. The issue is that the Unicode Standard introduces a layer of indirection between characters and bit patterns, i.e.\ the code point, which can be encoded differently by different encoding systems.} Underlyingly, the character encoding is
represented by a range of numerical values called a \textsc{code space}, which
is used to encode a set of characters. A \textsc{code point} is a unique
non-negative integer within a code space (i.e.~within a certain numerical
range). In the Unicode Standard character encoding system, an \textsc{abstract
character}, for example the \textsc{latin small letter p}, is mapped to a
particular code point, in this case the decimal value 112, normally represented in
hexadecimal, which then looks in Unicode parlance as
\uni{0070}.
%\footnote{Hexadecimal (base-16) 0070 is equivalent to decimal
%(base-10) 112, which can be calculated by considering that $(0\cdot16^3) +
%(0\cdot16^2) + (7\cdot16^1) + (0\cdot16^0) = 7\cdot16 = 112$. Underlyingly,
%computers may treat this code point as an 8-bit binary (base-2) sequence (11100000), as
%can be seen by calculating that $(1\cdot2^7) + (1\cdot2^6) + (1\cdot2^5) +
%(0\cdot2^4) + (0\cdot2^3) + (0\cdot2^2) + (0\cdot2^1) + (0\cdot2^0) = 64 + 32 +
%16 = 112$.} 
That encoded abstract character is rendered on a computer screen (or
printed page) as a \textsc{glyph}, e.g.\ <p>, depending on the \textsc{font} and
the context in which that character appears.

In Unicode Standard terminology, an (abstract) \textsc{character} is the basic
encoding unit. The term \textsc{character} can be quite confusing due to its
alternative definitions across different scientific disciplines and because in
general the word \textsc{character} means many different things to different
people. It is therefore often preferable to refer to Unicode characters simply
as \textsc{code points}, because there is a one-to-one mapping between Unicode
characters and their numeric representation. In the Unicode approach, a
character refers to the abstract meaning and/or general shape, rather than a
specific shape, though in code tables some form of visual representation is
essential for the reader's understanding. Unicode defines characters as
abstractions of orthographic symbols, and it does not define visualizations for
these characters (although it does present examples). In contrast, a
\textsc{glyph} is a concrete graphical representation of a character as it
appears when rendered (or rasterized) and displayed on an electronic device or
on printed paper. For example, <g {\large \textit{g}} \textbf{g}
{\fontspec{ArialMT} {\small g} \textit{g} \textbf{g}}> are different glyphs of the
same character, i.e.~they may be rendered differently depending on the
typography being used, but they all share the same code point.\footnote{The first three g's 
are regular, italic and bold versions of Linux Libertine, the main font used in this book, 
while the last three are regular, italic and bold version of Arial.} From the
perspective of Unicode they are \textit{the same thing}. In this approach, a
\textsc{font} is then simply a collection of glyphs connected to code points.
Allography is not specified in Unicode (barring a few exceptional cases, due
to legacy encoding issues), but can be specified in a font as a
\textsc{contextual variant} (aka presentation form).

Each code point in the Unicode Standard is associated with a set of
\textsc{character properties} as defined by the Unicode character property
model.\footnote{The character property model is described in
\url{http://www.unicode.org/reports/tr23/}, but the actual properties are
described in \url{http://www.unicode.org/reports/tr44/}. A simplified overview
of the properties is available at: 
\url{http://userguide.icu-project.org/strings/properties}. The actual code
tables listing all properties for all Unicode code points are available at: 
\url{http://www.unicode.org/Public/UCD/latest/ucd/}.} Basically, those
properties are just a long list of values for each character. For example, code
point \uni{0047} has the following properties (among many others): 
\begin{itemize}
	\item Name: LATIN CAPITAL LETTER G 
	\item Alphabetic: YES 
	\item Uppercase: YES 
	\item Script: LATIN 
	\item Extender: NO 
	\item Simple\_Lowercase\_Mapping: 0067 
\end{itemize}

These properties contain the basic information of the Unicode Standard and they
are necessary to define the correct behavior and conformance required for
interoperability in and across different software implementations (as defined in
the Unicode Standard Annexes). The character properties assigned to each code
point are based on each character's behavior in real-world writing
traditions. For example, the corresponding lowercase character to \uni{0047} is
\uni{0067}.\footnote{Note that the relation between uppercase and lowercase is in
many situations much more complex than this, and Unicode has further
specifications for those cases.} Another use of properties is to define the
script of a character.\footnote{The Glossary of Unicode Terms defines the term \textsc{script} as
a ``collection of letters and other written signs used to represent textual
information in one or more writing systems. For example, Russian is written with
a subset of the Cyrillic script; Ukrainian is written with a different subset.
The Japanese writing system uses several scripts.''} In practice, script is
simply defined for each character as the explicit \textsc{script} property in
the Unicode Character Database.

One frequently referenced property is the \textsc{block} property, which is
often used in software applications to impose some structure on the large number
of Unicode characters. Each character in Unicode belongs to a specific block.
These blocks are basically an organizational structure to alleviate the
administrative burden of keeping Unicode up-to-date. Blocks consist of
characters that in some way belong together, so that characters are easier to
find. Some blocks are connected with a specific script, like the Hebrew block or
the Gujarati block. However, blocks are predefined ranges of code points, and
often there will come a point after which the range is completely filled. Any
extra characters will have to be assigned somewhere else. There is, for example,
a block \textsc{Arabic}, which contains most Arabic symbols. However, there is
also a block \textsc{Arabic Supplement}, \textsc{Arabic Presentation Forms-A}
and \textsc{Arabic Presentation Forms-B}. The situation with Latin symbols is
even more extreme. In general, the names of blocks should not be taken as a
definitional statement. For example, many IPA symbols are not located in the
aptly-named block \textsc{IPA extensions}, but in other blocks
(see Section~\ref{pitfall-no-complete-ipa-block}).

\section{Grapheme clusters}

There are many cases in which a sequence of characters (i.e.~a sequence of more
than one code point) represents what a user perceives as an individual unit in a
particular orthographic writing system. For this reason the Unicode Standard
differentiates between \textsc{abstract character} and \textsc{user-perceived
character}. Sequences of multiple code points that correspond to a single
user-perceived characters are called \textsc{grapheme clusters} in Unicode parlance.
Grapheme clusters come in two flavors: (default) grapheme clusters and tailored
grapheme clusters.

The (default) \textsc{grapheme clusters} are locale-independent graphemes,
i.e.~they always apply when a particular combination of characters occurs
independent of the writing system in which they are used. These character
combinations are defined in the Unicode Standard as functioning as one
\textsc{text element}.\footnote{The Glossary of Unicode Terms defines \textsc{text element} as:
``A minimum unit of text in relation to a particular text process, in the
context of a given writing system. In general, the mapping between text elements
and code points is many-to-many.''} The simplest example of a grapheme cluster
is a base character followed by a letter modifier character. For example, the
sequence <n> + <\dia{0303}> (i.e.~\textsc{latin small letter n} at \uni{006E}, followed
by \textsc{combining tilde} at \uni{0303}) combines visually into <ñ>, a
user-perceived character in writing systems like that of Spanish. In effect, what the
user perceives as a single character actually involves a multi-code-point
sequence. Note that this specific sequence can also be represented with a single
so-called \textsc{precomposed code point}, the \textsc{latin small letter n with
tilde} at \uni{00F1}, but this is not the case for all multi-code-point
character sequences. A solution to the problem of multiple encodings for the
same text element was developed early on in the Unicode Standard. It is called 
\textsc{canonical equivalence}, e.g.~for <ñ>, the sequence \uni{006E} \uni{0303} 
should in all situations be treated identically to the precomposed \uni{00F1}. By doing so, 
Unicode can also support special or precomposed characters in legacy character sets. 
To determine canonical equivalence, the Unicode Standard offers different kinds of normalization to either 
decompose precomposed characters (called \textsc{NFD} for \textsc{normalization form
canonical decomposition}) or to combine sequences of code points into precomposed 
characters (called \textsc{NFC} for \textsc{normalization form canonical composition}).\footnote{See the 
Unicode Standard Annex \#15, Unicode Normalization Forms (\url{http://unicode.org/reports/tr15/}), 
which provides a detailed description of normalization algorithms and illustrated examples.} 
In current software development practice, NFC seems to be preferred in most situations
and is widely proposed as the preferred canonical form. We discuss Unicode normalization 
in detail in Section \ref{pitfall-canonical-equivalence}.

More difficult for text processing, because less standardized, is what the
Unicode Standard terms \textsc{tailored grapheme clusters}.\footnote{\url{http://unicode.org/reports/tr29/}} 
Tailored grapheme clusters are locale-dependent graphemes, i.e.~such combination of characters do
not function as text elements in all situations. Examples include the sequence
<c>~+~<h> for the Slovak digraph <ch> and the sequence <ky> in the Sisaala
practical orthography, which is pronounced as IPA /tʃ/ \citep{Moran2006}. These grapheme
clusters are \textsc{tailored} in the sense that they must be specified on a
language-by-language or writing-system-by-writing-system basis. They are also 
grapheme clusters in these orthographies for processes such as collation (i.e.\ sorting).\footnote{\url{https://www.unicode.org/glossary/\#collation}}

The Unicode Standard provides technical specifications for creating locale specific data
in so-called \textsc{Unicode Locales}, i.e.~specifications 
that define a set of language-specific elements (e.g.~tailored grapheme
clusters, collation order, capitalization equivalence), as well as other special
information, like how to format numbers, dates, or currencies. Locale
descriptions are saved in the \textsc{Common Locale Data Repository
(CLDR)},\footnote{More information about the CLDR can be found here:
\url{http://cldr.unicode.org/}.} a repository of
language-specific definitions of writing system properties, each of which
describes specific usages of characters. Each locale can be encoded in a
document using the \textsc{Locale Data Markup Language (LDML)}. LDML is an XML
format and vocabulary for the exchange of structured locale data. Unicode Locale
Descriptions allow users to define language- or even resource-specific writing
systems or orthographies.\footnote{The Glossary of Unicode Terms defines \textsc{writing
system} only very loosely, as it is not a central concept in the Unicode
Standard. A writing system is, ``A set of rules for using one or more scripts to
write a particular language. Examples include the American English writing
system, the British English writing system, the French writing system, and the
Japanese writing system.''} However, Unicode Locales have various drawbacks 
for the daily practice of scientific linguistic research in a multilingual setting (see Section~\ref{characterizing-writing-systems}).

\chapter{Unicode pitfalls}
\label{unicode-pitfalls}

% ==========================
\section{Wrong it ain't}
\label{wrong-it-is-not}
% ==========================

In this chapter we describe some of the most common pitfalls that we have
encountered when using the Unicode Standard in our own work, or in discussion
with other linguists. This section is not meant as a criticism of the decisions
made by the Unicode Consortium; rather we aim to highlight where the
technical aspects of the Unicode Standard diverge from many users'
intuitions. What have sometimes been referred to as problems or inconsistencies
in the Unicode Standard are mostly due to legacy compatibility issues, which can
lead to unexpected behavior by linguists using the standard. However, there are
also some cases in which the Unicode Standard has made decisions that
theoretically could have been made differently, but for some reason or another
(mostly very good reasons) were accepted as they are now. We call such behavior that
executes without error but does something different than the user
expected -- often unknowingly -- a \textsc{pitfall}.

In this context, it is important to realize that the Unicode Standard was not
developed to solve linguistic problems per se, but to offer a consistent
computational environment for written language. In those cases in which the
Unicode Standard behaves differently than expected, we think it is important not
to dismiss Unicode as wrong or deficient, because our
experience is that in almost all cases the behavior of the Unicode Standard has
been particularly well thought through. The Unicode Consortium has a 
wide-ranging view of matters and often examines important practical use cases
that are not normally considered from a linguistic point of view. Our general
guideline for dealing with the Unicode Standard is to accept it as it is, and
not to tilt at windmills. Alternatively, of course, it is possible to actively
engage in the development of the standard itself, an effort that is highly
appreciated by the Unicode Consortium.

% ==========================
\section{Pitfall: Characters are not glyphs}
\label{pitfall-characters-are-not-glyphs}
% ==========================

A central principle of Unicode is the distinction between character and glyph. A
character is the abstract notion of a symbol in a writing system, while a glyph
is the visual representation of such a symbol. In practice, there is a complex
interaction between characters and glyphs. A single Unicode character may of
course be rendered as a single glyph. However, a character may also be a piece
of a glyph, or vice-versa. Actually, all possible relations between glyphs and
characters are attested.

First, a single character may have different contextually determined glyphs. For
example, characters in writing systems like Hebrew and Arabic have different
glyphs depending on where they appear in a word. Some letters in Hebrew change
their form at the end of the word, and in Arabic, primary letters have four
contextually-sensitive variants (isolated, word initial, medial and final).

Second, a single character may be rendered as a sequence of multiple glyphs. For
example, in Tamil one Unicode character may result in a combination of a
consonant and vowel, which are rendered as two adjacent glyphs by fonts that
support Tamil, e.g.\ \textsc{tamil letter au} at \uni{0B94} represents a single 
character <\tamil{ஔ}>, composed of two glyphs <\tamil{ஓ}> and <\tamil{ன}>. Perhaps confusingly, 
in the Unicode Standard there are also two individual characters, 
i.e.\ \textsc{tamil letter oo} at \uni{0B93} and 
\uni{0BA9} \textsc{tamil letter nnna}, each of which is a glyph. Another example is 
Sinhala \textsc{sinhala vowel sign kombu deka} at \uni{0DDB} <\sinhala{ෛ}>, which is 
visually two glyphs, each represented by \textsc{sinhala vowel sign kombuva} 
at \uni{0DD9} <\sinhala{ ෙ}>.

Third, a single glyph may be a combination of multiple characters. For example, 
the ligature <fi>, a single glyph, is the result of two
characters, <f> and <i>, that have undergone glyph substitution by font
rendering (see also Section~\ref{pitfall-faulty-rendering}). Like
contextually-determined glyphs, ligatures are (intended) artifacts of text
processing instructions. Finally, a single glyph may be a part of a
character, as exemplified by diacritics like the diaeresis <\dia{0308}> in <ë>.

Further, the rendering of a glyph is dependent on the font being used. For
example, the Unicode character \textsc{latin small letter g} appears as <g> and
<{\fontspec{Courier}g}> in the Linux Libertine and Courier fonts, respectively,
because their typefaces are designed differently. Furthermore, the font face may
change the visual appearance of a character, for example Times New Roman
two-story <{\fontspec{Times New Roman}a}> changes to a single-story glyph in italics
<\emph{\fontspec{Times New Roman}a}>. This becomes a real problem for some
phonetic typesetting (see Section~\ref{pitfall-ipa-homoglyphs}).

In sum, character-to-glyph mappings are complex technical issues that the
Unicode Consortium has had to address in the development of the Unicode
Standard. However, they can can be utterly confusing for the lay user because visual
rendering does not (necessarily) indicate logical encoding.

% ==========================
\section{Pitfall: Characters are not graphemes}
\label{pitfall-characters-are-not-graphemes}
% ==========================

The Unicode Standard encodes characters. This becomes most clear with the notion of grapheme.
From a linguistic point of view, graphemes are the basic building blocks of a
writing system (see Section~\ref{linguistic-terminology}). It is extremely
common for writing systems to use 
combinations of multiple symbols (or letters) as a single grapheme, such as <sch>, <th> or <ei>.
There is no way to encode such complex graphemes using the Unicode Standard.

The Unicode Standard deals with complex graphemes only inasmuch as they consist of
base characters with diacritics (see
Section~\ref{pitfall-different-notions-of-diacritics} for a discussion of the
notion of diacritic). The Unicode Standard calls such combinations \textit{grapheme
clusters}. Complex graphemes consisting of multiple base characters,
like <sch>, are called \textit{tailored grapheme clusters} (see
Chapter~\ref{the-unicode-approach}).

There are special Unicode characters that 
can help determining groups of characters as being larger tailored grapheme clusters,
specifically the \textsc{zero width joiner} at \uni{200D} and the
\textsc{combining grapheme joiner} at \uni{034F}. However, these characters are
confusingly named (cf.~Section~\ref{pitfall-names}). Both code points actually do
not join characters, but explicitly separate them. The zero-width joiner
can be used to solve special problems related to ordering (called \textit{collation}
in Unicode parlance). The combining grapheme joiner can be used to
separate characters that are not supposed to form ligatures. 

To solve the issue of tailored grapheme clusters, Unicode offers some assistance
in the form of Unicode Locales.\footnote{\url{http://cldr.unicode.org/locale_faq-html}} 
However, in the practice of
linguistic research, this is not a real solution. To address this issue, we propose to
use orthography profiles (see Chapter~\ref{orthography-profiles}). Basically,
both orthography profiles and Unicode Locales offer a way to specify
tailored grapheme clusters. For example, for English one could specify that <sh>
is such a cluster. Consequently, this sequence of characters is then always
interpreted as a complex grapheme. For cases in which this is not the right
decision, like in the English word \textit{mishap}, the \textsc{zero width
joiner} at \uni{200D} can be entered between <s> and <h>.

% ==========================
\section{Pitfall: Missing glyphs}
\label{pitfall-missing-glyphs}
% ==========================

The Unicode Standard is often praised (and deservedly so) for solving many of
the perennial problems with the interchange and display of the world's writing
systems. Nevertheless, a common complaint from users is that some symbols do not display 
correctly, i.e. they might be displayed \textit{not at all} or only from a so-called 
\textit{fall back font}, e.g.\ showing a 
rectangle <▯>, question mark <?>, or the Unicode replacement character <�>. 
The reason for such behaviour is that the user's computer does not have the fonts 
installed that map the desired glyphs to Unicode characters. 
Therefore the glyphs cannot be displayed.
This is not the Unicode Standard's fault because it is a character 
encoding system and not a font. Computer-internally everything works as expected; 
any handling of Unicode code points works independently of how they 
are displayed on the screen. So although users might see
alien faces on display, they should not fret because everything is still 
technically in order below the surface.

There are two obstacles regarding missing glyphs. One is practical: 
designing glyphs includes many different considerations and 
it is a time-consuming process, especially when done well. 
Traditional expectations of what specific characters should look like need
to be taken into account when designing glyphs. Those expectations are often not
well documented, and it is mostly up to the knowledge and expertise of the font
designer to try and conform to them. Furthermore, the number of characters 
supported by Unicode is vast. Therefore, most designers 
produce fonts that only include glyphs for certain parts of the Unicode
Standard. 

The second obstacle is technical: the maximum number of glyphs that can be 
defined by the TrueType font standard and the OpenType specification 
(ISO/IEC 14496-22:2015) is 65,535. The current version of the Unicode Standard 
contains 137,374 characters. Thus, no single font can provide individual 
glyphs for all Unicode characters.

A simple solution to missing glyphs is to install additional fonts
providing additional glyphs. For broad coverage, there is the Noto font family, a project developed by Google, 
which covers over 100 scripts and nearly 64,000 characters.\footnote{\url{https://www.google.com/get/noto/}} 
The Unicode Consortium also provides, but does not endorse, an extensive list of fonts and font libraries online.\footnote{\url{http://unicode.org/resources/fonts.html}}

For the more exotic characters there is often not much choice. We have had success using 
Michael Everson's \textsc{Everson Mono} font, which has 9,756 different glyphs (not including 
Chinese)\footnote{\url{http://www.evertype.com/emono/}} and with the somewhat older \textsc{Titus Cyberbit Basic} font 
by Jost Gippert and Carl-Martin Bunz. It includes 10,044 different glyphs (not including 
Chinese).\footnote{\url{http://titus.fkidg1.uni-frankfurt.de/unicode/tituut.asp}} 

We also suggest installing at least one \textsc{fall-back
font}, which provides glyphs that show the user some information about
the underlying encoded character. Apple computers have such a font (which is
invisible to the user), which is designed by Michael Everson and made available
for other systems through the Unicode Consortium.\footnote{
\url{http://www.unicode.org/policies/lastresortfont\_eula.html}} Further, the
\textsc{GNU Unifont} is a clever way to produce bitmaps approximating the
intended glyph of each available character.\footnote{\url{http://unifoundry.com/unifont.html}} Finally,
SIL International provides a \textsc{SIL Unicode BMP Fallback
Font}. This font does not show a real 
glyph, but instead shows the hexadecimal code inside a box
for each character, so a user can at least see the Unicode code point of the
character intended for display.\footnote{\url{http://scripts.sil.org/UnicodeBMPFallbackFont}}

% ==========================
\section{Pitfall: Faulty rendering}
\label{pitfall-faulty-rendering}
% ==========================

A similar complaint to missing glyphs, discussed previously, is that while 
a glyph might be displayed, it does not look right. There are two
reasons for unexpected visual display, namely automatic font substitution and
faulty rendering. Like missing glyphs, any such problems are independent from
the Unicode Standard. The Unicode Standard only includes very general
information about characters and leaves the specific visual display for others to
decide on. Any faulty display is thus not to be blamed on the Unicode
Consortium, but on a complex interplay of different mechanisms happening in a
computer to turn Unicode code points into visual symbols. We will only sketch a
few aspects of this complex interplay here.

Most modern software applications (like Microsoft Word) offer some approach to
\textsc{automatic font substitution}. This means that when a text is written in
a specific font (e.g.~Times New Roman) and an inserted Unicode character does not
have a glyph within this font, then the software application will automatically
search for another font to display the glyph. The result will be that this
specific glyph will look slightly different from the others. This mechanism
works differently depending on the software application; only limited
user influence is usually expected and little feedback is given. This may be rather
frustrating to font-aware users.

% \footnote{For example, Apple Pages does not give any feedback that a font is being replaced, and the user does not seem to have any influence on the choice of replacement (except by manually marking all occurrences). In contrast, Microsoft Word does indicate the font replacement by showing the name in the font menu of the font replacement. However, Word simply changes the font completely, so any text written after the replacement is written in a different font as before. Both behaviors leave much to be desired.}

Another problem with visual display is related to so-called \textsc{font
rendering}. Font rendering refers to the process of the actual positioning of
Unicode characters on a page of written text. This positioning is actually a
highly complex challenge and many things can go wrong in the process. Well-known
rendering difficulties, like proportional glyph size or ligatures, are reasonably
well understood by developers. Nevertheless, the positioning of multiple diacritics relative to
a base character is still a widespread problem. Especially problematic is when 
more than one diacritic is supposed to be placed above (or
below) another. Even within the Latin script vertical placement 
often leads to unexpected effects in many modern software applications. 
The rendering problems arising in Arabic and in many scripts of Southeast
Asia (like Devanagari or Burmese) are even more complex. 

To understand why these problems arise it is important to realize that there are
basically three different approaches to font rendering. The most widespread is
Adobe's and Microsoft's \textsc{OpenType} system. This approach makes it
relatively easy for font developers, but the font itself does not include all
details about the precise placement of individual characters. For those details,
additional script descriptions are necessary. Each of these pieces of software can lead to
unexpected behavior.\footnote{For more details about OpenType, see
\url{http://www.adobe.com/products/type/opentype.html} and
\url{http://www.microsoft.com/typography/otspec/}. Additional systems for
complex text layout are, among others, Microsoft's DirectWrite
(\url{https://msdn.microsoft.com/library/dd368038.aspx}) and the open-source
project HarfBuzz (\url{http://www.freedesktop.org/wiki/Software/HarfBuzz/}).}
Alternative systems are \textsc{Apple Advanced Typography} (AAT) and the
open-source \textsc{Graphite} system produced and maintained by the Non-Roman Script Initiative of SIL International 
(SIL).\footnote{More information about AAT can be found at:
\url{https://developer.apple.com/fonts/}. \newline Graphite is described
in detail at:
\url{http://scripts.sil.org/default}.}
In these systems, a larger burden is placed on the description inside
the font.

There is no complete solution to the problems arising from faulty font rendering.
Switching to another software application that offers better handling is the
only real alternative, but this is normally not an option for daily work. 
Font rendering is developing quickly in the software industry, so we can expect 
the situation to only get better. % In the meantime one 
% can try to correct faulty layout by tweaking baseline and/or kerning (when such 
% option are available).

% ==========================
\section{Pitfall: Blocks}
\label{pitfall-blocks}
% ==========================

The Unicode code space is subdivided into blocks of contiguous code points. For
example, the block called \textsc{Cyrillic} runs from \uni{0400} till
\uni{04FF}. These blocks arose as an attempt at ordering the enormous number of
characters in Unicode, but the idea of blocks very quickly ran into problems.
First, the size of a block is fixed, so when a block is full, a new block will
have to be instantiated somewhere else in the code space. For example, this
led to the blocks \textsc{Cyrillic Supplement}, \textsc{Cyrillic Extended-A}
(both of which are already full) and \textsc{Cyrillic Extended-B}. Second,
when a specific character already exists, it is not duplicated in another
block, although the name of the block might indicate that a specific symbol
should be available there. In general, names of blocks are just an approximate
indication of the kind of characters that will be in the block.

The problem with blocks arises because finding the right character among the
thousands of Unicode characters is not easy. Many software applications present
blocks as a primary search mechanism, because the block names suggest where to
look for a particular character. However, when a user searches for an IPA
character in the block \textsc{IPA Extensions}, then many IPA characters will not
be found there. For example, the velar nasal <ŋ> is not part of the block
\textsc{IPA Extensions} because it was already included as \textsc{latin small letter
eng} at \uni{014B} in the block \textsc{Latin Extensions-A}.

In general, finding a specific character in the Unicode Standard is often non-trivial. 
The names of the blocks can help, but they are not (and were never supposed
to be) a foolproof structure. It is neither the goal nor the aim of the Unicode
Consortium to provide a user interface to the Unicode Standard. If one often
encounters the problem of needing to find a suitable character, there are
various other useful services for end-users available.\footnote{The Unicode
website offers a basic interface to the code charts at:
\url{http://www.unicode.org/charts/index.html}. As a more flexible interface, we
particularly like PopChar from Ergonis Software, available for both Mac and
Windows. There are also various free websites that offer search interfaces
to the Unicode code tables, like \url{http://unicode-search.net} or
\url{http://unicode-search.net}. Another useful approach for searching for characters
using shape matching \citep{Belongie2002} is: \url{http://shapecatcher.com}.}

% ==========================
\section{Pitfall: Names}
\label{pitfall-names}
% ==========================

The names of characters in the Unicode Standard are sometimes misnomers and
should not be misinterpreted as definitions. For example, the \textsc{combining
grapheme joiner} at \uni{034F} does not join characters into larger graphemes
(see Section~\ref{pitfall-characters-are-not-graphemes}) and the \textsc{latin
letter retroflex click} \uni{01C3} is actually not the IPA symbol for a
retroflex click, but for an alveolar click (see
Section~\ref{pitfall-ipa-homoglyphs}). In a sense, these names can be seen as
errors. However, it is probably better to realize that such names are just
convenience labels that are not going to be changed. Just like the block names
(Section~\ref{pitfall-blocks}), the character names are often helpful, but they
are not supposed to be definitions.

The actual intended meaning of a Unicode code point is a combination of the
name, the block and the character properties (see
Chapter~\ref{the-unicode-approach}). Further details about the underlying intentions 
with which a character should be used
are only accessible by perusing the actual decisions of the Unicode Consortium.
All proposals, discussions and decisions of the Unicode Consortium are publicly
available. Unfortunately there is not (yet) any way to easily find everything
that is ever proposed, discussed and decided in relation to a specific
code point of interest, so many of the details are often somewhat
hidden.\footnote{All proposals and other documents that are the basis of Unicode
decisions are available at: \url{http://www.unicode.org/L2/all-docs.html}. The
actual decisions that make up the Unicode Standard are documented in the minutes
of the Unicode Technical Committee, available at: 
\url{http://www.unicode.org/consortium/utc-minutes.html}.}

% ==========================
\section{Pitfall: Homoglyphs}
\label{pitfall-homoglyphs}
% ==========================

Homoglyphs are visually indistinguishable glyphs (or highly similar glyphs) that
have different code points in the Unicode Standard and thus different character
semantics. As a principle, the Unicode Standard does not specify how a character
appears visually on the page or the screen. So in most cases, a different
appearance is caused by the specific design of a font, or by user-settings like
size or boldface. Taking an example already discussed in
Section~\ref{character-encoding-system}, the following symbols <g {\large \textit{g}}
\textbf{g} {\fontspec{ArialMT} {\small g} \textit{g} \textbf{g}}> are different
glyphs of the same character, i.e.~they may be rendered differently depending on
the typography being used, but they all share the same code point (viz.
\textsc{latin small letter g} at \uni{0067}). In contrast, the symbols
<{\fontspec{EversonMono}AАΑᎪᗅᴀꓮ𐊠𝖠𝙰}> are all different code points,
although they look highly similar -- in some cases even sharing exactly the same
glyph in some fonts. All these different A-like characters include the following
code points in the Unicode Standard:

\begin{itemize}
	\item[] <{\fontspec{EversonMono}A}> \textsc{latin capital letter a}, at \uni{0041} 
	\item[] <{\fontspec{EversonMono}А}> \textsc{cyrillic capital letter a}, at \uni{0410} 
	\item[] <{\fontspec{EversonMono}Α}> \textsc{greek capital letter alpha}, at \uni{0391} 
	\item[] <{\fontspec{EversonMono}Ꭺ}> \textsc{cherokee letter go}, at \uni{13AA} 
	\item[] <{\fontspec{EversonMono}ᗅ}> \textsc{canadian syllabics carrier gho}, at \uni{15C5} 
	\item[] <{\fontspec{EversonMono}ᴀ}> \textsc{latin small letter capital a}, at \uni{1D00} 
	\item[] <{\fontspec{EversonMono}ꓮ}> \textsc{lisu letter a}, at \uni{A4EE} 
%	\item[] <{\fontspec{EversonMono}A}> \textsc{fullwidth latin capital letter a}, at \uni{FF21} 
	\item[] <{\fontspec{EversonMono}𐊠}> \textsc{carian letter a}, at \uni{102A0} 
%	\item[] <{\fontspec{EversonMono}̀}> \textsc{old italic letter a}, at \uni{10300} 
	\item[] <{\fontspec{EversonMono}𝖠}> \textsc{mathematical sans-serif capital a}, \uni{1D5A0} 
	\item[] <{\fontspec{EversonMono}𝙰}> \textsc{mathematical monospace capital a}, at \uni{1D670} 
\end{itemize}

The existence of such homoglyphs is partly due to legacy compatibility, but for
the most part these characters are simply different characters that happen to
look similar.\footnote{A particularly nice interface to look for homoglyphs is
\url{http://shapecatcher.com}, based on the principle of recognizing shapes
\citep{Belongie2002}.} Yet, they are suppose to behave differently from the
perspective of a font designer. For example, when designing a Cyrillic font, the
<A> will have different aesthetics and different traditional expectations
compared to a Latin <A>. Thus, the Unicode Standard has character properties 
associated with each code point which define certain expectations, e.g.\ characters 
belong to a specific script or they have different lower case variants (see 
Section \ref{character-encoding-system}).

Homoglyphs are a widespread problem for consistent encoding. Although for
most users it looks like the words <voces> and <νοсеѕ> are nearly identical, in 
fact they do not share any code points.\footnote{The first words
consists completely of Latin characters: \unif{0076}, \unif{006F},
\unif{0063}, \unif{0065} and \unif{0073}. The second is a mix of Cyrillic
and Greek characters: \unif{03BD}, \unif{03BF}, \unif{0041}, \unif{0435}
and \unif{0455}.} For computers these two words are completely different
entities. Sometimes when users with Cyrillic or Greek keyboards have to type
some Latin-based orthography, they mix similar looking Cyrillic or Greek
characters into their text, because those characters are so much easier to type.
Similarly, when users want to enter an unusual symbol, they normally search by
visual impression in their favorite software application, and just pick
something that looks reasonably alike to what they expect the glyph to look
like.

It is very easy to make errors during text entry and add characters that are 
not supposed to be included. Our proposals for orthography profiles (see
Chapter~\ref{orthography-profiles}) are a method for checking the consistency of 
any text. In situations in which interoperability is important, we consider it 
crucial to add such checks in any workflow.

% ==========================
\section{Pitfall: Canonical equivalence}
\label{pitfall-canonical-equivalence}
% ==========================

For some characters, there is more than one possible encoding in the Unicode
Standard. This means that for the computer
there exists multiple different entities, which for the user, may be visually the same. This
leads to, for example, problems with search. The computer searches for specific 
code points and by design does not return all visually similar characters.
As a solution, the Unicode Standard includes a notion of \textsc{canonical
equivalence}. Different encodings are explicitly declared as equivalent in the
Unicode Standard code tables. Further, to harmonize all encodings in a specific
piece of text, the Unicode Standard proposes a mechanism of
\textsc{normalization}. The process of normalization and the 
Unicode Normalization Forms are described 
in detail in the Unicode Standard Annex \#15 online.\footnote{\url{http://unicode.org/reports/tr15/}} 
Here we provide a brief summary of that material as it pertains to canonical equivalence.

Consider for example the characters and following Unicode code points:
\begin{enumerate}
	\def\labelenumi{\arabic{enumi}.} 
	\item <Å> \textsc{latin capital letter a with ring above} \uni{00C5} 
	\item <Å> \textsc{angstrom sign} \uni{212B}
	\item <Å> \textsc{latin capital letter a} \uni{0041}
	+ \textsc{combining ring above} \uni{030A}
\end{enumerate}

\noindent The character, represented here by glyph <Å>, is encoded in the Unicode Standard
in the first two examples by a single-character sequence; each is assigned a
different code point. In the third example, the glyph is encoded in a
multiple-character sequence that is composed of two character code points. All
three sequences are \textsc{canonical equivalent}, i.e.~they are strings that
represent the same abstract character and because they are not distinguishable
by the user, the Unicode Standard requires them to be treated the same in
regards to their behavior and appearance. Nevertheless, they are encoded
differently. For example, if one were to search an electronic text (with
software that does not apply Unicode Standard normalization) for
\textsc{angstrom sign} (\uni{212B}), then the instances of \textsc{latin 
capital letter a with ring above} (\uni{00C5}) would not be found.

In other words, there are equivalent sequences of Unicode characters that should
be normalized, i.e.~transformed into a unique Unicode-sanctioned representation
of a character sequence called a \textsc{normalization form}. Unicode provides a
Unicode Normalization Algorithm, which puts combining marks
into a specific logical order and it defines decomposition and composition
transformation rules to convert each string into one of four normalization
forms. We will discuss here the two most relevant normalization forms: NFC and
NFD.

The first of the three characters above is considered the \textsc{Normalization
Form C (NFC)}, where \textsc{C} stands for composition. When the process of NFC
normalization is applied to the characters in 2 and 3, both 
are normalized into the \textsc{pre-composed} character sequence in 1. Thus all
three canonical character sequences are standardized into one composition form
in NFC. The other frequently encountered Unicode normalization form is the
\textsc{Normalization Form D (NFD)}, where \textsc{D} stands for decomposition.
When NFD is applied to the three examples above, all three, including
importantly the single-character sequences in 1 and 2, are normalized into the
\textsc{decomposed} multiple-sequence of characters in 3. Again, all three are
then logically equivalent and therefore comparable and syntactically
interoperable.

As illustrated, some characters in the Unicode Standard have alternative
representations (in fact, many do), but the Unicode Normalization Algorithm can
be used to transform certain sequences of characters into canonical
forms to test for equivalency. To determine equivalence, each
character in the Unicode Standard is associated with a combining class, which is
formally defined as a character property called \textsc{canonical combining
class} which is specified in the Unicode Character Database. The combining class
assigned to each code point is a numeric value between 0 and 254 and is used by
the Unicode Canonical Ordering Algorithm to determine in what sequences they should 
occur (but see Section~\ref{pitfall-no-unique-diacritic-ordering} for the limitations 
of the Unicode combining class values). Normalization forms, as very briefly
described above, can be used to ensure character equivalence by ordering
character sequences so that they can be faithfully compared.

It is very important to note that any software application that is Unicode
Standard compliant is free to change the character stream from one
representation to another. This means that a software application may compose,
decompose or reorder characters as its developers desire; as long as the
resultant strings are canonical equivalent to the original. This might lead to
unexpected behavior for users. Various players, like the Unicode Consortium, the
W{\large 3}C, or the TEI recommend NFC in most user-directed situations, and
some software applications that we tested indeed seem to automatically convert
strings into NFC.\footnote{See the summary of various recommendation here:
\url{http://www.win.tue.nl/~aeb/linux/uc/nfc_vs_nfd.html}.} This means in
practice that if a user, for example, enters <a> and <\dia{0300}>, i.e.~\textsc{latin
small letter a} at \uni{0061} and \textsc{combining grave accent} at \uni{0300},
this might be automatically converted into <à>, i.e.~\textsc{latin small letter
a with grave} at \uni{00E0}.\footnote{The behavior of software applications can
be quite erratic in this respect. For example, Apple's TextEdit does not do any
conversion on text entry. However, when you copy and paste some text inside the
same document in rich text mode, it will be transformed into
NFC on paste. Saving a document does not do any conversion to the glyphs on
screen, but it will save the characters in NFC.}


% ==========================
\section{Pitfall: Absence of canonical equivalence}
\label{pitfall-absence-of-equivalence}
% ==========================

Although in most cases canonical equivalence will take care of alternative
encodings of the same character, there are some cases in which the Unicode
Standard decided against equivalence. This leads to identical characters that
are not equivalent, like <ø> \textsc{latin small letter o with stroke} at
\uni{00F8} and <o̷> a combination of \textsc{latin small letter o} at \uni{006F}
with \textsc{combining short solidus overlay} at \uni{0037}.
The general rule followed is that extensions of Latin characters that are
visually connected to the base character are not separated as combining diacritics. For
example, characters like <ŋ ɲ ɳ> or <ɖ ɗ> are obviously derived from <n> and <d>
respectively, but they are treated like new separate characters in the Unicode
Standard. Likewise, characters like <ø> and <ƈ> are not separated into a base 
character <o> and <c> with an attached combining diacritic.

Interestingly, and somewhat illogically, there are three elements which are
directly attached to their base characters, but which are still treated as
separable in the Unicode Standard. Such characters are decomposed (in NFD
normalization) into a base character with a combining diacritic. However, it is
these cases that should be considered the exceptions to the rule. These three 
elements are the following:

\begin{itemize}

  \item <\dia{0327}>: the \textsc{combining cedilla} at \uni{0327} \newline 
        This diacritic is
        for example attested in the precomposed character <ç> \textsc{latin
        small letter c with cedilla} at \uni{00E7}. This <ç> will thus be
        decomposed in NFD normalization.
  \item <\dia{0328}>: the \textsc{combining ogonek} at \uni{0328} \newline 
        This diacritic is
        for example attested in precomposed <ą> \textsc{latin small letter a
        with ogonek} at \uni{0105}. This <ą> will thus be decomposed in NFD
        normalization.
  \item <\dia{031B}>: the \textsc{combining horn} at \uni{031B} \newline 
        This diacritic is for
        example attested in precomposed <ơ> \textsc{latin small letter o with
        horn} at \uni{01A1}. This <ơ> will thus be decomposed in NFD
        normalization. 

\end{itemize}

There are further combinations that deserve special care because it is actually
possible to produce identical characters in different ways without them being
. In these situations, the general rule holds, namely that
characters with attached extras are not decomposed. However, in the following
cases the extras actually exist as combining diacritics, so there is also 
the possibility to construct a character by using a base character with those 
combining diacritics.

\begin{itemize}
  
  \item First, there are the combining characters designated as \textit{combining
        overlay} in the Unicode Standard, like <\dia{0334}>
        \textsc{combining tilde overlay} at \uni{0334} or <\dia{0335}>
        \textsc{combining short stroke overlay} at \uni{0335}. There are many
        characters that look like they are precomposed with such an overlay,
        for example <\charis{ɫ~ᵬ~ᵭ~ᵱ}> or <\charis{ƚ~ɨ~ɉ~ɍ}>, or also the
        example of <ø> given at the start of this section. However, they are 
        not decomposed in NFD normalization.
  \item Second, the same situation also occurs with combining characters
        designated as \textit{combining hook}, like 
        <{\fontspec{CharisSIL}{\large ◌}}\symbol{"0321}> \textsc{combining
        palatalized hook below} at \uni{0321}. This element seems to occur in
        precomposed characters like <\charis{ᶀ~ᶁ~ᶂ~ᶄ}>. However, they are 
        not decomposed in NFD normalization.
        
\end{itemize}

To harmonize the encoding in these cases it is not sufficient to use Unicode 
normalization. Additional checks are necessary, for example by using orthography 
profiles (see Chapter~\ref{orthography-profiles}).


% ==========================
\section{Pitfall: Encodings}
\label{encodings}
% ==========================

% Section "Pitfall: File formats" may profit most from a more transparent terminology, distinguishing levels of encoding, rather than talking about how texts "appear inside some kind of computer file".

Before we discuss the pitfall of different file formats in Section \ref{pitfall-file-formats}, it is pertinent to point out that the common usage of the term \textsc{encoding} unfortunately does not distinguish between \textit{encoded} sequences of code points and text \textit{encoded} as bit patterns. Recall, a code point is simply a numerical representation of some defined entity; in other words, a code point is a character encoding-specific unique identifier or ID. In the Unicode Standard encoding, code points are numbers that serve as unique identifiers, each of which is associated with a set of character properties defined by the Unicode Consortium in the Unicode Character Database.\footnote{\url{https://www.unicode.org/ucd/}} The number of each code point can be \textit{encoded} in various formats, including as a decimal integer (e.g.\ 112), as an 8-bit binary sequence (01110000) or hexadecimal (0070). This example Unicode code point, \uni{0070}, represents \textsc{latin small letter p} and its associated Unicode properties, such as it belongs to the category Letter, Lowercase [Ll], in the Basic Latin block, and that its title case and upper case is associated with code point \uni{0050}.\footnote{See also Chapter \ref{the-unicode-approach}.}

% 0070;LATIN SMALL LETTER P;Ll;0;L;;;;;N;;;0050;;0050

The other meaning of encoding has to do with the fact that computers represent data and instructions in patterns of bits. A bit pattern is a combination of binary digits arranged in a sequence. And how these sequences are carved up into bit patterns is determined by how they are \textit{encoded}. Thus the term \textsc{encoding} is used for both sequences of code points and text encoded as bit patterns. Hence a Unicode-aware programmer might prefer to say that UTF-8, UTF-16, etc., are Unicode encoding systems because they determine how sequences of bit patterns are determined, which are then mapped to characters.\footnote{UTF stands for Unicode Transformation Format. It a method for translating numbers into binary data and vice versa. There are several different UTF encoding formats, e.g.\ UTF-8 is a variable-length encoding that uses 8-bit code units, is compatible with ASCII, and is common on the web. UTF-16 is also variable-length, uses 16-bit code units, and is used system-internally by Windows and Java. See further discussion under \textit{Code units} in Section \ref{pitfall-file-formats}. For more in-depth discussion, refer to the Unicode Frequently Asked Questions and additional sources therein: \url{http://unicode.org/faq/utf_bom.html}.} The terminological issue here is that the Unicode Standard introduces a layer of indirection between characters and bit patterns, i.e.\ the code point, which can be encoded differently by different encoding systems.

% Characters encoded, but not seen.
Note also that all computer character encodings include so-called \textsc{control characters}, which are non-printable sometimes action-inducing characters, such as the null character, bell code, backspace, escape, delete, and line feed. Control characters can interact with encoding schemes. For example, some programming languages make use of the null character to mark the end of a string. Line breaks are part of the text, and as such as covered by the Unicode Standard. But they can be problematic because line breaks differ from operating system to operating system in how they are encoded. These variants are discussed in Section \ref{pitfall-file-formats}.

% This section should also mention that line breaks are actually part of the text, and as such also covered by the Unicode standard. Again regarding line breaks, the recommendation "that everybody use this encoding [only LF] whenever possible" (page 31) seems less well motivated (and specific) than most other recommendations in the book. Some common formats, e.g.\ CSV (as specified by  RFC 4180 [4]), specify "CRLF" as line break. Thus, the perceived "strong tendency" towards "LF" may be just that.


% ==========================
\section{Pitfall: File formats}
\label{pitfall-file-formats}
% ==========================

Unicode is a character encoding standard, but characters of course 
appear inside some kind of computer file. The most basic Unicode-based file
format is pure line-based text, i.e.~strings of Unicode-encoded characters
separated by line breaks (note that these line breaks are what for most people
intuitively corresponds to paragraph breaks). Unfortunately, even within this
apparently basic setting there exists a multitude of variants. In general these
different possibilities are well-understood in the software industry, and
nowadays they normally do not lead to any problems for the end user. However,
there are some situations in which a user is suddenly confronted with cryptic
questions in the user interface involving abbreviations like LF, CR, BE, LE or
BOM.\@ Most prominently this occurs with exporting or importing data in several
software applications from Microsoft. Basically, there are two different issues
involved. First, the encoding of line breaks and, second, the encoding of the
Unicode characters into code units and the related issue of endianness.

\subsubsection*{Line breaks}

The issue with \textsc{line breaks} originated with the instructions necessary
to direct a printing head of a physical printer to a new line. This involves two
movements, known as \textsc{carriage return} (CR, returning the printing head to
the start of the line on the page) and \textsc{line feed} (LF, moving the
printing head to the next line on the page). Physically, these are two different
events, but conceptually together they form one action. In the history of
computing, various encodings of line breaks have been used (e.g.~CR+LF, LF+CR,
only LF, or only CR). Currently, all Unix and Unix-derived systems use only LF
as code for a line break, while software from Microsoft still uses a combination
of CR+LF.\@ Today, most software applications recognize both options, and
are able to deal with either encoding of line breaks (until rather recently this
was not the case, and using the wrong line breaks would lead to unexpected
errors). Our impression is that there is a strong tendency in software
development to standardize on the simpler ``only LF'' encoding for line
breaks, and we suggest that everybody should use this encoding whenever possible.

\subsubsection*{Code units}

The issue with \textsc{code units} stems from the question how to separate a
stream of binary ones and zeros, i.e.~bits, into chunks representing Unicode
characters. A code unit is the sequence of bits used to encode a single
character in an encoding. The Unicode Standard offers three different approaches, 
called UTF-32, UTF-16 and UTF-8, that are intended for different use cases.\footnote{The
letters UTF stand for \textsc{Unicode Transformation Format}, but the notion of
``transformation'' is a legacy notion that does not have meaning anymore.
Nevertheless, the designation UTF (in capitals) has become an official
standard designation, but should probably best be read as simply ``Unicode
Format''.} The details of this issue are extensively explained in section 2.5 of
the Unicode Core Specification \citep{Unicode2018}. 

Basically, \textsc{UTF-32} encodes each character in 32 bits (32 \textit{bi}nary
uni\textit{ts}, i.e.~32 zeros or ones) and is the most disk-space-consuming
variant of the three. However, it is the most efficient encoding
processing-wise, because the computer simply has to separate each character
after 32 bits. 

In contrast, \textsc{UTF-16} uses only 16 bits per character, which is
sufficient for the large majority of Unicode characters, but not for all of
them. A special system of \textsc{surrogates} is defined within the Unicode
Standard to deal with these additional characters. The effect is a more
disk-space efficient encoding (approximately half the size), while adding a
limited computational overhead to manage the surrogates. 

Finally, \textsc{UTF-8} is a more complex system that dynamically encodes each
character with the minimally necessary number of bits, choosing either 8, 16 or
32 bits depending on the character. This represents again a strong reduction in
space (particularly due to the high frequency of data using erstwhile ASCII
characters, which need only 8 bits) at the expense of even more computation
necessary to process such strings. However, because of the ever growing
computational power of modern machines, the processing overhead is in most
practical situations a non-issue, while saving on space is still useful,
particularly for sending texts over the Internet. As a result, UTF-8 has become
the dominant encoding on the World Wide Web. We suggest that everybody uses
UTF-8 as their default encoding.

A related problem is a general issue about how to store information in computer
memory, which is known as \textsc{endianness}. The details of this issue go
beyond the scope of this book. It suffices to realize that there is a difference
between \textsc{big-endian} (BE) storage and \textsc{little-endian} (LE)
storage. The Unicode Standard offers a possibility to explicitly indicate what
kind of storage is used by starting a file with a so-called \textsc{byte order
mark} (BOM). However, the Unicode Standard does not require the use of BOM,
preferring other non-Unicode methods to signal to computers which kind of
endianness is used. This issue only arises with UTF-32 and UTF-16 encodings.
When using the preferred UTF-8, using a BOM is theoretically possible, but
strongly dispreferred according to the Unicode Standard. We suggest that
everyone tries to prevent the inclusion of BOM in their data.

% ==========================
%\section{Pitfall: Software}
%\label{software}
% ==========================

% issues we've encountered in our computing environments
%  save as UTF-8
%  conversion between software programs (or when is something really UTF-8?)
%  NFC / NFD in copy & paste(!)


% ==========================
\section{Pitfall: Incomplete implementations}
\label{incomplete-implementations}
% ==========================
Another pitfall that we encounter when using the Unicode Standard is its incomplete implementation in different standards and programming languages, e.g.\ SQL, XML, XLST, Python. For example, although the Unicode Standard mandates that the comparison of Unicode text be done using normalized text, this is not the case with the equality operator ``=='' in Python. Furthermore, it is not always transparent what the operating system or specific software applications do when text is being copied and pasted. For example, copy and pasting the character sequence \uni{0061} \textsc{latin small letter a} <a> and \uni{0301} \textsc{combining acute accent} <\dia{0301}>, visually <á>, into the text editor TextWrangler leaves the sequence decomposed as two characters. But when pasting the decomposed sequence into RStudio, and other software programs, the sequence becomes precomposed as \uni{00E1} \textsc{latin small letter a with acute}, i.e.\ <á>. Although inconvenient, we expect all such behaviour of software only become more consistent in the future.

% One Unicode pitfall which may be worth adding is the problem with incomplete implementations of the Unicode Standard. This is a problem most big-enough standards suffer from, e.g.\ SQL, XML, XSLT. So although the Unicode Standard mandates that comparison of Unicode text should always be done using normalized text, this is not how the equality operator "==" is implemented for Unicode text in the Python programming language.


% ==========================
\section{Recommendations}
\label{recommendations}
% ==========================

Summarizing the pitfalls discussed in this chapter, we propose the following 
recommendations:

\begin{itemize}
   \item To prevent strange boxes instead of nice glyphs, always install a few
         fonts with a large glyph collection and at least one fall-back font
         (see Section~\ref{pitfall-missing-glyphs}).
   \item Unexpected visual impressions of symbols do not necessarily mean that
         the actual encoding is wrong. It is mostly a problem of faulty
         rendering or font substitution (see Section~\ref{pitfall-faulty-rendering}).
   \item Do not trust the names of code points as a definition of the character
         (see Section~\ref{pitfall-names}). Also do not trust Unicode blocks as
         a strategy to find specific characters (see
         Section~\ref{pitfall-blocks}).
   \item To ensure consistent encoding of texts, apply Unicode normalization
         (NFC or NFD, see Section~\ref{pitfall-canonical-equivalence}).
   \item To prevent remaining inconsistencies after normalization, for example 
         stemming from homoglyphs (see Section~\ref{pitfall-homoglyphs}) 
         or from missing canonical equivalence in the Unicode Standard
         (see Section~\ref{pitfall-absence-of-equivalence}), 
         use orthography profiles (see Chapter~\ref{orthography-profiles}).
   \item To deal with tailored grapheme clusters
         (Section~\ref{pitfall-characters-are-not-graphemes}), use Unicode Locale 
         Descriptions, or orthography profiles 
         (see Chapter~\ref{orthography-profiles}).
   \item As a preferred file format, use Unicode Format UTF-8 in 
         Normalization Form Composition (NFC) with LF line endings, 
         but without byte order mark (BOM), whenever possible (see 
         Section~\ref{pitfall-file-formats}). This last nicely cryptic 
         recommendation has T-shirt potential:
  
\end{itemize}

\begin{center}
  I prefer it
  
  \textbf{UTF-8 NFC LF no BOM}
\end{center}



% ==========================
\chapter{The International Phonetic Alphabet}
\label{the-international-phonetic-alphabet}
% ==========================

In this chapter we present a brief history of the IPA 
(Section~\ref{IPAhistory}), which dates 
back to the late 19th century, not long after the creation of the first 
typewriter with a QWERTY 
keyboard. An understanding of the IPA and its premises and principles 
(Section~\ref{IPApremises-principles}) leads to a better 
appreciation of the challenges that the International Phonetic Association 
faced when digitally encoding the IPA's set of symbols and 
diacritics (Section~\ref{EncodingIPA}). Occurring a little over a hundred years after 
the inception of the IPA, its encoding was a major challenge 
(Section~\ref{need-for-multilingual-environment}); many 
linguists have encountered pitfalls when the two are used together 
(Chapter~\ref{ipa-meets-unicode}).

% ==========================
\section{Brief history}
\label{IPAhistory}
% ==========================

Established in 1886, the \textsc{international phonetic association} (henceforth
\textit{Association}) has long maintained a standard alphabet, the
\textsc{international phonetic alphabet} or IPA, which is a
standard in linguistics to transcribe sounds of spoken languages. It was
first published in 1888 as an international system of phonetic transcription for
oral languages and for pedagogical purposes. It contained phonetic values for
English, French and German. Diacritics for length and nasalization were already
present in this first version, and the same symbols are still used today. 
%\footnote{Also referred to as API, for \textit{Association Phonétique Internationale}.} 

Originally, the IPA was a list of symbols with pronunciation examples
using words in different languages. In 1900 the symbols were first organized into
a chart and were given phonetic feature labels, e.g.\ for manner of
articulation among others \textit{plosives}, \textit{nasales}, \textit{fricatives}, for place of
articulation among others \textit{bronchiales}, \textit{laryngales}, \textit{labiales} and for vowels
e.g.\ \textit{fermées}, \textit{mi-fermées}, \textit{mi-ouvertes}, \textit{ouvertes}. Throughout the last
century, the structure of the chart has changed with increases in phonetic
knowledge. Thus, similar to notational systems in other scientific disciplines,
the IPA reflects facts and theories of phonetic knowledge that have developed
over time. It is natural then that the IPA is modified occasionally to
accommodate scientific innovations and discoveries. In fact, updates are part of the
Association's mandate. These changes are captured in revisions to the IPA chart.\footnote{For a detailed history, 
we refer the reader to:
\url{https://en.wikipedia.org/wiki/History\_of\_the\_International\_Phonetic_Alphabet}.}

Over the years there have been several revisions, but mostly minor ones. Articulation 
labels -- what are often called \textit{features}, even though the IPA
deliberately avoids this term -- have changed, e.g.\ terms like \textit{lips}, \textit{throat}
or \textit{rolled} are no longer used. Phonetic symbol values have changed, e.g.\
voiceless is no longer marked by <h>. Symbols have been dropped, e.g.\ the
caret diacritic denoting `long and narrow' is no longer used. And many symbols
have been added to reflect contrastive sounds found in the world's very diverse
phonological systems. The use of the IPA is guided by principles outlined in 
the \textit{Handbook of the International Phonetic Association} \citep{IPA1999}, 
henceforth simply called \textit{Handbook}. 

Today, the IPA is designed to meet practical linguistic needs and is used to
transcribe the phonetic or phonological structure of languages. It is also used
increasingly as a foreign language learning tool, as a standard pronunciation
guide and as a tool for creating practical orthographies of previously unwritten
languages. The IPA suits many linguists' needs because:

\begin{itemize}

	\item it is intended to be a set of symbols for representing all possible
       sounds in the world's (spoken) languages;
	\item its chart has a linguistic basis (and specifically a phonological bias)
       rather than just being a general phonetic notation scheme;
	\item its symbols can be used to represent distinctive feature
       combinations;\footnote{Although the chart uses traditional manner and
       place of articulation labels, the symbols can be used as a representation
       of any defined bundle of features, binary or otherwise, to define
       phonetic dimensions.}
	\item its chart provides a summary of linguists' agreed-upon phonetic 
	knowledge.

\end{itemize}

Several styles of transcription with the IPA are possible, as illustrated in the
\textit{Handbook}, and they are all valid.\footnote{For an illustration of
the differences, see the 29 languages and their transcriptions in the
\textit{Illustrations of the IPA} \citep[41--154]{IPA1999}.} Therefore, there are 
different but equivalent transcriptions, or as noted by \citet[64]{Ladefoged1990a}, 
``perhaps now that the Association has been explicit in its eclectic approach, outsiders to the
Association will no longer speak of \textit{the} IPA transcription of a given
phenomenon, as if there were only one approved style.'' Clearly not all
phoneticians agree, nor are they likely to ever completely agree, on all aspects of the
IPA or on transcription approaches and practices in general. As noted above, 
there have been several revisions in the IPA's long history, but the current version (2005) is
strikingly similar to the 1926 version, which shows the viability of the IPA as a
common standard for linguistic transcription.

% ==========================

\section{Premises and principles}
\label{IPApremises-principles}
% ==========================
\subsection*{Premises}
\label{IPApremises}

Any IPA transcription is based on two premises: (i) that it is possible to
describe the acoustic speech signal (sound waves) in terms of sequentially
ordered discrete segments, and, (ii) that each segment can be characterized by
an articulatory target.

Once spoken language data are segmented, the IPA provides symbols to
unambiguously represent phonetic details. However, since phonetic detail could
potentially include anything, e.g.\ something like ``deep voice'', the IPA
restricts phonetic detail to linguistically relevant aspects of speech.
Phonological considerations thus inextricably play a roll in transcription. In
other words, phonetic observations beyond quantitative acoustic analysis are
always made in terms of some phonological framework.

Today, the IPA chart reflects a linguistic theory grounded in principles of
phonological contrast and in knowledge about the attested linguistic variation.
This fact is stated explicitly in several places, including in the
\textit{Report on the 1989 Kiel convention} published in the \textit{Journal of
the International Phonetic Association} \citep[67--68]{International1989report}:

\begin{quote}
The IPA is intended to be a set of symbols for representing all the possible 
sounds of the world's languages. The representation of these sounds uses a set 
of phonetic categories which describe how each sound is made. These categories 
define a number of natural classes of sounds that operate in phonological rules 
and historical sound changes. The symbols of the IPA are shorthand ways of 
indicating certain intersections of these categories.
\end{quote}

\noindent and in the \textit{Handbook} \citep[18]{IPA1999}: 

\begin{quote}
% The general value of the symbols in the chart is listed below. In each case 
[...] a symbol can be regarded as a shorthand equivalent to a phonetic
description, and a way of representing the contrasting sounds that occur in a
language. Thus [m] is equivalent to ``voiced bilabial nasal'', and is also a way
of representing one of the contrasting nasal sounds that occur in English and
other languages. [...] When a symbol is said to be suitable for the
representation of sounds in two languages, it does not necessarily mean that the
sounds in the two languages are identical.
\end{quote}

\noindent From its earliest days the Association aimed to provide ``a separate sign for
each distinctive sound; that is, for each sound which, being used instead of
another, in the same language, can change the meaning of a word''
\citep[27]{IPA1999}. Distinctive sounds became later known as \textsc{phonemes}
and the IPA has developed historically into a notational device with a strictly
segmental phonemic view. A phoneme is an abstract theoretical notion derived
from an acoustic signal as produced by speakers in the real world. Therefore the
IPA contains a number of theoretical assumptions about speech and how to
transcribe speech in written form. 

% TODO: cite Moran 2018

% ==========================
\subsection*{Principles}
\label{IPAprinciples}
% ==========================

Essentially, transcription has two parts: a text containing symbols and a set 
of conventions for interpreting those symbols (and their combinations). 
The symbols of the IPA distinguish between letter-like symbols and
diacritics (symbol modifiers). The use of the letter-like symbols to represent 
a language's sounds is guided by the principle of contrast; where two words 
are distinguishable by phonemic contrast, those contrasts should be transcribed 
with different letter symbols (and not just diacritics). Allophonic distinction 
falls under the rubric of diacritically-distinguished symbols, e.g.\ the 
difference in English between an aspirated /p/ in [pʰæt] and 
an unreleased /p/ in [stop̚]. 

\begin{itemize}

	\item Different letter-like symbols should be used whenever
          a language employs two sounds contrastively.
	\item When two sounds in a language are not known to be contrastive, the same
          symbol should be used to represent these sounds. Diacritics may
          be used to distinguish different articulations when necessary.
\end{itemize}          
          
\noindent Yet, in some situations diacritics are used to mark phonemic
contrasts. The \textit{Handbook} recommends to limit the use of phonemic
diacritics to the following situations: 

\begin{itemize}

 	\item denoting length, stress and pitch;
	\item obviating the design of a (large) number of new symbols when a 
		  single diacritic suffices (e.g.\ nasalized vowels, aspirated stops). 
               
\end{itemize}	

The interpretation of the IPA symbols in specific usage is not trivial. Although
the articulatory properties of the IPA symbols themselves are rather succinctly
summarized by the normative description in the \textit{Handbook}, it is common
in practical applications that the transcribed symbols do not precisely
represent the phonetic value of the sounds that they represent. So an IPA symbol
<t> in one transcription is not always the same as an IPA <t> in another
transcription (or even within a single transcription). The interpretation of any
particular <t> is mostly a language-specific convention (or even
resource-specific and possibly even context-specific), a fact which --
unfortunately -- is in most cases not made explicit by users of the IPA.

There are different reasons for this difficulty to interpret IPA symbols, all
officially sanctioned by the IPA. An important principle of the IPA is that
different representations resulting from different underlying analyses are
allowed. Because the IPA does not provide phonological analyses for specific
languages, the IPA does not define a single correct transcription system.
Rather, the IPA aims to provide a resource that allows users to express any
analysis so that it is widely understood. Basically, the IPA allows for both a 
\textit{narrow} phonetic transcription and a \textit{broad} phonological transcription. 
A narrow phonetic transcription may freely use all symbols in the IPA 
chart with direct reference to their phonetic value, i.e.\ the transcriber can 
indicate with the symbols <ŋ͡m> that the phonetic value of the attested sound 
is a simultaneous labial and velar closure which is voiced and contains nasal 
airflow, independently of the phonemic status of this sound in the language in 
question. 

% examples from Wells \url{https://www.phon.ucl.ac.uk/home/wells/transcription-ELL.pdf}

In contrast, the basic goal of a broad phonemic transcription is to distinguish all
words in a language with a minimal number of transcription symbols
\citep[19]{Abercrombie1964}. A phonemic transcription includes the conventions
of a particular language's phonological rules. These rules determine the
realization of that language's phonemes. For example, in the transcription of
German, Dutch, English and French a symbol <t> might be used for the voiceless
plosive in the alveolar and dental areas. This symbol is sufficient for a succinct
transcription of these languages because there is no further phonemic
subdivisions in this domain in either of these languages. However, the
language-specific realization of this consonant is closer to [t̪ʰ], [t], [tʰ]
and [t̪], respectively. Similarly, the five vowels of Greek can be represented
phonemically in IPA as /ieaou/, though phonetically they are closer to [iεaɔu].
The Japanese five-vowel system can also be transcribed in IPA as
/ieaou/, while the phonetic targets in this case are closer to [ieaoɯ].

Note also that there can be different systems of phonemic transcription for the
same variety of a language, so two different resources on the ``same'' language
might use different symbols that represent the same sound. The differences may
result from the fact that more than one phonetic symbol may be appropriate for a
phoneme, or the differences may be due to different phonemic analyses, e.g.\
Standard German's vowel system is arguably contrastive in length, tenseness or
vowel quality. Finally, even within a single phonemic transcription a specific
symbol can have different realizations because of allophonic variation which 
might not be explicitly transcribed.

In sum, there are three different reasons why phonemically-based IPA 
transcription is difficult to interpret:

\begin{itemize}
  
   \item A symbol represents the phonemic status, and not necessarily the
         precise phonetic realization. So, different transcriptions might use 
         the same symbol for different underlying sounds.
   \item Any symbol that is used for a specific phoneme is not necessarily
         unique, so different transcriptions might use different symbols for the
         same underlying sound.
   \item Allophonic variation can be disregarded in phonemic transcription, so
         the same symbol within a single transcription can represent different
         sounds.
  
\end{itemize}

Ideally, all such implicit conventions of a phonemic transcription would be
explicitly codified. This could very well be performed by using an orthography
profile (see Chapter~\ref{orthography-profiles}), linking the selected phonemic 
transcription symbols to narrow phonetic transcriptions, possibly also including 
specifications of contextual interpretation.

% ==========================

\section{IPA encodings}
\label{EncodingIPA}
% ==========================

In 1989, an IPA revision convention was held in Kiel, Germany. As in previous meetings, 
there were changes made to the repertoire of phonetic symbols in the IPA chart, which 
reflected what had been discovered, described and cataloged by linguists about the 
phonological systems in the world's languages in the interim. Personal computers 
were also becoming more commonplace, and linguists were using them to create databases. 
A cogent example is the UCLA Phonological 
Segment Inventory Database (UPSID; \citealt{Maddieson1984}), which was expanded 
\citep{MaddiesonPrecoda1990} and then encoded and distributed in a computer program 
\citep{MaddiesonPrecoda1992}.\footnote{It could be installed via 
floppy disk on an IBM PC, or compatible, running 
DOS with 1MB free disk space and 360K available RAM.} The programmers used 
only ASCII characters to maximize compatibility (e.g.\ <kpW> for $[$kpʷ$]$), but 
were faced with unavoidable arbitrary mappings between ASCII letters and 
punctuation, and the more than 900 segment types documented 
in their sample of world's languages' phonological systems. The developers 
devised a system of base characters with secondary diacritic marks 
(e.g.\ in the previous example <kp>, the base character, is modified with <W>). 
This encoding approach is 
also used in SAMPA and X-SAMPA (see below) and in the 
ASJP.\footnote{See the ASJP use case in the online supplementary 
materials to this book: \url{https://github.com/unicode-cookbook/recipes}.} 
But before UPSID, SAMPA and ASJP, IPA was encoded with numbers.
 
% \footnote{The marking of tone was extended (from characteristically Africanist practice) to include a second system for marking linguistic tones (the `Chao tone letters'), a much used convention based on musical staff to describe pitch in by Yuen Ren Chao (1892--1982).} 

\subsection*{IPA Numbers}
Prior to the Kiel Convention for the modern revision of the IPA in 1989,
\citet{Wells1987} collected and published practical approaches to coded
representations of the IPA, which dealt mainly with the assignment of characters
on the keyboard to IPA symbols. The process of assigning standardized computer
codes to phonetic symbols was given to the \textit{workgroup on computer
coding} (henceforth \textit{working group}) at the Kiel Convention. This working
group was given the following tasks
\citep{Esling1990,EslingGaylord1993}: 

\begin{itemize}
	\item determining how to represent the IPA numerically
	\item developing a set of numbers to refer to the IPA symbols unambiguously
	\item providing each symbol a unique name (intended to provide a mnemonic description of that character's shape)
\end{itemize}

\noindent The identification of IPA symbols with unique identifiers was 
a first step in formalizing the IPA computationally because it would give 
each symbol an unambiguous numerical identifier called an \textsc{ipa number}. 
The numbering system was to be comprehensive enough to support future revisions 
of the IPA, including symbol specifications and diacritic placement. The 
application of diacritics was also to be made explicit. 

Although the Association had never officially approved a set of names 
for the IPA symbols, each IPA symbol received a unique \textsc{ipa name}. 
Many symbols already had an informal name (or two) used by linguists, but 
consensus on symbol names was growing due to the recent publication of the 
\textit{Phonetic Symbol Guide} \citep{PullumLadusaw1986}. Thus most of the 
IPA symbol names were taken from that source \citep[31]{IPA1999}.

The working group decided insightfully that the computing-coding convention 
for the IPA should be independent of computer environments or formats, 
i.e.\ the \textsc{ipa number} was not meant to be encoded at the bit pattern level.
The working group report's declaration includes the following explanatory 
remarks \citep[82]{International1989report}:

\begin{quote}
The recommendation of a 7-bit ASCII or 8-bit extended-ASCII coding system 
would be short-sighted in view of development towards 16-bit and 32-bit 
processors. In fact, any specific recommendations would tie the Association 
to a stage of technological development which is bound to be outdated long 
before the next revision of the handbook.
\end{quote}

\noindent The coding convention was not meant to address the engineering 
aspects of the actual encoding in computers (cf.\ \cite{Anderson1984}). However, 
it was meant to serve as a basis for a interchange standard for 
creating mapping tables from various computer encodings, fonts, phonetic-character-set 
software, etc., to common IPA numbers, and therefore symbols.\footnote{Remember, at 
this time in the late 1980s there was no stable multilingual computing environment. 
But some solution was needed because scholars were increasingly using personal 
computers for their research and many were quickly adopting electronic mail or 
discussion boards like Usenet as a medium for international exchanges. 
Most of these systems ran on 8-bit hardware systems using a 7-bit ASCII character encoding.}

Furthermore, the assignment of computer codes to IPA symbols was meant to
represent an unbiased formulation. The Association here played the role of an
international advisory body and it stated that it should not recommend a
particular existing system of encoding. In fact, during this time there were a
number of coding systems used (see Section~\ref{encoding}), but none of them had
a dominant international position. The differences between systems were also
either too great or too subtle to warrant an attempt at combining them
\citep{International1989report}.

The working group assigned each IPA symbol to a unique three-digit IPA number. 
Encoded in this number scheme implicitly is information
about the status of each symbol (see below). The IPA numbers were listed with the
IPA symbols and they were also illustrated in IPA chart form (see
\cite[84]{EslingGaylord1993} or \cite[App. 2]{IPA1999}). The numbers were
assigned in linear order (e.g.\ [p] 101, [b] 102, [t] 103...) following the IPA
revision of 1989 and its update in 1996. Although the numbering scheme still 
exists, in practice it is superseded by the Unicode codification of symbols.

The working group made the decision that no IPA symbol, past or present, 
could be ignored. The comprehensive inclusion of all IPA symbols was to 
anticipate the possibility that some symbols might be added, withdrawn, 
or reintroduced into current or future usage. For example, in the 1989 
revision voiceless implosives <~ƥ,~ƭ,~ƈ,~ƙ,~ʠ~> were added; in the 1993 
revision they were removed. Ligatures like <~ʧ,~ʤ~> are included as formerly 
recognized IPA symbols; they are assigned to the 200 series of IPA numbers 
as members of the group of symbols formerly recognized by the IPA. To ensure 
backwards compatibility, legacy IPA symbols would retain an IPA number and 
an IPA name for reference purposes. As we discuss below, this decision is 
later reflected in the Unicode Standard as many legacy IPA symbols still reside in 
the \textsc{IPA Extensions} block.

The IPA number is expressed as a three-digit number. The first digit 
indicates the symbol's category \citep{Esling1990,EslingGaylord1993}:

\begin{itemize}
	\item 100s for accepted IPA consonants
	\item 200s for former IPA consonants and non-IPA symbols
	\item 300s for vowels
	\item 400s for segmental diacritics
	\item 500s for suprasegmental symbols
	\item 600s-800s for future specifications
	\item 900s for escape sequences
\end{itemize}

After a symbol is categorized, it is assigned a number sequentially, e.g.\ [i]
301, [e] 302, [ɛ] 303. The system allows for the addition of new symbols within
the various series by appending them, e.g.\ [\charis{ⱱ}] 184. Former non-IPA
symbols (or often-used but non-official IPA symbols) for consonants, vowels and
diacritics are numbered from 299 backwards. For example, the voiceless and
voiced postalveolar affricates and fricatives <~č,~ǰ,~š,~ž~> are assigned the
IPA numbers 299, 298, 297 and 296, respectively, because they are not sanctioned
IPA symbols.

The assignment of the IPA numbers to IPA symbols provided the basis for uniquely
identifying the set of past and present IPA symbols as a type of computational
representational standard of the IPA. Within each revision of the IPA, the
coding defines a closed and clearly defined set of characters. The benefits of
this standardization are clear in at least two ways: it is used in translation
tables that reference ASCII representations of the IPA, and this early
computational representation of the IPA became the basis for X-SAMPA and for the
inclusion of the IPA into the Unicode Standard version 1.0.

% ==========================
\subsection*{SAMPA and X-SAMPA}
\label{sampa-xsampa}
% ==========================

True to the working group's aim, the IPA numbers provided a mechanism for 
an interchange standard for creating mapping tables to various 
computer encodings. For example, the IPA coding system was used as a mapping 
system in the creation of SAMPA \citep{Wells_etal1992}, an ASCII representation 
of the IPA symbols. 

For a long time, linguists, like all other computer users, were
limited to ASCII-encoded 7-bit characters, which only includes Latin characters,
numbers and some punctuation and symbols. Restricted to these standard character
sets that lacked IPA support or other language-specific graphemes that they
needed, linguists devised their own solutions.\footnote{Early work addressing
the need for a universal computing environment for writing systems and their
computational complexity is discussed in \citet{Simons1989}. A more recent survey of
practical recommendations for language resources, including notes on encoding,
can be found in \citet{BirdSimons2003}.} For example, some chose to represent
unavailable graphemes with substitutes, e.g.~the combination of <ng> to
represent <ŋ>. Tech-savvy linguists redefined selected characters from a
character encoding by mapping custom-made fonts to specific code points.\footnote{For 
example, SIL's popular font SIL IPA 1990.} However,
one linguist's electronic text would not render properly on another linguist's
computer without access to the same font. Furthermore, if two character encodings
defined two character sets differently, then data could not be reliably and
correctly displayed. This is a commonly encountered example of the non-interoperability of
data and data formats.

One solution was the ASCII-ification of the IPA, which simply involved 
defining keyboard-able sequences consisting of ASCII combinations as IPA symbols. 
For example, \citet{Wells1987} provides an in-depth description of IPA
codings from country-to-country. Later ASCII-IPAs include Kirshenbaum (created
in 1992 in a Usenet group and named after its lead developer who was at
Hewlett-Packard Laboratories) and Worldbet (published by
\citet{Hieronymus1993}, who was at AT\&T Laboratories). 
The most successful effort was SAMPA (Speech Assessment
Methods Phonetic Alphabet), which was created between 1988--1991 in Europe to 
represent IPA symbols with ASCII
character sequences \citep{Wells1987,Wells_etal1992}, using e.g.\ <p\textbackslash> 
for [ɸ]. SAMPA was developed by a group of speech scientists from nine countries 
in Europe and it constituted the ASCII-IPA symbols needed for phonemic transcription 
of the principal European languages \citep{Wells1995}. It is still widely 
used in language technology.

Two problems with SAMPA are that (i) it is only a partial encoding of the IPA
and (ii) it encodes different languages in separate data tables, instead of
using a universal alphabet, like IPA.\@ SAMPA tables were developed as part of a
European Commission-funded project to address technical problems like electronic
mail exchange (what is now simply called email). SAMPA is essentially a hack to
work around displaying IPA characters, but it provided speech technology and
other fields a basis that has been widely adopted and often still used in code.
So, SAMPA is a collection of tables to be compared, instead of a large universal
table representing all languages. 

An extended version of SAMPA, called X-SAMPA, set out to include every symbol,
including all diacritics, in the IPA chart \citep{Wells1995}. X-SAMPA is
considered more universally applicable because it consists of one table that
encodes all characters in IPA. In line with the principles of the IPA, SAMPA and
X-SAMPA include a repertoire of symbols. These symbols are intended to represent
phonemes rather than all allophonic distinctions. Additionally, both
ASCII-ifications of IPA -- SAMPA and X-SAMPA -- are
(reportedly) uniquely parsable \citep{Wells1995}. However, like the IPA, X-SAMPA
has different notations for encoding the same phonetic phenomena (cf.\ Section~
\ref{pitfall-multiple-options-ipa}).

SAMPA and X-SAMPA have been widely used for speech technology and as an encoding
system in computational linguistics. In fact, they are still used in popular
software packages that require ASCII input, e.g.~RuG/L04 and SplitsTree4.\footnote{See
\url{http://www.let.rug.nl/kleiweg/L04/} and \url{http://www.splitstree.org/},
respectively.}

% ==========================

\section{The need for a single multilingual environment}
\label{need-for-multilingual-environment}
% ==========================

In hindsight it is easy to lose sight of how impactful 30 years of technological
development have been on linguistics, from theory development using quantitative
means to pure data collection and dissemination. But at the end of the 1970s,
virtually no ordinary working linguist was using a personal computer
\citep{Simons1996}. Personal computer usage, however, dramatically increased
throughout the 1980s. By 1990, dozens of character sets were in common use. They
varied in their architecture and in their character repertoires, which made
things a mess. 

During the 1980s, it became increasingly clear that an adequate solution 
to the problem of multilingual computing environments was needed. Linguists 
were on the forefront of addressing this issue because they faced these 
challenges head-on by wishing to publish and communicate electronic text 
with phonetic symbols which were not included in basic ASCII. One 
only needs to look at facsimiles of older electronic documents to see exotic 
symbols written in by hand after the preparation of typed version.

%There were two major players in the universal character set race:
%Unicode and the International Organization for Standardization (ISO).

%Long familiar were linguists already with the distinction between function 
%and form. Even in the context of the computer implementation of writing systems, 
%the necessity to distinguish form and function had been made \citep{Becker1984}. 
%The computer industry, on the other hand, did not consider, ignored, or simply 
%did not encode this principle when creating operating systems like MS-DOS, which 
%were limited to 256 code points (due to computer hardware architecture) and 
%encoded with one-to-one mappings from character codes to graphemes.

% This standard became the basis for a proposal to include the IPA in the first version of the Unicode Standard. Decisions by the Computer Coding working group and work they continued after the 1989 Kiel Convention were adopted by the International Phonetic Association. These decisions are directly reflected in the Unicode Standard's encoding of IPA, seeing as it was the Association who submitted the script proposal to the Unicode Consortium.

A major benefit of the standardization of the IPA in a computational
representation by the Kiel working group is that it provided the basis for a
formal proposal to be submitted to various international standards
organizations, several of which were trying to tackle (and in a sense win) the
multilingual computing environment problem (cf.\ Section~\ref{encoding}).
Basically, everyone -- from corporations to governments to language scientists
-- wanted a single unified multilingual character encoding set for all the
world's writing systems, even if they did not understand or appreciate the
challenges involved in creating and adopting a solution.

Industry was starting to tackle the issues involved in developing a single
multilingual computing environment on a variety of fronts, including the then
new technology of bitmap fonts and the creation of Font Manager and Script
Manager by Apple \citep{Apple1985,Apple1986,Apple1988}. As noted above, around
this time linguists were developing work-arounds such as SAMPA, so that they
could communicate IPA transcription and use ASCII-based software. Some linguists
formalized the issues of multilingual text processing from a computational
perspective \citep{Anderson1984,Becker1984,Simons1989}. The study of writing
systems was also being invigorated \citep[11--15]{Sampson1985} by the
computational challenges in making computers work in a multilingual environment.
The engineering problems and solutions had been spelled out years before, e.g.\
a two-byte encoding for multilingual text \citep{Anderson1984}. Although
languages vary to an astounding extent (cf.\ \cite{EvansLevinson2009}), writing
systems are quite similar formally and the issue of formal representation of the
world's orthographic systems had already been addressed \citep{Simons1989}. 

%A major obstacle in creating a single encoding multilingual environment from 
%the perspective of writing systems involves the distinction between function 
%and form \citep{Becker1984} This distinction is so central to basic linguistic 
%theory and that trained linguists and semiologists take it as second nature. 
%A central challenge in developing a universal character set was to combine a 
%technological solution with a formalization of writing systems proper.\footnote{Of 
%course there were additional practical issues to overcome, e.g.\ funding, creating 
%the formal and technological proposal, deciding which characters and writing systems 
%to include initially, while setting precedence of how to add new ones in the future.}


% "The set of IPA symbols and their numbers were used to draw up an entity set within SGML by the Text Encoding Initiative (TEl). The name of each entity is formed by 'IPA' preceding the number, e.g.\ IPA304 is the rEIentity name of lower-case A. These symbols can be processed as IPA symbols and represented on paper and screen with the appropriate local font by modifying the :entity replacement text. The advantage of the SGML entity set is that it is independent or the character set being used."
% "A TEl writing system declaration (wsd) has been drawn up for the IPA symbols."
% A TEl writing system declaration (wsd) has been drawn up for the IPA symbols. This document gives information about the symbol and its IPA function, as well as its encoding in the accompanying SGML document and in UnicodelUCS and in AFII. The writing system declaration can be read as a text d9cument or processed by machines in an SMGL process.

After the Kiel Convention in 1989, the working group assisted
the International Phonetic Association in representing the IPA to the
\textsc{international organization for standardization} (ISO) and to the \textsc{text
encoding initiative} (TEI) \citep{EslingGaylord1993}. The working group's
formalization of the IPA, i.e.\ a full listing of agreed-upon computer codings
for phonetic symbols, was used in developing writing system descriptions, which
were at the time being solicited for scripts to be included in the new
multilingual international character encoding standards. The working group for
ISO/IEC 10646 and Unicode were two such initiatives.

In the historical context of the IPA being considered for inclusion in 
ISO/IEC 10646, it is important to realize that there were a variety of 
sources (i.e.\ not just from the Association) which submitted character 
proposals for phonetic alphabets. These proposals, including the one from the 
Association via the Kiel working group, were considered as a whole by 
the ISO working groups that were responsible for incorporating a phonetic 
script into the universal character set (UCS). The ISO working groups that 
were responsible for assigning a phonetic character set then made their 
own submissions as part of a review process by ISO for approval based on 
both informatics and phonetic criteria \citep[86]{EslingGaylord1993}. 

Character set ISO/IEC 10646 was approved by ISO, including the phonetic
characters submitted to them in May 1993. The set of IPA characters were
assigned UCS codes in 16-bit representation (in hexadecimal) and were published
as Tables 2 and 3 in \cite{EslingGaylord1993}, which include a graphical
representation of the IPA symbol, its IPA name, phonetic description, IPA
number, UCS code and AFII code.\footnote{The Association for Font
Information Interchange (AFII) was an international database of glyphs created
to promote the standardization of font data required to produce ISO/IEC 10646.} When the
character sets of ISO/IEC 10646 and the Unicode Standard later converged (see
Chapter~\ref{the-unicode-approach}), the IPA proposal was
included in the Unicode Standard Version 1.0 -- largely as we know it
today.

With subsequent revisions to the IPA, one might have expected that the Unicode
Consortium would update the Unicode Standard in a way that is in line with the
development of linguistic insights. However, updates that go against the principle
of maintaining backwards compatibility lose out, i.e.\ it is more important to
deal with the pitfalls created along the way than it is to change the standard.
Therefore, many of the pitfalls we encounter today when using Unicode IPA are
historic relics that we have to come to grips with.

% https://en.wikipedia.org/wiki/Uralic_Phonetic_Alphabet
% http://www.unicode.org/conference/bulldog.html
% http://www-01.sil.org/computing/computing_environment.html

It was a long journey, but the goal of achieving a single multilingual computing
environment has largely been accomplished. As such, we should not dismiss the IPA numbers 
or pre-Unicode encoding attempts, such as SAMPA/X-SAMPA, as misguided. The parallels 
between the IPA numbers and Unicode Code points, for example, are striking because both the IPA and 
the Unicode Consortium came up with the solution of an additional layer of indirection (an abstraction layer) 
between symbols/characters and encoding on the bit pattern level. SAMPA/X-SAMPA is also still useful 
as an input method for IPA in ASCII and required by some software.

Current users of the Unicode Standard must cope
with the pitfalls that were dug along the way, as will be discussed in the next
chapter. As the Association foresightfully remarked about Unicode:

\begin{center} 

\textit{``When this character set is in wide use, \\
it will be the normal way to encode IPA symbols.''}

\ \\

\citep[164]{IPA1999}.

\end{center}


% ==========================

\begin{comment}
\section{Unicode and ISO 10646}
% ==========================

In the late 1980s, a universal character set was being developed by what 
is now referred to as the Unicode Consortium (officially incorporated in January 1991).
This consortium 
consisted largely, although not entirely, of major US corporations, with 
the aim of overcoming the inoperability of different coded character sets 
and their costly hinderance for developing multilingual software development 
and for internationalization efforts. Commercial importance of course drove 
the early inclusion of Latin, non-Latin, and some exotic scripts; see the 
table of commercial importance as measured by GDP of countries using certain 
writing systems \citep[2]{Becker1988}.

The original Unicode manifesto is \cite{Becker1988}.\footnote{http://www.unicode.org/history/unicode88.pdf} 
Its aim was for a reliable international multilingual text encoding standard 
that would encompass all scripts of the world, or in the author's own words, 
``a new, world-wide ASCII''. An in-depth history of Unicode, highlighting 
interesting facts like its first text prototypes at Apple and its incorporation 
into TrueType, is retold online.\footnote{\url{http://www.unicode.org/history/earlyyears.html}}

Unicode 88 provided the basic principles for the Unicode Standard's design -- 
pushing for 16-bit representations of characters with a clear distinction 
between characters and glyphs. Some of the contents of this status proposal 
of 1988 were reworked for inclusion in the early Unicode Standard pre-publication 
drafts and by August 1990, the proposal was in a (very) rough draft format. Its 
editors and the Unicode Working Group (the predecessor to the Unicode Technical 
Committee) worked together to lay out the proposed standard's structure and 
content. At this time, the proposal contained no code charts nor block descriptions. 

% http://www.unicode.org/history/earlyyears.html
% During this period of time, in addition to his co-authoring of Apple KanjiTalk, Davis was involved in further discussions of a universal character set which were being prompted by the development of the Apple File Exchange.

The other major player in developing a universal character set was the ISO 
working group from the International Standards Organization (ISO), based 
in Europe, which was responsible for ISO/IEC 10646. This character set 
standard was composed in 1989 and a draft was published in 
1990.\footnote{\url{http://www.iso.org/iso/catalogue_detail.htm?csnumber=56921}} 
The \textit{Universal Multiple-Octet Coded Character Set} or simply UCS was the first 
officially standardized character encoding with the aim of including all 
characters from all writing 
systems.\footnote{\url{http://www.nada.kth.se/i18n/ucs/unicode-iso10646-oview.html}}

ISO/IEC 10646 is partly based on ISO/IEC 8859, a series of ASCII-based 
standard character encodings published in 1987 that use a single bit 8-byte 
character set. Each part of the standard, e.g.\ 8859-1, 8859-5, 8859-6, 
encodes characters to support different languages' writing systems, e.g.\ 
Latin-1 Western European, Latin/Cyrillic, Latin/Arabic, respectively. Being 
a joint effort by the International Organization for Standardization (ISO) 
and the International Electrotechnical Commission (IEC), the aim of the 
standard is reliable information exchange. So, again, issues of phonetic 
symbol encoding, typography, etc., were ignored -- or perhaps more properly 
put, not commercially driven at this early stage.

Intended for the major Western European languages, ISO/IEC 8859 was an 
extension of the ASCII character encoding standard, which included the 
English alphabet, numerals and computer control characters (e.g.\ beep, 
space, carriage return). By extending ASCII's 7-bit system to 8-bit, the 
character repertoire of each of ISO/IEC 8859 character set was doubled 
from 128 to 256 characters. Each character set defined a mapping between 
digital bit patterns and characters, which are visually rendered on screen 
as graphic symbols. ASCII was shared between ISO/IEC 8859 character sets, 
but the characters in the extra bit patterns were different. Thus an aim 
of the ISO working group responsible for ISO/IEC 10646 was to bring all 
characters in all writings systems into a single unified encoding.

In 1991, the Unicode Consortium and the ISO Working Group for ISO/IEC 10646 
decided to create a single universal standard for encoding multilingual 
text.\footnote{http://unicode.org/book/appC.pdf} 
The two character sets converged, resulting in mutually acceptable changes 
to both, and each group keeps versions of their respective character codes 
and encoding forms synchronized.\footnote{http://www.unicode.org/versions/} 
Although each standard has its own form of reference and the terminology in 
each may differ slightly, the practical difference is that the Unicode Standard 
is a formal implementation of ISO/IEC 10646 and imposes additional constraints 
on its implementation. The Unicode Standard includes character data, algorithms 
and specifications, outside the scope of ISO/IEC 10646, which ensure, when 
properly implemented in software applications and platforms, that characters 
are treated uniformly. 

The incorporation of the Unicode Standard into the international encoding 
standard ISO 10646 was approved by ISO as an International Standard in June 
1992.\footnote{http://www.unicode.org/versions/Unicode1.0.0/Notice.pdf} 
The joint Unicode and ISO/IEC 10646 standard has become \textit{the} universal 
character set and it is a single multilingual environment for the majority 
of the world's written languages. Its formal implementation has also been 
vital to the rise of a multi-lingual Internet.

\end{comment}


\chapter{IPA meets Unicode}
\label{ipa-meets-unicode}

\section{The twain shall meet}

The International Phonetic Alphabet (IPA) is a common standard in linguistics to
transcribe sounds of spoken language into discrete segments using a Latin-based
alphabet. Although IPA is reasonably easily adhered to with pen and paper, it is
not trivial to encode IPA characters electronically. In this chapter we discuss
various pitfalls with the encoding of IPA in the Unicode Standard. We will
specifically refer to the 2015 version of the IPA \citep{IPA2015} and the 11.0.0 version of Unicode \citep{Unicode2018}.

As long as a transcription is only directed towards phonetically trained eyes,
then all the details of the Unicode-encoding are unimportant. For a linguist
reading an IPA transcription, many of the details that will be discussed in this
chapter might seem like hair-splitting trivialities. However, if IPA
transcriptions are intended to be used across resources (e.g.~searching similar
phenomena across different languages) then it becomes crucial that there are
strict encoding guidelines. Our main goal in this chapter is to present the
encoding issues and propose recommendations for a strict IPA encoding for
situations in which cross-resource consistency is crucial.

There are several pitfalls to be aware of when using the Unicode Standard to
encode IPA.\@ As we have said before, from a linguistic perspective it might
sometimes look like the Unicode Consortium is making incomprehensible decisions,
but it is important to realize that the consortium has tried and is continuing
to try to be as consistent as possible across a wide range of use cases, and it
does not place linguistic traditions above other orthographic choices. Furthermore,
when we look at the history of how the IPA met Unicode, we see that many of the
decisions for IPA symbols in the Unicode Standard come directly from the
International Phonetic Association itself. Therefore, many pitfalls that we will
encounter have their grounding in the history of the principles of the IPA, as
well as in the technological considerations involved in creating a single
multilingual encoding. In general, we strongly suggest to linguists to not
complain about any decisions in the Unicode Standard, but to try and understand
the rationale of the International Phonetic Association and the Unicode
Consortium (both of which are almost always well-conceived in our experience)
and devise ways to work with any unexpected behavior.

Many of the current problems derive from the fact that the IPA is clearly
historically based on the Latin script, but different enough from most other
Latin-based writing systems to warrant special attention. This ambivalent status
of the IPA glyphs (partly Latin, partly special) is unfortunately also attested
in the treatment of IPA in the Unicode Standard. In retrospect, it might have
been better to consider the IPA (and other transcription systems) to be a
special kind of script within the Unicode Standard, and treat the obvious
similarity to Latin glyphs as a historical relic. All IPA glyphs would then have
their own code points, instead of the current situation in which some IPA glyphs
have special code points, while others are treated as being identical to the
regular Latin characters. Yet, the current situation, however unfortunate, is
unlikely to change, so as linguists we must learn to deal with the specific
pitfalls of IPA within the Unicode Standard. 

% ==========================
\section{Pitfall: No complete IPA code block}
\label{pitfall-no-complete-ipa-block}
% ==========================

The ambivalent nature of IPA glyphs arises because, on the one hand, the IPA
uses Latin-based glyphs like <a>, <b> or <p>. From this perspective, the IPA
seems to be just another orthographic tradition using Latin characters, all of
which do not get a special treatment within the Unicode Standard (just like
e.g.~the French, German, or Danish orthographic traditions do not have a special
status). On the other hand, the IPA uses many special symbols (like turned <ɐ>,
mirrored <ɘ> and/or extended <ɧ> Latin glyphs) not found in any other Latin-based
writing system. For this reason a special block with code points, called
\textsc{IPA Extensions} was already included in the first version of the Unicode
Standard (Version 1.0 from 1991).

As explained in Section~\ref{pitfall-blocks}, the Unicode Standard code space is
subdivided into character blocks, which generally encode characters from a
single script. However, as is illustrated by the IPA, characters that form a
single writing system may be dispersed across several different character
blocks. With its diverse collection of symbols from various scripts and
diacritics, the IPA is spread across 12 blocks in the Unicode
Standard:\footnote{This number of blocks depends on whether only IPA-sanctioned
symbols are counted or if the phonetic symbols commonly found in the literature
are also included, see~\cite[Appendix~C]{Moran2012}. The 159 characters from 12 
code blocks shown here are the characters proposed for strict IPA encoding, 
as discussed in Section~\ref{ipa-recommendations}.}

\begin{itemize}[itemsep=4pt]
	\item \textsc{Basic Latin }(27 characters) \newline 
	a b c d e f h i j k l m n o p q r s t u v w x y z~.~|
	\item \textsc{Latin-1 Supplement} (4 characters) \newline 
	æ ç ð ø
	\item \textsc{Latin Extended-A} (3 characters) \newline 
	ħ ŋ œ
	\item \textsc{Latin Extended-B} (4 characters) \newline 
	ǀ ǁ ǂ ǃ
	\item \textsc{Latin Extended-C} (1 character): \newline
	\charis{ⱱ}
	\item \textsc{IPA Extensions} (67 characters) \newline 
	ɐ ɑ ɒ ɓ ɔ ɕ ɖ ɗ ɘ ə ɛ ɜ ɞ ɟ ɠ ɡ ɢ ɣ ɤ ɥ ɦ ɧ ɨ ɪ ɬ ɭ ɮ ɯ ɰ ɱ ɲ ɳ ɴ \newline
	ɵ ɶ ɸ ɹ ɺ ɻ ɽ ɾ ʀ ʁ ʂ ʃ ʄ ʈ ʉ ʊ ʋ ʌ ʍ ʎ ʏ ʐ ʑ ʒ ʔ ʕ ʘ ʙ ʛ ʜ ʝ ʟ ʡ ʢ 
	\item \textsc{Greek and Coptic} (3 characters) \newline 
	β θ χ
	\item \textsc{Spacing Modifier Letters} (17 characters) \newline
	\dia{02DE} \dia{02E1} \dia{02B7} \dia{02B2} \dia{02E0} \dia{02E4} 
	\dia{02B0} \dia{02BC} \dia{02D0} \dia{02D1} ˥ ˦ ˧ ˨ ˩ {\large ˈ ˌ}
	\item \textsc{Superscripts and Subscripts} (1 character) \newline
	\dia{207F} 
	\item \textsc{Combining Diacritical Marks} (25 characters) \newline 
	\dia{033C} \dia{032A} \dia{033B} \dia{033A} \dia{031F} \dia{0320} \dia{031D} 
	\dia{031E} \dia{0318} \dia{0319} \dia{031C} \dia{0339} \dia{032C} \dia{0325} 
	\dia{0330} \dia{0324} \dia{0329} \dia{032F} \dia{0334} \dia{0303} \dia{0308} 
	\dia{033D} \dia{0306} \dia{031A} \ \dia{0361}{\large\fontspec{CharisSIL}◌}
    \item \textsc{General Punctuation} (2 characters) \newline 
    ‖ \charis{‿}
%	\item \textsc{Modifier Tone Letters} (2 characters) \newline
%	{\large\fontspec{CharisSIL}ꜛ ꜜ}
	\item \textsc{Arrows} (4 characters) \newline 
	↑ ↓ ↗ ↘

\end{itemize}

% ==========================
\section{Pitfall: IPA homoglyphs in Unicode}
\label{pitfall-ipa-homoglyphs}
% ==========================

Another problem is the large number of homoglyphs, i.e.~different characters
that have highly similar glyphs (or even completely identical glyphs, depending
on the font rendering). For example, a user of a Cyrillic computer keyboard
should ideally not use the <а> \textsc{cyrillic small letter a} at code point
\uni{0430} for IPA transcriptions, but instead use the <a> \textsc{latin small
letter a} at code point \uni{0061}, although visually they are mostly
indistinguishable, and the Cyrillic character is more easily typed on a Cyrillic
keyboard. Some further problematic homoglyphs related to encoding IPA in the
Unicode Standard are the following:

\begin{itemize}

   \item The uses of the apostrophe have led to long discussions on the Unicode
        Standard email list. An English keyboard inputs the symbol
        <\dia{0027}> \textsc{apostrophe} at \uni{0027}, although the preferred Unicode
        apostrophe is the <\dia{2019}> \textsc{right single quotation mark} at
        \uni{2019}.\footnote{Note that many word processors (like Microsoft
        Word) by default will replace straight quotes by curly quotes,
        depending on the whitespace around it.} However, the glottal
        stop/glottalization/ejective marker is yet another completely different
        character, namely <\dia{02BC}>, i.e.~the \textsc{modifier letter apostrophe} 
        at \uni{02BC}, which unfortunately looks extremely similar to
        \uni{2019}. 
  \item Another problem is the <\dia{02C1}> \textsc{modifier letter reversed glottal
        stop} at \uni{02C1} vs.\@ the <\dia{02E4}> \textsc{modifier letter small reversed
        glottal stop} at \uni{02E4}. Both appear in various resources
        representing phonetic data online. This is thus a clear example for
        which the Unicode Standard does not solve the linguistic standardization
        problem.
  \item Linguists are also unlikely to distinguish between the <ə>
        \textsc{latin small letter schwa} at code point \uni{0259} and <ǝ>
        \textsc{latin small letter turned e} at \uni{01DD}.
  \item The alveolar click <ǃ> at \uni{01C3} is of course often simply
        typed as <!> \textsc{exclamation mark} at \uni{0021}.\footnote{In the
        Unicode Standard the <ǃ> at \unif{01C3} is labeled \textsc{latin letter
        retroflex click}, but in IPA that glyph is used for an alveolar or
        postalveolar click (not retroflex). This naming is probably best seen as
        an error in the Unicode Standard. For the real retroflex click, see 
        Section~\ref{ipa-additions}.}
  \item The dental click <ǀ>, in Unicode known as \textsc{latin letter dental
        click} at \uni{01C0}, is often simply typed as <|> \textsc{vertical
        line} at \uni{007C}.
  \item For the marking of length there is a special Unicode character, namely
        <\dia{02D0}> \textsc{modifier letter triangular colon} at \uni{02D0}. However,
        typing <\dia{003A}> \textsc{colon} at \uni{003A} is of course much easier.        
  \item There are two mostly identical-looking Unicode characters for the superscript
        <ʰ>: the \textsc{combining latin small letter h} at \uni{036A} and the
        \textsc{modifier letter small h} at \uni{02B0}. Making the situation 
        even more problematic is that they have different behavior (see 
        Section~\ref{pitfall-different-notions-of-diacritics}). To harmonize the 
        behavior of <ʰ> with other superscript letters, we propose to 
        standardize on the modifier letter at \uni{02B0} (see 
        Section~\ref{pitfall-no-unique-diacritic-ordering}).
  
\end{itemize} 

Conversely, non-linguists are unlikely to distinguish any semantic difference
between an open back unrounded vowel, which is encoded in IPA with a
``single-story'' <ɑ> \textsc{latin small letter alpha} at \uni{0251}, and the open
front unrounded vowel, which is encoded in IPA with a ``double-story'' <a>
\textsc{latin small letter a} at \uni{0061}, basically treating them as
homoglyphs, although they are different phonetic symbols. But even among
linguists this distinction leads to problems. For example, as pointed out by
\citet{Mielke2009}, there is a problem stemming from the fact that about 75\% of
languages are reported to have a five-vowel system \citep{Maddieson1984}.
Historically, linguistic descriptions tend not to include precise audio
recording and measurements of formants, so this may lead one to ask if the many
<a> characters that are used in phonological description reflects a
transcriptional bias. The common use of <a> in transcriptions could be in part
due to the ease of typing the letter on an English keyboard (or for older
descriptions, the typewriter). We found it to be exceedingly rare that a
linguist uses <ɑ> for a low back unrounded vowel.\footnote{One example is
\citet[75]{Vidal2001}, in which the author states: ``The definition of Pilagá
/a/ as [+back] results from its behavior in certain phonological contexts. For
instance, uvular and pharyngeal consonants only occur around /a/ and /o/. Hence,
the characterization of /a/ and /o/ as a natural class of (i.e.\ [+back]
vowels), as opposed to /i/ and /e/.''} They simply use <a> as long as there is
no opposition to <ɑ>.

%\footnote{See Thomason's Language Log post, ``Why I don't love the International Phonetic Alphabet'' at:\url{http://itre.cis.upenn.edu/~myl/languagelog/archives/005287.html}.}

Making things even more problematic, there is an old typographic tradition that
the double-story <a> uses a single-story <ɑ> in italics. This leads to the
unfortunate effect that even in many well-designed fonts the italics of <a> and
<ɑ> use the same glyph. For example, in Linux Libertine (the font of this book)
the italics of these characters are highly similar, <\textit{a}> and
<\textit{ɑ}>, while in Charis SIL they are identical: <\textit{\charis{a}}> and
<\textit{\charis{ɑ}}>. If this distinction has to be kept upright in italics,
the only solution we can currently offer is to use \textsc{slanted} glyphs
(i.e.~artificially italicized glyphs) instead of real italics (i.e.~special
italics glyphs designed by a typographer).\footnote{For example, the widely used
IPA font Doulos SIL
(\url{https://software.sil.org/doulos/}) does not
have real italics. This leads some word-processing software, like Microsoft
Word, to produce slanted glyphs instead. That particular combination of font and
software application will thus lead to the desired effect distinguishing <a>
from <ɑ> in italics. However, note that when the text is transferred to another
font (i.e.~one that includes real italics) and/or to another software
application (like Apple Pages, which does not perform slanting), then this
visual appearance will be lost. In this case we are thus still in the
pre-Unicode situation in which the choice of font and rendering software
actually matters. The ideal solution from a linguistic point of view would be
the introduction of a new IPA code point for a different kind of <a>, which
explicitly specifies that it should still be rendered as a double-story
character when italicized. After informal discussion with various Unicode
players, our impression is that this highly restricted problem is not
sufficiently urgent to introduce even more <a> homographs in Unicode (which
already lead to much confusion, see Section~\ref{pitfall-homoglyphs}).} This 
approach was taken by the Language Science Press to distinguish 
between the two different orthographic <a>'s in Chakali in 
\cite{Brindle2017}.\footnote{\url{http://langsci-press.org/catalog/book/74}}

Lastly, before we move on from the pitfall of IPA homoglyphs in Unicode to the pitfall of 
homoglyphs in IPA, we are aware of one example that illustrates both pitfalls. 
Consider for example what one reviewer coined \textsc{i dot-suppression}. 
When combining, say the \textsc{latin small letter i} <i> at \uni{0069} with 
the \textsc{combining acute accent} <\dia{0301}> at \uni{0301}, the result is 
 the combination of these two characters into <í> or the associated 
precomposed form \textsc{latin small letter i with acute} <í> at \uni{00ED}. 
Typographically, the accent mark replaces the dot. In IPA, the <í> denotes 
a high front unrounded vowel with high tone. However, the result of losing 
the dot makes this IPA symbol look very similar to the near-high near-front 
unrounded vowel <ɪ>, when it also has the high tone marker: <ɪ́>. To boot, 
when an accent mark is added to \textsc{latin small letter i with stroke} <ɨ> at 
\uni{0268}, the dot is not suppressed but retained, i.e.\ <ɨ́>.\footnote{According 
to the Unicode Standard, \textsc{latin small letter i with stroke} <ɨ> at 
U+0268 cannot be decomposed into, say, \textsc{latin small letter i} <i> at 
U+0069 and \textsc{combining short stroke overlay} <\dia{0335}> at U+0335. 
We discuss the pitfall of missing decomposition forms 
in Section \ref{pitfall-missing-decomposition}.}


% ==========================
\section{Pitfall: Homoglyphs in IPA}
\label{pitfall-homoglyphs-in-IPA}
% ==========================

Reversely, there are a few cases in which the IPA distinguishes different
phonetic concepts, but the visual characters used by the IPA look very much
alike. Such cases are thus homoglyphs in the IPA itself, which of course need
different encodings.

\begin{itemize}
  
   \item The dental click <ǀ> and the indication of a minor group break <|>
           look almost the same in
           most fonts. For a proper encoding, the \textsc{latin letter dental
           click} at \uni{01C0} and the \textsc{vertical line }at \uni{007C} 
           should be used, respectively.
   \item Similarly, the alveolar lateral click <ǁ>	should be encoded with a
           \textsc{latin letter lateral click} at \uni{01C1}, different from <‖>, 
           which according to the IPA is the character to by used for a major group 
           break (by intonation), to be encoded by \textsc{double vertical line} 
           at \uni{2016}.
   \item The marking of primary stress < ˈ > looks like an apostrophe, and
           is often typed with the same symbol as the ejective <\dia{02BC}>. For a
           proper encoding, these two symbols should be typed as 
           \textsc{modifier letter vertical line} at \uni{02C8} and
           \textsc{modifier letter apostrophe} at \uni{02BC}, respectively. 
   \item There are two different dashed-l characters in IPA, namely the <ɫ> 	      \textsc{latin small letter l with middle tilde} at \uni{026B} and the <ɬ>	      \textsc{latin small letter l with belt} at \uni{026C}. These of course
          look highly similar, although they are different sounds. As a solution, 
          we will actually propose to not use the middle tilde at all 
          (see Section~\ref{pitfall-multiple-options-ipa}).     
   
\end{itemize}

% ==========================
\section{Pitfall: Multiple encoding options in IPA}
\label{pitfall-multiple-options-ipa}     
% ========================== 

It is not just the Unicode Standard that offers multiple options for encoding
the IPA.\@ Even the IPA specification itself offers some flexibility in how
transcriptions have to be encoded. There are a few cases in which the IPA
explicitly allows for different options of transcribing the same phonetic
content. This is understandable from a transcriber's point of view, but it is
not acceptable when the goal is interoperability between resources written in
IPA.\@ We consider it crucial to distinguish between valid IPA, for which it
is sufficient that any phonetically-trained reader is able to understand the
transcription, and strict IPA, which should be standardized on a single
unique encoding for each sound, so search will work across resources. We are
aware of the following non-unique encoding options in the IPA, which will be
discussed in turn below:

\begin{itemize}
  \item The marking of tone
  \item The marking of <g>
  \item The marking of velarization and pharyngealization
  \item The placement of diacritics
\end{itemize}

\noindent The first case in which the IPA allows for different encodings is the question
of how to transcribe tone (cf.\ \cite{Maddieson1990}). There is an old tradition 
to use diacritics on vowels to mark different tone levels \citep{IPA1949}. Prior to the 1989 Kiel 
convention, IPA-approved tone symbols included diacritics above or below the vowel or syllable, 
e.g.\ high and low tones marked with macrons (<\diaf{0304}>, <\diaf{0331}>), and acute and grave 
accents for high rising tone <\diaf{0301}>, low rising tone <\diaf{0317}>, high falling tone 
<\diaf{0300}> and low falling tone <\diaf{0316}>. These tone symbols, however,  
had failed to catch on (probably) due to aesthetic objections and matters of adequacy for 
transcription \citep[29]{Maddieson1990}.

After the 1989 Kiel convention, the accent tone symbols were updated to the tradition 
that we are familiar with today and which was already in wide use by Africanists and others, 
namely level tones <\diaf{030B}, \diaf{0301}, \diaf{0304}, \diaf{0300}, \diaf{030F}> 
and contour tones <\diaf{030C}, \diaf{0302}, \diaf{1DC4}, \diaf{1DC5}, \diaf{1DC8}>.\footnote{To make things even more
complicated, there are at least two different Unicode homoglyphs for the low and
high level tones, namely <\diaf{0340}> \textsc{combining grave tone mark} at
\unif{0340} vs.~<\diaf{0300}> \textsc{combining grave accent} at \unif{0300} for
low tone, and <\diaf{0341}> \textsc{combining acute tone mark} at \unif{0341}
vs.<\diaf{0301}> \textsc{combining acute accent} at \unif{0301} for high tone.} 
In addition, the IPA also adopted tone letters developed by \citet{Chao1930}, e.g.\ <˥˦˧˨˩>, 
which were in wide use by Asianists.\footnote{Not sanctioned
by the IPA, but nevertheless widely attested, is the use of superscript
numbers for marking tones, also proposed by \citet{Chao1930}. One issue to note here is 
that superscript numbers can be either regular numbers that are formatted as 
superscript with a text processor, or they can be separate superscript characters, 
as defined in the Unicode Standard (see: \url{https://en.wikipedia.org/wiki/Superscripts_and_Subscripts}). 
This divide means that searching text is dependent 
on how the author formatted or encoded the superscript numbers.} 
Tone letters in the IPA have five different levels,
and sequences of these letters can be used to indicate contours. Well-designed
fonts will even merge a sequence of tone letters into a contour. For example,
compare the font Linux Libertine, which does not merge tone letters
<{\fontspec{LinLibertineO}˥˨˧˩}>, with the font CharisSIL, which merges this
sequence of four tone letters into a single contour <\charis{˥˨˧˩}>. For strict
IPA encoding we propose to standardize on tone letters.

% IPA1999 pg 19
% "Either of the variant letter shapes [g] and [g] may be used to represent the voiced velar plosive."
% https://www.cambridge.org/core/journals/journal-of-the-international-phonetic-association/information/instructions-contributors

Second, we commonly encounter the use of <g> \textsc{latin small letter g} at
\uni{0067}, instead of the Unicode Standard IPA character for the voiced velar
stop <ɡ> \textsc{latin small letter script g} at \uni{0261}. One begins to
question whether this issue is at all apparent to the working linguist, or if
they simply use the \uni{0067} because it is easily keyboarded and thus saves
time, whereas the latter must be cumbersomely inserted as a special symbol in
most software. The International Phonetic Association has taken the stance that
both the keyboard \textsc{latin small letter g} and the \textsc{latin small
letter script g} are valid input characters for the voiced velar plosive 
\citep[19]{IPA1999}.\footnote{Note however that the current instructions for contributors to 
the Journal of the International Phonetic Association requires the use of 
opentail <ɡ> and not looptail <g>.}
Unfortunately, this decision further introduces ambiguity for linguists trying
to adhere to a strict Unicode Standard IPA encoding. For strict IPA encoding we
propose to standardize on the more idiosyncratic \textsc{latin small letter
script g} at \uni{0261}.

Third, the IPA has special markers for velarization <\dia{02E0}> and
pharyngealization <\dia{02E4}>. Confusingly, there is also a marker for
``velarized or pharyngealized'', using the <\dia{0334}> \textsc{combining tilde
overlay} at \uni{0334}. The tilde overlay seems to be extremely rarely used. We 
suggest to try and avoid using the tilde overlay, though for reasons of backward 
compatibility we will allow it in valid-IPA.\@

Finally, the IPA states that ``diacritics may be placed above a symbol with a
descender''. For example, for marking of voiceless pronunciation of
voiced segments the IPA uses the ring diacritic. Originally, the ring should be
placed below the base character, like in <m̥>, using the \textsc{combining ring
below} at \uni{0325}. However, in letters with long descenders the IPA also
allows to put the ring above the base, like in <ŋ̊>, using the \textsc{combining
ring above} at \uni{030A}. Yet, proper font design does not have any problem
with rendering the ring below the base character, like in <ŋ̥>, so for strict
IPA encoding we propose to standardize on the ring below. As a principle, for
strict IPA encoding only one option should be allowed for all diacritics.

The variable encoding as allowed by the IPA becomes even more troublesome for
the tilde and diaeresis diacritics. In these cases, the IPA itself attaches
different semantics to the symbols above and below a base characters. The tilde
above a character (like in <ã>, using the \textsc{combining tilde} at
\uni{0303}) is used for nasalization, while the tilde below a character (like in
<a̰>, using the \textsc{combining tilde below} at \uni{0330}) indicates creaky
voice. Likewise, the diaeresis above (like in <ä>, using the \textsc{combining
diaeresis} at \uni{0308}) is used for centralization, while the diaeresis below
a character (like in <a̤>, using the \textsc{combining diaeresis below} at
\uni{0324}) indicates breathy voice. These cases strengthen our plea to not
allow diacritics to switch position for typographic convenience.

% length IPA 1999:22
% "Note: as in Finnish orthography, length can also be indicated in phonetic
% transcription by double letters: e.g . Finnish maatto [rncctto] 'electrical
% earth/ground'."

% syllable breaks and word boundaries <\s, ., |, ||, tie-bar below>
% White spaces can be used to indicate word boundaries. Syllable breaks can be
% marked when required. The other two boundary symbols are used to mark the domain
% of larger prosodic units. There is also a linking symbol that can be used for
% explicitly indicating the lack of a boundary.

% (In all these transcriptions, no spaces between words have been included. This
% is inevitable in an impressionistic transcription where it is not yet known how
% the utterance divides into words. In phonemic and allophonic transcriptions it
% is common to include spaces to aid legibility, but their theoretical validity is
% problematic.)


% ==========================
\section{Pitfall: Tie bar}
\label{pitfall-tie-bar}
% ==========================

% Wells1995:10 "The underscore could in principle also be pressed into service
% to represent the IPA tie bar. The current chart mentions its use only for
% affricates and double articulations, and then only "if necessary"."

In the major revision of the IPA in 1932, affricates were represented by two
consonants <tʃ>, ligatures <ʧ>, or with the tie bar <t͡ʃ>. In the
1938 revision the tie bar's semantics were broadened to indicate simultaneous
articulation, as for example in labial velars such as <k͡p>. Thus, the tie bar is a
convenient diacritic for visually tokenizing input strings into chunks of
phonetically salient groups, including affricates, doubly articulated consonants
or diphthongs. 

The tie bar can be placed above or below the base characters, e.g.
<\charis{t͡s}> or <\charis{t͜s}>. IPA allows both options. The choice between
the two symbols is purely for legible rendering; there is no difference in
semantics between the two symbols. However, rendering is such a problematic
issue for tie bars in general that many linguists simply do not use them. Just
looking at a few different fonts already indicates that actually no font
designer really gets the placement right in combination with superscripts and
subscripts. If really necessary, we propose to standardize on the tie bar above
the base characters, using a \textsc{combining double inverted breve} at
\uni{0361}.\footnote{Also note that the \textsc{undertie} at \unif{203F} looks
like the tie bar below and is easily confused with it. However, it is a
different character and has a different function in IPA. The undertie is used as
a linking symbol to indicate the lack of a boundary, e.g.\ French \textit{petit
ami} [pətit\charis{‿}ami] `boyfriend'.}

\begin{itemize}[itemsep=6pt]
  \item[] {\fontspec{Times New Roman}Times new Roman: t̥ʰ͡s t̥ʰ͜s}
  \item[] {\small \fontspec{CharisSIL}CharisSIL:\@ t̥ʰ͡s t̥ʰ͜s}
  \item[] {\footnotesize \fontspec{Monaco}Monaco: t̥ʰ͡s t̥ʰ͜s}
  \item[] {\fontspec{DoulosSIL}DoulosSIL:\@ t̥ʰ͡s t̥ʰ͜s}
  \item[] Linux Libertine: t̥ʰ͡s t̥ʰ͜s
\end{itemize}

Tie bars are a special type of character in the sense that they do not belong to
a segment, but bind two graphemes together. This actually turns out to be rather
different from Unicode conceptions. The Unicode encoding of this character
belong to the Combining Diacritical Marks, namely either \textsc{combining double
inverted breve} at \uni{0361} or \textsc{combining double breve below} at
\uni{035C}. Such a combining mark is by definition tied to the character in
front, but not the character following it. The Unicode treatment of this
character thus only partly corresponds to the IPA conception, which ideally
would have the tie bar linked both to the character in front and to the
character following. 

Further, according to the spirit of the IPA, it would also be possible to
combine more than two base characters into one tie bar, but this is not possible
with Unicode (i.e.~there is no possibility to draw a tie bar over three or four
characters). It is possible to indicate such larger groups by repeating the tie
bar, like for a triphthong <a͡ʊ͡ə> in the English word \textit{hour}. If really
necessary, we consider this possible, even though the rendering will never look
good. 

Most importantly though, in comparison to normal Unicode processing, the tie bar
actually takes a reversed approach to complex graphemes. Basically, the Unicode
principle (see Section~\ref{pitfall-characters-are-not-graphemes}) is that fixed
sequences in a writing system have to be specified as tailored grapheme
clusters. Only in case the sequence is not a cluster, then this has to be
explicitly indicated. IPA takes a different approach. In IPA by default
different base letters are not connected into larger clusters; only when it is
specified in the string itself (using the tie bar).

% ==========================
\section{Pitfall: Ligatures and digraphs}
\label{pitfall-ligatures-digraphs}     
% ==========================   

% TODO: link back to discussion on IPA principles and how ligatures got brought
% along historically

One important distinction to acknowledge is the difference between multigraphs
and ligatures. Multigraphs are groups of characters (in the context of IPA e.g.
<tʃ> or <ou>) while ligatures are single characters (e.g.\ <ʧ> \textsc{latin
small letter tesh digraph} at \uni{02A7}). Ligatures arose in the context of
printing easier-to-read texts, and are included in the Unicode Standard for
reasons of legacy encoding. However, their usage is discouraged by the Unicode
core specification. Specifically related to IPA, various phonetic combinations
of characters (typically affricates) are available as single code points in the
Unicode Standard, but are designated \textsc{digraphs}. Such glyphs might be used by
software to produce a pleasing display, but they should not be hard-coded into
the text itself. In the context of IPA, characters like the following ligatures
should thus \emph{not} be used. Instead a combination of two characters is
preferred:
      
\begin{itemize} 
	\item[] <ʣ> \textsc{latin small letter dz digraph} at \uni{02A3} 
	  (use <dz>) 
    \item[] <ʤ> \textsc{latin small letter dezh digraph} at \uni{02A4}
      (use <dʒ>)
    \item[] <ʥ> \textsc{latin small letter dz digraph with curl} at \uni{02A5}
      (use <dʑ>)
    \item[] <ʦ> \textsc{latin small letter ts digraph} at \uni{02A6} 
      (use <ts>)
	\item[] <ʧ> \textsc{latin small letter tesh digraph} at \uni{02A7} 
	  (use <tʃ>) 
    \item[] <ʨ> \textsc{latin small letter tc digraph with curl} at \uni{02A8}
      (use <tɕ>)
   	\item[] <ʩ> \textsc{latin small letter feng digraph} at \uni{02A9}
	  (use <fŋ>) 
\end{itemize}

However, there are a few Unicode characters that are historically ligatures, but
which are today considered as simple characters in the Unicode Standard and thus
should be used when writing IPA, namely:

\begin{itemize}
	\item[] <ɮ> \textsc{latin small letter lezh} at \uni{026E} 
	\item[] <œ> \textsc{latin small ligature oe} at \uni{0153} 
	\item[] <ɶ> \textsc{latin letter small capital oe} at \uni{0276} 
	\item[] <æ> \textsc{latin small letter ae} at \uni{00E6} 
\end{itemize}

% ==========================
\section{Pitfall: Missing decomposition}
\label{pitfall-missing-decomposition}
% ==========================

Although many combinations of base character with diacritic are treated as
 with precomposed characters, there are a few combinations
in IPA that allow for multiple, apparently identical, encodings that are not
 (see Section~\ref{pitfall-canonical-equivalence} on the
principle of canonical equivalence). For that reason, the following elements
should not be treated as diacritics when encoding IPA in Unicode:
\begin{itemize}
  \item[] <{\fontspec{CharisSIL}{\large ◌}}\symbol{"0321}> \textsc{combining palatalized hook below} at \uni{0321}
  \item[] <{\fontspec{CharisSIL}{\large ◌}}\symbol{"0322}> \textsc{combining retroflex hook below} at \uni{0322}
  \item[] <\dia{0335}> \textsc{combining short stroke overlay} at \uni{0335}
  \item[] <\dia{0337}> \textsc{combining short solidus overlay} at \uni{0337}
\end{itemize} 

There turn out to be a lot of characters in the IPA that could be conceived as
using any of these elements, like <ɲ>, <ɳ>, <ɨ> or <ø>. However, all such
characters exist as well as precomposed combination in Unicode, and these
precomposed characters should preferably be used. When combinations of a base character with
diacritic are used instead, then these combinations are not  to the
precomposed combinations. This means that any search will not find both at the
same time.

% \footnote{The IPA does not
% describe any character for a voiced retroflex implosive, which would
% transparently be \charis{ᶑ}. We propose to add this character to the IPA, see
% Section~\ref{ipa-additions}.} 

A similar problem arises with the rhotic hook. There are two precomposed
characters in Unicode with a rhotic hook, which are not  
with a combination of the vowel with a separately encoded hook:
\begin{itemize}
  \item[] <ɚ> \textsc{latin small letter schwa with hook} at \uni{025A}
  \item[] <ɝ> \textsc{latin small letter reversed open e with hook} at \uni{025D}
\end{itemize}
All other combinations of vowels with rhotic hooks will have to be made by using
<\dia{02DE}> \textsc{modifier letter rhotic hook} at \uni{02DE}, because there
is no complete set of precomposed characters with rhotic hooks in Unicode. For
that reason we propose to not use the two precomposed characters with hooks
mentioned above, but always use the separate rhotic hook at \uni{02DE} in IPA.\@

A similar situation arises with <\dia{0334}> \textsc{combining tilde overlay} at
\uni{0334}. The main reason some phoneticians like to use this in IPA is to mark
the dark <l> in English codas, using the character <ɫ> \textsc{latin small
letter l with middle tilde} at \uni{026B}. This character is not canonically
equivalent to the combination <l>~+~<\dia{0334}>, so one of the two possible
encodings has to be chosen. Because the tilde overlay is described as a general
mechanism by the IPA, we propose to use the separated <\dia{0334}>
\textsc{combining tilde overlay} at \uni{0334}. However, note that phonetically 
this seems to be (almost) superfluous (see Section~\ref{pitfall-multiple-options-ipa}) 
and the typical usage in the form of <ɫ> is (almost) a homoglyph with <ɬ> (see 
Section~\ref{pitfall-homoglyphs-in-IPA}). For these reasons we also suggest to try 
and avoid the tilde overlay completely.

Reversely, note that the <ç> \textsc{latin small letter c with cedilla} at
\uni{00E7} is  with <c> with <\dia{0327}>
\textsc{combining cedilla} at \uni{0327}, so it will be separated into two
characters by Unicode canonical decomposition, also if such a decomposition is
not intended in the IPA.\@ However, because of the nature of canonical
equivalence (see Section~\ref{pitfall-canonical-equivalence}), these two
encodings are completely identical in any computational treatment, so this
decomposition does not have any practical consequences.

% ==========================
\section{Pitfall: Different notions of diacritics}
\label{pitfall-different-notions-of-diacritics}
% ==========================

% TODO: this section is wrong wrt IPA diacritics and needs updating @SM

Another pitfall relates to the question, what is a diacritic? The problem is that
the meaning of the term \textsc{diacritic} as used by the IPA is not the same as it is used
in the Unicode Standard. Specifically, diacritics in the IPA-sense are either
so-called \textsc{combining diacritical marks} or \textsc{spacing modifier
letters} in the Unicode Standard. Crucially, Combining Diacritical Marks are by
definition combined with the character before them (to form so-called default
grapheme clusters, see Chapter~\ref{the-unicode-approach}). In contrast, Spacing
Modifier Letters are by definition \emph{not} combined into grapheme clusters
with the preceding character, but simply treated as separate letters. In the
context of the IPA, the following IPA-diacritics are actually Spacing Modifier
Letters in the Unicode Standard:

\begin{itemize}
  
	\item[] Length marks, namely: 
    	\begin{itemize}
    	  \item[] <\dia{02D0}> \textsc{modifier letter triangular colon} at \uni{02D0}
    	  \item[] <\dia{02D1}> \textsc{modifier letter half triangular colon} at \uni{02D1}
    	\end{itemize}
	 
	\item[] Tone letters, including but not limited to: 
	\begin{itemize} 
	  \item[] <˥> \textsc{modifier letter extra-high tone bar} at \uni{02E5}
	  \item[] <˨> \textsc{modifier letter low tone bar} at \uni{02E8}
	\end{itemize}
	
	\item[] Superscript letters, including but not limited to:\footnote{The Unicode Standard defines the 
	  \textit{Phonetic Extensions} block that defines symbols used in phonetic notation 
	  systems, including the Uralic Phonetic Alphabet, Americanist and Russianist phonetic notations, 
	  Oxford English and American dictionaries, etc. Among other symbols, the 
	  \textit{Phonetic Extensions} block includes 
	  the superscript letters <\textsuperscript{m, ŋ, b}>, which are not valid 
	  IPA characters, although we have seen them used in linguistic practice.}
	\begin{itemize}
	  \item[] <\dia{02B0}> \textsc{modifier letter small h} at \uni{02B0}
	  \item[] <\dia{02E4}> \textsc{modifier letter small reversed glottal stop} at \uni{02E4}
	  \item[] <\dia{207F}> \textsc{superscript latin small letter n} at \uni{207F}
	\end{itemize}
	
	\item[] The rhotic hook:\footnote{It is really unfortunate that the rhotic hook
         in Unicode is classified as a Spacing Modifier, and not as a Combining 
         Diacritical Mark. Although the rhotic hook is placed to the right of its 
         base character (and not above or below), it still is always connected 
         to the character in front, even physically connected to it. We cannot 
         find any reason for this treatment, and consider it an error in 
         Unicode. We hope it will be possible to change this classification in 
         the future.}
	\begin{itemize}
	  \item[] <\dia{02DE}> \textsc{modifier letter rhotic hook} at \uni{02DE}
	\end{itemize}
	
\end{itemize}

Although linguists might expect these characters to belong together with the
character in front of them, at least for tone letters, stress symbols and <ʰ>
\textsc{modifier letter small h} at \uni{02B0} the Unicode Consortium's decision
to treat it as a separate character is also linguistically correct.

\begin{itemize}
  
   \item According to the IPA, <ʰ> can be used both as <\dia{02B0}> for
         post-aspiration (following the base character) and as
         <\diareverse{02B0}> for pre-aspiration (preceding the base
         character), so there is no consistent direction in which this
         diacritic should bind. Note that there is yet another homoglyph,
         namely the \textsc{combining latin small letter h} at \uni{036A}.
         We propose not to use this combining diacritical mark, but to
         standardize on Unicode modifier letters for all superscript
         letters in IPA.
  
   \item Tone letters <˥, ˦, ˧, ˨, ˩> from \uni{02E5}--\uni{02E9} 
   		 are normally written at the end of the syllable,
         possibly occurring immediately adjacent to a consonant in the coda of
         the syllable. Such tone markers should of course not be treated as
         belonging to this consonant, so we propose to treat tone letters as 
         separate segments.
 
   \item Stress markers <\diareverse{02C8}> at \uni{02C8} and
         <\diareverse{02CC}> at \uni{02CC} have a very similar
         distribution in that they normally are written at the start of
         the stressed syllable. In a sense, they thus belong to the
         characters following the stress marker, but it would be wrong to
         cluster them together with whatever segment is at the start of
         the syllable. So, like tone letters, we propose to treat stress
         markers as separate segments.
 
\end{itemize}

\noindent If intended, then any default combination of Spacing Modifiers
with the preceding character can be specified in orthography profiles (see
Chapter~\ref{orthography-profiles}). 

% ==========================
\section{Pitfall: No unique diacritic ordering}
\label{pitfall-no-unique-diacritic-ordering}
% ==========================

Also related to diacritics is the question of ordering. To our knowledge, the
International Phonetic Association does not specify an ordering for
diacritics that combine with phonetic base symbols; this exercise is left to the
reasoning of the transcriber. However, such marks have to be explicitly ordered
if sequences of them are to be interoperable and compatible computationally. An example is a
labialized aspirated alveolar plosive: <tʷʰ>. There is nothing holding linguists
back from using <tʰʷ> instead (with exactly the same intended meaning). However,
from a technical standpoint, these two sequences are different; if both
sequences are used in a document, searching for <tʷʰ> will not find any
instances of <tʰʷ>, and vice versa. Likewise, a creaky voiced syllabic dental
nasal can be encoded in various orders, e.g.\ <n̪̰̩>, <n̩̰̪> or <n̩̪̰>.

\subsubsection*{Canonical combining classes}

In accordance with the absence of any specification of ordering in the IPA, the
Unicode Standard likewise does not propose any standardized orders. Both leave it
to the user to be consistent; this approach naturally invites inconsistency across 
different authored resources.

There is one (minor) aspect of ordering for which the Unicode Standard does
present a canonical solution. Fortunately, this is uncontroversial from a
linguistic perspective. Diacritics in the Unicode Standard (i.e.~Combining
Diacritical Marks, see Section~\ref{pitfall-different-notions-of-diacritics})
are classified in so-called \textsc{canonical combining classes}. In practice,
the diacritics are distinguished by their position relative to the base
character.\footnote{For a detailed description, see: 
\url{http://unicode.org/reports/tr44/\#Canonical\_Combining\_Class\_Values}.} 
When applying a Unicode normalization (NFC or NFD, see
Section~\ref{pitfall-canonical-equivalence}), the diacritics in different
positions are put in a specified order. This process therefore harmonizes the
difference between different encodings in some situations, for example in the
case of an extra-short creaky voice vowel <ḛ̆>. This grapheme cluster can be
encoded either as <e>+<\dia{0306}>+<\dia{0330}> or as
<e>+<\dia{0330}>+<\dia{0306}>. To prevent this twofold encoding, the Unicode
Standard specifies the second ordering as canonical (namely, diacritics
below are put before diacritics above).

% The next paragraph does not belong here - we should put it somewhere else

When encoding a string according to the Unicode Standard, it is possible to do
this either using the NFC (composition) or NFD (decomposition) normalization (see
Section~\ref{pitfall-canonical-equivalence}).
Decomposition implies that precomposed characters (like <á> \textsc{latin small
letter a with acute} at \uni{00E1}) will be split into its parts. This might
sound preferable for a linguistic analysis, as the different diacritics are
separated from the base characters. However, note that most attached elements
like strokes (e.g.~in the <ɨ>), retroflex hooks (e.g.~in <ʐ>) or rhotic hooks
(e.g.~in <ɝ>) will not be decomposed. In
general, Unicode decomposition does not behave like a feature decomposition as
expected from a linguistic perspective. It is thus important to consider Unicode
decomposition only as a technical procedure, and not assume that it is
linguistically sensible.

\subsubsection*{Proposal for diacritic ordering}

Facing the problem of specifying a consistent ordering of diacritics while
developing a large database of phonological inventories from the world's
languages, \citet[540]{Moran2012} defined a set of diacritic ordering
conventions. The conventions are influenced by the linguistic literature, though
some ad-hoc decisions had to be taken given the vast variability of phonological
segments described by linguists. The most recent version of the conventions 
is published online by~\citet{MoranMcCloy2014}.\footnote{\url{http://phoible.github.io/conventions/}}

According to Unicode Canonical Combining Classes, overlay diacritics 
like <\dia{0334}> (Combining Class number 1) always come before diacritics
below (Combining Class number 220), which in turn always come before diacritics
above (Combining Class number 230), which in turn come before diacritics over
multiple characters like the tie bar <\dia{0361}{\large\fontspec{CharisSIL}◌}>
(Combining Class number 233). We follow this order, but add the other IPA
diacritics (which are not diacritics in the Unicode sense) between diacritics
below and the tie bar. Further, \textit{within} all these classes of diacritics
there is no canonical ordering specified by Unicode, so we propose an explicit
ordering here.

Starting with the diacritics below: if a character sequence contains more than
one diacritic below the base character, then the place features are applied
first (linguolabial, dental, apical, laminal, advanced, retracted), followed by
the manner features (raised, lowered, advanced and retracted tongue root), then
secondary articulations (more round, less round), laryngeal settings (creaky,
breathy, voiced, devoiced), and finally the syllabic or non-syllabic marker. So,
the order that is proposed is the following, where <\textbar{}> indicates
\textit{or} and <→> indicates \textit{precedes}. Note that the groups of
alternatives (as marked by <\textbar{}>) are supposed never to occur together
with the same base character. In effect, this represents yet another restriction
on possible diacritic sequences.

\begin{itemize}
	\item[] \textsc{Combining Diacritical Marks (below) ordering:}
	\begin{itemize}	
	  \item[→] linguolabial <\dia{033C}> \textbar{} dental <\dia{032A}> \textbar{} apical <\dia{033A}> \textbar{} laminal <\dia{033B}>
	  \item[→] advanced <\dia{031F}> \textbar{} retracted <\dia{0320}> 
	  \item[→] raised <\dia{031D}> \textbar{} lowered <\dia{031E}>
	  \item[→] advanced tongue root <\dia{0318}> \textbar{} retracted tongue root <\dia{0319}>
	  \item[→] more rounded <\dia{0339}> \textbar{} less rounded <\dia{031C}>
	  \item[→] creaky voiced <\dia{0330}> \textbar{} breathy voiced <\dia{0324}> \textbar{} voiced <\dia{032C}> \textbar{} voiceless <\dia{0325}>
	  \item[→] syllabic <\dia{0329}> \textbar{} non-syllabic <\dia{032F}>
	\end{itemize}
 \end{itemize}

\noindent Next, if a character sequence contains more than one diacritic above the base
character, we propose the following order:

\begin{itemize}
	\item[] \textsc{Combining Diacritical Marks (above) ordering:}
	\begin{itemize}
	  \item[→] nasalized <\dia{0303}>
	  \item[→] centralized <\dia{0308}> \textbar{} mid-centralized <\dia{033D}>
	  \item[→] extra short <\dia{0306}>
	  \item[→] no audible release <\dia{031A}\ >
 \end{itemize} \end{itemize}

\noindent Then, when a character sequence contains more than one character of the Spacing
Modifier Letters, these will be placed after all combining diacritical marks in the
following order:

\begin{itemize}
	\item[] \textsc{Spacing Modifier Letters ordering:}
	\begin{itemize}
	  \item[→] rhotic hook <\dia{02DE}>
	  \item[→] lateral release <\dia{02E1}> \textbar{} nasal release <\dia{207F}>
	  \item[→] labialized <\dia{02B7}>
	  \item[→] palatalized <\dia{02B2}>
	  \item[→] velarized <\dia{02E0}>
	  \item[→] pharyngealized <\dia{02E4}>
	  \item[→] aspirated <\dia{02B0}> \textbar{} ejective <\dia{02BC}>
	  \item[→] long <\dia{02D0}> \textbar{} half-long <\dia{02D1}>
	\end{itemize}
\end{itemize}

\noindent Finally, the tie bar follows at the very end of any such sequence:

\begin{itemize}
  \item[] \textsc{Tie bar:}
  \begin{itemize}
%    \item[→] tone letters <˥ ˦ ˧ ˨ ˩>
    \item[→] tie bar <\dia{0361}{\large\fontspec{CharisSIL}◌}>
  \end{itemize}
\end{itemize}

% ==========================
\section{Pitfall: Revisions to the IPA}
\label{ipa-revisions}
% ==========================

With each revision of the IPA, many decisions need to be made by 
the Association as to which symbols should be added, removed or 
changed. For example, in the 1989 revision of the IPA at the Kiel Convention, 
changes to specific symbols (in previous charts) were debated and 
the Association's members made certain decisions. The prevailing mood at 
the convention was not to change specific symbols unless a strong 
case was made \citep{Ladefoged1990a}. For example, two such decisions 
included:

\begin{itemize}
	\item Symbols for clicks were changed from <ʇ~ʖ~ʗ> to <ǀ~ǁ~ǃ>
       because the latter were the symbols used by nearly all Khoisanists and
       Bantuists.
	\item The Americanist tradition of using using <\dia{030C}>, a
       \textsc{combining caron}\\ at \uni{030C} for all postalveolar sounds, like
       in <š~ž~č~ǰ>, was not adopted because the Association members at the
       convention ``were not sufficiently impressed by arguments ... to the
       effect that these sounds formed a natural class, and thus is would be
       appropriate to recognize this by maintaining a common aspect to their
       symbolism'' \citep[62]{Ladefoged1990a}. 
\end{itemize}

\noindent These decisions have practical consequences for transcribers 
of IPA, particularly those who wish to follow current recommended practices of 
encoding electronic text in the Unicode Standard. For example, the Unicode 
Standard contains \textsc{latin small letter turned t} <ʇ> at U+0287, 
which is no longer part of the IPA. It still exists, however, in the Unicode 
\textsc{IPA Extensions} block, with the comment ``dental click (sound of `tsk tsk')''. 
In such cases, the IPA transcriber must know the status of legacy symbols in the current 
version of the IPA and the correct characters in the Unicode Standard.

The most controversial issue regarding symbols debated at the convention was the
representation for voiceless implosives \citep[62]{Ladefoged1990a}. In
accordance with the principles of the IPA, as outlined in Section
\ref{the-international-phonetic-alphabet}, distinct symbols are favored for
cases of phonological contrast. Further, convenience of display in the chart 
must also be taken into account when arguing for or against the inclusion or deletion
of IPA symbols in the IPA chart. Finally, the inclusion or deletion of symbols
should consider the current state of phonetic knowledge of the world's
languages.

\citet{Ladefoged1990a} argued against the inclusion of the symbols < ƥ, ƭ, ƈ, ƙ,
ʠ > for voiceless implosives, noting (i) that they are not contrastive (e.g.\ in
Mayan languages); (ii) that there is no instrumental evidence supporting
voiceless implosives in Africa; and (iii) that the sounds are sufficiently rare
so as not to need a whole new row of symbols in the chart. Ladefoged favored
symbolizing the sounds using a voiceless diacritic ring below voiced implosives,
e.g.\ <ɓ̥>. Nevertheless, in the 1989 IPA chart there is indeed a row for
implosives containing voiceless and voiced
pairs.\footnote{https://en.wikipedia.org/wiki/File:IPA\_as\_of\_1989.png} 
But in the next revision, in 1993 (with an update in 1996), the voiceless
implosives were dropped. The implosives row from the IPA consonantal chart
disappeared and voiced implosives were given a column in the non-pulmonic
consonants table (which is still reflected in the latest revision to date, IPA
2005).

The \textit{Journal of the International Phonetic Association} follows its own 
published standard for the IPA at the time of publication, even when it 
may conflict with the Association's principle of using different symbols 
for contrastive sounds and diacritics for phonetic variation. For example, 
in the case of voiceless implosives, \citet{McLaughlin2005} shows that 
Seereer-Siin (Niger-Congo, Atlantic; ISO 639-3 srr) has a phonologically 
contrastive set of voiced and voiceless implosive stops at the labial, 
coronal and palatal places of articulation. These symbols are transcribed 
in an \textit{Illustrations of the IPA} article in the IPA journal as 
< ɓ̥, ɗ̥, ʄ̥ >.

The point of this pitfall is to highlight that revisions to the IPA will 
continue into the future, albeit infrequently. Nevertheless, 
given the Unicode Standard's principle of maintaining backwards compatibility 
(at all costs), transcribers and consumers of IPA cannot rely solely on 
remarks in the Unicode Standard to reflect current standard IPA usage. 
There is the possibility that at a later revision of the IPA, symbols that 
are not currently encoded in the Unicode Standard will be added to the IPA -- 
although we think this is unlikely. 

% ==========================
\section{Additions to the IPA}
\label{ipa-additions}
% ==========================

In the course of collecting a large sample of phoneme systems across the world's
languages, \citet{Moran_etal2014} found that in order 
to preserve distinctions both within and across language descriptions, 
additions to the approved IPA glyph set were needed. Wherever possible 
these additions were drawn from the extIPA symbols for disordered
speech \citep{Duckworth_etal1990}.\footnote{\url{https://www.internationalphoneticassociation.org/sites/default/files/extIPAChart2008.pdf}}
This section describes these proposed additions to the IPA glyph set. The
additions are not part of the official IPA recommendations, so they should be 
used with caution.

\begin{itemize}
  
\item \textsc{Retroflex click} \newline
      Retroflex clicks can be represented by <‼> \textsc{double exclamation
      mark} at \uni{203C}. Note that the (post-)alveolar click <ǃ> at \uni{01C3}
      is confusingly referred to as \textsc{latin letter retroflex click} in the
      Unicode Standard, which is probably best considered an error.
\item \textsc{Voiced retroflex implosive} \newline 
      Although the IPA includes a
      series of voiced implosives (marked with a hook on top, see
      Section~\ref{pitfall-missing-decomposition}), there is no voiced retroflex
      implosive. Following the spirit of the IPA, we propose to use <\charis{ᶑ}>
      \textsc{latin small letter d with hook and tail} at \uni{1D91} for this
      sound.
\item \textsc{Fortis/lenis} \newline
      Languages described as having a fortis/plain/lenis distinction that
      corresponds poorly with the traditional
      voiced/voiceless-unaspirated/voiceless-aspirated continuum can be marked
      using the voiceless glyph for the plain phoneme, and then 
      <\dia{0348}> \textsc{combining double vertical line below} at \\
      \uni{0348} to mark the fortis articulation, and/or <\dia{0349}>
      \textsc{combining left angle below} at \uni{0349} for the lenis
      articulation.
\item \textsc{Frictionalization} \newline 
      The diacritic <\dia{0353}>
      \textsc{combining x below} at \uni{0353} can be used to represent three
      types of frictionalized sounds: First, click consonants where the release
      of the anterior closure involves an ingressive sucking sound similar
      to a fricative, for example <kǃ͓ʰ>; second, frictionalized vowels
      (sounds that are phonologically vocalic, but with sufficiently close
      closures to create buzzing); and third, fricative sounds at places of
      articulation that do not have dedicated fricative glyphs, for example
      sounds with voiceless velar lateral frication, like <ʟ̥͓>.
\item \textsc{Derhoticization} \newline 
      For derhoticization we propose to use
      <\dia{032E}> \textsc{combining breve below} at \uni{032E}. 
\item \textsc{Coronal non-sibilant} \newline
      Languages described as having a sibilant/non-sibilant distinction among
      coronal fricatives and affricates can be handled using the subscript
      <\dia{0347}> \textsc{combining equals sign below} at \uni{0347} to mark
      the non-sibilant \\ phoneme.
\item \textsc{Glottalization} \newline 
      Glottalized sounds can be indicated using
      <\dia{02C0}> \textsc{modifier letter glottal stop} at \uni{02C0}, unless
      it is clear that either ejective or creaky voicing are the
      intended sounds (in which cases the standard IPA diacritics should be
      used). Pre-glottalized sounds can be marked with
      <\diareverse{02C0}> to the left of the base
      glyph, for example <\charis{ˀt}>.
\item \textsc{Voiced pre-aspiration} \newline Voiced sounds having
      pre-aspiration can be marked with
      <\diareverse{02B1}> \textsc{modifier letter
      small h with hook} at \uni{02B1} to the left of the base glyph, for
      example <\charis{ʱd}>.
\item \textsc{Epilaryngeal phonation} \newline 
      There are some rare articulations that make
      use of an epilaryngeal phonation mechanism (e.g.\ the “sphincteric vowels”
      of~!Xóõ). To represent these vowels, we propose to use the modifier <\dia{1D31}>
      \textsc{modifier letter capital e} at \uni{1D31} to denote such sphincteric
      phonation.

\end{itemize}


% ==========================
\section{Unicode IPA Recommendations}
\label{ipa-recommendations}
% ==========================

% Great quote for standardization -- last sentence(s) in Ladefoged 1990:552

Summarizing the pitfalls as discussed in this chapter, we propose to define
three different IPA encodings: strict-IPA, valid-IPA and widened-IPA.\@
Informally speaking, valid-IPA represents the current state of the IPA
\citep{IPA2015}. Strict-IPA represents a more constrained version of IPA, while
widened-IPA is a slightly extended version of IPA, allowing a few more symbols.

\ 

\noindent Strict-IPA encoding is supposed to be used when interoperability of
phonetic resources is intended. It is a strongly constrained subset of IPA
geared towards uniqueness of encoding. Ideally, for each transcription there
should be exactly one possible strict-IPA encoding. For each phonetic feature
there is only one possibility (see Section~\ref{pitfall-multiple-options-ipa})
and the IPA diacritics are forced into a canonical ordering (see
Section~\ref{pitfall-no-unique-diacritic-ordering}).

Valid-IPA does allow alternative symbols with the same phonetic meaning, as 
specified in the official IPA specifications. Also, valid-IPA does not enforce a 
specific ordering of diacritics, because the IPA does not propose any such 
ordering. This means that in valid-IPA the same phonetic intention can be 
encoded in multiple ways. This is sufficient for phonetically trained human 
eyes, but it is not sufficient for automatic interoperability.

Finally, widened-IPA includes a few more symbols which seem to be useful for
various special cases (see Section~\ref{ipa-additions}).

\ 

\noindent At the end of this chapter we have added a few longish tables summarizing all
159 different Unicode code points that form the basis of strict-IPA encoding
(107 letters, 36 diacritics and 16 remaining symbols). We also make these tables available 
online in CSV format.\footnote{\url{https://github.com/unicode-cookbook/cookbook/tree/master/book/tables}} 
Each of these tables shows a typical glyph, and then lists the Unicode Code point,
Unicode Name and IPA description for each symbol. Further, there is a table with 
the additional options for valid-IPA and a table with the additional options for 
widened-IPA.\@

\begin{itemize}[itemsep=6pt]

  \item \textsc{strict-IPA letters} \newline
        The 107 different IPA letters as allowed in strict-IPA encoding are
        listed in Table~\ref{tab:ipa_letters} starting on
        page~\pageref{tab:ipa_letters}.
  \item \textsc{strict-IPA diacritics} \newline The 36 different IPA diacritics and
        tone markers (both Unicode Modifier Letters and Combining Diacritical
        Marks) as allowed in strict-IPA encoding are listed in
        Table~\ref{tab:ipa_diacritics} starting on
        page~\pageref{tab:ipa_diacritics}.
  \item \textsc{strict-IPA remainders} \newline The 16 remaining IPA symbols
        (boundary, stress, tone letters and intonation markers) as allowed in strict-IPA
        encoding are listed in Table~\ref{tab:ipa_leftovers} on
        page~\pageref{tab:ipa_leftovers}.
  \item \textsc{valid-IPA additions} \newline The 16 additional symbols as allowed in
        valid-IPA encoding are listed in Table~\ref{tab:ipa_lax} on
        page~\pageref{tab:ipa_lax}.     
  \item \textsc{widened-IPA additions} \newline
        The 10 proposed additions to the IPA are listed in
        Table~\ref{tab:ipa_additions} on page~\pageref{tab:ipa_additions}.
  
\end{itemize}

\newpage
\tablecaption{Strict-IPA letters with Unicode encodings}\label{tab:ipa_letters}
 \tablefirsthead{
   \toprule
   & Code & Unicode name & IPA name \\ 
   \midrule
   }
 \tablehead{
   \multicolumn{4}{c}{
     \small\tablename\ \thetable{} 
     Strict-IPA letters with Unicode encodings --- \textit{continued}
     } \\
   \toprule
   & Code & Unicode name & IPA name \\ 
   \midrule
   }
 \tabletail{
   \bottomrule
   \multicolumn{4}{r}{
     \small\textit{continued on next page}
     } \\
   }
 \tablelasttail{\bottomrule}
  
\begin{center}
\begin{xtabular}{ l L{1.1cm} L{5.4cm} L{4.2cm} }
a & \unif{0061} & \textsc{latin small letter a} & open front unrounded \\ 
æ & \unif{00E6} & \textsc{latin small letter ae} & raised open front unrounded \\ 
ɐ & \unif{0250} & \textsc{latin small letter turned a} & lowered schwa \\ 
ɑ & \unif{0251} & \textsc{latin small letter alpha} & open back unrounded \\ 
ɒ & \unif{0252} & \textsc{latin small letter turned alpha} & open back rounded \\ 
b & \unif{0062} & \textsc{latin small letter b} & voiced bilabial plosive \\ 
ʙ & \unif{0299} & \textsc{latin letter small capital b} & voiced bilabial trill \\ 
ɓ & \unif{0253} & \textsc{latin small letter b with hook} & voiced bilabial implosive \\ 
c & \unif{0063} & \textsc{latin small letter c} & voiceless palatal plosive \\ 
ç & \unif{00E7} & \textsc{latin small letter c with cedilla} & voiceless palatal fricative \\ 
ɕ & \unif{0255} & \textsc{latin small letter c with curl} & voiceless alveolo-palatal fricative \\ 
d & \unif{0064} & \textsc{latin small letter d} & voiced alveolar plosive \\ 
ð & \unif{00F0} & \textsc{latin small letter eth} & voiced dental fricative \\ 
ɖ & \unif{0256} & \textsc{latin small letter d with tail} & voiced retroflex plosive \\ 
ɗ & \unif{0257} & \textsc{latin small letter d with hook} & voiced dental/alveolar implosive \\ 
e & \unif{0065} & \textsc{latin small letter e} & close-mid front unrounded \\ 
ə & \unif{0259} & \textsc{latin small letter schwa} & mid-central schwa \\ 
ɛ & \unif{025B} & \textsc{latin small letter open e} & open-mid front unrounded \\ 
ɘ & \unif{0258} & \textsc{latin small letter reversed e} & close-mid central unrounded \\ 
ɜ & \unif{025C} & \textsc{latin small letter reversed open e} & open-mid central unrounded \\ 
ɞ & \unif{025E} & \textsc{latin small letter closed reversed open e} & open-mid central rounded \\ 
f & \unif{0066} & \textsc{latin small letter f} & voiceless labiodental fricative \\ 
ɡ & \unif{0261} & \textsc{latin small letter script g} & voiced velar plosive \\ 
ɢ & \unif{0262} & \textsc{latin letter small capital g} & voiced uvular plosive \\ 
ɠ & \unif{0260} & \textsc{latin small letter g with hook} & voiced velar implosive \\ 
ʛ & \unif{029B} & \textsc{latin letter small capital g with hook} & voiced uvular implosive \\ 
ɤ & \unif{0264} & \textsc{latin small letter rams horn} & close-mid back unrounded \\ 
ɣ & \unif{0263} & \textsc{latin small letter gamma} & voiced velar fricative \\ 
h & \unif{0068} & \textsc{latin small letter h} & voiceless glottal fricative \\ 
ħ & \unif{0127} & \textsc{latin small letter h with stroke} & voiceless pharyngeal fricative \\ 
ʜ & \unif{029C} & \textsc{latin letter small capital h} & voiceless epiglottal fricative \\ 
ɦ & \unif{0266} & \textsc{latin small letter h with hook} & voiced glottal fricative \\ 
ɧ & \unif{0267} & \textsc{latin small letter heng with hook} & simultaneous voiceless postalveolar+velar fricative \\ 
ɥ & \unif{0265} & \textsc{latin small letter turned h} & voiced labial-palatal approximant \\ 
i & \unif{0069} & \textsc{latin small letter i} & close front unrounded \\ 
ɪ & \unif{026A} & \textsc{latin letter small capital i} & lax close front unrounded \\ 
ɨ & \unif{0268} & \textsc{latin small letter i with stroke} & close central unrounded \\ 
j & \unif{006A} & \textsc{latin small letter j} & voiced palatal approximant \\ 
ʝ & \unif{029D} & \textsc{latin small letter j with crossed tail} & voiced palatal fricative \\ 
ɟ & \unif{025F} & \textsc{latin small letter dotless j with stroke} & voiced palatal plosive \\ 
ʄ & \unif{0284} & \textsc{latin small letter dotless j with stroke and hook} & voiced palatal implosive \\ 
k & \unif{006B} & \textsc{latin small letter k} & voiceless velar plosive \\ 
l & \unif{006C} & \textsc{latin small letter l} & voiced alveolar lateral approximant \\ 
ʟ & \unif{029F} & \textsc{latin letter small capital l} & voiced velar lateral approximant \\ 
ɬ & \unif{026C} & \textsc{latin small letter l with belt} & voiceless alveolar lateral fricative \\ 
ɭ & \unif{026D} & \textsc{latin small letter l with retroflex hook} & voiced retroflex lateral approximant \\ 
ɮ & \unif{026E} & \textsc{latin small letter lezh} & voiced alveolar lateral fricative \\ 
ʎ & \unif{028E} & \textsc{latin small letter turned y} & voiced palatal lateral approximant \\ 
m & \unif{006D} & \textsc{latin small letter m} & voiced bilabial nasal \\ 
ɱ & \unif{0271} & \textsc{latin small letter m with hook} & voiced labiodental nasal \\ 
n & \unif{006E} & \textsc{latin small letter n} & voiced alveolar nasal \\ 
ɴ & \unif{0274} & \textsc{latin letter small capital n} & voiced uvular nasal \\ 
ɲ & \unif{0272} & \textsc{latin small letter n with left hook} & voiced palatal nasal \\ 
ɳ & \unif{0273} & \textsc{latin small letter n with retroflex hook} & voiced retroflex nasal \\ 
ŋ & \unif{014B} & \textsc{latin small letter eng} & voiced velar nasal \\ 
o & \unif{006F} & \textsc{latin small letter o} & close-mid back rounded \\ 
ø & \unif{00F8} & \textsc{latin small letter o with stroke} & close-mid front rounded \\ 
œ & \unif{0153} & \textsc{latin small ligature oe} & open-mid front rounded \\ 
ɶ & \unif{0276} & \textsc{latin letter small capital oe} & open front rounded \\ 
ɔ & \unif{0254} & \textsc{latin small letter open o} & open-mid back rounded \\ 
ɵ & \unif{0275} & \textsc{latin small letter barred o} & close-mid central rounded \\ 
p & \unif{0070} & \textsc{latin small letter p} & voiceless bilabial plosive \\ 
ɸ & \unif{0278} & \textsc{latin small letter phi} & voiceless bilabial fricative \\ 
q & \unif{0071} & \textsc{latin small letter q} & voiceless uvular plosive \\ 
r & \unif{0072} & \textsc{latin small letter r} & voiced alveolar trill \\ 
ʀ & \unif{0280} & \textsc{latin letter small capital r} & voiced uvular trill \\ 
ɹ & \unif{0279} & \textsc{latin small letter turned r} & voiced alveolar approximant \\ 
ɺ & \unif{027A} & \textsc{latin small letter turned r with long leg} & voiced alveolar lateral flap \\ 
ɻ & \unif{027B} & \textsc{latin small letter turned r with hook} & voiced retroflex approximant \\ 
ɽ & \unif{027D} & \textsc{latin small letter r with tail} & voiced retroflex tap \\ 
ɾ & \unif{027E} & \textsc{latin small letter r with fishhook} & voiced alveolar tap \\ 
ʁ & \unif{0281} & \textsc{latin letter small capital inverted r} & voiced uvular fricative \\ 
s & \unif{0073} & \textsc{latin small letter s} & voiceless alveolar fricative \\ 
ʂ & \unif{0282} & \textsc{latin small letter s with hook} & voiceless retroflex fricative \\ 
ʃ & \unif{0283} & \textsc{latin small letter esh} & voiceless postalveolar fricative \\ 
t & \unif{0074} & \textsc{latin small letter t} & voiceless alveolar plosive \\ 
ʈ & \unif{0288} & \textsc{latin small letter t with retroflex hook} & voiceless retroflex plosive \\ 
u & \unif{0075} & \textsc{latin small letter u} & close back rounded \\ 
ʉ & \unif{0289} & \textsc{latin small letter u bar} & close central rounded \\ 
ɯ & \unif{026F} & \textsc{latin small letter turned m} & close back unrounded \\ 
ɰ & \unif{0270} & \textsc{latin small letter turned m with long leg} & voiced velar approximant \\ 
ʊ & \unif{028A} & \textsc{latin small letter upsilon} & lax close back rounded \\ 
v & \unif{0076} & \textsc{latin small letter v} & voiced labiodental fricative \\ 
ʋ & \unif{028B} & \textsc{latin small letter v with hook} & voiced labiodental approximant \\ 
\charis{ⱱ} & \unif{2C71} & \textsc{latin small letter v with right hook} & voiced labiodental tap \\ 
ʌ & \unif{028C} & \textsc{latin small letter turned v} & open-mid back unrounded \\ 
w & \unif{0077} & \textsc{latin small letter w} & voiced labial-velar approximant \\ 
ʍ & \unif{028D} & \textsc{latin small letter turned w} & voiceless labial-velar fricative \\ 
x & \unif{0078} & \textsc{latin small letter x} & voiceless velar fricative \\ 
y & \unif{0079} & \textsc{latin small letter y} & close front rounded \\ 
ʏ & \unif{028F} & \textsc{latin letter small capital y} & lax close front rounded \\ 
z & \unif{007A} & \textsc{latin small letter z} & voiced alveolar fricative \\ 
ʐ & \unif{0290} & \textsc{latin small letter z with retroflex hook} & voiced retroflex fricative \\ 
ʑ & \unif{0291} & \textsc{latin small letter z with curl} & voiced alveolo-palatal fricative \\ 
ʒ & \unif{0292} & \textsc{latin small letter ezh} & voiced postalveolar fricative \\ 
ʔ & \unif{0294} & \textsc{latin letter glottal stop} & voiceless glottal plosive \\ 
ʕ & \unif{0295} & \textsc{latin letter pharyngeal voiced fricative} & voiced pharyngeal fricative \\ 
ʡ & \unif{02A1} & \textsc{latin letter glottal stop with stroke} & epiglottal plosive \\ 
ʢ & \unif{02A2} & \textsc{latin letter reversed glottal stop with stroke} & voiced epiglottal fricative \\ 
ǀ & \unif{01C0} & \textsc{latin letter dental click} & voiceless dental click \\ 
ǁ & \unif{01C1} & \textsc{latin letter lateral click} & voiceless alveolar lateral click \\ 
ǂ & \unif{01C2} & \textsc{latin letter alveolar click} & voiceless palatoalveolar click \\ 
ǃ & \unif{01C3} & \textsc{latin letter retroflex click} & voiceless (post)alveolar click \\ 
ʘ & \unif{0298} & \textsc{latin letter bilabial click} & voiceless bilabial click \\ 
β & \unif{03B2} & \textsc{greek small letter beta} & voiced bilabial fricative \\ 
θ & \unif{03B8} & \textsc{greek small letter theta} & voiceless dental fricative \\ 
χ & \unif{03C7} & \textsc{greek small letter chi} & voiceless uvular fricative \\
\end{xtabular}
\end{center}

\tablecaption{Strict-IPA diacritics with Unicode encodings}\label{tab:ipa_diacritics}
 \tablefirsthead{
   \toprule
   & Code & Unicode name & IPA name \\ 
   \midrule
   }
 \tablehead{
   \multicolumn{4}{c}{
     \small\tablename\ \thetable{} 
     Strict-IPA diacritics with Unicode encodings --- \textit{continued}
     } \\
   \toprule
   & Code & Unicode name & IPA name \\ 
   \midrule
   }
 \tabletail{
   \bottomrule
   \multicolumn{4}{r}{
     \small\textit{continued on next page}
     } \\
   }
 \tablelasttail{\bottomrule}

\begin{center}
\begin{xtabular}{ l L{1.1cm} L{6.2cm} L{3.6cm} }
\dia{0334} & \unif{0334} & \textsc{combining tilde overlay} & velarized or pharyngealized \\
\dia{033C} & \unif{033C} & \textsc{combining seagull below} & linguolabial \\
\dia{032A} & \unif{032A} & \textsc{combining bridge below} & dental \\
\dia{033B} & \unif{033B} & \textsc{combining square below} & laminal \\
\dia{033A} & \unif{033A} & \textsc{combining inverted bridge below} & apical \\
\dia{031F} & \unif{031F} & \textsc{combining plus sign below} & advanced \\
\dia{0320} & \unif{0320} & \textsc{combining minus sign below} & retracted \\
\dia{031D} & \unif{031D} & \textsc{combining up tack below} & raised \\
\dia{031E} & \unif{031E} & \textsc{combining down tack below} & lowered \\
\dia{0318} & \unif{0318} & \textsc{combining left tack below} & advanced tongue root \\
\dia{0319} & \unif{0319} & \textsc{combining right tack below} & retracted tongue root \\
\dia{031C} & \unif{031C} & \textsc{combining left half ring below} & less rounded \\
\dia{0339} & \unif{0339} & \textsc{combining right half ring below} & more rounded \\
\dia{032C} & \unif{032C} & \textsc{combining caron below} & voiced \\
\dia{0325} & \unif{0325} & \textsc{combining ring below} & voiceless \\
\dia{0330} & \unif{0330} & \textsc{combining tilde below} & creaky voiced \\
\dia{0324} & \unif{0324} & \textsc{combining diaeresis below} & breathy voiced \\
\dia{0329} & \unif{0329} & \textsc{combining vertical line below} & syllabic \\
\dia{032F} & \unif{032F} & \textsc{combining inverted breve below} & non-syllabic \\
\dia{0303} & \unif{0303} & \textsc{combining tilde} & nasalized \\
\dia{0308} & \unif{0308} & \textsc{combining diaeresis} & centralized \\
\dia{033D} & \unif{033D} & \textsc{combining x above} & mid-centralized \\
\dia{0306} & \unif{0306} & \textsc{combining breve} & extra-short \\
\dia{031A} & \unif{031A} & \textsc{combining left angle above} & no audible release \\
\dia{02DE} & \unif{02DE} & \textsc{modifier letter rhotic hook} & rhotacized \\
\dia{02E1} & \unif{02E1} & \textsc{modifier letter small l} & lateral release \\
\dia{207F} & \unif{207F} & \textsc{superscript latin small letter n} & nasal release \\
\dia{02B7} & \unif{02B7} & \textsc{modifier letter small w} & labialized \\
\dia{02B2} & \unif{02B2} & \textsc{modifier letter small j} & palatalized \\
\dia{02E0} & \unif{02E0} & \textsc{modifier letter small gamma} & velarized \\
\dia{02E4} & \unif{02E4} & \textsc{modifier letter small reversed glottal stop} & pharyngealized \\
\dia{02B0} & \unif{02B0} & \textsc{modifier letter small h} & aspirated \\
\dia{02BC} & \unif{02BC} & \textsc{modifier letter apostrophe} & ejective \\
\dia{02D0} & \unif{02D0} & \textsc{modifier letter triangular colon} & long \\
\dia{02D1} & \unif{02D1} & \textsc{modifier letter half triangular colon} & half-long \\
\dia{0361}{\large\fontspec{CharisSIL}◌} & \unif{0361} & \textsc{combining double inverted breve} & tie bar \\
\end{xtabular}
\end{center}
\tablecaption{Other Strict-IPA symbols with Unicode encodings}\label{tab:ipa_leftovers}
 \tablefirsthead{
   \toprule
   & Code & Unicode name & IPA name \\ 
   \midrule
   }
 \tablehead{
   \multicolumn{4}{c}{
     \small\tablename\ \thetable{} 
     Other Strict-IPA symbols with Unicode encodings --- \textit{continued}
     } \\
   \toprule
   & Code & Unicode name & IPA name \\ 
   \midrule
   }
 \tabletail{
   \bottomrule
   \multicolumn{4}{r}{
     \small\textit{continued on next page}
     } \\
   }
 \tablelasttail{\bottomrule}

\begin{center}
\Needspace{6cm}
\begin{xtabular}{ l L{1.2cm} L{5.4cm} L{4cm} }
{\large ˈ} & \unif{02C8} & \textsc{modifier letter vertical line} & primary stress \\
{\large ˌ} & \unif{02CC} & \textsc{modifier letter low vertical line} & secondary stress \\
{˥} & \unif{02E5} & \textsc{modifier letter extra-high tone bar} & extra high tone \\
{˦} & \unif{02E6} & \textsc{modifier letter high tone bar} & high tone \\
{˧} & \unif{02E7} & \textsc{modifier letter mid tone bar} & mid tone \\
{˨} & \unif{02E8} & \textsc{modifier letter low tone bar} & low tone \\
{˩} & \unif{02E9} & \textsc{modifier letter extra-low tone bar} & extra low tone \\
%{\large\fontspec{CharisSIL}ꜛ} & \unif{A71B} & \textsc{modifier letter raised up arrow} & upstep \\
%{\large\fontspec{CharisSIL}ꜜ} & \unif{A71C} & \textsc{modifier letter raised down arrow} & downstep \\
{↑} & \unif{2191} & \textsc{upwards arrow} & global rise \\
{↓} & \unif{2193} & \textsc{downwards arrow} & global fall \\
{↗} & \unif{2197} & \textsc{north east arrow} & global rise \\
{↘} & \unif{2198} & \textsc{south east arrow} & global fall \\
 & \unif{0020} & \textsc{space} & word break \\
{\large.} & \unif{002E} & \textsc{full stop} & syllable break \\
{|} & \unif{007C} & \textsc{vertical line} & minor group break (foot) \\
{‖} & \unif{2016} & \textsc{double vertical line} & major group break (intonation) \\
\charis{‿} & \unif{203F} & \textsc{undertie} & linking (absence of a break) \\
\end{xtabular}
\end{center}
\tablecaption{Additional characters for valid-IPA with Unicode encodings}\label{tab:ipa_lax}
 \tablefirsthead{
   \toprule
   & Code & Unicode name & Phonetic description \\ 
   \midrule
   }
 \tablehead{
   \multicolumn{4}{c}{
     \small\tablename\ \thetable{} 
     Additional characters for valid-IPA with Unicode encodings --- \textit{continued}
     } \\
   \toprule
   & Code & Unicode name & Phonetic description \\ 
   \midrule
   }
 \tabletail{
   \bottomrule
   \multicolumn{4}{r}{
     \small\textit{continued on next page}
     } \\
   }
 \tablelasttail{\bottomrule}

\begin{center}
\begin{xtabular}{ l L{1.2cm} L{6.1cm} L{3.5cm} }
\dia{030A} & \unif{030A} & \textsc{combining ring above} & voiceless (above) \\
g & \unif{0067} & \textsc{latin small letter g} & voiced velar plosive \\
\dia{030B} & \unif{030B} & \textsc{combining double acute accent} & extra high tone \\
\dia{0301} & \unif{0301} & \textsc{combining acute accent} & high tone \\
\dia{0304} & \unif{0304} & \textsc{combining macron} & mid tone \\
\dia{0300} & \unif{0300} & \textsc{combining grave accent} & low tone \\
\dia{030F} & \unif{030F} & \textsc{combining double grave accent} & extra low tone \\
\dia{0302} & \unif{0302} & \textsc{combining circumflex accent} & falling \\
\dia{030C} & \unif{030C} & \textsc{combining caron} & rising \\
\dia{1DC4} & \unif{1DC4} & \textsc{combining macron-acute} & high rising \\
\dia{1DC5} & \unif{1DC5} & \textsc{combining grave-macron} & low rising \\
\dia{1DC6} & \unif{1DC6} & \textsc{combining macron-grave} & low falling \\
\dia{1DC7} & \unif{1DC7} & \textsc{combining acute-macron} & high falling \\
\dia{1DC8} & \unif{1DC8} & \textsc{combining grave-acute-grave} & rising-falling \\
\dia{1DC9} & \unif{1DC9} & \textsc{combining acute-grave-acute} & falling-rising \\ 
\dia{035C}{\large\fontspec{CharisSIL}◌} & \unif{035C} & \textsc{combining double breve below} & tie bar (below) \\
\end{xtabular}
\end{center}

\tablecaption{Additions to widened-IPA with Unicode encodings}\label{tab:ipa_additions}
 \tablefirsthead{
   \toprule
   & Code & Unicode name & Phonetic description \\ 
   \midrule
   }
 \tablehead{
   \multicolumn{4}{c}{
     \small\tablename\ \thetable{} 
     Additions to widened-IPA with Unicode encodings --- \textit{continued}
     } \\
   \toprule
   & Code & Unicode name & Phonetic description \\ 
   \midrule
   }
 \tabletail{
   \bottomrule
   \multicolumn{4}{r}{
     \small\textit{continued on next page}
     } \\
   }
 \tablelasttail{\bottomrule}

\begin{center}
\begin{xtabular}{ l L{1.1cm} L{6.1cm} L{3.5cm} }
‼ & \unif{203C} & \textsc{double exclamation mark} & retroflex click \\
\charis{ᶑ} & \unif{1D91} & \textsc{latin small letter d with hook and tail} & voiced retroflex implosive \\
\dia{0348} & \unif{0348} & \textsc{combining double vertical line below} & fortis \\
\dia{0349} & \unif{0349} & \textsc{combining left angle below} & lenis \\
\dia{0353} & \unif{0353} & \textsc{combining x below} & frictionalized \\
\dia{032E} & \unif{032E} & \textsc{combining breve below} & derhoticized \\
\dia{0347} & \unif{0347} & \textsc{combining equals sign below} & non-sibilant \\
\dia{02C0} & \unif{02C0} & \textsc{modifier letter glottal stop} & glottalized \\
\diareverse{02B1} & \unif{02B1} & \textsc{modifier letter small h with hook} & voiced pre-aspirated \\
\dia{1D31} & \unif{1D31} & \textsc{modifier letter capital e} & epilaryngeal phonation \\
\end{xtabular}
\end{center}

\chapter{Practical recommendations}
\label{practical-recommendations}

% A section which is missing in something called a "Cookbook" would be
% $ practical recommendations on how to input Unicode characters. There are
% various character selection tools, shortcuts on the keyboard, the
% shapecatcher website references at several places, or the Wikipedia
% lists of glyphs and fileformat.info. Having all this in one section
% would be handy for the user. It is of course unrelated to the
% orthography profiles, but I imagine that many people will use this book
% as a primer on IPA+Unicode and actually disregard the last two chapters.
% For this group, such a summary would be useful.

This chapter is meant to be a short guide for novice users who are not interested in the programmatic aspects presented in Chapters \ref{orthography-profiles} and \ref{implementation}. Instead, we provide links to quickly find general information about the Unicode Standard and the International Phonetic Alphabet (IPA). And we target ordinary working linguists who want to know how to easily insert special characters into their digital documents and applications.

\section{Unicode}
We discussed the Unicode Consortium's approach to computationally encoding writing systems in Chapter \ref{the-unicode-approach}. The common pitfalls that we have encountered when using the Unicode Standard are discussed in detail in Chapter \ref{unicode-pitfalls}. Together these chapters provide users with an in-depth background about the hurdles they may encounter when using the Unicode Standard for encoding their data or for developing multilingual applications. For general background information about Unicode and character encodings, see these resources:

\begin{itemize}
	\item \url{http://www.unicode.org/standard/WhatIsUnicode.html}
	\item \url{https://en.wikipedia.org/wiki/Unicode}
	\item \url{https://www.w3.org/International/articles/definitions-characters/}
\end{itemize}

% \section{Unicode character pickers}
For practical purposes, users need a way to insert special characters (i.e.\ characters that are not easily entered via their keyboards) into documents and software applications. There are a few basic approaches for inserting special characters. One way is to use software-specific functionality, when it is available. For example, Microsoft Word has an insert-special-symbol-or-character function that allows users to scroll through a table of special characters across different scripts. Special characters can be then inserted into the document by clicking on them. Another way is to install a system-wide application for special character insertion. We have long been fans of the PopChar application from Ergonis Software, which is a small program that can insert most Unicode characters (note however that the full version requires a paid subscription).\footnote{\url{http://www.ergonis.com/products/popcharx/}}

There are also Web-based Unicode character pickers available through the browser that allow for the creation and insert of special characters, which can then be copied \& pasted into documents or software applications. For example, try:

\begin{itemize}
	\item \url{https://unicode-table.com/en/}
	\item \url{https://r12a.github.io/pickers/}
\end{itemize}

Yet another option for special character insertion includes operating system-specific shortcuts. For example on the Mac, holding down a key on the keyboard for a second, say <u>, triggers a pop up with the options <û, ü, ù, ú, ū> which can then be inserted by keying the associated number (1--5). This method is convenient for occasionally inserting type accented characters, but the full range of special characters is limited and this method is burdensome for rapidly inserting many different characters. For complete access to special characters, Mac provides a Keyboard Viewer application available in the Keyboard pane of the System Preferences.

On Windows, accented characters can be inserted by using alt-key shortcuts, i.e.\ holding down the alt-key and keying in a sequence of numbers (which typically reflect the Unicode character's decimal representation). For example, \textsc{latin small letter c with cedilla} at \uni{00E7} with the decimal code 231 can be inserted by holding the alt-key and keying the sequence 0231. Again, this method is burdensome for rapidly inserting characters. For access to the full range of Unicode characters, the Character Map program comes preinstalled on all Microsoft Operating systems.

There are also many third party applications that provide virtual keyboards. These programs typically override keys or keystrokes on the user's keyboard  allowing them to quickly keyboard special characters (once the layout of the new keyboard is mastered). They can be language-specific or devoted specifically to IPA. Two popular programs are:

\begin{itemize}
	\item \url{https://keyman.com/}
	\item \url{http://scripts.sil.org/ipa-sil_keyboard}
\end{itemize}


\section{IPA}
In Chapter \ref{the-international-phonetic-alphabet} we described in detail the history and principles of the International Phonetic Alphabet (IPA) and how it became encoded in the Unicode Standard. In Chapter \ref{ipa-meets-unicode} we describe the resulting pitfalls from their marriage. These two chapters provide a detailed overview of the challenges that users face when working with the two standards.

For general information about the IPA, the standard text is the \textit{Handbook of the International Phonetic Association: A Guide to the Use of the International Phonetic Alphabet} \citep{IPA2007}. The handbook describes in detail the principles and premises of the IPA, which we have summarized in Section \ref{IPApremises-principles}. The handbook also provides many examples of how to use the IPA. The Association also makes available information about itself online\footnote{\url{https://www.internationalphoneticassociation.org/}} and it provides the most current IPA charts.\footnote{\url{https://www.internationalphoneticassociation.org/content/ipa-chart}} Wikipedia also has a comprehensive article about the IPA.\footnote{\url{https://en.wikipedia.org/wiki/International_Phonetic_Alphabet}}

There are several good Unicode IPA character pickers available on the Web and through the browser, including:

\begin{itemize}
	\item \url{https://r12a.github.io/pickers/ipa/}
	\item \url{https://westonruter.github.io/ipa-chart/keyboard/}
	\item \url{http://ipa.typeit.org/}
\end{itemize}

\noindent Various linguistics departments also provide information about IPA fonts, software, and inserting Unicode IPA characters. Two useful resources are:

\begin{itemize}
	\item \url{http://www.phon.ucl.ac.uk/resource/phonetics/}
	\item \url{https://www.york.ac.uk/language/current/resources/freeware/ipa-fonts-and-software/}
\end{itemize}	

Regarding fonts that display Unicode IPA correctly, many linguists turn to the IPA Unicode fonts developed by SIL Internnational. The complete SIL font list is available online.\footnote{\url{http://scripts.sil.org/SILFontList}} There is also a page that describes IPA transcription using the SIL fonts and provides an informative discussion on deciding which font to use.\footnote{\url{http://scripts.sil.org/ipahome}} Traditionally, IPA fonts popular with linguists were created and maintained by SIL International, so it is often the case in our experience that we encounter linguistics data in legacy IPA fonts, i.e.\ pre-Unicode fonts such as SIL IPA93.\footnote{\url{http://scripts.sil.org/FontFAQ_IPA93}} SIL International does a good job of describing how to convert from legacy IPA fonts to Unicode IPA. The most popular Unicode IPA fonts are Doulos SIL and Charis SIL:

\begin{itemize}
	\item \url{https://software.sil.org/doulos/}}
	\item \url{https://software.sil.org/charis/}
\end{itemize}	

Lastly, here are some online resources that we find particularly useful for finding more information about individual Unicode characters and also for converting between encodings:

\begin{itemize}
	\item \url{http://www.fileformat.info/}
	\item \url{https://unicodelookup.com/}
	\item \url{https://r12a.github.io/scripts/featurelist/}
	\item \url{https://r12a.github.io/app-conversion/}
\end{itemize}


\section{For programmers and potential programmers}
If you have made it this far, and you are eager to know more about the technological aspects of the Unicode Standard and how they relate to software programming, we recommend two light-hearted blog posts on the topic. The classic blog post about what programmers should know about the Unicode Standard is Joel Spolsky's \textit{The Absolute Minimum Every Software Developer Absolutely, Positively Must Know About Unicode and Character Sets (No Excuses!)}.\footnote{\url{https://www.joelonsoftware.com/2003/10/08/the-absolute-minimum-every-software-developer-absolutely-positively-must-know-about-unicode-and-character-sets-no-excuses/}} A more recent blogpost, with a bit more of the technical details, is by David C. Zentgraf and is titled, \textit{What Every Programmer Absolutely, Positively Needs To Know About Encodings And Character Sets To Work With Text}.\footnote{\url{http://kunststube.net/encoding/}} This post is aimed at software developers and uses the PHP language for examples.

For users of Python, see the standard documentation on how to use Unicode in your programming applications.\footnote{\url{https://docs.python.org/3/howto/unicode.html}} For R users we recommend the \textsc{stringi} library.\footnote{\url{https://cran.r-project.org/web/packages/stringi/index.html}} For \LaTeX users, the TIPA package is useful for inserting IPA characters into your typeset documents. See these resources:

\begin{itemize}
	\item \url{http://www.tug.org/tugboat/tb17-2/tb51rei.pdf}
	\item \url{https://ctan.org/pkg/tipa}
	\item \url{http://ptmartins.info/tex/tipacheatsheet.pdf}
\end{itemize}

\noindent But we find it much easier to use the Unicode-aware XeTeX/XeLaTeX typesetting system.\footnote{\url{http://xetex.sourceforge.net/}} Unicode characters can be directly inserted into your Tex documents and compiled into typeset PDF with XeLaTeX.

Lastly, we leave you with some Unicode humor for making it this far:

\begin{itemize}
	\item \url{https://xkcd.com/380/}
	\item \url{https://xkcd.com/1137/}
	\item \url{http://www.commitstrip.com/en/2014/06/17/unicode-7-et-ses-nouveaux-emoji/}
	\item \url{http://www.i18nguy.com/humor/unicode-haiku.html}
\end{itemize}




\chapter{Orthography profiles}
\label{orthography-profiles}

\section{Characterizing writing systems}
\label{characterizing-writing-systems}

The Unicode Standard offers a very detailed technical approach 
for characterizing writing systems computationally. As such, it is 
sometimes too complex for the day-to-day practice of many linguists, 
as exemplified by the need to understand the common pitfalls that we discussed in Chapters 
\ref{unicode-pitfalls} \& \ref{ipa-meets-unicode}. Therefore, in this section we propose some simple 
guidelines for linguists working in multilingual environments.

Our aims for adopting a Unicode-based
solution are: (i) to improve the consistency of the encoding of sources, (ii)
to transparently document knowledge about the writing system (including
transliteration), and (iii) to do all of that in a way that is easy and quick to
manage for many different sources with many different writing systems. The
central concept in our proposal is the \textsc{orthography profile}, a simple
delimited text file, that characterizes and documents a writing system.
We also offer basic implementations in Python and R to assist with the
production of such files, and to apply orthography profiles for consistency
testing, grapheme tokenization and transliteration. Not only can orthography
profiles be helpful in the daily practice of linguistics, they also succinctly
document the orthographic details of a specific source, and, as such, might
fruitfully be published alongside sources (e.g.~in digital archives). Also, in
high-level linguistic analyses in which the graphemic detail is of central
importance (e.g.~phonotactic or comparative-historical studies), orthography
profiles can transparently document the decisions that have been taken in the
interpretation of the orthography in the sources used.

Given these goals, Unicode Locales (see
Chapter~\ref{the-unicode-approach}) might seem like the ideal orthography
profiles. However, there are various practical obstacles preventing the use of
Unicode Locales in the daily linguistic practice, namely: (i) the
XML structure\footnote{\url{http://unicode.org/reports/tr35/}} is too verbose to easily and quickly produce or correct manually,
(ii) Unicode Locales are designed for a wide scope of information (like date
formats or names of weekdays) most of which is not applicable for documenting
writing systems, and (iii) most crucially, even if someone made the effort to
produce a technically correct Unicode Locale for a specific source at hand,
then it is well-nigh impossible to deploy the description. This is because a locale
description has to be submitted to and accepted by the Unicode Common Locale
Data Repository. The repository is (rightly so) not interested in descriptions
that only apply to a limited set of sources (e.g.~descriptions for only a single
dictionary).

The major challenge, then, is developing an infrastructure to identify the
elements that are individual graphemes in a source, specifically for the
enormous variety of sources using some kind of alphabetic writing system.
Authors of source documents (e.g.~dictionaries, wordlists, corpora) use a
variety of writing systems that range from their own idiosyncratic
transcriptions to already well-established practical or longstanding
orthographies. Although the IPA is one practical choice as a sound-based
normalization for writing systems (which can act as an interlingual pivot to
attain interoperability across writing systems), graphemes in each writing
system must also be identified and standardized if interoperability across
different sources is to be achieved. In most cases, this amounts to more than
simply mapping a grapheme to an IPA segment because graphemes must first be
identified in context (e.g.~is the sequence one sound or two sounds or both?)
and strings must be tokenized, which may include taking orthographic rules into
account (e.g.~a nasal sound may be transcribed as <n> when it appears between 
two vowels, but when it appears between a vowel and a consonant it becomes 
a nasalized vowel <Ṽ>).

In our experience, data from each source must be
individually tokenized into graphemes so that its orthographic structure can be 
identified and its contents can be extracted. To extract data for analysis, a
source-by-source approach is required before an orthography profile can be
created. For example, almost every available lexicon on the world's languages is
idiosyncratic in its orthography and thus requires lexicon-specific approaches
to identify graphemes in the writing system and to map graphemes to phonemes, if
desired.

Our key proposal for the characterization of a writing system is to use a
grapheme tokenization as an inter-orthographic pivot. Basically, any source
document is tokenized by graphemes, and only then a mapping to IPA (or any other
orthographic transliteration) is performed. An \textsc{orthography profile} then is a
description of the units and rules that are needed to adequately model a
graphemic tokenization for a language variety as described in a particular
source document. An orthography profile summarizes the Unicode (tailored)
graphemes and orthographic rules used to write a language (the details of the
structure and assumptions of such a profile will be presented in the next
section).

% TODO: add tsoshi figure here?

As an example of graphemic tokenization, note the three different levels of
technical and linguistic elements that interact in the hypothetical lexical
form <tsʰṍ̰shi>:

\begin{enumerate}
	\def\labelenumi{\arabic{enumi}.} 
	\item code points (10 text elements): t s ʰ o \dia{0303} \dia{0330} \dia{0301} s h i 
	\item grapheme clusters (7 text elements): t s ʰ ṍ̰ s h i 
	\item tailored grapheme clusters (4 text elements): tsʰ ṍ̰ sh i 
\end{enumerate}

In (1), the string <tsʰṍ̰shi> has been tokenized into ten Unicode code points
(using NFD normalization), delimited here by space. Unicode normalization is
required because sequences of code points can differ in their visual and logical
orders. For example, <õ̰> is ambiguous to whether it is the sequence of <o> +
<\dia{0303}> + <\dia{0330}> or <o> + <\dia{0330}> + <\dia{0303}>. Although these two variants are visually homoglyphs,
computationally they are different (see Sections \ref{pitfall-ipa-homoglyphs} \& \ref{pitfall-homoglyphs-in-IPA}). 
Unicode normalization should be applied to
this string to reorder the code points into a canonical order, allowing the data
to be treated  for search and comparison. 

In (2), the
Unicode code points have been logically normalized and visually organized into
grapheme clusters, as specified by the Unicode Standard. The combining character
sequence <õ̰> is normalized and visually grouped together. Note that the
\textsc{modifier letter small h} at \uni{02B0} is not grouped with any 
other character. This is because it
belongs to the Spacing Modifier Letters category. The Unicode Standard 
does not specify the direction that these characters modify a host
character. For example, it can indicate either pre- or post-aspiration (whereas the
nasalization or creaky diacritic is defined in the Unicode Standard to apply to
a specified base character). 

Finally, to arrive at the graphemic tokenization in (3), tailored grapheme
clusters are needed, possibly as specified in an orthography profile. For example,
an orthography profile might specify that the sequence of characters <tsʰ> form
a single grapheme. The orthography profile could also specify orthographic
rules, e.g.~when tokenizing graphemes in English, the sequences <sh> in the
forms <mishap> and <mishmash> should be treated as distinct sequences depending
on their contexts.

\section{Informal description}
\label{informal-description-of-orthography-profiles}

An orthography profile describes the Unicode code points, characters, graphemes
and orthographic rules in a writing system. An orthography profile is a
language-specific (and often even resource-specific) description of the units
and rules that are needed to adequately model a writing system. An important
assumption is that we assume a resource is encoded in Unicode or
has been converted to Unicode. Any data source that the Unicode Standard is
unable to capture will also not be captured by an orthography profile.

Informally, an orthography profile specifies the graphemes -- in Unicode
parlance \textsc{tailored grapheme clusters} -- that are expected to occur in any
data to be analyzed or checked for consistency. These graphemes are first
identified throughout the whole data, a step which we call
\textsc{tokenization}, and simply returned as such, possibly including
error messages about any parts of the data that are not specified by the
orthography profile. Once the graphemes are identified, they might also be
changed into other graphemes -- a step which we call \textsc{transliteration}.
When a grapheme has different possible transliterations, then these differences
should be separated by contextual specification, possibly down to listing
individual exceptional cases.

The crucial difference between our current proposal and traditional
computational approaches to transliteration is the strict separation between
tokenization and transliteration. Most computational approaches to
transliteration are based on finite-state transducers (including the
transliteration as described in the Unicode Locale Data Markup
Language).\footnote{\url{http://www.unicode.org/reports/tr35/}} Finite-state 
transducers attempt to describe the mapping from input to output string directly 
as a set of rewrite rules. Although such systems are computationally well 
understood, we feel that they are not well-suited for day-to-day linguistic 
practice. First, by forcing a first step of grapheme tokenization, our system 
tries to keep close to the logic of the writing system. Second, by separating 
tokenization from transliteration there is no problem with `feeding' and 
`bleeding' of rules, common with transducers (cf.~Section~\ref{r-implementation}).

Note that to deal with ambiguous parsing cases, it is still possible to use the
Unicode approach of including the \textsc{zero-width non-joiner} character at
\uni{200C} into the text. The idea is to add this character into the text to
identify cases in which a sequence of characters is \textit{not} supposed to be
a complex grapheme cluster -- even though the sequence is in the orthography
profile.

In practice, we foresee a workflow in which orthography profiles are iteratively
refined, while at the same time inconsistencies and errors in the data to be
tokenized are corrected. In some more complex use cases there might even be a
need for multiple different orthography profiles to be applied in sequence (see
Sections~\ref{python-implementations} \& \ref{r-implementation} on various exemplary use cases). The result of any such
workflow will normally be a cleaned dataset and an explicit description of the
orthographic structure in the form of an orthography profile. Subsequently, the
orthography profiles can be easily distributed in scholarly channels alongside
the cleaned data, for example in supplementary material added to journal papers
or in electronic archives.

\section{Formal specification}
\label{formal-specification-of-orthography-profiles}

\subsection*{File Format}
The formal specifications of an orthography profile (or simply \textsc{profile}
for short) are the following:

\begin{enumerate}
	\def\labelenumi{A\arabic{enumi}.} 
	\item \textsc{A profile is a unicode utf-\scalebox{0.8}{8} encoded text file} 
	   % normalized following NFC (or NFD if specified in the metadata)
       that includes information pertinent to the orthography.\footnote{See 
	   Section~\ref{pitfall-file-formats} in which we suggest to use NFC, 
	   no-BOM and LF line breaks because of the pitfalls they avoid. Specifying 
	   a convention for line endings and BOM is often overly strict because most 
	   computing environments (now) transparently handle both alternatives. 
	   For example, using Python a file can be decoded using the encoding 
	   ``utf-8-sig'', which strips away the BOM (if present) and reads 
	   an input full in text mode, so that both line feed variants ``LF'' and 
	   ``CRLF'' will be stripped. However, note that most shells (e.g. bash) will not 
	   behave properly with CRLF line endings.}

	\item \textsc{A profile is a delimited text file with an obligatory header
       line}. A minimal profile must have a single column with the header \texttt{Grapheme}. 
	   For any additional columns, the name in the header 
       must be specified. The actual ordering of the columns is unimportant. 
	   The header list must be delimited 
	   in the same way as the rest of the file's contents. Each record must be kept 
	   on a separate line. Separate lines with comments are not allowed. Comments that
       belong to specific lines must be put in a separate column of
       the file, e.g.~add a column called \textsc{comments}.

\begin{comment}	   
	\item \textsc{Metadata should be added in a separate utf-\scalebox{0.8}{8} text file} using a basic
       \textsc{tag: value} format. Metadata about the orthographic description
       given in the orthography profile includes, minimally, (i) author, (ii)
       date, (iii) title of the profile, (iv) a stable language identifier
       encoded in BCP 47/ISO 639-3 of the target language of the profile, and (v)
       bibliographic data for resource(s) that illustrate the orthography
       described in the profile. Further, (vi) the tokenization method and (vii)
       the Unicode normalization used should be documented here (see below).
\end{comment}

	\item \textsc{Metadata should be added in a separate utf-\scalebox{0.8}{8} text file} using 
	   the JSON-LD dialect specified in \textit{Metadata Vocabulary for Tabular 
	   Data}.\footnote{\url{https://www.w3.org/TR/tabular-metadata/}} This metadata 
	   format allows for easy inclusion of Dublin Core metadata,\footnote{\url{http://dublincore.org/}} 
	   which should be used to specify information about the orthographic description 
	   in the orthography profile.\footnote{\url{http://w3c.github.io/csvw/metadata/\#dfn-common-property}} 
	   The orthography profile metadata should minimally include provenance information including: 
	   (i) author, (ii) date, (iii) title of the profile, and (iv)
       bibliographic data for resource(s) that illustrate the orthography
       described in the profile. Crucially, the metadata should also specify 
	   (v) a stable language identifier of the target language of the profile
       using BCP 47/ISO 639-3 or Glottocode as per the CLDF ontology.\footnote{\url{http://cldf.clld.org/v1.0/terms.rdf}}	   	   
	   Further, the metadata file should provide information about the orthography profile's structure and contents, 
	   including: (vi) its dialect description,\footnote{\url{http://w3c.github.io/csvw/metadata/\#dfn-dialect-descriptions}} 
	   and (vii) proper column descriptions,\footnote{\url{http://w3c.github.io/csvw/metadata/\#dfn-datatype-description}} 
	   which describe how a column should be interpreted and processed (e.g.\ whether they 
	   should be processed as regular expressions; see below).
	   Finally, in accordance with the \textit{Metadata Vocabulary for Tabular 
	   Data}, the metadata's filename should consist of the orthography 
	   profile's filename appended with ``-metadata.json''.\footnote{JSON-LD metadata 
	   is also the choice for datasets conforming to the Cross-Linguistic Data Formats standard, 
	   see: \url{http://cldf.clld.org/}.}

\end{enumerate}

\noindent The content of a profile consists of lines, each describing a grapheme
of the orthography, using the following columns:

\begin{enumerate}
	\def\labelenumi{A\arabic{enumi}.} \setcounter{enumi}{4} 
	\item \textsc{A minimal profile consists of a single column} with a header
       called \texttt{Grapheme}, listing each of the different graphemes in a
       separate line. The name of this column is crucial for automatic 
       processing.
	\item \textsc{Optional columns can be used to specify the left and right
       context of the grapheme}, to be designated with the headers \texttt{Left}
       and \texttt{Right} respectively. The same grapheme can occur multiple
       times with different contextual specifications, for example to
       distinguish different pronunciations depending on the context. 
	\item \textsc{The columns \texttt{Grapheme}, \texttt{Left} and \texttt{Right}
       can use regular expression \\ metacharacters.} If regular expressions are
       used, then they must be specified in the metadata file as such, 
	   and all literal usage of the special symbols, like full stops <.>
       or dollar signs <\$> (so-called \textsc{metacharacters}) have to be
       explicitly escaped by adding a backslash before them (i.e.~use
       <\textbackslash.> or <\textbackslash\$>). Note that any specification of
       context automatically expects regular expressions, so it is 
       better to always escape all regular expression metacharacters when used
       literally in the orthography. The following symbols will need to be
       preceded by a backslash: {[} {]} ( ) \{ \} | ~+ * . - ! ? \^{} \$ and the
       backslash \textbackslash~itself. 
	\item \textsc{An optional column can be used to specify classes of graphemes},
       to be identified by the header \texttt{Class}. For example, this column
       can be used to define a class of vowels. Users can simply add ad-hoc
       identifiers in this column to indicate a group of graphemes, which can
       then be used in the description of the graphemes or the context. The
       identifiers should of course be chosen so that they do not conflate
       with any symbols used in the orthography. Note that such
       classes only refer to the graphemes, not to the context. 
	\item \textsc{Columns describing transliterations for each graphemes can be
       added and named at will}. Often more than a single possible
       transliteration will be of interest. Any software application using these
       profiles should prompt the user to name any of these columns to select a
       specific transliteration. 
	\item \textsc{Any other columns can be added freely, but will be typically ignored
       by any software application using the profiles}. As orthography profiles
       are also intended to be read and interpreted by humans, it is often
       very useful to add extra information about the graphemes in further
       columns, such as Unicode code points, Unicode names, frequency of
       occurrence, examples of occurrence, explanation of contextual
       restrictions, or comments. 
 \end{enumerate}

\subsection*{Processing}
For the automated processing of the profiles, the following technical standards
will be expected:

\begin{enumerate}
	\def\labelenumi{B\arabic{enumi}.} 
	\item \textsc{Each line of a profile will be interpreted according to the content type of the column as specified in the profile metadata}. Content types include literal and regular expression.
	\item \textsc{The \textsc{class} column will be used to produce explicit
       \textsc{or} chains of regular expressions}, which will then be inserted
       in the \texttt{Grapheme}, \texttt{Left} and \texttt{Right} columns at
       the position indicated by the class-identifiers. For example, a class
       called \texttt{V} as a context specification might be replaced by a regular
       expression like:
       \texttt{(au\textbar{}ei\textbar{}a\textbar{}e\textbar{}i\textbar{}o\textbar{}u}).
       Only the graphemes themselves are included here, not any contexts
       specified for the elements of the class. Note that the 
       ordering inside this regular expression is crucial (e.g.\ regular expressions are greedy, so
	   longest matches should be placed before matching substrings).
	\item \textsc{The \textsc{left} and \textsc{right} contexts will be included
       into the regular expressions by using lookbehind and lookahead}.
       Basically, the actual regular expression syntax of lookbehind and
       lookahead is simply hidden to the users by allowing them to only specify
       the contexts themselves. Internally, the contexts in the columns
       \texttt{Left} and \texttt{Right} are combined with the column
       \texttt{Grapheme} to form a complex regular expression like:\\ 
       \texttt{(?\textless{}=Left)Grapheme(?=Right)}. 
	\item \textsc{The regular expressions will be applied in the order as specified
       in the profile, from top to bottom.} A software implementation can offer
       help in figuring out the optimal ordering of the regular expressions, but
       then it should be made explicit in the orthography profile because regular 
	   expressions are executed in order from top to bottom.
\end{enumerate}

\noindent The actual implementation of the profile on some text-string will function as
follows:

\begin{enumerate}
	\def\labelenumi{B\arabic{enumi}.} \setcounter{enumi}{4} 
	\item \textsc{All graphemes are matched in the text before they are tokenized
       or transliterated}. In this way, there is no necessity for the user to
       consider \textit{feeding} and \textit{bleeding} situations, in which the application of
       a rule either changes the text so another rule suddenly applies (feeding)
       or prevents another rule from applying (bleeding). 
	\item \textsc{The matching of the graphemes can occur either globally or
       linearly.} From a computer science perspective, the most natural way to
       match graphemes from a profile in some text is by walking linearly
       through the text-string from left to right, and at each position going
       through all graphemes in the profile to see which one matches, then go to
       the position at the end of the matched grapheme and start over. This is
       basically how a finite state transducer works, which is a
       well-established technique in computer science. However, from a
       linguistic point of view, our experience is that most linguists find it
       more natural to think from a global perspective. In this approach, the
       first grapheme in the profile is matched everywhere in the text-string
       first, before moving to the next grapheme in the profile. Theoretically,
       these approaches will lead to different results, though in practice of
       actual natural language orthographies they almost always lead to the same
       result. Still, we suggest that any software application using orthography
       profiles should offer both approaches (i.e.\ \textsc{global} or
       \textsc{linear}) to the user. The approach used should be documented in
       the metadata as \textsc{tokenization method}. 
	\item \textsc{The matching of the graphemes can occur either in nfc or nfd.} 
	   The Unicode Standard states that software is free to compose or decompose the character stream from one 
	   representation to another. However, Unicode conformant software must treat canonically equivalent sequences in 
	   NFC and NFD as the same. It is up to the orthography profile creator how they choose to encode their profile. 
	   Several sources suggest to use NFC when possible for text encoding,\footnote{\url{http://www.win.tue.nl/~aeb/linux/uc/nfc_vs_nfd.html}} 
	   including SIL International with regard to data archiving.\footnote{\url{http://scripts.sil.org/cms/scripts/page.php?item_id=NFC_vs_NFD}} 
	   In our experience, in some use cases it turns out to
       be practical to treat both text and profile as NFD. This typically
       happens when many different combinations of diacritics occur in the
       data. An NFD profile can then be used to first check which individual
       diacritics are used, before turning to the more cumbersome inspection of
       all combinations. We suggest that any software application using
       orthography profiles should offer both approaches (i.e.\ \textsc{NFC} or
       \textsc{NFD}) to the user. The approach used can be documented in the
       metadata as \textsc{unicode normalization}. 
	\item \textsc{The text-string is always returned in tokenized form} by
       separating the matched graphemes by a user-specified symbols-string. Any
       transliteration will be returned on top of the tokenization. 
	\item \textsc{Leftover characters}, i.e.~\textsc{characters that are not
       matched by the profile, should be reported to the user as errors.}
       Typically, the unmatched characters are replaced in the tokenization by a
       user-specified symbol-string.       
 \end{enumerate}

\subsection*{Software applications}

Any software application offering to use orthography profile:

\begin{enumerate}
	\def\labelenumi{\arabic{enumi}.} 
	\item \textsc{should offer user-options} to specify:
	\begin{enumerate}
		\def\labelenumii{C\arabic{enumii}.} 
		\item \textsc{the name of the column to be used for transliteration} (if any). 
		\item \textsc{the symbol-string to be inserted between graphemes.} Optionally,
        a warning might be given if the chosen string includes characters from
        the orthography itself. 
		\item \textsc{the symbol-string to be inserted for unmatched strings} in the
        tokenized and transliterated output. 
		\item \textsc{the tokenization method}, i.e.~whether the tokenization should
        proceed as \textsc{global} or \textsc{linear} (see B6 above). 
		\item \textsc{unicode normalization}, i.e.~whether the text-string and profile
        should use \textsc{NFC} or \textsc{NFD}. 
    \end{enumerate}
	\item \textsc{might offer user-options}:
	\begin{enumerate}
		\def\labelenumii{C\arabic{enumii}.} \setcounter{enumii}{5} 
		\item \textsc{to assist in the ordering of the graphemes.} In our experience 
		working with idiosyncratic transcriptions and orthographies from low-resource languages, it
        is helpful to identify multi-sequence graphemes before single graphemes, and to
        identify graphemes with context before graphemes without context. Further,
        frequently relevant rules might be applied after rarely relevant rules
        (though frequency is difficult to establish in practice, as it depends
        on the available data). Also, if this all fails to give any decisive
        ordering between rules, it seems useful to offer linguists the option to
        reverse the ordering from any manual specified ordering, because
        linguists tend to write the more general rule first, before turning to
        exceptions or special cases. 
		\item \textsc{to assist in dealing with upper and lower case characters.} It
        seems practical to offer some basic case matching, so characters like
        <a> and <A> are treated equally. This will be useful in many concrete
        cases (such as search or collation), although the user should be warned that case matching does not
        function universally in the same way across orthographies.\footnote{For example 
		compare the different first-letter capitalization practices of the digraphs 
		<Nj> and <IJ> (single-character ligatures in the Unicode Standard) 
		in the Latin-based scripts of Southern-Slavic languages and 
		Dutch, respectively.} Ideally,
        users should prepare orthography profiles with all lowercase and
        uppercase variants explicitly mentioned, so by default no case matching
        should be performed. 
		\item \textsc{to treat the profile literally}, i.e.~to not interpret regular
        expression metacharacters. Matching graphemes literally often leads to
        significant speed increase, and ensures that users do not have to worry
        about escaping metacharacters. However, in our experience all actually
        interesting use cases of orthography profiles include some contexts,
        which automatically prevents any literal interpretation.
    \end{enumerate}
	\item \textsc{should return the following information} to the user:
	\begin{enumerate}
		\def\labelenumii{C\arabic{enumii}.} \setcounter{enumii}{8} 
		\item \textsc{the original text-strings} to be processed in the specified
        Unicode normalization, i.e.~in either NFC or NFD as specified by the
        user. 
		\item \textsc{the tokenized strings}, with additionally any transliterated
        strings, if transliteration is requested. 
		\item \textsc{a survey of all errors encountered}, ideally both (i) in which
        text-strings any errors occurred and (ii) which characters in the
        text-strings lead to errors. 
		\item \textsc{a reordered profile}, when any automatic reordering is offered. 
	\end{enumerate}
\end{enumerate}



\chapter{Implementation}
\label{implementation}

\section{Overview}
To illustrate the practical applications of orthography profiles, we have implemented two versions of the specifications presented in Chapter~\ref{orthography-profiles}: one in Python\footnote{\url{https://pypi.python.org/pypi/segments}} and one in R.\footnote{\url{https://github.com/cysouw/qlcData}} In this chapter, we introduce these two software libraries and provide practical step-by-step guidelines for installing and using them. Various simple and sometimes somewhat abstract examples will be discussed to show the different options available, and to illustrate the intended usage of orthography profiles in general. 

Note that our two libraries have rather different implementation histories, thus they may not give the same results in all situations (as discussed in Chapter~\ref{orthography-profiles}). However, we do provide extensive test suites for each implementation that follow standard practices to make sure that results are correct. Users should refer to these tests and to the documentation in each release for specifics about each implementation. Note that due to the different naming convention practices in Python and R, function names differ between the two libraries. Also, the performance with larger datasets may not be comparable between the Python and R implementations. In sum, our two libraries should be considered as proofs of concept and not as the final word on the practical application of the specifications discussed in the previous chapter. In our experience, the current versions are sufficiently fast and stable to be useful for academic practice (e.g.\ checking data consistency, or analyzing and transliterating small to medium sized data sets), but they should probably not be used for full-scale industry applications without adaptation.

First, in Section \ref{installing-python-and-r} we explain how to install Python\footnote{\url{https://www.python.org/}} and R.\footnote{\url{https://www.r-project.org/}} Then in Sections \ref{python-implementations} \& \ref{r-implementation}, we discuss our Python and R software packages, respectively. In addition to the material presented here to get users started, we maintain several case studies online that illustrate how to use orthography profiles in action. For convenience, we make these recipes available as Jupyter Notebooks\footnote{\url{http://jupyter.org/}} in our GitHub repository.\footnote{\url{https://github.com/unicode-cookbook/}} In the final section in this chapter, we also briefly describe a few recipes that we do not go into detail in this book.

\section{How to install Python and R}
\label{installing-python-and-r}

When one encounters problems installing software, or bugs in programming code, search engines are your friend! Installation problems and incomprehensible error messages have typically been encountered and solved by other users. Try simply copying and pasting the output of an error message into a search engine; the solution is often already somewhere online. We are fans of Stack Exchange\footnote{\url{https://stackexchange.com/}} -- a network of question-and-answer websites -- which are extremely helpful in solving issues regarding software installation, bugs in code, etc.

Searching the web for ``install r and python'' returns numerous tutorials on how to set up your machine for scientific data analysis. Note that there is no single correct setup for a particular computer or operating system. Both Python and R are available for Windows, Mac, and Linux operating systems from the Python and R project websites. Another option is to use a so-called package manager, i.e.\ a software program that allows the user to manage software packages and their dependencies. On Mac, we use Homebrew,\footnote{\url{https://brew.sh/}} a simple-to-install (via the Terminal App) free and open source package management system. Follow the instructions on the Homebrew website and then use Homebrew to install R and Python (as well as other software packages such as Git and Jupyter Notebooks). 

Alternatively for R, RStudio\footnote{\url{https://www.rstudio.com/}} provides a free and open source integrated development environment (IDE). This application can be downloaded and installed (for Mac, Windows and Linux) and it includes its own R installation and R libraries package manager. For developing in Python, we recommend the free community version of PyCharm,\footnote{\url{https://www.jetbrains.com/pycharm/}} an IDE which is available for Mac, Windows, and Linux. 

Once you have R or Python (or both) installed on your computer, you are ready to use the orthography profiles software libraries presented in the next two sections. As noted above, we make this material available online on GitHub,\footnote{\url{https://github.com/}} a web-based version control system for source code management. GitHub repositories can be cloned or downloaded,\footnote{\url{https://help.github.com/articles/cloning-a-repository/}} so that you can work through the examples on your local machine. Use your favorite search engine to figure out how to install Git on your computer and learn more about using Git.\footnote{\url{https://git-scm.com/}} In our GitHub repository, we make the material presented below (and more use cases described briefly in Section \ref{use-cases}) available as Jupyter Notebooks. Jupyter Notebooks provide an interface where you can run and develop source code using the browser as an interface. These notebooks are easily viewed in our GitHub repository of use cases.\footnote{\url{https://github.com/unicode-cookbook/recipes}}



\section{Python package: segments}
\label{python-implementations}

The Python package \texttt{segments} is available both as a command line interface (CLI) and as an application programming interface (API).


\subsection*{Installation}

To install the Python package \texttt{segments} \citep{ForkelMoran2018} from the Python Package Index (PyPI) run:

\begin{lstlisting}[language=bash, basicstyle=\myfont]
  $ pip install segments
\end{lstlisting}

\noindent on the command line. This will give you access to both the CLI and programmatic functionality in Python scripts, when you import the \texttt{segments} library.

You can also install the \texttt{segments} package from the GitHub repository,\footnote{\url{https://github.com/cldf/segments}} in particular if you would like to contribute to the code base:\footnote{\url{https://github.com/cldf/segments/blob/master/CONTRIBUTING.md}}

\begin{lstlisting}[language=bash, basicstyle=\myfont]
  $ git clone https://github.com/cldf/segments
  $ cd segments
  $ python setup.py develop
\end{lstlisting}


\subsection*{Application programming interface}
The \texttt{segments} API can be accessed by importing the package into Python. Here is an example of how to import the library, create a tokenizer object, tokenize a string, and create an orthography profile. Begin by importing the \texttt{Tokenizer} from the \texttt{segments} library.

\begin{lstlisting}[basicstyle=\myfont]
>>> from segments.tokenizer import Tokenizer
\end{lstlisting}

\noindent Next, instantiate a tokenizer object, which takes optional arguments for an orthography profile and an orthography profile rules file.

\begin{lstlisting}[basicstyle=\myfont]
>>> t = Tokenizer()
\end{lstlisting}

\noindent The default tokenization strategy is to segment some input text at the Unicode Extended Grapheme Cluster boundaries,\footnote{\url{http://www.unicode.org/reports/tr18/tr18-19.html\#Default_Grapheme_Clusters}} and to return, by default, a space-delimited string of graphemes. White space between input string sequences is by default separated by a hash symbol <\#>, which is a linguistic convention used to denote word boundaries. The default grapheme tokenization is useful when you encounter a text that you want to tokenize to identify potential orthographic or transcription elements.

\begin{lstlisting}[extendedchars=false, escapeinside={(*@}{@*)}, basicstyle=\myfont]
>>> result = t('(*@ĉháɾã̌ctʼɛ↗ʐː| k͡p@*)')
>>> print(result)
>>> '(*@ĉ h á ɾ ã̌ c t ʼ ɛ ↗ ʐ ː | \# k͡ p@*)'
\end{lstlisting}

\begin{lstlisting}[extendedchars=false, escapeinside={(*@}{@*)}, basicstyle=\myfont]
>>> result = t('(*@ĉháɾã̌ctʼɛ↗ʐː| k͡p@*)', segment_separator='(*@-@*)')
>>> print(result)
>>> '(*@ĉ-h-á-ɾ-ã̌-c-t-ʼ-ɛ-↗-ʐ-ː-| \# k͡ -p@*)'
\end{lstlisting}

\begin{lstlisting}[extendedchars=false, escapeinside={(*@}{@*)}, basicstyle=\myfont, showstringspaces=false]
>>> result = t('(*@ĉháɾã̌ctʼɛ↗ʐː| k͡p@*)', separator=' // '))
>>> print(result)
>>> '(*@ĉ h á ɾ ã̌ c t ʼ ɛ ↗ ʐ ː | // k͡ p@*)'
\end{lstlisting}

\noindent The optional \texttt{ipa} parameter forces grapheme segmentation for IPA strings.\footnote{\url{https://en.wikipedia.org/wiki/International\_Phonetic\_Alphabet}} Note here that Unicode Spacing Modifier Letters,\footnote{\url{https://en.wikipedia.org/wiki/Spacing\_Modifier\_Letters}} such as <ː> and <\dia{0361}{\large\fontspec{CharisSIL}◌}>, will be segmented together with base characters (although you might need orthography profiles and rules to correct these in your input source; see Section \ref{pitfall-different-notions-of-diacritics} for details).

\begin{lstlisting}[extendedchars=false, escapeinside={(*@}{@*)}, basicstyle=\myfont]
>>> result = t('(*@ĉháɾã̌ctʼɛ↗ʐː| k͡p@*)', ipa=True)
>>> print(result)
>>> '(*@ĉ h á ɾ ã̌ c t ʼ ɛ ↗ ʐː | \# k͡p@*)'
\end{lstlisting}

\noindent You can also load an orthography profile and tokenize input strings with it. In the data directory,\footnote{https://github.com/unicode-cookbook/recipes/tree/master/Basics/data} we've placed an example orthography profile. Let's have a look at it using \texttt{more} on the command line.


\begin{lstlisting}[extendedchars=false, escapeinside={(*@}{@*)}, basicstyle=\myfont, showstringspaces=false]
  $ more data/orthography-profile.tsv
  Grapheme  IPA   XSAMPA  COMMENT
  a         a     a
  aa        (*@aː@*)    (*@a:@*)
  b         b     b
  c         c     c
  ch        (*@tʃ@*)    tS
  (*@-@*)         NULL  NULL    "comment with   tab"
  on        (*@õ@*)     o~
  n         n     n
  ih        (*@í@*)     i_H
  inh       (*@ĩ́@*)     i~_H
\end{lstlisting}


\noindent An orthography profile is a delimited UTF-8 text file (here we use tab as a delimiter for reading ease). The first column must be labeled \texttt{Grapheme}, as discussed in Section \ref{formal-specification-of-orthography-profiles}. Each row in the \texttt{Grapheme} column specifies graphemes that may be found in the orthography of the input text. In this example, we provide additional columns \texttt{IPA} and \texttt{XSAMPA}, which are mappings from our graphemes to their IPA and X-SAMPA transliterations. The final column \texttt{COMMENT} is for comments; if you want to use a tab ``quote that     string''!

Let's load the orthography profile with our tokenizer.

\begin{lstlisting}[basicstyle=\myfont]
>>> from segments.tokenizer import Profile
>>> t = Tokenizer('data/orthography-profile.tsv')
\end{lstlisting}

\noindent Now let's segment the graphemes in some input strings with our orthography profile. The output is segmented given the definition of graphemes in our orthography profile, e.g.\ we specified the sequence of two <a a> should be a single unit <aa>, and so should the sequences <c h>, <o n> and <i h>.


\begin{lstlisting}[extendedchars=false, escapeinside={(*@}{@*)}, basicstyle=\myfont]
>>> t('(*@aabchonn-ih@*)')
>>> '(*@aa b ch on n - ih@*)'
\end{lstlisting}

\noindent This example shows how we can tokenize input text into our orthographic specification. We can also segment graphemes and transliterate them into other forms, which is useful when you have sources with different orthographies, but you want to be able to compare them using a single representation like IPA or X-SAMPA.

\begin{lstlisting}[extendedchars=false, escapeinside={(*@}{@*)}, basicstyle=\myfont]
>>> t('(*@aabchonn-ih@*)', column='IPA')
>>> '(*@aː b tʃ õ n í@*)'
\end{lstlisting}

% For some reason that fails me, this does not work:
% \begin{lstlisting}[extendedchars=false, escapeinside={(*@}{@*)}]
% >>> t.transform('aabchonn-ih', 'XSAMPA')
% >>> '(*@a: b tS o~ n i_H@*)'
% \end{lstlisting}

\begin{lstlisting}[basicstyle=\myfont, showstringspaces=false, escapeinside={(*@}{@*)}]
>>> t('aabchonn(*@-@*)ih', column='XSAMPA')
>>> 'a: b tS o~ n i_H'
\end{lstlisting}


\noindent It is also useful to know which characters in your input string are not in your orthography profile. By default, missing characters are displayed with the Unicode \textsc{replacement character} at \uni{FFFD}, which appears below as a white question mark within a black diamond.

\begin{lstlisting}[extendedchars=false, escapeinside={(*@}{@*)}, basicstyle=\myfont, language=bash]
>>> t('(*@aa b ch on n - ih x y z@*)')
>>> '(*@aa b ch on n - ih � � �@*)'
\end{lstlisting}

\noindent You can change the default by specifying a different replacement character when you load the orthography profile with the tokenizer.

\begin{lstlisting}[basicstyle=\myfont, extendedchars=false, escapeinside={(*@}{@*)}, showstringspaces=false]
>>> t = Tokenizer('data/orthography(*@-@*)profile.tsv', 
	errors_replace=lambda c: '?')
>>> t('aa b ch on n (*@-@*) ih x y z')
>>> 'aa b ch on n (*@-@*) ih ? ? ?'
\end{lstlisting}

\begin{lstlisting}[basicstyle=\myfont, extendedchars=false, escapeinside={(*@}{@*)}, showstringspaces=false]
>>> t = Tokenizer('data/orthography(*@-@*)profile.tsv', 
	errors_replace=lambda c: '<{0}>'.format(c))
>>> t('aa b ch on n (*@-@*) ih x y z')
>>> 'aa b ch on n (*@-@*) ih <x> <y> <z>'
\end{lstlisting}

\noindent Perhaps you want to create an initial orthography profile that also contains those graphemes <x>, <y>, and <z>? Note that the space character and its frequency are also captured in this initial profile.

\begin{lstlisting}[basicstyle=\myfont, extendedchars=false, escapeinside={(*@}{@*)}, showstringspaces=false]
>>> profile = Profile.from_text('aa b ch on n (*@-@*) ih x y z')
>>> print(profile)
\end{lstlisting}


\begin{lstlisting}[language=bash, texcl=true, basicstyle=\myfont, extendedchars=false, escapeinside={(*@}{@*)}]
  Grapheme  frequency  mapping
            9
  a         2          a
  h         2          h
  n         2          n
  b         1          b
  c         1          c
  o         1          o
  (*@-@*)         1          (*@-@*)
  i         1          i
  x         1          x
  y         1          y
  z         1          z
\end{lstlisting}


\subsection*{Command line interface}

From the command line, access \texttt{segments} and its 
various arguments. For help, run:

\begin{lstlisting}[language=bash, basicstyle=\myfont, extendedchars=false, escapeinside={(*@}{@*)}]
  $ segments (*@-@*)h

usage: segments [(*@-@*)h] [(*@--@*)verbosity VERBOSITY] 
                     [(*@--@*)encoding ENCODING]
                     [(*@--@*)profile PROFILE]
                     [(*@--@*)mapping MAPPING]
                     command ...

Main command line interface of the segments package.

positional arguments:
  command               tokenize | profile
  args

optional arguments:
  (*@-@*)h, (*@--@*)help            show this help message and exit
  (*@--@*)verbosity VERBOSITY
                        increase output verbosity
  (*@--@*)encoding ENCODING   input encoding
  (*@--@*)profile PROFILE     path to an orthography profile
  (*@--@*)mapping MAPPING     column name in ortho profile to map 
                        graphemes

Use 'segments help <cmd>' to get help about individual commands.  
\end{lstlisting}

\noindent We have created some test data\footnote{\url{https://github.com/unicode-cookbook/recipes/tree/master/Basics/sources}} with the German word \textit{Schächtelchen}, which is the diminutive form of \textit{Schachtel}, meaning `box', `packet', or `carton' in English.

\begin{lstlisting}[language=bash, basicstyle=\myfont]
  $ more sources/german.txt

  Schächtelchen
\end{lstlisting}

\noindent We can create an initial orthography profile of the German text by passing it to the \texttt{segments profile} command. The initial profile tokenizes the text on Unicode grapheme clusters, lists the frequency of each grapheme, and provides an initial mapping column by default.

\begin{lstlisting}[language=bash, extendedchars=false, escapeinside={(*@}{@*)}, basicstyle=\myfont]
  $ cat sources/german.txt | segments profile

  Grapheme  frequency  mapping
  c         3          c
  h         3          h
  e         2          e
  S         1          S
  (*@ä@*)         1          (*@ä@*)
  t         1          t
  l         1          l
  n         1          n
\end{lstlisting}

\noindent Next, we know a bit about German orthography and which characters combine to form German graphemes. We can use the information from our initial orthography profile to hand-curate a more precise German orthography profile that takes into account capitalization (German orthography obligatorily capitalizes nouns) and grapheme clusters, such as <sch> and <ch>. We can use the initial orthography profile above as a starting point (note that, in large texts, the frequency column may signal errors in the input, such as typos, if a grapheme occurs with very low frequency). The initial orthography profile can be edited with a text editor or spreadsheet program. As per the orthography profile specifications (see Chapter \ref{orthography-profiles}), we can adjust rows in the \texttt{Grapheme} column and then add additional columns for transliterations or comments.


\begin{lstlisting}[language=bash, extendedchars=false, escapeinside={(*@}{@*)}, basicstyle=\myfont]
  $ more data/german-orthography-profile.tsv

  Grapheme  IPA  XSAMPA  COMMENT                     
  Sch       (*@ʃ@*)    S       German nouns are capitalized
  (*@ä@*)         (*@ɛː@*)   E:                                  
  ch        (*@ç@*)    C                                   
  t         t    t                                   
  e         e    e                                   
  l         l    l                                   
  n         n    n                                    
\end{lstlisting}

\noindent Using the command line \texttt{segments} function and passing it our orthography profile, we can now segment our German text example into graphemes.

\begin{lstlisting}[language=bash, extendedchars=false, escapeinside={(*@}{@*)}, basicstyle=\myfont]
  $ cat sources/german.txt | segments 
    (*@--@*)profile=data/german(*@-@*)orthography(*@-@*)profile.tsv tokenize

  '(*@Sch ä ch t e l ch e n@*)'
\end{lstlisting}

\noindent By providing \texttt{segments} a column for transliteration, we can convert the text into IPA.

\begin{lstlisting}[language=bash, extendedchars=false, escapeinside={(*@}{@*)}, basicstyle=\myfont]
  $ cat sources/german.txt | segments (*@--@*)mapping=IPA 
	(*@--@*)profile=data/german(*@-@*)orthography(*@-@*)profile.tsv tokenize

  '(*@ʃ ɛː ç t e l ç e n@*)'
\end{lstlisting}

\noindent And we can transliterate to X-SAMPA.

\begin{lstlisting}[language=bash, basicstyle=\myfont, extendedchars=false, escapeinside={(*@}{@*)}]
  $ cat sources/german.txt | segments (*@--@*)mapping=XSAMPA 
	(*@--@*)profile=data/german(*@-@*)orthography(*@-@*)profile.tsv tokenize

  'S E: C t e l C e n'
\end{lstlisting}

\noindent More examples are available online.\footnote{\url{https://github.com/unicode-cookbook/recipes}}


%%%%%%%%%%%%%%%%%%%%%%%%%%%%%%%%%%%%%%%%%%%%%%%%%%%%%%%%%%%%%%%%%
% Old example that reviewer didn't like!
\begin{comment}
\begin{lstlisting}[language=bash]
  $ more text.txt
  aäaaöaaüaa
\end{lstlisting}

\noindent Here is an example of how to create and use an orthography 
profile for segmentation. Create a text file:

\begin{lstlisting}[language=bash]
  $ more text.txt
  aäaaöaaüaa
\end{lstlisting}

\noindent Now look at the profile:

\begin{lstlisting}[language=bash,texcl=true]
  $ cat text.txt | segments profile
  Grapheme frequency mapping
  a        7         a
  ä        1         ä
  ü        1         ü
  ö        1         ö
\end{lstlisting}

\noindent Write the profile to a file:

\begin{lstlisting}[language=bash]
  $ cat text.txt | segments profile > profile.prf
\end{lstlisting}

\noindent Edit the profile:

\begin{lstlisting}[language=bash]
  $ more profile.prf
  Grapheme frequency mapping
  aa       0         x
  a        7         a
  ä        1         ä
  ü        1         ü
  ö        1         ö
\end{lstlisting}

\noindent Now tokenize the text without profile:

\begin{lstlisting}[language=bash]
  $ cat text.txt | segments tokenize
  a ä a a ö a a ü a a	
\end{lstlisting}

\noindent And with profile:
\begin{lstlisting}[language=bash]
  $ cat text.txt | segments --profile=profile.prf tokenize
  a ä aa ö aa ü aa

  $ cat text.txt | segments --mapping=mapping 
    --profile=profile.prf tokenize
  a ä x ö x ü x
\end{lstlisting}
\end{comment}
% \end old example that reviewer didn't like!
%%%%%%%%%%%%%%%%%%%%%%%%%%%%%%%%%%%%%%%%%%%%%%%%%%%%%%%%%%%%%%%%%


\section{R library: qlcData}
\label{r-implementation}

% \subsection*{Installing the R implementation}
\subsection*{Installation}

The R implementation is available in the package \texttt{qlcData} \citep{Cysouw2018}, which is 
directly available from the central R repository CRAN (Comprehensive R Archive 
Network). The R software environment itself has to be downloaded from its 
website.\footnote{\url{https://www.r-project.org}} After starting the included 
R program, the \texttt{qlcData} package for dealing with orthography profiles can be 
simply installed as follows:

\begin{knitrout}\footnotesize
\definecolor{shadecolor}{rgb}{1, 1, 1}\color{fgcolor}\begin{kframe}
\begin{alltt}
\hlcom{# download and install the qlcData software}
\hlkwd{install.packages}\hlstd{(}\hlstr{'qlcData'}\hlstd{)}
\hlcom{# load the software, so it can be used}
\hlkwd{library}\hlstd{(qlcData)}
\end{alltt}
\end{kframe}
\end{knitrout}

The version available through CRAN is the latest stable version.
To obtain the most recent bug-fixes and experimental additions, please use the
development version, which is available on
GitHub.\footnote{\url{http://github.com/cysouw/qlcData}} This development
version can be easily installed using the github-install helper software from the
\texttt{devtools} package.

\begin{knitrout}\footnotesize
\definecolor{shadecolor}{rgb}{1, 1, 1}\color{fgcolor}\begin{kframe}
\begin{alltt}
\hlcom{# download and install helper software}
\hlkwd{install.packages}\hlstd{(}\hlstr{'devtools'}\hlstd{)}
\hlcom{# install the qlcData package from GitHub}
\hlstd{devtools}\hlopt{::}\hlkwd{install_github}\hlstd{(}\hlstr{'cysouw/qlcData'}\hlstd{,} \hlkwc{build_vignettes} \hlstd{=} \hlnum{TRUE}\hlstd{)}
\hlcom{# load the software, so it can be used }
\hlkwd{library}\hlstd{(qlcData)}
\end{alltt}
\end{kframe}
\end{knitrout}

Inside the \texttt{qlcData} package, there are two functions for
orthography processing, \texttt{write.profile} and \texttt{tokenize}. The package includes
help files with illustrative examples, and also a so-called vignette with
explanations and examples.

\begin{knitrout}\footnotesize
\definecolor{shadecolor}{rgb}{1, 1, 1}\color{fgcolor}\begin{kframe}
\begin{alltt}
\hlcom{# view help files}
\hlkwd{help}\hlstd{(write.profile)}
\hlkwd{help}\hlstd{(tokenize)}
\hlcom{# view vignette with explanation and examples}
\hlkwd{vignette}\hlstd{(}\hlstr{'orthography_processing'}\hlstd{)}
\end{alltt}
\end{kframe}
\end{knitrout}

Basically, the idea is to use \texttt{write.profile} to produce a
basic orthography profile from some data and then \texttt{tokenize} to apply the
(possibly edited) profile on some data, as exemplified in the next section. This
can of course be performed though R, but additionally there are two more
interfaces to the R code supplied in the \texttt{qlcData} package: (i) \texttt{Bash}
executables and (ii) \texttt{Shiny} webapps.

The Bash executables are little files providing an interface to the R code that
can be used in a shell on a UNIX-like machine. The exact location of these
executables is best found after installation of R the packages. The
location can be found by the following command in R. 

\begin{knitrout}\footnotesize
\definecolor{shadecolor}{rgb}{1, 1, 1}\color{fgcolor}\begin{kframe}
\begin{alltt}
\hlcom{# show the path to the bash executables}
\hlkwd{file.path}\hlstd{(}\hlkwd{find.package}\hlstd{(}\hlstr{'qlcData'}\hlstd{),} \hlstr{'exec'}\hlstd{)}
\end{alltt}
\end{kframe}
\end{knitrout}

These executables can be 
used in the resulting file path, or they can be linked and/or copied to any location as wanted. 
For example, a good way to use the executables in a terminal is to
make softlinks (using \texttt{ln}) from the executables to a directory in your
PATH, e.g.\ to \texttt{/usr/local/bin/}. The two executables are named
\texttt{tokenize} and \texttt{writeprofile}, and the links can be made directly 
by using Rscript to get the paths to the executables within the terminal.

\begin{knitrout}\footnotesize
\definecolor{shadecolor}{rgb}{1, 1, 1}\color{fgcolor}\begin{kframe}
\begin{alltt}
pathT=`Rscript -e 'cat(system.file("exec/tokenize", package="qlcData"))'`
pathW=`Rscript -e 'cat(system.file("exec/writeprofile", package="qlcData"))'`
\end{alltt}
\end{kframe}
\end{knitrout}

Then you can make softlinks to the R executables in \texttt{/usr/local/bin} by using the following command in the terminal:

\begin{knitrout}\footnotesize
\definecolor{shadecolor}{rgb}{1, 1, 1}\color{fgcolor}\begin{kframe}
\begin{alltt}
sudo ln -is $pathT $pathW /usr/local/bin
\end{alltt}
\end{kframe}
\end{knitrout}

You can also do this within R by using the following commands, again possible replacing \texttt{/user/local/bin} with a suitable location on your system:

\begin{knitrout}\footnotesize
\definecolor{shadecolor}{rgb}{1, 1, 1}\color{fgcolor}\begin{kframe}
\begin{alltt}
\hlcom{# link executables from within R}
\hlkwd{file.symlink}\hlstd{(}\hlkwd{system.file}\hlstd{(}\hlstr{"exec/tokenize"}\hlstd{,} \hlkwc{package}\hlstd{=}\hlstr{"qlcData"}\hlstd{),} \hlstr{"/usr/local/bin"}\hlstd{)}
\hlkwd{file.symlink}\hlstd{(}\hlkwd{system.file}\hlstd{(}\hlstr{"exec/writeprofile"}\hlstd{,} \hlkwc{package}\hlstd{=}\hlstr{"qlcData"}\hlstd{),} \hlstr{"/usr/local/bin"}\hlstd{)}
\end{alltt}
\end{kframe}
\end{knitrout}

After inserting this softlink it should be possible to access the
\texttt{tokenize} function from the shell. Try \texttt{tokenize --help} to test
the functionality.

% TODO:
% <<size='scriptsize', engine='bash', tidy=FALSE>>=
% tokenize --help
% @

% TODO
% To make the functionality even more accessible, we have prepared webapps with 
% the \texttt{shiny} framework for the R functions. These webapps are available 
% online at \url{TODO}. The webapps are also included inside the \texttt{qlcData} 
% package and can be started with the following helper function:

% To make the functionality even more accessible, we have prepared webapps with 
% the \texttt{Shiny} framework for the R functions. The webapps are 
% included inside the \texttt{qlcData} package and can be started with the 
% helper function (in R): \texttt{launch\_shiny('tokenize')}.

% <<eval=FALSE>>=
% launch_shiny('tokenize')
% @



\subsection*{Profiles and error reporting}
\label{error-reporting}

The first example of how to use these functions concerns finding errors in the
encoding of texts. In the following example, it looks as if we have two
identical strings, \texttt{AABB}. However, this is just a surface-impression
delivered by the current font, which renders Latin and Cyrillic capitals
identically. We can identify this problem when we produce an orthography profile
from the strings. Using the R implementation of orthography profiles, we
first assign the two strings to a variable \texttt{test}, and then produce an
orthography profile with the function \texttt{write.profile}. As it turns out,
some of the letters are Cyrillic.

\begin{knitrout}\footnotesize
\definecolor{shadecolor}{rgb}{1, 1, 1}\color{fgcolor}\begin{kframe}
\begin{alltt}
\hlstd{( test} \hlkwb{<-} \hlkwd{c}\hlstd{(}\hlstr{'AABB'}\hlstd{,} \hlstr{'\textbackslash{}u0041\textbackslash{}u0410\textbackslash{}u0042\textbackslash{}u0412'}\hlstd{) )}
\end{alltt}
\begin{verbatim}
## [1] "AABB" "AАBВ"
\end{verbatim}
\begin{alltt}
\hlkwd{write.profile}\hlstd{(test)}
\end{alltt}
\begin{verbatim}
##   Grapheme Frequency Codepoint                UnicodeName
## 1        A         3    U+0041     LATIN CAPITAL LETTER A
## 2        B         3    U+0042     LATIN CAPITAL LETTER B
## 3        А         1    U+0410  CYRILLIC CAPITAL LETTER A
## 4        В         1    U+0412 CYRILLIC CAPITAL LETTER VE
\end{verbatim}
\end{kframe}
\end{knitrout}

The function of error-message reporting can also nicely be illustrated
with this example. Suppose we made an orthography profile with just the two
Latin letters <A> and <B> as possible graphemes, then this profile would not be
sufficient to tokenize the strings. There are graphemes in the data that are not
in the profile, so the tokenization produces an error, which can be used to fix
the encoding (or the profile). In the example below, we can see that the
Cyrillic encoding is found in the second string of the \texttt{test} input.

\begin{knitrout}\footnotesize
\definecolor{shadecolor}{rgb}{1, 1, 1}\color{fgcolor}\begin{kframe}
\begin{alltt}
\hlstd{test} \hlkwb{<-} \hlkwd{c}\hlstd{(}\hlstr{'AABB'}\hlstd{,} \hlstr{'\textbackslash{}u0041\textbackslash{}u0410\textbackslash{}u0042\textbackslash{}u0412'}\hlstd{)}
\hlkwd{tokenize}\hlstd{( test,} \hlkwc{profile} \hlstd{=} \hlkwd{c}\hlstd{(}\hlstr{'A'}\hlstd{,} \hlstr{'B'}\hlstd{) )}
\end{alltt}


{\ttfamily\noindent\color{warningcolor}{\#\# Warning in tokenize(test, profile = c("{}A"{}, "{}B"{})): \\\#\# There were unknown characters found in the input data.\\\#\# Check output\$errors for a table with all problematic strings.}}\begin{verbatim}
## $strings
##   originals tokenized
## 1      AABB   A A B B
## 2      AАBВ   A ⁇ B ⁇
## 
## $profile
##   Grapheme Frequency
## 1        B         3
## 2        A         3
## 
## $errors
##   originals  errors
## 2      AАBВ A ⁇ B ⁇
## 
## $missing
##   Grapheme Frequency Codepoint                UnicodeName
## 1        А         1    U+0410  CYRILLIC CAPITAL LETTER A
## 2        В         1    U+0412 CYRILLIC CAPITAL LETTER VE
\end{verbatim}
\end{kframe}
\end{knitrout}

\subsection*{Different ways to write a profile}
\label{write-profile}

The function \texttt{write.profile} can be used to prepare a skeleton for an
orthography profile from some data. The preparation of an orthography profile
from some data might sound like a trivial problem, but actually there are
various different ways in which strings can be separated into graphemes by
\texttt{write.profile}. Consider the following string of characters called
\texttt{example} below. The default settings of \texttt{write.profile} separates
the string into Unicode graphemes according to grapheme clusters (called user-perceived characters; see Chapter~\ref{the-unicode-approach} for an explanation). The results are shown 
in Table \ref {tab:profile1}. As it 
turns out, some of these graphemes are single code points, others are combinations
of two code points (see Section~\ref{pitfall-characters-are-not-glyphs}).

\begin{knitrout}\footnotesize
\definecolor{shadecolor}{rgb}{1, 1, 1}\color{fgcolor}\begin{kframe}
\begin{alltt}
\hlstd{example} \hlkwb{<-} \hlstr{'\textbackslash{}u00d9\textbackslash{}u00da\textbackslash{}u00db\textbackslash{}u0055\textbackslash{}u0300\textbackslash{}u0055\textbackslash{}u0301\textbackslash{}u0055\textbackslash{}u0302'}
\hlstd{profile_1} \hlkwb{<-} \hlkwd{write.profile}\hlstd{(example)}
\end{alltt}
\end{kframe}
\end{knitrout}

% latex table generated in R 3.5.1 by xtable 1.8-2 package
% Thu Aug  9 20:07:08 2018
\begin{table}[H]
\centering
\begingroup\scriptsize
\begin{tabular}{llll}
  \toprule
Gr. & Freq. & Codepoint & Unicode Name \\ 
  \midrule
Ú & 1 & U+00DA & LATIN CAPITAL LETTER U WITH ACUTE \\ 
  Ú & 1 & U+0055, U+0301 & LATIN CAPITAL LETTER U, COMBINING ACUTE ACCENT \\ 
  Ù & 1 & U+00D9 & LATIN CAPITAL LETTER U WITH GRAVE \\ 
  Ù & 1 & U+0055, U+0300 & LATIN CAPITAL LETTER U, COMBINING GRAVE ACCENT \\ 
  Û & 1 & U+00DB & LATIN CAPITAL LETTER U WITH CIRCUMFLEX \\ 
  Û & 1 & U+0055, U+0302 & LATIN CAPITAL LETTER U, COMBINING CIRCUMFLEX ACCENT \\ 
   \bottomrule
\end{tabular}
\endgroup
\caption{Profile 1 (default settings, splitting grapheme clusters)} 
\label{tab:profile1}
\end{table}


By specifying the splitting separator as the empty string
\texttt{sep~=~""}, it is possible to split the string into Unicode code points,
thus separating the combining diacritics. The idea behind this option
\texttt{sep} is that separating by a character allows for user-determined
separation. The most extreme choice here is the empty string \texttt{sep~=~""},
which is interpreted as separation everywhere. The other extreme is the default
setting \texttt{sep~=~NULL}, which means that the separation is not
user-defined, but relegated to the Unicode grapheme definitions. The result is 
shown in Table \ref{tab:profile2}.

\begin{knitrout}\footnotesize
\definecolor{shadecolor}{rgb}{1, 1, 1}\color{fgcolor}\begin{kframe}
\begin{alltt}
\hlstd{profile_2} \hlkwb{<-} \hlkwd{write.profile}\hlstd{(example,} \hlkwc{sep} \hlstd{=} \hlstr{""}\hlstd{)}
\end{alltt}
\end{kframe}
\end{knitrout}

% latex table generated in R 3.5.1 by xtable 1.8-2 package
% Thu Aug  9 20:07:08 2018
\begin{table}[H]
\centering
\begingroup\scriptsize
\begin{tabular}{llll}
  \toprule
Grapheme & Frequency & Codepoint & Unicode Name \\ 
  \midrule
́ & 1 & U+0301 & COMBINING ACUTE ACCENT \\ 
  ̀ & 1 & U+0300 & COMBINING GRAVE ACCENT \\ 
  ̂ & 1 & U+0302 & COMBINING CIRCUMFLEX ACCENT \\ 
  U & 3 & U+0055 & LATIN CAPITAL LETTER U \\ 
  Ú & 1 & U+00DA & LATIN CAPITAL LETTER U WITH ACUTE \\ 
  Ù & 1 & U+00D9 & LATIN CAPITAL LETTER U WITH GRAVE \\ 
  Û & 1 & U+00DB & LATIN CAPITAL LETTER U WITH CIRCUMFLEX \\ 
   \bottomrule
\end{tabular}
\endgroup
\caption{Profile 2 (splitting by code points)} 
\label{tab:profile2}
\end{table}


Some characters look identical, although they are encoded differently.
Unicode offers different ways of normalization (see
Section~\ref{pitfall-canonical-equivalence}), which can be invoked here as well
using the option \texttt{normalize}. NFC normalization turns everything into the
precomposed characters, while NFD normalization separates everything into base
characters with combining diacritics. Splitting by code points (i.e.\ \texttt{sep~=~""}) 
shows the results of these two normalizations in Tables \ref{tab:profile3} \& \ref{tab:profile4}.

\begin{knitrout}\footnotesize
\definecolor{shadecolor}{rgb}{1, 1, 1}\color{fgcolor}\begin{kframe}
\begin{alltt}
\hlcom{# after NFC normalization Unicode code points have changed}
\hlstd{profile_3} \hlkwb{<-} \hlkwd{write.profile}\hlstd{(example,} \hlkwc{normalize} \hlstd{=} \hlstr{"NFC"}\hlstd{,} \hlkwc{sep} \hlstd{=} \hlstr{""}\hlstd{)}
\hlcom{# NFD normalization gives another structure of the code points}
\hlstd{profile_4} \hlkwb{<-} \hlkwd{write.profile}\hlstd{(example,} \hlkwc{normalize} \hlstd{=} \hlstr{"NFD"}\hlstd{,} \hlkwc{sep} \hlstd{=} \hlstr{""}\hlstd{)}
\end{alltt}
\end{kframe}
\end{knitrout}

% latex table generated in R 3.5.1 by xtable 1.8-2 package
% Thu Aug  9 20:07:08 2018
\begin{table}[H]
\centering
\begingroup\scriptsize
\begin{tabular}{llll}
  \toprule
Grapheme & Frequency & Codepoint & Unicode Name \\ 
  \midrule
Ú & 2 & U+00DA & LATIN CAPITAL LETTER U WITH ACUTE \\ 
  Ù & 2 & U+00D9 & LATIN CAPITAL LETTER U WITH GRAVE \\ 
  Û & 2 & U+00DB & LATIN CAPITAL LETTER U WITH CIRCUMFLEX \\ 
   \bottomrule
\end{tabular}
\endgroup
\caption{Profile 3 (splitting by NFC code points)} 
\label{tab:profile3}
\end{table}


% latex table generated in R 3.5.1 by xtable 1.8-2 package
% Thu Aug  9 20:07:08 2018
\begin{table}[H]
\centering
\begingroup\scriptsize
\begin{tabular}{llll}
  \toprule
Grapheme & Frequency & Codepoint & Unicode Name \\ 
  \midrule
́ & 2 & U+0301 & COMBINING ACUTE ACCENT \\ 
  ̀ & 2 & U+0300 & COMBINING GRAVE ACCENT \\ 
  ̂ & 2 & U+0302 & COMBINING CIRCUMFLEX ACCENT \\ 
  U & 6 & U+0055 & LATIN CAPITAL LETTER U \\ 
   \bottomrule
\end{tabular}
\endgroup
\caption{Profile 4 (splitting by NFD code points)} 
\label{tab:profile4}
\end{table}


It is important to realize that for Unicode grapheme definitions, NFC
and NFD normalization are equivalent. This can be shown by normalizing the
example in either NFD or NFC, as shown in Tables \ref{tab:profile5} \& \ref{tab:profile6}, 
by using the default separation in
\texttt{write.profile}. To be precise, default separation means setting
\texttt{sep~=~NULL}, but that has not be added explicitly below.

\begin{knitrout}\footnotesize
\definecolor{shadecolor}{rgb}{1, 1, 1}\color{fgcolor}\begin{kframe}
\begin{alltt}
\hlcom{# note that NFC and NFD normalization are identical}
\hlcom{# for Unicode grapheme definitions}
\hlstd{profile_5} \hlkwb{<-} \hlkwd{write.profile}\hlstd{(example,} \hlkwc{normalize} \hlstd{=} \hlstr{"NFD"}\hlstd{)}
\hlstd{profile_6} \hlkwb{<-} \hlkwd{write.profile}\hlstd{(example,} \hlkwc{normalize} \hlstd{=} \hlstr{"NFC"}\hlstd{)}
\end{alltt}
\end{kframe}
\end{knitrout}

% latex table generated in R 3.5.1 by xtable 1.8-2 package
% Thu Aug  9 20:07:08 2018
\begin{table}[H]
\centering
\begingroup\scriptsize
\begin{tabular}{llll}
  \toprule
Gr. & Freq. & Codepoint & Unicode Name \\ 
  \midrule
Ú & 2 & U+0055, U+0301 & LATIN CAPITAL LETTER U, COMBINING ACUTE ACCENT \\ 
  Ù & 2 & U+0055, U+0300 & LATIN CAPITAL LETTER U, COMBINING GRAVE ACCENT \\ 
  Û & 2 & U+0055, U+0302 & LATIN CAPITAL LETTER U, COMBINING CIRCUMFLEX ACCENT \\ 
   \bottomrule
\end{tabular}
\endgroup
\caption{Profile 5 (splitting by graphemes after NFD)} 
\label{tab:profile5}
\end{table}


% latex table generated in R 3.5.1 by xtable 1.8-2 package
% Thu Aug  9 20:07:08 2018
\begin{table}[H]
\centering
\begingroup\scriptsize
\begin{tabular}{llll}
  \toprule
Gr. & Freq. & Codepoint & Unicode Name \\ 
  \midrule
Ú & 2 & U+00DA & LATIN CAPITAL LETTER U WITH ACUTE \\ 
  Ù & 2 & U+00D9 & LATIN CAPITAL LETTER U WITH GRAVE \\ 
  Û & 2 & U+00DB & LATIN CAPITAL LETTER U WITH CIRCUMFLEX \\ 
   \bottomrule
\end{tabular}
\endgroup
\caption{Profile 6 (splitting by graphemes after NFC)} 
\label{tab:profile6}
\end{table}


These different profiles can also be produced using the bash
executable \texttt{writeprofile} (see above for how to install the Bash executable). 
This example is also included in the help file of the executable.

% TODO: fix
% <<size='scriptsize', engine='bash'>>=
% writeprofile --help
% @


\subsection*{Using an orthography profile skeleton}
\label{profile-skeleton}

A common workflow to use these functions is to first make a skeleton for an
orthography profile and then edit this profile by hand. For example, Table
\ref{tab:profile_skeleton1} shows the profile skeleton after a few graphemes have
been added to the file. Note that in this example, the profile is written to the
desktop, and this file has to be edited manually. We simply add a few
multigraphs to the column \texttt{Grapheme} and leave the other columns empty.
These new graphemes are then included in the graphemic parsing.

\begin{knitrout}\footnotesize
\definecolor{shadecolor}{rgb}{1, 1, 1}\color{fgcolor}\begin{kframe}
\begin{alltt}
\hlcom{# a few words to be graphemically parsed}
\hlstd{example} \hlkwb{<-} \hlkwd{c}\hlstd{(}\hlstr{"mishmash"}\hlstd{,} \hlstr{"mishap"}\hlstd{,} \hlstr{"mischief"}\hlstd{,} \hlstr{"scheme"}\hlstd{)}
\hlcom{# write a profile skeleton to a file}
\hlkwd{write.profile}\hlstd{(example,} \hlkwc{file} \hlstd{=} \hlstr{"~/Desktop/profile_skeleton.txt"}\hlstd{)}
\hlcom{# edit the profile, and then use the edited profile to tokenize}
\hlkwd{tokenize}\hlstd{(example,} \hlkwc{profile} \hlstd{=} \hlstr{"~/Desktop/profile_skeleton.txt"}\hlstd{)}\hlopt{$}\hlstd{strings}
\end{alltt}
\end{kframe}
\end{knitrout}

\begin{knitrout}\footnotesize
\definecolor{shadecolor}{rgb}{1, 1, 1}\color{fgcolor}\begin{kframe}
\begin{verbatim}
##   originals    tokenized
## 1   shampoo  sh a m p oo
## 2    mishap   m i sh a p
## 3  mischief m i sch ie f
## 4    scheme    sch e m e
\end{verbatim}
\end{kframe}
\end{knitrout}

% latex table generated in R 3.5.1 by xtable 1.8-2 package
% Thu Aug  9 20:07:08 2018
\begin{table}[htb]
\centering
\begingroup\scriptsize
\begin{tabular}{llll}
  \toprule
Grapheme & Frequency & Codepoint & UnicodeName \\ 
  \midrule
sh &  &  &  \\ 
  ch &  &  &  \\ 
  sch &  &  &  \\ 
  ie &  &  &  \\ 
  oo &  &  &  \\ 
  a & 2 & U+0061 & LATIN SMALL LETTER A \\ 
  c & 2 & U+0063 & LATIN SMALL LETTER C \\ 
  e & 3 & U+0065 & LATIN SMALL LETTER E \\ 
  f & 1 & U+0066 & LATIN SMALL LETTER F \\ 
  h & 4 & U+0068 & LATIN SMALL LETTER H \\ 
  i & 3 & U+0069 & LATIN SMALL LETTER I \\ 
  m & 4 & U+006D & LATIN SMALL LETTER M \\ 
  o & 2 & U+006F & LATIN SMALL LETTER O \\ 
  p & 2 & U+0070 & LATIN SMALL LETTER P \\ 
  s & 4 & U+0073 & LATIN SMALL LETTER S \\ 
   \bottomrule
\end{tabular}
\endgroup
\caption{Manually edited profile skeleton} 
\label{tab:profile_skeleton1}
\end{table}


To leave out the Unicode information in
the profile skeleton, use the option \texttt{info = FALSE}. It is also
possible not to use a separate file at all, but process everything within R. In
simple situations this is often useful (see below), but in general we prefer to
handle everything through a separately saved orthography profile. This profile
often contains highly useful information that is nicely coded and saved inside
this one file, and can thus be easily distributed and shared. Doing the same as
above completely within R might look as follows:

\begin{knitrout}\footnotesize
\definecolor{shadecolor}{rgb}{1, 1, 1}\color{fgcolor}\begin{kframe}
\begin{alltt}
\hlcom{# make a profile, just select the column 'Grapheme'}
\hlstd{profile} \hlkwb{<-} \hlkwd{write.profile}\hlstd{(example)[,}\hlstr{"Grapheme"}\hlstd{]}
\hlcom{# extend the profile with multigraphs}
\hlstd{profile} \hlkwb{<-} \hlkwd{c}\hlstd{(}\hlstr{"sh"}\hlstd{,} \hlstr{"ch"}\hlstd{,} \hlstr{"sch"}\hlstd{,} \hlstr{"ie"}\hlstd{,} \hlstr{"oo"}\hlstd{, profile)}
\hlcom{# use the profile to tokenize}
\hlkwd{tokenize}\hlstd{(example, profile)}\hlopt{$}\hlstd{strings}
\end{alltt}
\begin{verbatim}
##   originals    tokenized
## 1   shampoo  sh a m p oo
## 2    mishap   m i sh a p
## 3  mischief m i sch ie f
## 4    scheme    sch e m e
\end{verbatim}
\end{kframe}
\end{knitrout}

\subsection*{Rule ordering}
\label{rule-ordering}

Everything is not yet correct with the graphemic parsing of the example discussed
previously. The sequence <sh> in `mishap' should not be a digraph, and
conversely the sequence <sch> in `mischief' should of course be separated into
<s> and <ch>. One of the important issues to get the graphemic parsing right is
the order in which graphemes are parsed. For example, currently the grapheme
<sch> is parsed before the grapheme <ch>, leading to <m\ i\ sch\ ie\ f> instead
of the intended <m\ i\ s\ ch\ ie\ f>. The reason that <sch> is parsed before
<ch> is that by default longer graphemes are parsed before shorter ones. Our
experience is that in most cases this is expected behavior. You can change the
ordering by specifying the option \texttt{ordering}. Setting this option to
\texttt{NULL} results in no preferential ordering, i.e.\ the graphemes are parsed
in the order of the profile, from top to bottom. Now `mischief' is parsed
correctly, but `scheme' is wrong. So this ordering is not the solution in this
case.

\begin{knitrout}\footnotesize
\definecolor{shadecolor}{rgb}{1, 1, 1}\color{fgcolor}\begin{kframe}
\begin{alltt}
\hlcom{# do not reorder the profile}
\hlcom{# just apply the graphemes from top to bottom}
\hlkwd{tokenize}\hlstd{( example}
         \hlstd{,} \hlkwc{profile} \hlstd{=} \hlstr{"~/Desktop/profile_skeleton.txt"}
         \hlstd{,} \hlkwc{ordering} \hlstd{=} \hlkwa{NULL}
        \hlstd{)}\hlopt{$}\hlstd{strings}
\end{alltt}
\end{kframe}
\end{knitrout}

\begin{knitrout}\footnotesize
\definecolor{shadecolor}{rgb}{1, 1, 1}\color{fgcolor}\begin{kframe}
\begin{verbatim}
##   originals     tokenized
## 1   shampoo   sh a m p oo
## 2    mishap    m i sh a p
## 3  mischief m i s ch ie f
## 4    scheme    s ch e m e
\end{verbatim}
\end{kframe}
\end{knitrout}

There are various additional options for rule ordering implemented. Please check
the help description in R, i.e.\ \texttt{help(tokenize)}, for more details on the
possible rule ordering specifications. In summary, there are four different 
ordering options, that can also be combined:

\begin{itemize}
  
   \item \textsc{size}\\
         This option orders the lines in the profile by the size of the
         grapheme, largest first. Size is measured by number of Unicode
         characters after normalization as specified in the option
         \texttt{normalize}. For example, <é> has a size of 1 with
         \texttt{normalize = "NFC"}, but a size of 2 with
         \texttt{normalize = "NFD"}.

   \item \textsc{context}\\ This option orders the lines by whether they have
           any context specified (see next section). Lines with context will
           then be used first. Note that this only works when the option
           \texttt{regex = TRUE} is also chosen (otherwise context
           specifications are not used).

   \item \textsc{reverse}\\ This option orders the lines from bottom to top.
         Reversing order can be useful because hand-written profiles tend to put
         general rules before specific rules, which mostly should be applied in
         reverse order.

  \item \textsc{frequency}\\
         This option orders the lines by the frequency with which they
         match in the specified strings before tokenization, least frequent
         coming first. This frequency of course depends crucially on the
         available strings, so it will lead to different orderings when applied
         to different data. Also note that this frequency is (necessarily)
         measured before graphemes are identified, so these ordering frequencies
         are not the same as the final frequencies shown in the output.
         Frequency of course also strongly differs on whether context is used
         for the matching through \texttt{regex = TRUE}.
  
\end{itemize}

By specifying more than one ordering, these orderings are used to break ties,
e.g.\ the default setting \texttt{ordering = c("size", "context", "reverse")}
will first order by size, and for those with the same size, it will order by
whether there is any context specified or not. For lines that are still tied
(i.e.\ have the same size and both/neither have context) the order will be
reversed compared to the order as attested in the profile, because most
hand-written specifications of graphemes will first write the general rule,
followed by more specific regularities. To get the right tokenization, these 
rules should in most cases be applied in reverse order.

Note that different ordering of the rules does not result in 
feeding and bleeding effects found with finite-state rewrite
rules.\footnote{Bleeding is the effect that the application of a rule changes
the string, so as to prevent a following rule from applying. Feeding is the opposite: a
specific rule will only be applied because a previous rule changed the string
already. The interaction of rules with such feeding and bleeding effects is
extremely difficult to predict.} The graphemic parsing advocated here is 
crucially different from rewrite rules in that there is nothing being rewritten:
each line in an orthography profile specifies a grapheme to be captured in the 
string. All lines in the profile are processed in a specified order (as determined
by the option \texttt{ordering}). At the processing of a specific line, all 
matching graphemes in the data are marked as captured, but not changed. 
Captured parts cannot be captured again, but they can still be used to match 
contexts of other lines in the profile. Only when all lines are processed the 
captured graphemes are separated (and possibly transliterated). In this way the 
result of the applied rules is rather easy to predict.

To document a specific case of graphemic parsing, it is highly useful to save
all results of the tokenization to file by using the option \texttt{file.out},
for example as follows: 

\begin{knitrout}\footnotesize
\definecolor{shadecolor}{rgb}{1, 1, 1}\color{fgcolor}\begin{kframe}
\begin{alltt}
\hlcom{# save the results to various files}
\hlkwd{tokenize}\hlstd{( example}
         \hlstd{,} \hlkwc{profile} \hlstd{=} \hlstr{"~/Desktop/profile_skeleton.txt"}
         \hlstd{,} \hlkwc{file.out} \hlstd{=} \hlstr{"~/Desktop/result"}
        \hlstd{)}
\end{alltt}
\end{kframe}
\end{knitrout}

This will lead to the following four files being written. Crucially, a
new profile is produced with the re-ordered orthography profile. To reproduce
the tokenization, this re-ordered profile can be used with the option
\texttt{ordering~=~NULL}.

\begin{itemize}
  
   \item \textsc{result\_strings.tsv}:\\ A tab-separated file with the original
         and the tokenized/transliterated strings.

   \item \textsc{result\_profile.tsv}:\\ A tab-separated file with the
         graphemes with added frequencies of occurrence in the data. The lines
         in the file are re-ordered according to the order that resulted from the
         ordering specifications (see Section~\ref{rule-ordering}).

   \item \textsc{result\_errors.tsv}:\\ A tab-separated file with all original
         strings that contain unmatched parts. Unmatched parts are indicated
         with the character as specified with the option \texttt{missing}. By
         default the character \textsc{double question mark} <⁇> at
         \uni{2047} is used. When there are no errors, this file is 
         absent.

    \item \textsc{result\_missing.tsv}:\\ A tab-separated file with the graphemes
          that are missing from the original orthography profile, as indicated in
          the errors. When there are no errors, then this file is absent.
          
\end{itemize}

\subsection*{Contextually specified graphemes}
\label{contextual-specification}

To refine a profile, it is also possible to add graphemes with contextual
specifications. An orthography profile can have columns called \texttt{Left} and
\texttt{Right} to specify the context in which the grapheme is to be
separated.\footnote{The column names \textttf{Left}, \textttf{Right} and
\textttf{Grapheme} are currently hard-coded, so these exact column names
should be used for these effects to take place. The position of the columns in
the profile is unimportant. So the column \textttf{Left} can occur anywhere.}
For example, we are adding an extra line to the profile from above, resulting in
the profile shown in Table~\ref{tab:profile_skeleton2}. The extra line specifies
that <s> is a grapheme when it occurs after <mi>. Such contextually-specified
graphemes are based on regular expressions so you can also use regular
expressions in the description of the context. For such contextually specified
graphemes to be included in the graphemic parsing we have to specify the option
\texttt{regex = TRUE}. This contextually specified grapheme should actually be
handled first, so we could try \texttt{ordering = NULL}. However, we can also
explicitly specify that rules with contextual information should be applied
first by using \texttt{ordering = "context"}. That gives the right results for
this toy example, as shown in Table \ref{tab:profile_skeleton2}.

\begin{knitrout}\footnotesize
\definecolor{shadecolor}{rgb}{1, 1, 1}\color{fgcolor}\begin{kframe}
\begin{alltt}
\hlcom{# add a contextual grapheme, and then use the edited }
\hlcom{# profile to tokenize}
\hlkwd{tokenize}\hlstd{( example}
         \hlstd{,} \hlkwc{profile} \hlstd{=} \hlstr{"~/Desktop/profile_skeleton.txt"}
         \hlstd{,} \hlkwc{regex} \hlstd{=} \hlnum{TRUE}
         \hlstd{,} \hlkwc{ordering} \hlstd{=} \hlstr{"context"}
        \hlstd{)}\hlopt{$}\hlstd{strings}
\end{alltt}
\end{kframe}
\end{knitrout}

\begin{knitrout}\footnotesize
\definecolor{shadecolor}{rgb}{1, 1, 1}\color{fgcolor}\begin{kframe}
\begin{verbatim}
##   originals     tokenized
## 1   shampoo   sh a m p oo
## 2    mishap   m i s h a p
## 3  mischief m i s ch ie f
## 4    scheme    s ch e m e
\end{verbatim}
\end{kframe}
\end{knitrout}

% latex table generated in R 3.5.1 by xtable 1.8-2 package
% Thu Aug  9 20:07:08 2018
\begin{table}[htb]
\centering
\begingroup\scriptsize
\begin{tabular}{lllll}
  \toprule
Left & Grapheme & Frequency & Codepoint & UnicodeName \\ 
  \midrule
mi & s &  &  &  \\ 
   & sh &  &  &  \\ 
   & ch &  &  &  \\ 
   & sch &  &  &  \\ 
   & ie &  &  &  \\ 
   & oo &  &  &  \\ 
   & a & 2 & U+0061 & LATIN SMALL LETTER A \\ 
   & c & 2 & U+0063 & LATIN SMALL LETTER C \\ 
   & e & 3 & U+0065 & LATIN SMALL LETTER E \\ 
   & f & 1 & U+0066 & LATIN SMALL LETTER F \\ 
   & h & 4 & U+0068 & LATIN SMALL LETTER H \\ 
   & i & 3 & U+0069 & LATIN SMALL LETTER I \\ 
   & m & 4 & U+006D & LATIN SMALL LETTER M \\ 
   & o & 2 & U+006F & LATIN SMALL LETTER O \\ 
   & p & 2 & U+0070 & LATIN SMALL LETTER P \\ 
   & s & 4 & U+0073 & LATIN SMALL LETTER S \\ 
   \bottomrule
\end{tabular}
\endgroup
\caption{Orthography profile with contextual specification for <s>} 
\label{tab:profile_skeleton2}
\end{table}


Note that with the option \texttt{regex = TRUE} all
content in the profile is treated as regular expressions, so the characters with
special meaning in regular expressions should be either omitted or escaped (by
putting a <\ \backslash\ > \textsc{reverse solidus} at \uni{005C} before the
character). Specifically, this concerns the following characters:

\begin{itemize}
  
  \item[] <-> \textsc{hyphen-minus} at \uni{002D}
  \item[] <!> \textsc{exclamation mark} at \uni{0021}
  \item[] <?> \textsc{question mark} at \uni{003F}
  \item[] <.> \textsc{full stop} at \uni{002E}
  \item[] <(> \textsc{left parenthesis} at \uni{0028}
  \item[] <)> \textsc{right parenthesis} at \uni{0029}
  \item[] <[> \textsc{left square bracket} at \uni{005B}
  \item[] <]> \textsc{right square bracket} at \uni{005D}
  \item[] <\{> \textsc{left curly bracket} at \uni{007B}
  \item[] <\}> \textsc{right curly bracket} at \uni{007D}
  \item[] <|> \textsc{vertical line} at \uni{007C}
  \item[] <*> \textsc{asterisk} at \uni{002A}
  \item[] <\backslash> \textsc{reverse solidus} at \uni{005C}
  \item[] <ˆ> \textsc{circumflex accent} at \uni{005E}
  \item[] <+> \textsc{plus sign} at \uni{002B}
  \item[] <\$> \textsc{dollar sign} at \uni{0024}
  
\end{itemize}

\subsection*{Profile skeleton with columns for editing}
\label{profile-editing}

When it is expected that context might be important for a profile, then the
profile skeleton can be created with columns prepared for the contextual
specifications. This is done by using the option \texttt{editing = TRUE}~(cf.\
Table~\ref{tab:profile_editing_1} for a toy profile of some Italian words).

\begin{knitrout}\footnotesize
\definecolor{shadecolor}{rgb}{1, 1, 1}\color{fgcolor}\begin{kframe}
\begin{alltt}
\hlstd{example} \hlkwb{<-} \hlkwd{c}\hlstd{(}\hlstr{'cane'}\hlstd{,} \hlstr{'cena'}\hlstd{,} \hlstr{'cine'}\hlstd{)}
\hlkwd{write.profile}\hlstd{(example}
              \hlstd{,} \hlkwc{file} \hlstd{=} \hlstr{"~/Desktop/profile_skeleton.txt"}
              \hlstd{,} \hlkwc{editing} \hlstd{=} \hlnum{TRUE}
              \hlstd{,} \hlkwc{info} \hlstd{=} \hlnum{FALSE}
              \hlstd{)}
\end{alltt}
\end{kframe}
\end{knitrout}

% latex table generated in R 3.5.1 by xtable 1.8-2 package
% Thu Aug  9 20:07:08 2018
\begin{table}[htb]
\centering
\begingroup\scriptsize
\begin{tabular}{lllll}
  \toprule
Left & Grapheme & Right & Class & Replacement \\ 
  \midrule
 & a &  &  & a \\ 
   & c &  &  & c \\ 
   & e &  &  & e \\ 
   & i &  &  & i \\ 
   & n &  &  & n \\ 
   \bottomrule
\end{tabular}
\endgroup
\caption{Orthography profile with empty columns for editing contexts} 
\label{tab:profile_editing_1}
\end{table}


Besides the columns \texttt{Left}, \texttt{Grapheme}, and \texttt{Right} as
discussed in the previous sections, there are also columns \texttt{Class} and
\texttt{Replacement}. The column \texttt{Class} can be used to specify classes
of graphemes that can then be used in the contextual specification. The column
\texttt{Replacement} is just a copy of the column \texttt{Grapheme}, providing a
skeleton to specify transliteration. The name of the column
\texttt{Replacement} is not fixed -- there can actually be multiple columns with 
different kinds of transliterations in a single profile.

To achieve contextually determined replacements it is possible to use a regular
expression in the contextual column. For example, consider the edited toy
profile for Italian in Table~\ref{tab:profile_editing_2} (where <c> becomes /k/
except before <i,e>, then it becomes /tʃ/). 

% latex table generated in R 3.5.1 by xtable 1.8-2 package
% Thu Aug  9 20:07:08 2018
\begin{table}[htb]
\centering
\begingroup\scriptsize
\begin{tabular}{lllll}
  \toprule
Left & Grapheme & Right & Class & IPA \\ 
  \midrule
 & c & [ie] &  & tʃ \\ 
   & a &  &  & a \\ 
   & n &  &  & n \\ 
   & c &  &  & k \\ 
   & e &  &  & e \\ 
   & i &  &  & i \\ 
   \bottomrule
\end{tabular}
\endgroup
\caption{Orthography profile with regex as context} 
\label{tab:profile_editing_2}
\end{table}


To use this profile, you have to add the option \texttt{regex = TRUE}. Also note
that we have changed the name of the transliteration column, so we have to tell
the tokenization process to use this column to transliterate. This is done by
adding the option \texttt{transliterate = "IPA"}.

\begin{knitrout}\footnotesize
\definecolor{shadecolor}{rgb}{1, 1, 1}\color{fgcolor}\begin{kframe}
\begin{alltt}
\hlcom{# add a contextual grapheme, and then use the edited }
\hlcom{# profile to tokenize}
\hlkwd{tokenize}\hlstd{( example}
         \hlstd{,} \hlkwc{profile} \hlstd{=} \hlstr{"~/Desktop/profile_skeleton.txt"}
         \hlstd{,} \hlkwc{regex} \hlstd{=} \hlnum{TRUE}
         \hlstd{,} \hlkwc{transliterate} \hlstd{=} \hlstr{"IPA"}
        \hlstd{)}\hlopt{$}\hlstd{strings}
\end{alltt}
\end{kframe}
\end{knitrout}

\begin{knitrout}\footnotesize
\definecolor{shadecolor}{rgb}{1, 1, 1}\color{fgcolor}\begin{kframe}
\begin{verbatim}
##   originals tokenized transliterated
## 1      cane   c a n e        k a n e
## 2      cena   c e n a       tʃ e n a
## 3      cine   c i n e       tʃ i n e
\end{verbatim}
\end{kframe}
\end{knitrout}

Another equivalent possibility is to use a column \texttt{Class} to specify a
class of graphemes, and then use this class in the specification of context.
This is useful to keep track of recurrent classes in larger profiles. You are
free to use any class-name you like, as long as it does not clash with the rest
of the profile. The example shown in Table~\ref{tab:profile_editing_3} should 
give the same result as obtained previously by using a regular expression.

% latex table generated in R 3.5.1 by xtable 1.8-2 package
% Thu Aug  9 20:07:08 2018
\begin{table}[htb]
\centering
\begingroup\scriptsize
\begin{tabular}{lllll}
  \toprule
Left & Grapheme & Right & Class & IPA \\ 
  \midrule
 & c & Vfront &  & tʃ \\ 
   & a &  &  & a \\ 
   & n &  &  & n \\ 
   & c &  &  & k \\ 
   & e &  & Vfront & e \\ 
   & i &  & Vfront & i \\ 
   \bottomrule
\end{tabular}
\endgroup
\caption{Orthography profile with Class as context} 
\label{tab:profile_editing_3}
\end{table}


\begin{knitrout}\footnotesize
\definecolor{shadecolor}{rgb}{1, 1, 1}\color{fgcolor}\begin{kframe}
\begin{alltt}
\hlcom{# add a class, and then use the edited profile to tokenize}
\hlkwd{tokenize}\hlstd{( example}
         \hlstd{,} \hlkwc{profile} \hlstd{=} \hlstr{"~/Desktop/profile_skeleton.txt"}
         \hlstd{,} \hlkwc{regex} \hlstd{=} \hlnum{TRUE}
         \hlstd{,} \hlkwc{transliterate} \hlstd{=} \hlstr{"IPA"}
        \hlstd{)}\hlopt{$}\hlstd{strings}
\end{alltt}
\end{kframe}
\end{knitrout}

\begin{knitrout}\footnotesize
\definecolor{shadecolor}{rgb}{1, 1, 1}\color{fgcolor}\begin{kframe}
\begin{verbatim}
##   originals tokenized transliterated
## 1      cane   c a n e        k a n e
## 2      cena   c e n a       tʃ e n a
## 3      cine   c i n e       tʃ i n e
\end{verbatim}
\end{kframe}
\end{knitrout}

\subsection*{Formatting grapheme separation}
\label{formattingseparation}

In all examples above we have used the default formatting for grapheme
separation using space as a separator, which is obtained by the default setting
\texttt{sep~=~"~"}. It is possible to specify any other separator here,
including the empty string, i.e.\ \texttt{sep = ""}. This will not show the
graphemic tokenization anymore (although it has of course been used in the
background).

\begin{knitrout}\footnotesize
\definecolor{shadecolor}{rgb}{1, 1, 1}\color{fgcolor}\begin{kframe}
\begin{alltt}
\hlcom{# Use the empty string as separator}
\hlkwd{tokenize}\hlstd{( example}
         \hlstd{,} \hlkwc{profile} \hlstd{=} \hlstr{"~/Desktop/profile_skeleton.txt"}
         \hlstd{,} \hlkwc{regex} \hlstd{=} \hlnum{TRUE}
         \hlstd{,} \hlkwc{transliterate} \hlstd{=} \hlstr{"IPA"}
         \hlstd{,} \hlkwc{sep} \hlstd{=} \hlstr{""}
        \hlstd{)}\hlopt{$}\hlstd{strings}
\end{alltt}
\end{kframe}
\end{knitrout}

\begin{knitrout}\footnotesize
\definecolor{shadecolor}{rgb}{1, 1, 1}\color{fgcolor}\begin{kframe}
\begin{verbatim}
##   originals tokenized transliterated
## 1      cane      cane           kane
## 2      cena      cena          tʃena
## 3      cine      cine          tʃine
\end{verbatim}
\end{kframe}
\end{knitrout}

Normally, the separator specified should not occur in the data. If it does,
unexpected things might happen, so consider removing the chosen separator from
your strings first. However, there is also an option \texttt{sep.replace} to
replace the separator with something else. When \texttt{sep.replace} is specified,
this mark is inserted in the string at those places where the separator occurs.
Typical usage in linguistics would be \texttt{sep = " ", sep.replace = "\#"} adding
spaces between graphemes and replacing spaces in the input string by hashes in
the output string.

\begin{knitrout}\footnotesize
\definecolor{shadecolor}{rgb}{1, 1, 1}\color{fgcolor}\begin{kframe}
\begin{alltt}
\hlcom{# Replace separator in string to be tokenized}
\hlkwd{tokenize}\hlstd{(} \hlstr{"test test test"}
         \hlstd{,} \hlkwc{sep} \hlstd{=} \hlstr{" "}
         \hlstd{,} \hlkwc{sep.replace} \hlstd{=} \hlstr{"#"}
        \hlstd{)}\hlopt{$}\hlstd{strings}\hlopt{$}\hlstd{tokenized}
\end{alltt}
\begin{verbatim}
## [1] "t e s t # t e s t # t e s t"
\end{verbatim}
\end{kframe}
\end{knitrout}

\subsection*{Remaining issues}

Given a set of graphemes, there are at least two different methods to tokenize
strings. The first is called \texttt{method = "global"}. This approach
takes the first grapheme in the profile, then matches this grapheme globally at
all places in the string, and then turns to process the next string in the profile. The 
other approach is called \texttt{method = "linear"}. This
approach walks through the string from left to right. At the first character it
looks through all graphemes whether there is any match, and then walks further
to the end of the match and starts again. This approach is more akin to
finite-state rewrite rules (though note that it still works differently from
such rewrite rules, as previously stated). The global method is used 
by default in the R implementation.

In some special cases these two tokenization methods can lead to different
results, but these special situations are very unlikely to happen in natural
language. The example below shows that a string \texttt{'abc'} can be parsed
differently in case of a very special profile with a very special ordering of
the graphemes.

\begin{knitrout}\footnotesize
\definecolor{shadecolor}{rgb}{1, 1, 1}\color{fgcolor}\begin{kframe}
\begin{alltt}
\hlcom{# different parsing methods can lead to different results}
\hlcom{# the global method first catches 'bc'}
\hlkwd{tokenize}\hlstd{(} \hlstr{"abc"}
         \hlstd{,} \hlkwc{profile} \hlstd{=} \hlkwd{c}\hlstd{(}\hlstr{"bc"}\hlstd{,}\hlstr{"ab"}\hlstd{,}\hlstr{"a"}\hlstd{,}\hlstr{"c"}\hlstd{)}
         \hlstd{,} \hlkwc{order} \hlstd{=} \hlkwa{NULL}
         \hlstd{,} \hlkwc{method} \hlstd{=} \hlstr{"global"}
         \hlstd{)}\hlopt{$}\hlstd{strings}
\end{alltt}
\begin{verbatim}
##   originals tokenized
## 1       abc      a bc
\end{verbatim}
\begin{alltt}
         
\hlcom{# the linear method catches the first grapheme, which is 'ab'}
\hlkwd{tokenize}\hlstd{(} \hlstr{"abc"}
         \hlstd{,} \hlkwc{profile} \hlstd{=} \hlkwd{c}\hlstd{(}\hlstr{"bc"}\hlstd{,}\hlstr{"ab"}\hlstd{,}\hlstr{"a"}\hlstd{,}\hlstr{"c"}\hlstd{)}
         \hlstd{,} \hlkwc{order} \hlstd{=} \hlkwa{NULL}
         \hlstd{,} \hlkwc{method} \hlstd{=} \hlstr{"linear"}
         \hlstd{)}\hlopt{$}\hlstd{strings}
\end{alltt}
\begin{verbatim}
##   originals tokenized
## 1       abc      ab c
\end{verbatim}
\end{kframe}
\end{knitrout}

Further, the current R implementation has a limitation when regular expressions
are used. The problem is that overlapping matches are not captured when using
regular expressions.\footnote{This restriction is an effect of the underlyingly
used ICU implementation of the Unicode Standard as implemented in R through the
package \textttf{stringi}.} Everything works as expected without regular
expressions, but there might be warnings/errors in case of \texttt{regex =
TRUE}. However, just as in the previous issue, this problem should only very
rarely (when at all) happen in natural language data.

The problem can be exemplified by a sequence <bbbb> in which a grapheme <bb>
should be matched. With the default \texttt{regex = FALSE} there are three
possible matches, but with \texttt{regex = TRUE} only the first two <b>'s or the
last two <b>'s are matched. The middle two <b>'s are not matched because they
overlap with the other matches. In the example below this leads to an error,
because the second <bb> is not matched. However, we have not been able to
produce a real example in any natural language in which this limitation might
lead to an error.

\begin{knitrout}\footnotesize
\definecolor{shadecolor}{rgb}{1, 1, 1}\color{fgcolor}\begin{kframe}
\begin{alltt}
\hlcom{# Everything perfect without regular expressions}
\hlkwd{tokenize}\hlstd{(} \hlstr{"abbb"}
        \hlstd{,} \hlkwc{profile} \hlstd{=} \hlkwd{c}\hlstd{(}\hlstr{"ab"}\hlstd{,}\hlstr{"bb"}\hlstd{)}
        \hlstd{,} \hlkwc{order} \hlstd{=} \hlkwa{NULL}
        \hlstd{,} \hlkwc{regex} \hlstd{=} \hlnum{FALSE}
        \hlstd{)}\hlopt{$}\hlstd{strings}
\end{alltt}
\begin{verbatim}
##   originals tokenized
## 1      abbb     ab bb
\end{verbatim}
\begin{alltt}
        
\hlcom{# Matching with regular expressions does not catch overlap}
\hlkwd{tokenize}\hlstd{(} \hlstr{"abbb"}
        \hlstd{,} \hlkwc{profile} \hlstd{=} \hlkwd{c}\hlstd{(}\hlstr{"ab"}\hlstd{,}\hlstr{"bb"}\hlstd{)}
        \hlstd{,} \hlkwc{order} \hlstd{=} \hlkwa{NULL}
        \hlstd{,} \hlkwc{regex} \hlstd{=} \hlnum{TRUE}
        \hlstd{)}\hlopt{$}\hlstd{strings}
\end{alltt}


{\ttfamily\noindent\color{warningcolor}{\#\# Warning in tokenize("{}abbb"{}, profile = c("{}ab"{}, "{}bb"{}), order = NULL, regex = TRUE): \\\#\# There were unknown characters found in the input data.\\\#\# Check output\$errors for a table with all problematic strings.}}\begin{verbatim}
##   originals tokenized
## 1      abbb    ab ⁇ ⁇
\end{verbatim}
\end{kframe}
\end{knitrout}


\section{Recipes online}
\label{use-cases}

We provide several use cases online -- what we refer to as \textit{recipes} -- that illustrate the applications of orthography profiles using our implementations in Python and R.\footnote{\url{https://github.com/unicode-cookbook/recipes}} Here we briefly describe these use cases and we encourage users to try them out using Git and Jupyter Notebooks.

First, as we discussed above, we provide a basic tutorial on how to use the Python \texttt{segments}\footnote{\url{https://pypi.python.org/pypi/segments}} and R \texttt{qlcData}\footnote{\url{https://github.com/cysouw/qlcData}} libraries. This recipe simply shows the basic functions of each library to get you started.\footnote{\url{https://github.com/unicode-cookbook/recipes/tree/master/Basics}}

The two recipes using the Python \texttt{segments} package include a tutorial on how to segment graphemes in IPA text:

\begin{itemize}
	\item https://github.com/unicode-cookbook/recipes/tree/master/JIPA
\end{itemize}
	
\noindent and an example of how to create an orthography profile to tokenize fieldwork data from a large comparative wordlist.

\begin{itemize}
	\item https://github.com/unicode-cookbook/recipes/tree/master/Dogon
\end{itemize}

\noindent The JIPA recipes uses excerpts from \textit{The North Wind and the Sun} passages from the Illustrations of the IPA published in the Journal of the International Phonetic Alphabet. Thus the recipe shows how a user might tokenize IPA proper. The Dogon recipe uses fieldwork data from the Dogon languages of Mali language documentation project.\footnote{\url{http://dogonlanguages.org/}} This recipe illustrates how a user might tokenize fieldwork data from numerous linguists using different transcription practices by defining these practices with an orthography profile to make the output unified and comparable.

The two recipes using the R \texttt{qlcData} library include a use case for tokenizing wordlist data from the Automated Similarity Judgment Program (ASJP):\footnote{\url{http://asjp.clld.org/}}

\begin{itemize}
	\item https://github.com/unicode-cookbook/recipes/tree/master/ASJP
\end{itemize}

\noindent and for tokenizing a corpus of text in Dutch orthography:

\begin{itemize}
	\item https://github.com/unicode-cookbook/recipes/tree/master/Dutch
\end{itemize}

\noindent The ASJP use case shows how to download the full set of ASJP wordlists, to combine them into a single large CSV file, and to tokenize the ASJP orthography. The Dutch use case takes as input the 10K corpus for Dutch (``nld'') from the Leipzig Corpora Collection,\footnote{\url{http://wortschatz.uni-leipzig.de/en/download/}} which is then cleaned and tokenized with an orthography profile that captures the intricacies of Dutch orthography.

\section{Closing words}

In closing, we hope that these rather elaborate musings on writing systems, Unicode and the IPA will help readers appreciate the progress that has been made over the last decades, but also acknowledge the many pitfalls that are still lurking under the surface. But mainly we hope that our proposals point towards a way forward in sharing scientific data, interpretations and analyses in a more transparent manner.

GitHub (or any similar platform) provides a platform for sharing scientific results and it also promotes a means for scientific replicability of results. Not only is all our code on Github, we actually used Github to collaboratively write this book and we will continue to use it to make corrections and update the book when it will become necessary.

Moreover, we find that in cases where the scientists are building tools for analysis, open repositories and data help to ensure that what you see is what you get.

% \include{chapters/use_cases}
% knit_child("chapters/codetesting.tex")


%%%%%%%%%%%%%%%%%%%%%%%%%%%%%%%%%%%%%%%%%%%%%%%%%%%%
%%%                                              %%%
%%%             Backmatter                       %%%
%%%                                              %%%
%%%%%%%%%%%%%%%%%%%%%%%%%%%%%%%%%%%%%%%%%%%%%%%%%%%%

% There is normally no need to change the backmatter section
\input{backmatter.tex}
\end{document}

% you can create your book by running
% xelatex lsp-skeleton.tex
%
% you can also try a simple 
% make
% on the commandline
